% Options for packages loaded elsewhere
\PassOptionsToPackage{unicode}{hyperref}
\PassOptionsToPackage{hyphens}{url}
\documentclass[
]{article}
\usepackage{enumitem}
\usepackage{xcolor}
\usepackage{amsmath,amssymb}
\setcounter{secnumdepth}{-\maxdimen} % remove section numbering
\usepackage{iftex}
\ifPDFTeX
  \usepackage[T1]{fontenc}
  \usepackage[utf8]{inputenc}
  \usepackage{textcomp} % provide euro and other symbols
\else % if luatex or xetex
  \usepackage{unicode-math} % this also loads fontspec
  \defaultfontfeatures{Scale=MatchLowercase}
  \defaultfontfeatures[\rmfamily]{Ligatures=TeX,Scale=1}
\fi
\usepackage{lmodern}
\ifPDFTeX\else
  % xetex/luatex font selection
\fi
% Use upquote if available, for straight quotes in verbatim environments
\IfFileExists{upquote.sty}{\usepackage{upquote}}{}
\IfFileExists{microtype.sty}{% use microtype if available
  \usepackage[]{microtype}
  \UseMicrotypeSet[protrusion]{basicmath} % disable protrusion for tt fonts
}{}
\makeatletter
\@ifundefined{KOMAClassName}{% if non-KOMA class
  \IfFileExists{parskip.sty}{%
    \usepackage{parskip}
  }{% else
    \setlength{\parindent}{0pt}
    \setlength{\parskip}{6pt plus 2pt minus 1pt}}
}{% if KOMA class
  \KOMAoptions{parskip=half}}
\makeatother
\usepackage{longtable,booktabs,array}
\newcounter{none} % for unnumbered tables
\usepackage{calc} % for calculating minipage widths
% Correct order of tables after \paragraph or \subparagraph
\usepackage{etoolbox}
\makeatletter
\patchcmd\longtable{\par}{\if@noskipsec\mbox{}\fi\par}{}{}
\makeatother
% Allow footnotes in longtable head/foot
\IfFileExists{footnotehyper.sty}{\usepackage{footnotehyper}}{\usepackage{footnote}}
\makesavenoteenv{longtable}
\usepackage{graphicx}
\makeatletter
\newsavebox\pandoc@box
\newcommand*\pandocbounded[1]{% scales image to fit in text height/width
  \sbox\pandoc@box{#1}%
  \Gscale@div\@tempa{\textheight}{\dimexpr\ht\pandoc@box+\dp\pandoc@box\relax}%
  \Gscale@div\@tempb{\linewidth}{\wd\pandoc@box}%
  \ifdim\@tempb\p@<\@tempa\p@\let\@tempa\@tempb\fi% select the smaller of both
  \ifdim\@tempa\p@<\p@\scalebox{\@tempa}{\usebox\pandoc@box}%
  \else\usebox{\pandoc@box}%
  \fi%
}
% Set default figure placement to htbp
\def\fps@figure{htbp}
\makeatother
\ifLuaTeX
  \usepackage{luacolor}
  \usepackage[soul]{lua-ul}
\else
  \usepackage{soul}
\fi
\setlength{\emergencystretch}{3em} % prevent overfull lines
\providecommand{\tightlist}{%
  \setlength{\itemsep}{0pt}\setlength{\parskip}{0pt}}
\usepackage{bookmark}
\IfFileExists{xurl.sty}{\usepackage{xurl}}{} % add URL line breaks if available
\urlstyle{same}
\hypersetup{
  hidelinks,
  pdfcreator={LaTeX via pandoc}}

\author{}
\date{}

\begin{document}

\includegraphics[width=8.5in,height=11.00005in]{images/media/image1.png}

\textbf{\hfill\break
}

OpenStax Anatomy and Physiology 2e

Guided Lecture Notes and Study Guide

\section{}\label{section}

Chapter 1 \hyperref[section-1]{2}

Chapter 2 \hyperref[chapter-2]{7}

Chapter 3 \hyperref[chapter-3]{17}

Chapter 4 \hyperref[section-2]{28}

Chapter 5 \hyperref[section-3]{36}

Chapter 6 \hyperref[section-4]{45}

Chapter 7 \hyperref[section-5]{55}

Chapter 8 \hyperref[section-6]{67}

Chapter 9 \hyperref[chapter-9]{70}

Chapter 10 \hyperref[chapter-10]{78}

Chapter 11 \hyperref[section-7]{86}

Chapter 12 \hyperref[section-8]{92}

Chapter 13 \hyperref[section-9]{96}

Chapter 14 \hyperref[section-10]{101}

Chapter 15 \hyperref[section-11]{109}

Chapter 16 \hyperref[section-12]{113}

Chapter 17 \hyperref[section-13]{121}

Chapter 18 \hyperref[section-17]{131}

Chapter 19 \hyperref[section-18]{142}

Chapter 20 \hyperref[section-19]{164}

Chapter 21 \hyperref[section-20]{188}

Chapter 22 \hyperref[section-23]{200}

Chapter 23 \hyperref[section-24]{210}

Chapter 24 \hyperref[section-25]{225}

Chapter 25 \hyperref[section-26]{240}

Chapter 26 \hyperref[section-27]{254}

Chapter 27
\hyperref[this-figure-shows-the-different-organs-in-the-male-reproductive-system.-the-top-panel-shows-the-side-view-of-a-man-and-an-uncircumcised-and-a-circumcised-penis.-the-bottom-panel-shows-the-lateral-view-of-the-male-reproductive-system-and-the-major-parts-are-labeled.chapter-27]{263}

Chapter 28 \hyperref[section-29]{274}

\section{}\label{section-1}

\section{\texorpdfstring{Chapter 1 }{Chapter 1 }}\label{chapter-1}

I. Overview of Anatomy and Physiology

a. Define anatomy:

b. Anatomy is broken down into 4 different branches:

i. Macroscopic or gross anatomy:

ii. Microscopic anatomy:

iii. Regional anatomy:

iv. Systemic anatomy:

1. Which of the types of anatomy are specialization areas?

2. Which of the types of anatomy are approaches of studying anatomy?

3. How are all the branches of anatomy similar?

c. Define physiology:

d. Define homeostasis:

II. Structural Organization of Human Body

a. Levels of organization in the human body order from smallest to most
complex:

i. Subatomic particles:

ii. Atoms:

iii. Molecules:

iv. Organelles:

v. Cells:

vi. Tissues:

vii. Organs:

viii. Organ system:

ix. Organisms:

x. Biosphere:

b. Compare and contrast chemical levels of organization from the human
body's level of organization.

c. A pure substance or \_\_\_\_\_\_\_\_\_\_\_\_\_\_\_ contain atoms,
which are made up of protons, \_\_\_\_\_\_\_\_\_\_\_, and
\_\_\_\_\_\_\_\_\_\_\_\_\_\_\_ or subatomic particles.

d. Cells make-up \_\_\_\_\_\_\_\_\_, \_\_\_\_\_\_\_\_\_\_\_ make-up
organs, \_\_\_\_\_\_\_\_\_\_\_\_ make-up organ systems.

e. The human body is made up of multiple different body systems. Organ
systems are made up of \_\_\_\_\_\_\_\_\_ that work together.

f. Organ Systems of the human body:

{\def\LTcaptype{none} % do not increment counter
\begin{longtable}[]{@{}
  >{\raggedright\arraybackslash}p{(\linewidth - 4\tabcolsep) * \real{0.3244}}
  >{\raggedright\arraybackslash}p{(\linewidth - 4\tabcolsep) * \real{0.3116}}
  >{\raggedright\arraybackslash}p{(\linewidth - 4\tabcolsep) * \real{0.3148}}@{}}
\toprule\noalign{}
\endhead
\bottomrule\noalign{}
\endlastfoot
\textbf{Organ System} & \textbf{Organs involved} & \textbf{Functions} \\
Integumentary & Hair, skin and nails & \\
Muscular & & \\
Skeletal & & \\
Endocrine & & \\
Nervous & & \\
Cardiovascular & & \\
Lymphatic & & \\
Respiratory & & \\
Digestive & & \\
Urinary & & \\
Male Reproductive & & \\
Female Reproductive & & \\
\end{longtable}
}

III. Functions of Human Life

a. Laws of thermodynamic

i. First law:

b. Metabolism= \_\_\_\_\_\_\_\_\_\_\_\_\_\_\_ +
\_\_\_\_\_\_\_\_\_\_\_\_\_\_\_

c. Define catabolism:

d. Define anabolism:

e. The cellular energy currency is \_\_\_\_\_\_\_\_\_\_\_
\_\_\_\_\_\_\_\_\_\_\_\_ \_\_\_\_\_\_\_\_\_\_\_\_\_\_.

IV. Requirements for human life

a. Compare and contrast oxygen versus carbon dioxide for human life.

b. Nutrients

i. Water:

ii. Micronutrients:

iii. Energy-Yielding and Body Building Nutrients:

c. Optimal temperature range for human body

i. When would controlled hypothermia be used?

ii. How does it aid in medical treatment?

d. Atmospheric pressure

i. Pressure helps blood gas by

e. Decompression Sickness:

i. Acclimation:

ii. Adaptation:

V. Homeostasis

a. Negative feedback:

i. Example.

b. Positive feedback:

i. Example.

VI. Anatomical Terminology

\includegraphics[width=4.875in,height=4.29167in,alt={This illustration shows an anterior and posterior view of the human body. The cranial region encompasses the upper part of the head while the facial region encompasses the lower half of the head beginning below the ears. The eyes are referred to as the ocular region. The cheeks are referred to as the buccal region. The ears are referred to as the auricle or otic region. The nose is referred to as the nasal region. The chin is referred to as the mental region. The neck is referred to as the cervical region. The trunk of the body contains, from superior to inferior, the thoracic region encompassing the chest, the mammary region encompassing each breast, the abdominal region encompassing the stomach area, the coxal region encompassing the belt line, and the pubic region encompassing the area above the genitals. The umbilicus, or naval, is located at the center of the abdomen. The pelvis and legs contain, from superior to inferior, the inguinal or groin region between the legs and the genitals, the pubic region surrounding the genitals, the femoral region encompassing the thighs, the patellar region encompassing the knee, the crural region encompassing the lower leg, the tarsal region encompassing the ankle, the pedal region encompassing the foot and the digital/phalangeal region encompassing the toes. The great toe is referred to as the hallux. The regions of the upper limbs, from superior to inferior, are the axillary region encompassing the armpit, the brachial region encompassing the upper arm, the antecubital region encompassing the front of the elbow, the antebrachial region encompassing the forearm, the carpal region encompassing the wrist, the palmar region encompassing the palm, and the digital/phalangeal region encompassing the fingers. The thumb is referred to as the pollux. The posterior view contains, from superior to inferior, the cervical region encompassing the neck, the dorsal region encompassing the upper back and the lumbar region encompassing the lower back. The regions of the back of the arms, from superior to inferior, include the cervical region encompassing the neck, acromial region encompassing the shoulder, the brachial region encompassing the upper arm, the olecranal region encompassing the back of the elbow, the antebrachial region encompasses the back of the arm, and the manual region encompassing the palm of the hand. The posterior regions of the legs, from superior to inferior, include the gluteal region encompassing the buttocks, the femoral region encompassing the thigh, the popliteus region encompassing the back of the knee, the sural region encompassing the back of the lower leg, and the plantar region encompassing the sole of the foot. Some regions are combined into larger regions. These include the trunk, which is a combination of the thoracic, mammary, abdominal, naval, and coxal regions. The cephalic region is a combination of all of the head regions. The upper limb region is a combination of all of the arm regions. The lower limb region is a combination of all of the leg regions.}]{images/media/image3.png}

c. Directions

\includegraphics[width=2.67708in,height=2.70833in,alt={This illustration shows two diagrams: one of a side view of a female and the other of an anterior view of a female. Each diagram shows directional terms using double-sided arrows. The cranial-distal arrow runs vertically behind the torso and lower abdomen. The cranial arrow is pointing toward the head while the caudal arrow is pointing toward the tail bone. The posterior/anterior arrow is running horizontally through the back and chest. The posterior or dorsal arrow is pointing toward the back while the anterior, or ventral arrow, is pointing toward the abdomen. On the anterior view, the proximal/distal arrow is on the right arm. The proximal arrow is pointing up toward the shoulder while the distal arrow is pointing down toward the hand. The lateral-medial arrow is a horizontal arrow on the abdomen. The medial arrow is pointing toward the navel while the lateral arrow is pointing away from the body to the right. Right refers to the right side of the woman's body from her perspective while left refers to the left side of the woman's body from her perspective."\textgreater{} \textless img src="/resources/e356a7098a9911af5537206fe5da40e8d44c32d1" data-media-type="image/jpg" alt="This illustration shows two diagrams: one of a side view of a female and the other of an anterior view of a female. Each diagram shows directional terms using double-sided arrows. The cranial-distal arrow runs vertically behind the torso and lower abdomen. The cranial arrow is pointing toward the head while the caudal arrow is pointing toward the tail bone. The posterior/anterior arrow is running horizontally through the back and chest. The posterior or dorsal arrow is pointing toward the back while the anterior, or ventral arrow, is pointing toward the abdomen. On the anterior view, the proximal/distal arrow is on the right arm. The proximal arrow is pointing up toward the shoulder while the distal arrow is pointing down toward the hand. The lateral-medial arrow is a horizontal arrow on the abdomen. The medial arrow is pointing toward the navel while the lateral arrow is pointing away from the body to the right. Right refers to the right side of the woman's body from her perspective while left refers to the left side of the woman's body from her perspective." width="450" id="6"\textgreater{} \textless/span\textgreater{}}]{images/media/image4.png}

d. Body planes

\includegraphics[width=2.51042in,height=2.58333in,alt={This illustration shows a female viewed from her right, front side. The anatomical planes are depicted as blue rectangles passing through the woman's body. The frontal or coronal plane enters through the right side of the body, passes through the body, and exits from the left side. It divides the body into front (anterior) and back (posterior) halves. The sagittal plane enters through the back and emerges through the front of the body. It divides the body into right and left halves. The transverse plane passes through the body perpendicular to the frontal and sagittal planes. This plane is a cross section which divides the body into upper and lower halves."\textgreater{} \textless img src="/resources/0cd3b78d2a4d73ef5d67a4070c4e3c4280de99db" data-media-type="image/jpg" alt="This illustration shows a female viewed from her right, front side. The anatomical planes are depicted as blue rectangles passing through the woman's body. The frontal or coronal plane enters through the right side of the body, passes through the body, and exits from the left side. It divides the body into front (anterior) and back (posterior) halves. The sagittal plane enters through the back and emerges through the front of the body. It divides the body into right and left halves. The transverse plane passes through the body perpendicular to the frontal and sagittal planes. This plane is a cross section which divides the body into upper and lower halves.}]{images/media/image5.png}

e. Cavities and serous membranes

\includegraphics[width=5.6866in,height=3.00901in,alt={This illustration shows a lateral and anterior view of the body and highlights the body cavities with different colors. The cranial cavity is a large, bean-shaped cavity filling most of the upper skull where the brain is located. The vertebral cavity is a very narrow, thread-like cavity running from the cranial cavity down the entire length of the spinal cord. Together the cranial cavity and vertebral cavity can be referred to as the dorsal body cavity. The thoracic cavity consists of three cavities that fill the interior area of the chest. The two pleural cavities are situated on both sides of the body, anterior to the spine and lateral to the breastbone. The superior mediastinum is a wedge-shaped cavity located between the superior regions of the two thoracic cavities. The pericardial cavity within the mediastinum is located at the center of the chest below the superior mediastinum. The pericardial cavity roughly outlines the shape of the heart. The diaphragm divides the thoracic and the abdominal cavities. The abdominal cavity occupies the entire lower half of the trunk, anterior to the spine. Just under the abdominal cavity, anterior to the buttocks, is the pelvic cavity. The pelvic cavity is funnel shaped and is located inferior and anterior to the abdominal cavity. Together the abdominal and pelvic cavity can be referred to as the abdominopelvic cavity while the thoracic, abdominal, and pelvic cavities together can be referred to as the ventral body cavity."\textgreater{} \textless img src="/resources/31b2c67b26e49d57f60ae5244033f5b4a0e7da6b" data-media-type="image/jpg" alt="This illustration shows a lateral and anterior view of the body and highlights the body cavities with different colors. The cranial cavity is a large, bean-shaped cavity filling most of the upper skull where the brain is located. The vertebral cavity is a very narrow, thread-like cavity running from the cranial cavity down the entire length of the spinal cord. Together the cranial cavity and vertebral cavity can be referred to as the dorsal body cavity. The thoracic cavity consists of three cavities that fill the interior area of the chest. The two pleural cavities are situated on both sides of the body, anterior to the spine and lateral to the breastbone. The superior mediastinum is a wedge-shaped cavity located between the superior regions of the two thoracic cavities. The pericardial cavity within the mediastinum is located at the center of the chest below the superior mediastinum. The pericardial cavity roughly outlines the shape of the heart. The diaphragm divides the thoracic and the abdominal cavities. The abdominal cavity occupies the entire lower half of the trunk, anterior to the spine. Just under the abdominal cavity, anterior to the buttocks, is the pelvic cavity. The pelvic cavity is funnel shaped and is located inferior and anterior to the abdominal cavity. Together the abdominal and pelvic cavity can be referred to as the abdominopelvic cavity while the thoracic, abdominal, and pelvic cavities together can be referred to as the ventral body cavity.}]{images/media/image6.png}

\includegraphics[width=3.08333in,height=1.59375in,alt={This illustration has two parts. Part A shows the abdominopelvic regions. These regions divide the abdomen into nine squares. The upper right square is the right hypochondriac region and contains the base of the right ribs. The upper left square is the left hypochondriac region and contains the base of the left ribs. The epigastric region is the upper central square and contains the bottom edge of the liver as well as the upper areas of the stomach. The diaphragm curves like an upside down U over these three regions. The central right region is called the right lumbar region and contains the ascending colon and the right edge of the small intestines. The central square contains the transverse colon and the upper regions of the small intestines. The left lumbar region contains the left edge of the transverse colon and the left edge of the small intestine. The lower right square is the right iliac region and contains the right pelvic bones and the ascending colon. The lower left square is the left iliac region and contains the left pelvic bone and the lower left regions of the small intestine. The lower central square contains the bottom of the pubic bones, upper regions of the bladder and the lower region of the small intestine. Part B shows four abdominopelvic quadrants. The right upper quadrant (RUQ) includes the lower right ribs, right side of the liver, and right side of the transverse colon. The left upper quadrant (LUQ) includes the lower left ribs, stomach, and upper left area of the transverse colon. The right lower quadrant (RLQ) includes the right half of the small intestines, ascending colon, right pelvic bone and upper right area of the bladder. The left lower quadrant (LLQ) contains the left half of the small intestine and left pelvic bone."\textgreater{} \textless img src="/resources/11b9d612bd3cb9db6996846022f8359664d9bcd8" data-media-type="image/jpg" alt="This illustration has two parts. Part A shows the abdominopelvic regions. These regions divide the abdomen into nine squares. The upper right square is the right hypochondriac region and contains the base of the right ribs. The upper left square is the left hypochondriac region and contains the base of the left ribs. The epigastric region is the upper central square and contains the bottom edge of the liver as well as the upper areas of the stomach. The diaphragm curves like an upside down U over these three regions. The central right region is called the right lumbar region and contains the ascending colon and the right edge of the small intestines. The central square contains the transverse colon and the upper regions of the small intestines. The left lumbar region contains the left edge of the transverse colon and the left edge of the small intestine. The lower right square is the right iliac region and contains the right pelvic bones and the ascending colon. The lower left square is the left iliac region and contains the left pelvic bone and the lower left regions of the small intestine. The lower central square contains the bottom of the pubic bones, upper regions of the bladder and the lower region of the small intestine. Part B shows four abdominopelvic quadrants. The right upper quadrant (RUQ) includes the lower right ribs, right side of the liver, and right side of the transverse colon. The left upper quadrant (LUQ) includes the lower left ribs, stomach, and upper left area of the transverse colon. The right lower quadrant (RLQ) includes the right half of the small intestines, ascending colon, right pelvic bone and upper right area of the bladder. The left lower quadrant (LLQ) contains the left half of the small intestine and left pelvic bone." width="550" id="12"\textgreater{} \textless/span\textgreater{}}]{images/media/image7.png}

VII. Medical imaging

a. X-rays were discovered by

i. Function of x-rays

b. Computed tomography:

i. When are CAT scans requested?

c. Magnetic resonance imagining (MRI):

d. Positron emission tomography (PET):

e. Ultrasonography:

\section{\texorpdfstring{Chapter 2 }{Chapter 2 }}\label{chapter-2}

\begin{enumerate}
\def\labelenumi{\Roman{enumi}.}
\item
  Elements and Atoms

  \begin{enumerate}
  \def\labelenumii{\Alph{enumii}.}
  \item
    Define:

    \begin{enumerate}
    \def\labelenumiii{\arabic{enumiii}.}
    \item
      Matter:
    \item
      Mass:

      \begin{enumerate}
      \def\labelenumiv{\alph{enumiv})}
      \item
        Compare human mass and weight.
      \end{enumerate}
    \item
      Elements:

      \begin{enumerate}
      \def\labelenumiv{\alph{enumiv})}
      \item
        Atom is the smallest quantity of an element
      \item
        Nucleus contains protons and neutrons
      \item
        How will electron shells affect molecule formation?
      \item
        Atomic number:
      \item
        Atomic mass:
      \end{enumerate}
    \end{enumerate}
  \end{enumerate}
\end{enumerate}

\begin{quote}
\includegraphics[width=2.36874in,height=3.59896in]{images/media/image8.png}

\includegraphics[width=5.40773in,height=4.20313in,alt={The top panel of this figure shows two electrons orbiting around the nucleus of a Helium atom. The bottom panel of this figure shows a cloud of electrons surrounding the nucleus of a Helium atom." width="285" id="6"\textgreater{}}]{images/media/image9.png}
\end{quote}

\begin{enumerate}
\def\labelenumi{\arabic{enumi}.}
\setcounter{enumi}{3}
\item
  Compounds:

  \begin{enumerate}
  \def\labelenumii{\alph{enumii})}
  \item
    Glucose is composed of carbon, \_\_\_\_\_\_\_ and \_\_\_\_\_\_\_\_\_
  \end{enumerate}
\item
  Atoms:
\item
  Subatomic particles:

  \begin{enumerate}
  \def\labelenumii{\alph{enumii})}
  \item
    Carbon 12, 13 and 14 have the same \_\_\_\_\_\_\_\_\_\_\_ and
    different numbers of \_\_\_\_\_\_\_\_\_\_\_\_\_\_
  \item
    Radioactive isotopes:

    \begin{enumerate}
    \def\labelenumiii{(\arabic{enumiii})}
    \item
      Used in medicine for PET scans
    \end{enumerate}
  \item
    Electrons circle the nucleus of an atom in
    \_\_\_\_\_\_\_\_\_\_\_\_\_\_\_\_\_\_\_
  \end{enumerate}
\end{enumerate}

\begin{enumerate}
\def\labelenumi{\Alph{enumi}.}
\setcounter{enumi}{1}
\item
  Define radiology:

  \begin{enumerate}
  \def\labelenumii{\arabic{enumii}.}
  \item
    How are radioactive isotopes used medically? What are the pros and
    cons of the use? What training would it take to use radioactive
    isotopes for medical use?
  \end{enumerate}
\end{enumerate}

\begin{enumerate}
\def\labelenumi{\Roman{enumi}.}
\setcounter{enumi}{1}
\item
  Chemical Bonds

  \begin{enumerate}
  \def\labelenumii{\Alph{enumii}.}
  \item
    Valence electrons are important for compound formation because
    \_\_\_\_\_\_\_\_\_\_\_\_\_\_\_\_\_\_\_\_\_\_\_\_\_\_\_\_\_\_\_\_\_\_\_\_\_\_\_\_\_\_\_\_\_\_\_\_\_\_\_\_\_\_\_\_\_\_\_.
  \item
    Compounds are formed by \_\_\_\_\_\_\_\_\_\_\_\_\_

    \begin{enumerate}
    \def\labelenumiii{\arabic{enumiii}.}
    \item
      Elements that lose or gain electrons become
      \_\_\_\_\_\_\_\_\_\_\_.

      \begin{enumerate}
      \def\labelenumiv{\alph{enumiv})}
      \item
        What type of bond can be formed as a result?

        \begin{enumerate}
        \def\labelenumv{(\arabic{enumv})}
        \item
          K+ is an example of what type of ion? What causes the K
          element to become an ion in this case?
        \item
          Can more than one \_\_\_\_\_\_\_\_\_\_ be lost or gained to
          produce ions?
        \item
          Mg++ is an example of what type of ion?
        \item
          Mg++ will form a \_\_\_\_\_\_\_\_\_\_\_\_\_\_ bond with
          \_\_\_\_\_\_\_\_\_\_\_\_ to form a compound
        \item
        \end{enumerate}
      \end{enumerate}
    \item
      The strongest \_\_\_\_\_\_\_\_\_\_ is a \_\_\_\_\_\_\_\_\_\_\_
      \_\_\_\_\_\_\_\_\_\_\_.

      \begin{enumerate}
      \def\labelenumiv{\alph{enumiv})}
      \item
        Electrons are \_\_\_\_\_\_\_\_\_\_\_ between elements that form
        molecules.
      \item
        Define polar:
      \item
        How are polar and nonpolar bonds different and similar?
      \item
        Water is an example of a \_\_\_\_\_\_\_\_\_ bond. Is polarity
        intramolecular or intermolecular?
      \item
        Draw a water molecule to illustrate polarity.
      \item
        What happens to polar molecules in water?
      \item
        What happens to nonpolar molecules in water?
      \end{enumerate}
    \item
      The weakest of the bonds is the \_\_\_\_\_\_\_\_\_\_
      \_\_\_\_\_\_\_\_\_\_\_

      \begin{enumerate}
      \def\labelenumiv{\alph{enumiv})}
      \item
        List all the types of bond found in a glass of water. Draw
        examples of the different bonds
      \item
        Looking at the periodic table, how does electronegativity change
        in elements from left to right?
      \item
        Does this affect hydrogen bonds? How?
      \end{enumerate}
    \end{enumerate}
  \end{enumerate}
\item
  Chemical Reactions

  \begin{enumerate}
  \def\labelenumii{\Alph{enumii}.}
  \item
    Define the following and give and example of each

    \begin{enumerate}
    \def\labelenumiii{\arabic{enumiii}.}
    \item
      Kinetic energy:
    \item
      Potential energy:
    \item
      Chemical energy:
    \item
      Mechanical energy:
    \item
      Radiant energy:
    \item
      Electrical energy:
    \end{enumerate}
  \item
    Chemical reactions are broken down into substance that enter the
    reaction or the \_\_\_\_\_\_\_\_\_\_ and substances that are
    produced in the reaction or the \_\_\_\_\_\_\_\_\_.

    \begin{enumerate}
    \def\labelenumiii{\arabic{enumiii}.}
    \item
      Types of chemical reactions

      \begin{enumerate}
      \def\labelenumiv{\alph{enumiv})}
      \item
        Synthesis reaction

        \begin{enumerate}
        \def\labelenumv{(\arabic{enumv})}
        \item
          A + B→
        \end{enumerate}
      \item
        Decomposition reaction

        \begin{enumerate}
        \def\labelenumv{(\arabic{enumv})}
        \item
          AB→
        \end{enumerate}
      \item
        Exchange reaction

        \begin{enumerate}
        \def\labelenumv{(\arabic{enumv})}
        \item
          A +BC→
        \item
          AB + CD→
        \end{enumerate}
      \item
        What causes chemical reactions to reverse?
      \end{enumerate}
    \end{enumerate}
  \item
    Properties that affect the rate of a chemical reaction

    \begin{enumerate}
    \def\labelenumiii{\arabic{enumiii}.}
    \item
      Properties of the reactants

      \begin{enumerate}
      \def\labelenumiv{\alph{enumiv})}
      \item
        List properties
      \end{enumerate}
    \item
      Temperature

      \begin{enumerate}
      \def\labelenumiv{\alph{enumiv})}
      \item
        Higher temperature affect?
      \item
        Lower temperature affect?
      \end{enumerate}
    \item
      Concentration and Pressure

      \begin{enumerate}
      \def\labelenumiv{\alph{enumiv})}
      \item
        Abundance?
      \item
        Amount of space?
      \end{enumerate}
    \item
      Enzyme and other catalysts

      \begin{enumerate}
      \def\labelenumiv{\alph{enumiv})}
      \item
        Explain how enzymes lower activation energy of a chemical
        reaction using the figures below:
      \end{enumerate}
    \end{enumerate}
  \end{enumerate}
\end{enumerate}

\begin{quote}
\includegraphics[width=6.5in,height=2.625in,alt={The left panel shows a graph of energy versus progress of reaction in the absence of enzymes. The right panel shows the graph in the presence of enzymes."}]{images/media/image10.png}
\end{quote}

\begin{enumerate}
\def\labelenumi{\Roman{enumi}.}
\setcounter{enumi}{3}
\item
  Inorganic Compounds Essential to Human Functioning

  \begin{enumerate}
  \def\labelenumii{\Alph{enumii}.}
  \item
    Compare and contrast inorganic and organic compounds
  \item
    Properties of water:

    \begin{enumerate}
    \def\labelenumiii{\arabic{enumiii}.}
    \item
      Solution= \_\_\_\_\_\_\_\_\_\_\_ + \_\_\_\_\_\_\_\_\_\_\_\_\_
    \item
      Give an example of hydrophobic molecules
    \item
      Give an example of hydrophilic molecules
    \item
      \_\_\_\_\_\_\_\_\_\_\_\_\_\_\_\_reaction used to break molecules
      into monomers and \_\_\_\_\_\_\_\_\_\_\_\_\_\_ reaction used to
      build molecules
    \end{enumerate}
  \end{enumerate}
\end{enumerate}

\begin{quote}
\includegraphics[width=4.19091in,height=2.17188in]{images/media/image11.png}
\end{quote}

\begin{enumerate}
\def\labelenumi{\Alph{enumi}.}
\setcounter{enumi}{2}
\item
  How are solute concentrations measured?

  \begin{enumerate}
  \def\labelenumii{\arabic{enumii}.}
  \item
    Define molarity

    \begin{enumerate}
    \def\labelenumiii{\alph{enumiii})}
    \item
      What is the molarity of NaCl?
    \item
      What is the difference between a mole and molarity?
    \end{enumerate}
  \item
    Define colloid:
  \item
    How is a suspension different from a colloid?
  \end{enumerate}
\item
  Salts are formed when \_\_\_\_\_\_\_\_\_\_ form \_\_\_\_\_\_\_\_\_\_
  bonds. Salts also \_\_\_\_\_\_\_\_\_\_\_\_\_\_\_\_\_ in water
  resulting in separate \_\_\_\_\_\_\_\_\_\_.
\item
  Define pH:

  \begin{enumerate}
  \def\labelenumii{\arabic{enumii}.}
  \item
    What is the pH of human blood?

    \begin{enumerate}
    \def\labelenumiii{\alph{enumiii})}
    \item
      Is human blood acid or basic?
    \item
      \_\_\_\_\_\_\_\_\_\_ causes the human blood to become
      \_\_\_\_\_\_\_\_\_\_\_. However, homeostatic mechanisms
      \_\_\_\_\_\_\_\_\_\_\_\_\_\_\_\_\_\_\_\_\_\_\_\_\_\_back to the
      normal range.
    \item
      \_\_\_\_\_\_\_\_\_\_\_\_ solution contains a weak acid and its
      conjugate base and work to help maintain
      \_\_\_\_\_\_\_\_\_\_\_\_\_.
    \end{enumerate}
  \item
    Compare and contrast acids, bases and salts.
  \item
    What is the pH of stomach acid?
  \item
    How will acid react in water, use HCl as the example?
  \item
    How is the reaction of HCl in water different from
    H\textsubscript{3}PO\textsubscript{4} ?
  \item
    How does a base pH differ from an acid?
  \item
    Define base:
  \end{enumerate}
\end{enumerate}

\begin{enumerate}
\def\labelenumi{\Roman{enumi}.}
\setcounter{enumi}{4}
\item
  Organic Compounds Essential to Human Functioning

  \begin{enumerate}
  \def\labelenumii{\Alph{enumii}.}
  \item
    Organic compounds typically contain which elements?

    \begin{enumerate}
    \def\labelenumiii{\arabic{enumiii}.}
    \item
      How many bonds will carbon form? Why
    \end{enumerate}
  \end{enumerate}
\end{enumerate}

\subsection{Functional Groups Important in Human
Physiology}\label{functional-groups-important-in-human-physiology}

\begin{longtable}[]{@{}
  >{\raggedright\arraybackslash}p{(\linewidth - 4\tabcolsep) * \real{0.1333}}
  >{\raggedright\arraybackslash}p{(\linewidth - 4\tabcolsep) * \real{0.1183}}
  >{\centering\arraybackslash}p{(\linewidth - 4\tabcolsep) * \real{0.7007}}@{}}
\caption{Table 2.1}\tabularnewline
\toprule\noalign{}
\endfirsthead
\endhead
\bottomrule\noalign{}
\endlastfoot
\begin{minipage}[t]{\linewidth}\raggedright
\begin{quote}
\textbf{Functional group}
\end{quote}
\end{minipage} & \begin{minipage}[t]{\linewidth}\raggedright
\begin{quote}
\textbf{Structural formula}
\end{quote}
\end{minipage} & \begin{minipage}[t]{\linewidth}\centering
\begin{quote}
\textbf{Importance}
\end{quote}
\end{minipage} \\
\begin{minipage}[t]{\linewidth}\raggedright
\begin{quote}
Hydroxyl
\end{quote}
\end{minipage} & \begin{minipage}[t]{\linewidth}\raggedright
\begin{quote}
---O---H
\end{quote}
\end{minipage} & \begin{minipage}[t]{\linewidth}\centering
\begin{quote}
Hydroxyl groups are polar. They are components of all four types of
organic compounds discussed in this chapter. They are involved in
dehydration synthesis and hydrolysis reactions.
\end{quote}
\end{minipage} \\
\begin{minipage}[t]{\linewidth}\raggedright
\begin{quote}
Carboxyl
\end{quote}
\end{minipage} & \begin{minipage}[t]{\linewidth}\raggedright
\begin{quote}
O---C---OH
\end{quote}
\end{minipage} & \begin{minipage}[t]{\linewidth}\centering
\begin{quote}
Carboxyl groups are found within fatty acids, amino acids, and many
other acids.
\end{quote}
\end{minipage} \\
\begin{minipage}[t]{\linewidth}\raggedright
\begin{quote}
Amino
\end{quote}
\end{minipage} & \begin{minipage}[t]{\linewidth}\raggedright
\begin{quote}
---N---H2
\end{quote}
\end{minipage} & \begin{minipage}[t]{\linewidth}\centering
\begin{quote}
Amino groups are found within amino acids, the building blocks of
proteins.
\end{quote}
\end{minipage} \\
\begin{minipage}[t]{\linewidth}\raggedright
\begin{quote}
Methyl
\end{quote}
\end{minipage} & \begin{minipage}[t]{\linewidth}\raggedright
\begin{quote}
---C---H3
\end{quote}
\end{minipage} & \begin{minipage}[t]{\linewidth}\centering
\begin{quote}
Methyl groups are found within amino acids.
\end{quote}
\end{minipage} \\
\begin{minipage}[t]{\linewidth}\raggedright
\begin{quote}
Phosphate
\end{quote}
\end{minipage} & \begin{minipage}[t]{\linewidth}\raggedright
\begin{quote}
---P---O 2--

4
\end{quote}
\end{minipage} & \begin{minipage}[t]{\linewidth}\centering
\begin{quote}
Phosphate groups are found within phospholipids and nucleotides.
\end{quote}
\end{minipage} \\
\end{longtable}

\begin{enumerate}
\def\labelenumi{\Alph{enumi}.}
\setcounter{enumi}{1}
\item
  What are Four organic compounds essential for human functioning?

  \begin{enumerate}
  \def\labelenumii{\arabic{enumii}.}
  \item
    Carbohydrates

    \begin{enumerate}
    \def\labelenumiii{\alph{enumiii})}
    \item
      Chemical formula for a generic carbohydrate is
      \_\_\_\_\_\_\_\_\_\_\_\_\_\_\_\_
    \item
      Another name for carbohydrate?

      \begin{enumerate}
      \def\labelenumiv{(\arabic{enumiv})}
      \item
        Maltose is an example of which type of carbohydrate?
      \item
        Glucose is an example of which type of carbohydrate?
      \end{enumerate}
    \item
      Are starches considered to be carbohydrates? Why or why not?
    \item
      Why is glucose stored as glycogen?
    \item
      Carbohydrates are used for cellular fuel in the form of
      \_\_\_\_\_\_\_\_\_\_ \_\_\_\_\_\_\_\_\_\_ \_\_\_\_\_\_\_\_\_\_\_
    \end{enumerate}
  \item
    Lipids

    \begin{enumerate}
    \def\labelenumiii{\alph{enumiii})}
    \item
      Are made of repeating units of hydrogens and carbons

      \begin{enumerate}
      \def\labelenumiv{(\arabic{enumiv})}
      \item
        Are hydrocarbons polar or nonpolar? Explain.
      \end{enumerate}
    \item
      Fat or \_\_\_\_\_\_\_\_\_\_\_\_\_\_\_\_\_ are commonly found in
      tissues and consist of a \_\_\_\_\_\_\_\_\_\_\_\_ core and
      \_\_\_\_\_\_\_\_\_\_\_\_\_\_ fatty acids
    \item
      Which reaction is illustrated below to form triglyceride from
      monomers?
    \item
      Saturated vs. unsaturated how are they different?
    \end{enumerate}
  \end{enumerate}
\end{enumerate}

\begin{quote}
\includegraphics[width=4.86979in,height=2.58247in,alt={"This diagram shows the chain structures of a saturated and an unsaturated fatty acid}]{images/media/image12.png}
\end{quote}

\begin{enumerate}
\def\labelenumi{(\arabic{enumi})}
\item
  Which is solid at room temperature?
\item
  Plant oils contain which type(s) of fatty acid?
\item
  The best diets will be low in \_\_\_\_\_\_\_\_\_\_\_\_ fats.
\item
  Why are trans fats harmful? Give examples of foods with the most trans
  fats?
\end{enumerate}

\begin{enumerate}
\def\labelenumi{\alph{enumi})}
\setcounter{enumi}{4}
\item
  Human cell membranes are made up of \_\_\_\_\_\_\_\_\_\_\_\_\_, which
  are \_\_\_\_\_\_\_\_\_\_\_\_\_\_\_\_. This means they contain
  \_\_\_\_\_\_\_\_\_\_\_\_\_\_ or ``water loving'' polar head and
  \_\_\_\_\_\_\_\_\_\_\_\_\_ or ``water hating'' nonpolar tails
\item
  Describe a typical steroid. Give an example.
\end{enumerate}

\_\_\_\_\_\_\_\_\_\_\_\_ are cell signaling molecules and can aid in
regulation of blood pressure and inflammation when
\_\_\_\_\_\_\_\_\_\_\_\_\_\_\_\_\_\_\_\_\_\_\_\_\_\_\_\_\_\_\_\_

\begin{enumerate}
\def\labelenumi{\arabic{enumi}.}
\setcounter{enumi}{2}
\item
  Proteins

  \begin{enumerate}
  \def\labelenumii{\alph{enumii})}
  \item
    Monomers of proteins are \_\_\_\_\_\_\_\_\_\_ linked together via
    \_\_\_\_\_\_\_\_\_\_\_ bonds, which are types of
    \_\_\_\_\_\_\_\_\_\_\_ bonds.
  \item
    Structural components of an amino acid:
  \end{enumerate}
\end{enumerate}

\begin{quote}
\includegraphics[width=3.83854in,height=2.81187in]{images/media/image13.png}

\includegraphics[width=3.32266in,height=2.41146in]{images/media/image14.png}
\end{quote}

\begin{enumerate}
\def\labelenumi{\alph{enumi})}
\setcounter{enumi}{2}
\item
  Hierarchy of protein shape

  \begin{enumerate}
  \def\labelenumii{(\arabic{enumii})}
  \item
    Primary:
  \item
    Secondary:
  \item
    Tertiary:
  \item
    Quaternary:
  \end{enumerate}
\end{enumerate}

\includegraphics[width=5.03646in,height=4.28297in,alt={This figure shows the secondary structure of peptides. The top panel shows a straight chain, the middle panel shows an alpha-helix and a beta sheet. The bottom panel shows the tertiary structure and fully folded protein.}]{images/media/image15.png}

\begin{enumerate}
\def\labelenumi{\alph{enumi})}
\setcounter{enumi}{3}
\item
  Enzymes

  \begin{enumerate}
  \def\labelenumii{(\arabic{enumii})}
  \item
    Describe the jigsaw puzzle model:
  \end{enumerate}
\end{enumerate}

\begin{quote}
\includegraphics[width=6.51395in,height=2.43229in,alt={This image shows the steps in which an enzyme can act. The substrate is shown binding to the enzyme, forming a product, and the detachment of the product."}]{images/media/image16.png}
\end{quote}

\begin{enumerate}
\def\labelenumi{\alph{enumi})}
\setcounter{enumi}{4}
\item
  How are enzymes affected by reactions?
\end{enumerate}

\begin{enumerate}
\def\labelenumi{\arabic{enumi}.}
\setcounter{enumi}{3}
\item
  Nucleic Acid

  \begin{enumerate}
  \def\labelenumii{\alph{enumii})}
  \item
    DNA:
  \item
    \includegraphics[width=5.52604in,height=4.10911in]{images/media/image17.png}RNA:

    \begin{enumerate}
    \def\labelenumiii{(\arabic{enumiii})}
    \item
      How is it different from DNA?
    \item
      What are the major types?

      \begin{enumerate}
      \def\labelenumiv{(\alph{enumiv})}
      \item
        What are the functions of the types?
      \end{enumerate}
    \item
      Which nucleotide is present in DNA but not RNA?
    \end{enumerate}
  \item
    ATP:

    \begin{enumerate}
    \def\labelenumiii{(\arabic{enumiii})}
    \item
      Define phosphorylation:
    \end{enumerate}
  \end{enumerate}
\end{enumerate}

\begin{quote}
\includegraphics[width=3.56771in,height=3.12174in]{images/media/image18.png}
\end{quote}

\section{\texorpdfstring{Chapter 3 }{Chapter 3 }}\label{chapter-3}

\begin{enumerate}
\def\labelenumi{\Roman{enumi}.}
\item
  The Cell Membrane
\end{enumerate}

Function:

\begin{enumerate}
\def\labelenumi{\arabic{enumi}.}
\item
  Plasma membrane is \_\_\_\_\_\_\_\_\_\_\_ \_\_\_\_\_\_\_\_\_\_\_\_\_,
  allowing for certain molecules to enter the cells while others cannot.
\end{enumerate}

Cell membrane characteristics:

\begin{enumerate}
\def\labelenumi{\arabic{enumi}.}
\setcounter{enumi}{1}
\item
  Phobicity

  \begin{enumerate}
  \def\labelenumii{\alph{enumii})}
  \item
    Hydrophobic
  \item
    Hydrophilic
  \item
    Amphipathic

    \begin{enumerate}
    \def\labelenumiii{(\arabic{enumiii})}
    \item
      Membrane lipid
    \end{enumerate}
  \end{enumerate}
\end{enumerate}

\begin{quote}
\includegraphics[width=2.40033in,height=2.86979in,alt={This diagram shows the structure of a phospholipid. The hydrophilic head group is shown as a pink sphere and the two tails are shown as yellow rectangles.}]{images/media/image19.png}
\end{quote}

\begin{enumerate}
\def\labelenumi{\arabic{enumi}.}
\setcounter{enumi}{2}
\item
  Fluids

  \begin{enumerate}
  \def\labelenumii{\alph{enumii})}
  \item
    Intracellular fluid (ICF) is found in the \_\_\_\_\_\_\_\_\_\_\_
  \item
    Interstitial fluid (IF) is found in the \_\_\_\_\_\_\_\_\_\_\_\_\_
  \item
    Extracellular fluid (ECF) is found in the \_\_\_\_\_\_\_\_\_
  \end{enumerate}
\item
  Membrane proteins

  \begin{enumerate}
  \def\labelenumii{\alph{enumii})}
  \item
    Integral proteins:

    \begin{enumerate}
    \def\labelenumiii{(\arabic{enumiii})}
    \item
      Channel:
    \item
      Recognition proteins

      \begin{enumerate}
      \def\labelenumiv{(\alph{enumiv})}
      \item
        Receptor:

        \begin{enumerate}
        \def\labelenumv{(\roman{enumv})}
        \item
          Ligand:
        \end{enumerate}
      \end{enumerate}
    \item
      Glycoproteins:

      \begin{enumerate}
      \def\labelenumiv{(\alph{enumiv})}
      \item
        Glycocalyx:
      \end{enumerate}
    \end{enumerate}
  \item
    Peripheral proteins:
  \end{enumerate}
\item
  Membrane transport

  \begin{enumerate}
  \def\labelenumii{\alph{enumii})}
  \item
    Passive transport:

    \begin{enumerate}
    \def\labelenumiii{(\arabic{enumiii})}
    \item
      Simple diffusion:

      \begin{enumerate}
      \def\labelenumiv{(\alph{enumiv})}
      \item
        How is spraying Lysol in one corner of your room an example of
        simple diffusion?
      \end{enumerate}
    \item
      Osmosis:

      \begin{enumerate}
      \def\labelenumiv{(\alph{enumiv})}
      \item
        How is osmosis different from simple diffusion?
      \end{enumerate}
    \end{enumerate}
  \end{enumerate}
\end{enumerate}

\begin{quote}
\includegraphics[width=3.11335in,height=1.76563in,alt={This figure shows the diffusion of water through osmosis. The left panel shows a beaker with water and different solute concentrations. A semipermeable membrane is present in the middle of the beaker. In the right panel, the water concentration is higher to the right of the semipermeable membrane.}]{images/media/image20.png}
\end{quote}

\begin{enumerate}
\def\labelenumi{(\alph{enumi})}
\setcounter{enumi}{1}
\item
  Tonicity

  \begin{enumerate}
  \def\labelenumii{(\roman{enumii})}
  \item
    If a patient comes into the hospital dehydrated, what percentage
    saline should be in the IV?
  \end{enumerate}
\end{enumerate}

\begin{quote}
\includegraphics[width=3.39242in,height=1.75521in,alt={This image shows how a red blood cell responds to the tonicity of solution. The left panel shows the hypertonic case, the middle panel shows the isotonic case and the right panel shows the hypotonic case.}]{images/media/image21.png}
\end{quote}

\begin{enumerate}
\def\labelenumi{(\arabic{enumi})}
\setcounter{enumi}{2}
\item
  Facilitated diffusion:
\item
  Filtration:
\end{enumerate}

\begin{enumerate}
\def\labelenumi{\alph{enumi})}
\setcounter{enumi}{1}
\item
  Active transport:

  \begin{enumerate}
  \def\labelenumii{(\arabic{enumii})}
  \item
    Protein pump

    \begin{enumerate}
    \def\labelenumiii{(\alph{enumiii})}
    \item
      Na\textsuperscript{+}/K\textsuperscript{+} pump:

      \begin{enumerate}
      \def\labelenumiv{(\roman{enumiv})}
      \item
        How many sodium ions are moved out of the cell?
      \item
        How many potassium ions are moved into the cell?
      \item
        What is the charge inside of the cell after?
      \item
        Electrical gradient:
      \end{enumerate}
    \end{enumerate}
  \end{enumerate}
\end{enumerate}

\begin{quote}
\includegraphics[width=6.5in,height=3.08333in,alt={This diagram shows many sodium potassium pumps embedded in the membrane. Potassium is pumped into the cytoplasm and sodium is pumped out of the cytoplasm." width="550" id="18"\textgreater{}}]{images/media/image22.png}
\end{quote}

\begin{enumerate}
\def\labelenumi{(\alph{enumi})}
\setcounter{enumi}{1}
\item
  Endocytosis:

  \begin{enumerate}
  \def\labelenumii{(\roman{enumii})}
  \item
    Phagocytosis:
  \item
    Pinocytosis:
  \item
    Receptor mediated endocytosis:

    \begin{enumerate}
    \def\labelenumiii{(\alph{enumiii})}
    \item
      Give examples of when receptor mediated endocytosis used instead
      of other forms of endocytosis?
    \end{enumerate}
  \end{enumerate}
\item
  Exocytosis:
\end{enumerate}

\begin{enumerate}
\def\labelenumi{(\arabic{enumi})}
\setcounter{enumi}{1}
\item
  How is transport affected by cystic fibrosis?

  \begin{enumerate}
  \def\labelenumii{(\alph{enumii})}
  \item
    Which ion is mostly affected? How?
  \item
    What are the results for patients?
  \end{enumerate}
\end{enumerate}

\begin{enumerate}
\def\labelenumi{\Roman{enumi}.}
\setcounter{enumi}{1}
\item
  The Cytoplasm and Cellular Organelles
\end{enumerate}

Cytosol is the \_\_\_\_\_\_ \_\_\_\_\_\_\_ portion of the cytoplasmic
compartment where \_\_\_\_\_\_\_\_\_\_ \_\_\_\_\_\_\_\_\_\_\_\_\_\_\_\_
occur.

\begin{enumerate}
\def\labelenumi{\arabic{enumi}.}
\item
  Describe how the cell is similar to a factor producing and shipping
  products. Use each organelle as a part of the factory.
\end{enumerate}

{\def\LTcaptype{none} % do not increment counter
\begin{longtable}[]{@{}
  >{\raggedright\arraybackslash}p{(\linewidth - 2\tabcolsep) * \real{0.5000}}
  >{\raggedright\arraybackslash}p{(\linewidth - 2\tabcolsep) * \real{0.5000}}@{}}
\toprule\noalign{}
\endhead
\bottomrule\noalign{}
\endlastfoot
\textbf{Organelle} & \textbf{Functions} \\
nucleus & \\
mitochondria & \\
lysosomes & \\
Smooth Endoplasmic reticulum & \\
Rough endoplasmic reticulum & \\
peroxisome & \\
vacuole & \\
Golgi apparatus & \\
nucleolus & \\
ribosomes & \\
\end{longtable}
}

Define apoptosis:

How are free radicals involved in this process? Include oxidative stress
in your explanation.

\includegraphics[width=4.28125in,height=3.39269in,alt={This diagram shows an animal cell with all the intracellular organelles labeled.}]{images/media/image23.png}

Cytoskeleton

\begin{enumerate}
\def\labelenumi{\arabic{enumi}.}
\setcounter{enumi}{1}
\item
  Compare and contrast cilia and flagella.
\item
  Centrioles or \_\_\_\_\_\_\_\_\_\_\_\_\_\_ are cytoskeletal elements
  found near the nucleus and are made up of tubulin. Myosin is an
  example of this cytoskeletal element.
\item
  Actin or \_\_\_\_\_\_\_\_\_\_\_\_\_, which are thinner cytoskeletal
  elements. The diameters of these elements range from 3-6 nm. Actin is
  found in \_\_\_\_\_\_\_\_\_\_\_\_\_\_\_ tissue.
\item
  \_\_\_\_\_\_\_\_\_\_\_\_\_ \_\_\_\_\_\_\_\_\_\_\_\_\_\_ have a
  diameter of 10 nm. These elements are made up of
  \_\_\_\_\_\_\_\_\_\_\_\_\_. These elements help to anchor
  \_\_\_\_\_\_\_\_\_\_\_\_\_ within the cell to other cells by
  \_\_\_\_\_\_\_\_\_\_\_ junctions.
\end{enumerate}

\begin{enumerate}
\def\labelenumi{\Roman{enumi}.}
\setcounter{enumi}{2}
\item
  The Nucleus and DNA Replication
\end{enumerate}

The computer of the cell is the \_\_\_\_\_\_\_\_\_\_\_\_\_\_\_\_.

Red blood cells are anucleate because
\_\_\_\_\_\_\_\_\_\_\_\_\_\_\_\_\_\_\_\_\_\_\_\_\_\_\_\_\_\_\_\_\_\_\_\_\_\_\_\_\_\_\_\_\_\_\_.

The membrane surrounding the nucleus is the \_\_\_\_\_\_\_\_\_\_\_\_\_
\_\_\_\_\_\_\_\_\_\_\_\_\_\_, which is a double layer containing
\_\_\_\_\_\_\_\_ \_\_\_\_\_\_\_ that allow for movement of molecules in
and out of the nucleus.

How is the nucleoplasm different from the cytoplasm?

Prepare a venn diagram to compare and contrast chromatid, chromatin and
chromosome.

Define nucleosome:

DNA strands run \_\_\_\_\_\_\_\_\_\_\_\_ to each other, with
the\_\_\_\_\_\_\_\_strand from 5' to 3' and the\_\_\_\_\_\_\_\_\_ strand
from 3'to 5'.

Prior to cell division, \_\_\_\_\_\_\_\_\_\_ \_\_\_\_\_\_\_\_\_\_\_\_
occurs to duplicate the genetic material in the cells and prepare for
new daughter cell formation.

\subsection{THE THREE MAIN STEPS OF DNA
REPLICATION:}\label{the-three-main-steps-of-dna-replication}

\begin{enumerate}
\def\labelenumi{\arabic{enumi}.}
\item
  Initiation:
\item
  Elongation:
\item
  Termination:
\end{enumerate}

\includegraphics[width=6.5in,height=4.01389in,alt={This image shows the process of DNA replication. A chromosome is shown expanding into the original template DNA and unwinding at the replication fork. The helicase is present at the replication fork. DNA polymerases are shown adding nucleotides to the leading and lagging strands.}]{images/media/image24.png}

\begin{enumerate}
\def\labelenumi{\Roman{enumi}.}
\setcounter{enumi}{3}
\item
  Protein Synthesis
\end{enumerate}

\textbf{True/False}. One gene=one protein.

The active products of gene expression are
\_\_\_\_\_\_\_\_\_\_\_\_\_\_\_\_.

The full complement of proteins in the cell are the
\_\_\_\_\_\_\_\_\_\_\_\_\_\_\_\_.

Triplet:

\includegraphics[width=6.5in,height=3.16667in,alt={This diagram shows the translation of RNA into proteins. A DNA template strand is shown to become an RNA strand through transcription. Then the RNA strand undergoes translation and becomes proteins.}]{images/media/image25.png}

\subsubsection{Central Dogma: DNA-transcribed to mRNA-translated to
protein}\label{central-dogma-dna-transcribed-to-mrna-translated-to-protein}

Since mRNA is transcribed from DNA, does that mean they have the same
language or monomers?

\begin{enumerate}
\def\labelenumi{\arabic{enumi}.}
\item
  What are those monomers?
\end{enumerate}

Is the triplet code used as the template for protein synthesis?

\begin{enumerate}
\def\labelenumi{\arabic{enumi}.}
\setcounter{enumi}{1}
\item
  What is a codon? When is it used?
\item
  How are DNA and RNA polymerases different?
\end{enumerate}

\paragraph{Stages of transcriptions}\label{stages-of-transcriptions}

\begin{enumerate}
\def\labelenumi{\arabic{enumi}.}
\setcounter{enumi}{3}
\item
  Initiation:
\item
  Elongation:
\item
  Termination:
\end{enumerate}

\includegraphics[width=6.5in,height=3.73611in,alt={In this diagram, RNA polymerase is shown transcribing a DNA template strand into its corresponding RNA transcript.}]{images/media/image26.png}

\subsubsection{Where does transcription occur in eukaryotic
cells?}\label{where-does-transcription-occur-in-eukaryotic-cells}

Post transcriptional modification of mRNA includes:

\begin{enumerate}
\def\labelenumi{\arabic{enumi}.}
\setcounter{enumi}{6}
\item
  Spliceosome:
\item
  Introns:
\item
  Exons:
\item
  Poly A tail:
\item
  5' cap:
\end{enumerate}

Types of RNA in eukaryotic cells:

\begin{enumerate}
\def\labelenumi{\arabic{enumi}.}
\setcounter{enumi}{11}
\item
  mRNA:
\item
  rRNA:
\item
  tRNA:
\end{enumerate}

\paragraph{Stages of Translation:}\label{stages-of-translation}

\begin{quote}
\includegraphics[width=3.44792in,height=4.625in,alt={"In this diagram, a pre-mRNA transcript is shown in the top of a flowchart. This pre-mRNA transcript contains introns and exons. In the next step, the intron is in a structure called the spliceosome. In the last step, the intron is shown separated from the spliced RNA." }]{images/media/image27.png}
\end{quote}

\begin{enumerate}
\def\labelenumi{\arabic{enumi}.}
\setcounter{enumi}{14}
\item
  Initiation:
\item
  Elongation:
\item
  Termination:
\item
  Where does translation occur?
\item
  When will free ribosomes complete protein synthesis versus the rough
  endoplasmic reticulum?
\item
  How will the anticodon result in an amino acid's production?
\end{enumerate}

\includegraphics[width=4.08125in,height=3.90996in,alt={This figure shows a schematic of a cell where transcription from DNA to mRNA takes place inside the nucleus and translation from mRNA to protein takes place in the cytoplasm." }]{images/media/image28.png}

\begin{enumerate}
\def\labelenumi{\Roman{enumi}.}
\setcounter{enumi}{4}
\item
  Cell Growth and Division
\end{enumerate}

Gametes:

\begin{enumerate}
\def\labelenumi{\arabic{enumi}.}
\item
  haploid, n, cells contain how many chromosomes?
\end{enumerate}

Somatic cells:

\begin{enumerate}
\def\labelenumi{\arabic{enumi}.}
\setcounter{enumi}{1}
\item
  Diploid, 2n, cells contain how many chromosomes?
\item
  Two copies of a single chromosome is a
  \_\_\_\_\_\_\_\_\_\_\_\_\_\_\_\_\_\_ pair.
\end{enumerate}

Cell cycle:

\begin{enumerate}
\def\labelenumi{\arabic{enumi}.}
\setcounter{enumi}{3}
\item
  Mitosis

  \begin{enumerate}
  \def\labelenumii{\alph{enumii})}
  \item
    Interphase:

    \begin{enumerate}
    \def\labelenumiii{(\arabic{enumiii})}
    \item
      G1
    \item
      S phase
    \item
      G2
    \item
      G0
    \item
      M
    \end{enumerate}
  \item
    Prophase:

    \begin{enumerate}
    \def\labelenumiii{(\arabic{enumiii})}
    \item
      Prometaphase:
    \end{enumerate}
  \item
    Metaphase:
  \item
    Anaphase:
  \item
    Telophase:
  \end{enumerate}
\item
  Cytokinesis:
\item
  Checkpoints:

  \begin{enumerate}
  \def\labelenumii{\alph{enumii})}
  \item
    CDK- \_\_\_\_\_\_\_\_\_\_ \_\_\_\_\_\_\_\_\_\_\_
    \_\_\_\_\_\_\_\_\_\_\_\_\_,molecules that work together with cyclins
    to \_\_\_\_\_\_\_\_\_\_\_\_\_\_\_\_\_\_\_\_\_\_\_\_\_\_\_.
  \end{enumerate}
\end{enumerate}

\subsubsection{Cancer and homeostatic
imbalance}\label{cancer-and-homeostatic-imbalance}

\begin{enumerate}
\def\labelenumi{\arabic{enumi}.}
\setcounter{enumi}{6}
\item
  Cell cycle control=\_\_\_\_\_\_\_\_\_\_\_\_\_\_\_
  \_\_\_\_\_\_\_\_\_\_\_\_\_\_
\item
  Cyclins:

  \begin{enumerate}
  \def\labelenumii{\alph{enumii})}
  \item
    Proto-oncogene products:
  \end{enumerate}
\item
  Cancer

  \begin{enumerate}
  \def\labelenumii{\alph{enumii})}
  \item
    Define proto-oncogene:
  \item
    Tumor suppressor genes:
  \end{enumerate}
\end{enumerate}

\begin{enumerate}
\def\labelenumi{\Roman{enumi}.}
\setcounter{enumi}{5}
\item
  Cellular Differentiation
\end{enumerate}

Stem cells

\begin{enumerate}
\def\labelenumi{\arabic{enumi}.}
\item
  Most undifferentiated stem cells \_\_\_\_\_\_\_\_\_\_\_\_\_\_\_, or
  first embryonic stem cell
\item
  Stem cells that become any human tissue
  \_\_\_\_\_\_\_\_\_\_\_\_\_\_\_\_\_\_\_.
\item
  \_\_\_\_\_\_\_\_\_\_\_\_\_\_\_\_\_ can differentiate into any type of
  cell of a certain type.

  \begin{enumerate}
  \def\labelenumii{\alph{enumii})}
  \item
    \_\_\_\_\_\_\_\_\_\_\_\_\_\_\_\_\_\_\_ cells limited to becoming
    only a few different cell types.
  \item
    \_\_\_\_\_\_\_\_\_\_\_\_\_\_\_ already specialized and can only
    become more of the same type of cell.
  \end{enumerate}
\end{enumerate}

Define differentiation:

\begin{quote}
\includegraphics[width=2.5706in,height=2.93125in]{images/media/image29.png}
\end{quote}

\section{}\label{section-2}

\section{\texorpdfstring{Chapter 4 }{Chapter 4 }}\label{chapter-4}

\begin{enumerate}
\def\labelenumi{\Roman{enumi}.}
\item
  Types of tissue

  \begin{enumerate}
  \def\labelenumii{\Alph{enumii}.}
  \item
    What are the differences between cells and tissues?
  \item
    Four major types in the human body:

    \begin{enumerate}
    \def\labelenumiii{\arabic{enumiii}.}
    \item
      Epithelial tissue

      \begin{enumerate}
      \def\labelenumiv{\alph{enumiv})}
      \item
        General locations:
      \item
        Major functions:
      \end{enumerate}
    \item
      Connective tissue:

      \begin{enumerate}
      \def\labelenumiv{\alph{enumiv})}
      \item
        General locations:
      \item
        Major functions:
      \end{enumerate}
    \item
      Muscle tissue:

      \begin{enumerate}
      \def\labelenumiv{\alph{enumiv})}
      \item
        General locations:
      \item
        Major functions:
      \end{enumerate}
    \item
      Nervous tissue:

      \begin{enumerate}
      \def\labelenumiv{\alph{enumiv})}
      \item
        General locations:
      \item
        Major functions:
      \end{enumerate}
    \end{enumerate}
  \end{enumerate}
\end{enumerate}

\includegraphics[width=5.99269in,height=4.63021in,alt={This is a two column-table containing both text and illustrations. The left column is titled germ layer while the right column is titled ``Gives rise to.'' The germ layer in the first row is ectoderm. Ectoderm gives rise to epidermis, glands on the skin, some cranial bones, the pituitary and adrenal medulla, the nervous system, the tissue between the cheeks and gums, and the anus. This row contains three pictures. The leftmost picture illustrates several layers of yellow, oval-shaped skin cells with purple nuclei. The middle diagram shows a neuron, which is a yellow, star shaped cell with finger like branches at its corners. The neuron also has a purple nucleus and a yellow tube that connects to the bottom of the cell. The right image in this row shows a brown pigment cell embedded at the bottom layer of several skin cells. It is secreting dark-colored pigment into the skin cells from tentacle-like projections. The germ layer in the second row is mesoderm. Mesoderm gives rise to connective tissues, bone, cartilage, blood, the endothelium of blood vessels, muscle, synovial membranes, serous membranes that line body cavities, the kidneys, and the lining of the gonads. Five images are given in this row to illustrate. The leftmost image is cardiac muscle, which is cylindrical and curved. There are many open spaces between neighboring cardiac muscles. The next image shows skeletal muscle, which is a series of closely stacked cylinders with well defined horizontal striping. The middle image shows three tubule cells of the kidney, which are square shaped and contain a brown nucleus. The fourth image shows a series of red blood cells, which are red and saucer shaped with a slight depression at the center. The fifth image shows smooth muscles which are tightly packed, diamond shaped cells with oval-shaped nuclei. Endoderm gives rise to the lining of the airways and digestive system (except the mouth and distal part of digestive system). Also, the rectum and anal canal, digestive glands, endocrine glands, and adrenal cortex all develop from endoderm. The leftmost image in this row shows a lung cell, which is a large, purple, trapezoid-shaped cell. The middle image shows a pair of thyroid cells, which are rectangle-shaped with the upper edge of each cell having a row of finger like projections, similar in appearance to carpet. The rightmost image in this row shows a pancreatic cell, which is large and wedge-shaped. The pancreatic cell has small indentations throughout its cell membrane." width="520" id="6"\textgreater{}}]{images/media/image30.png}

\begin{enumerate}
\def\labelenumi{\Alph{enumi}.}
\setcounter{enumi}{2}
\item
  Which germ layers give rise to the different types of tissue?
\item
  How are tissue membranes different from tissues?

  \begin{enumerate}
  \def\labelenumii{\arabic{enumii}.}
  \item
    Epithelial membrane:

    \begin{enumerate}
    \def\labelenumiii{\alph{enumiii})}
    \item
      Which tissues make up the membrane?
    \item
      Examples of membrane locations.
    \item
      Three types of epithelial membranes:

      \begin{enumerate}
      \def\labelenumiv{(\arabic{enumiv})}
      \item
        \_\_\_\_\_\_\_\_\_\_ membranes produces mucus which lines
        \_\_\_\_\_\_\_\_\_\_\_\_\_\_\_\_\_\_\_\_\_\_\_\_\_\_\_\_\_\_\_
      \item
        \_\_\_\_\_\_\_\_\_\_\_\_membranes line the
        \_\_\_\_\_\_\_\_\_\_\_\_\_ cavity of the body and contains
        \_\_\_\_\_\_\_\_\_\_\_\_\_\_ fluid.
      \item
        \_\_\_\_\_\_\_\_\_\_\_\_\_membranes is unique because
        \_\_\_\_\_\_\_\_\_\_\_\_\_\_\_\_\_\_\_\_\_\_\_\_\_\_\_\_\_\_\_\_\_\_\_
      \end{enumerate}
    \end{enumerate}
  \item
    Connective membrane:

    \begin{enumerate}
    \def\labelenumiii{\alph{enumiii})}
    \item
      Synovial membrane is found in
    \item
      How are synovial membranes different from other membranes?
    \end{enumerate}
  \end{enumerate}
\end{enumerate}

\begin{enumerate}
\def\labelenumi{\Roman{enumi}.}
\setcounter{enumi}{1}
\item
  Epithelial Tissue

  \begin{enumerate}
  \def\labelenumii{\Alph{enumii}.}
  \item
    Which embryonic layer will the epithelial cells differentiate from?
  \item
    The polarity of epithelial cells refers to the
    \_\_\_\_\_\_\_\_\_\_\_\_\_ surface which is the exposed surface and
    the \_\_\_\_\_\_\_\_\_\_\_\_\_ surface which is exposed to the
    underlying body structures.

    \begin{enumerate}
    \def\labelenumiii{\arabic{enumiii}.}
    \item
      \_\_\_\_\_\_\_\_\_\_ \_\_\_\_\_\_\_\_\_\_\_ is an important
      mixture of glycoproteins and collagen fibers. This allows
      epithelial cells to attach to \_\_\_\_\_\_\_\_\_\_\_\_\_\_\_\_\_
      tissue, which is always found under epithelial cells.
    \item
      Compare and contrast basal lamina and reticular lamina?
    \item
      Will you find blood vessels that innervate epithelial tissue? Why
      or why not?
    \end{enumerate}
  \item
    List general functions of epithelial tissue.

    \begin{enumerate}
    \def\labelenumiii{\arabic{enumiii}.}
    \item
      Give examples of locations where each function is prevalent.
    \item
      When would epithelial cells need cilia? Would they be flagellated?
      Why or why not?

      \begin{enumerate}
      \def\labelenumiv{\alph{enumiv})}
      \item
        Why are cilia important for breathing?
      \end{enumerate}
    \end{enumerate}
  \item
    Epithelial cells lack intercellular material, but are tightly
    connected via
    \_\_\_\_\_\_\_\_\_\_\_\_\_\_\_\_\_\_\_\_\_\_\_\_\_\_\_\_\_\_\_\_\_\_\_\_\_\_\_\_\_\_\_\_\_\_\_\_\_\_\_\_,
    which all aid in connecting cells and allowing for intercellular
    communication.

    \begin{enumerate}
    \def\labelenumiii{\arabic{enumiii}.}
    \item
      Cell polarity is further solidified by the
      \_\_\_\_\_\_\_\_\_\_\_\_ junctions.
    \item
      \_\_\_\_\_\_\_\_\_\_\_\_ junctions stabilize epithelial cells.
      How?

      \begin{enumerate}
      \def\labelenumiv{\alph{enumiv})}
      \item
        List and explain the differences between the types of
        \_\_\_\_\_\_\_\_\_ junctions.
      \item
        What role will cadherins play in cell junctions? How are
        cadherins different from integrins?
      \item
        Microfilaments present in the cells' cytoplasm aid in the
        \_\_\_\_\_\_\_ and \_\_\_\_\_\_\_\_\_\_\_ of tissue.
      \end{enumerate}
    \item
      Cells communicate by sending molecules through \_\_\_\_\_\_\_\_
      junctions.
    \end{enumerate}
  \item
    Epithelial cells are characterized by shape and cell number:

    \begin{enumerate}
    \def\labelenumiii{\arabic{enumiii}.}
    \item
      Three major cell shapes:

      \begin{enumerate}
      \def\labelenumiv{\alph{enumiv})}
      \item
        Fish scale like or \_\_\_\_\_\_\_\_\_\_\_\_.
      \item
        Cube shape or \_\_\_\_\_\_\_\_\_\_\_.
      \item
        Rectangular shape or taller than wide or \_\_\_\_\_\_\_\_\_\_\_.
      \end{enumerate}
    \item
      One layer of cells is considered \_\_\_\_\_\_\_\_\_\_\_\_\_\_.
    \item
      Two or more layers of cells is considered
      \_\_\_\_\_\_\_\_\_\_\_\_.
    \end{enumerate}
  \item
    Squamous epithelia

    \begin{enumerate}
    \def\labelenumiii{\arabic{enumiii}.}
    \item
      \_\_\_\_\_\_\_\_\_\_\_\_\_ lines lymphatic and cardiovascular
      systems. This tissue type is important for which feature in these
      two systems?
    \item
      Simple squamous epithelium also makes up the
      \_\_\_\_\_\_\_\_\_\_\_ membrane lining body cavities and internal
      organs.
    \item
      \_\_\_\_\_\_\_\_\_\_ squamous epithelium contains normal squamous
      cells at the \_\_\_\_\_\_\_\_\_ surface with
      \_\_\_\_\_\_\_\_\_\_\_\_\_ or \_\_\_\_\_\_\_\_\_\_\_ cells at the
      basal surface.

      \begin{enumerate}
      \def\labelenumiv{\alph{enumiv})}
      \item
        Where would you find this tissue type in the body? What would be
        the function?
      \item
        What is the best way to identify the differences?
      \item
        Which type of epithelium would make up the surface of the body?
        What are the functions of the tissue?
      \end{enumerate}
    \item
      Cuboidal epithelia

      \begin{enumerate}
      \def\labelenumiv{\alph{enumiv})}
      \item
        Major functions:
      \item
        Found?
      \item
        What are the differences between the simple and stratified
        layers?
      \end{enumerate}
    \item
      Columnar epithelia

      \begin{enumerate}
      \def\labelenumiv{\alph{enumiv})}
      \item
        \_\_\_\_\_\_\_\_\_\_\_\_ columnar found in the digestive tract
        and female reproductive tract. Why are some ciliated?

        \begin{enumerate}
        \def\labelenumv{(\arabic{enumv})}
        \item
          How does this aid in the reproductive process?
        \end{enumerate}
      \item
        How are simple columnar and pseudostratified columnar epithelium
        different?

        \begin{enumerate}
        \def\labelenumv{(\arabic{enumv})}
        \item
          Where would pseudostratified columnar epithelium be found in
          the human body?
        \item
          Why are \_\_\_\_\_\_\_\_\_ cells important?
        \item
          Are cilia needed on all columnar epithelial cells that contain
          goblet cells? Why or why not?
        \end{enumerate}
      \end{enumerate}
    \end{enumerate}
  \end{enumerate}
\end{enumerate}

\begin{quote}
\includegraphics[width=1.95573in,height=3.63299in,alt={This illustration shows a diagram of a goblet cell. The goblet cell is shaped roughly like an upside down vase. The enlarged end at the top contains six finger like projections labeled microvilli. Between the microvilli, secretary vesicles containing mucin are moving from the upper half of the cell toward the microvilli. Below the secretory vesicles are several rough endoplasmic reticula and an irregularly shaped Golgi apparatus with secretory vesicles budding off of it. The narrow, lower half of the cell contains the oval-shaped nucleus as well as a few mitochondria and segments of the endoplasmic reticulum." }]{images/media/image31.png}
\end{quote}

\begin{enumerate}
\def\labelenumi{\alph{enumi})}
\setcounter{enumi}{2}
\item
  Where would you find stratified columnar epithelium in the human body?
\end{enumerate}

\begin{enumerate}
\def\labelenumi{\arabic{enumi}.}
\setcounter{enumi}{5}
\item
  In the urinary system, \_\_\_\_\_\_\_\_\_\_\_\_ epithelium tissue is
  found. This tissue is important in aiding the
  \_\_\_\_\_\_\_\_\_\_\_\_\_\_\_\_\_\_\_\_\_\_\_ of the bladder.
\end{enumerate}

\begin{enumerate}
\def\labelenumi{\Alph{enumi}.}
\setcounter{enumi}{6}
\item
  Glandular Epithelia:

  \begin{enumerate}
  \def\labelenumii{\arabic{enumii}.}
  \item
    \_\_\_\_\_\_\_\_\_\_\_ glands or \_\_\_\_\_\_\_\_\_\_ are diffused
    through the bloodstream to their target cells. Examples:
  \item
    \_\_\_\_\_\_\_\_\_\_ glands release their content outside of the
    body such as breast milk, sweat and mucous.

    \begin{enumerate}
    \def\labelenumiii{\alph{enumiii})}
    \item
      These glands can be unicellular or multicellular. \_\_\_\_\_\_\_\_
      cells are examples of unicellular cells.
    \item
      \_\_\_\_\_\_\_\_ glands can either excrete their content via a
      \_\_\_\_\_\_\_\_\_\_ duct or \_\_\_\_\_\_\_\_\_\_\_.
    \end{enumerate}
  \end{enumerate}
\end{enumerate}

\includegraphics[width=6.08073in,height=4.69792in,alt={This table shows the different types of exocrine glands: alveolar (acinar) versus tubular and those with simple ducts versus compound ducts. Each diagram shows a single layer of columnar epithelial cells with a line of cells travelling along the surface of a tissue (surface epithelium) and then dipping into a hole in the tissue. The cells travel down the right side of the hole until they reach the bottom, then curve around the bottom of the hole and then travel up the left side. Finally, the cells emerge back onto the surface of the tissue. The surface epithelial cells are those that are on the surface of the tissue; the duct cells are those that line both walls of the hole. The gland cells are those that line the bottom of the hole. The shape of the hole differs in each gland. In the simple alvelolar (acinar) gland, the duct and gland cells are bulb shaped with the gland cells being the larger end of the bulb. Simple alveolar glands are not found in adults, as these represent an early developmental stage of simple, branched glands. In simple tubular glands, the duct and gland cells are U shaped. Simple tubular glands are found in the intestinal glands. In simple branched alveolar glands, the gland cells form three bulbs at the end of the duct, similar in appearance to a clover leaf. The sebaceous (oil) glands are examples of simple branched alveolar glands. In simple coiled tubular glands, the duct and gland cells form a U, however, the bottom of the U, which is all gland cells, is curved up to the right. Merocrine sweat glands are examples of simple coiled tubular glands. In simple branched tubular glands, the duct is very short and the gland cells divide into three lobes, similar in appearance to a bird's foot. The gastric glands of the stomach and mucous glands of the esophagus, tongue and duodenum are examples of simple branched tubular glands. Among the glands with compound ducts, compound alveolar (acinar) glands have three sets of clover leaf bulbs, for a total of six bulbs. Two of the clover leaf shaped structures extend parallel to the surface epithelium in opposite directions to each other. The third clover leaf extends down into the tissue, perpendicular to the surface. The duct is cross-shaped. The mammary glands are an example of compound alveolar glands. Compound tubular glands have a similar structure to compound alveolar glands. However, instead of three cloverleaf shaped bulbs, the compound tubular gland has three bird's foot shaped bulbs. The duct is also cross-shaped in the compound tubular gland. The mucous glands of the mouth and the bulbourethral glands of the male reproductive system are examples of compound tubular glands, which are also found in the seminiferous tubules of the testis. Compound tubuloalveolar glands are a hybrid between the compound alveolar gland and the compound tubular gland. The two sets of bulbs that run parallel to the surface are bird-foot shaped; however, the set of bulbs that runs perpendicularly below the surface is cloverleaf shaped. The salivary glands, glands of the respiratory passages and glands of the pancreas are all compound tubuloalveolar glands." width="550" id="13"\textgreater{}}]{images/media/image32.png}

\begin{enumerate}
\def\labelenumi{\alph{enumi})}
\setcounter{enumi}{2}
\item
  Modes of glandular secretion:
\end{enumerate}

\begin{quote}
\includegraphics[width=3.47656in,height=3.54278in,alt={These three diagrams show the three modes of secretion. All three diagrams show three orange cells in a line with attached to a basement membrane. Each cell has a large nucleus in its lower half. The upper half of each cell contains a Golgi apparatus, which appears like an upside down jellyfish. Yellow secretory vesicles are budding from the top end of the Golgi apparatus. Each vesicle contains several orange circles, which are the secreted substance. In merocrine secretion, the secretory vesicles travel to the top edge of the cells and release the secretion from the cell by melding with the cell membrane. In apocrine secretion, the top third of the cell, which contains the secretory vesicles, pinches in at the sides and then completely disconnects above the Golgi complex. The pinched off portion of the cell is the secretion, as it contains the majority of the secretory vesicles. In holocrine secretion, the upper third of the cell, just above the Golgi complex, forms many finger like projections. Each projection contains several vesicles. The tips of the projections that contain secretory vesicles bud off from the cell. In this method of secretion, the mature cell eventually dies and becomes the secretory product." width="400" id="15"\textgreater{}}]{images/media/image33.png}
\end{quote}

\begin{enumerate}
\def\labelenumi{\Roman{enumi}.}
\setcounter{enumi}{2}
\item
  Connective Tissue Supports and Protects

  \begin{enumerate}
  \def\labelenumii{\Alph{enumii}.}
  \item
    The most abundant type of tissue in the human body.
  \item
    Commonly found under \_\_\_\_\_\_\_\_\_\_ tissue.
  \item
    Distinguish by the presence of a \_\_\_\_\_\_\_\_\_\_\_\_ which
    contains extracellular substances and aids in tissue function.

    \begin{enumerate}
    \def\labelenumiii{\arabic{enumiii}.}
    \item
      The \_\_\_\_\_\_\_\_\_ substance is the major component of the
      matrix.
    \end{enumerate}
  \item
    Tissue types are listed below: List a functions and location
  \end{enumerate}
\end{enumerate}

{\def\LTcaptype{none} % do not increment counter
\begin{longtable}[]{@{}
  >{\raggedright\arraybackslash}p{(\linewidth - 4\tabcolsep) * \real{0.3333}}
  >{\raggedright\arraybackslash}p{(\linewidth - 4\tabcolsep) * \real{0.3333}}
  >{\raggedright\arraybackslash}p{(\linewidth - 4\tabcolsep) * \real{0.3333}}@{}}
\toprule\noalign{}
\endhead
\bottomrule\noalign{}
\endlastfoot
Connective tissue proper & Supportive connective tissue & Fluid
connective tissue \\
\begin{minipage}[t]{\linewidth}\raggedright
Loose connective tissue:

\begin{enumerate}
\def\labelenumi{\arabic{enumi}.}
\item
  Areolar

  \begin{enumerate}
  \def\labelenumii{\alph{enumii}.}
  \item
    Functions:
  \item
    Locations:
  \end{enumerate}
\item
  Adipose:

  \begin{enumerate}
  \def\labelenumii{\alph{enumii}.}
  \item
    Functions:
  \item
    Locations:
  \end{enumerate}
\item
  Reticular:

  \begin{enumerate}
  \def\labelenumii{\alph{enumii}.}
  \item
    Functions:
  \item
    locations:
  \end{enumerate}
\end{enumerate}
\end{minipage} & \begin{minipage}[t]{\linewidth}\raggedright
Cartilage:

\begin{enumerate}
\def\labelenumi{\arabic{enumi}.}
\item
  Hyaline:

  \begin{enumerate}
  \def\labelenumii{\alph{enumii}.}
  \item
    Functions:
  \item
    locations:
  \end{enumerate}
\item
  Fibrocartilage

  \begin{enumerate}
  \def\labelenumii{\alph{enumii}.}
  \item
    Functions:
  \item
    locations:
  \end{enumerate}
\item
  Elastic

  \begin{enumerate}
  \def\labelenumii{\alph{enumii}.}
  \item
    Functions:
  \item
    locations:
  \end{enumerate}
\end{enumerate}
\end{minipage} & \begin{minipage}[t]{\linewidth}\raggedright
Blood

\begin{enumerate}
\def\labelenumi{\arabic{enumi}.}
\item
  Functions:
\item
  locations:
\end{enumerate}
\end{minipage} \\
\begin{minipage}[t]{\linewidth}\raggedright
Dense connective tissue:

\begin{enumerate}
\def\labelenumi{\arabic{enumi}.}
\item
  Regular elastic

  \begin{enumerate}
  \def\labelenumii{\alph{enumii}.}
  \item
    Functions:
  \item
    locations:
  \end{enumerate}
\item
  Irregular elastic:

  \begin{enumerate}
  \def\labelenumii{\alph{enumii}.}
  \item
    Functions:
  \item
    locations:
  \end{enumerate}
\end{enumerate}
\end{minipage} & \begin{minipage}[t]{\linewidth}\raggedright
Bones:

\begin{enumerate}
\def\labelenumi{\arabic{enumi}.}
\item
  Compact bone:

  \begin{enumerate}
  \def\labelenumii{\alph{enumii}.}
  \item
    Functions:
  \item
    locations:
  \end{enumerate}
\item
  Cancellous bone:

  \begin{enumerate}
  \def\labelenumii{\alph{enumii}.}
  \item
    Functions:
  \item
    Locations:
  \end{enumerate}
\end{enumerate}
\end{minipage} & \begin{minipage}[t]{\linewidth}\raggedright
Lymph

\begin{enumerate}
\def\labelenumi{\arabic{enumi}.}
\item
  Functions:
\item
  locations:
\end{enumerate}
\end{minipage} \\
\end{longtable}
}

\begin{enumerate}
\def\labelenumi{\Alph{enumi}.}
\item
  What is tendinitis?

  \begin{enumerate}
  \def\labelenumii{\arabic{enumii}.}
  \item
    What causes tendinitis?
  \item
    Where is tendinitis mostly likely to occur in the body?
  \end{enumerate}
\end{enumerate}

\begin{enumerate}
\def\labelenumi{\Roman{enumi}.}
\setcounter{enumi}{3}
\item
  Muscle Tissue and Motion

  \begin{enumerate}
  \def\labelenumii{\Alph{enumii}.}
  \item
    Muscle tissue like nervous tissue is excitable. What does
    excitability mean?
  \item
    Three types of muscle tissue:

    \begin{enumerate}
    \def\labelenumiii{\arabic{enumiii}.}
    \item
      \_\_\_\_\_\_\_\_\_\_ muscle is found in the heart and works to
      \_\_\_\_\_\_\_\_\_\_\_\_\_\_\_\_\_\_\_\_\_\_\_.

      \begin{enumerate}
      \def\labelenumiv{\alph{enumiv})}
      \item
        Cardiomyocytes attach to each other via cell junctions called
        \_\_\_\_\_\_\_\_\_\_\_\_\_\_\_ disc.
      \item
        Is blood pumping through the body voluntary or involuntary?
      \end{enumerate}
    \item
      \_\_\_\_\_\_\_\_\_\_\_ muscle is found in the skeleton and works
      to \_\_\_\_\_\_\_\_\_\_\_\_\_\_\_\_\_\_\_\_\_\_\_.

      \begin{enumerate}
      \def\labelenumiv{\alph{enumiv})}
      \item
        These cells are striated which means
      \item
        How are these cells different from cardiac muscle?
      \end{enumerate}
    \item
      \_\_\_\_\_\_\_\_ muscle is found lining hollow organs and works to
      \_\_\_\_\_\_\_\_\_\_\_\_\_\_\_\_\_\_\_\_\_\_\_\_\_\_\_\_\_\_\_\_\_\_\_\_\_\_\_\_\_\_\_\_.
    \end{enumerate}
  \end{enumerate}
\item
  Nervous Tissue Mediates Perception and Response

  \begin{enumerate}
  \def\labelenumii{\Alph{enumii}.}
  \item
    What is the main function of nervous tissue?

    \begin{enumerate}
    \def\labelenumiii{\arabic{enumiii}.}
    \item
      How are signals sent and received?
    \end{enumerate}
  \item
    Nervous tissue contains two main cell types:

    \begin{enumerate}
    \def\labelenumiii{\arabic{enumiii}.}
    \item
      Neurons:
    \item
      Neuroglia:
    \end{enumerate}
  \item
    Neurons are exist in three main types:

    \begin{enumerate}
    \def\labelenumiii{\arabic{enumiii}.}
    \item
      Multipolar:
    \item
      Unipolar:
    \item
      Bipolar:
    \end{enumerate}
  \item
    What are the differences in the morphology of the three types?

    \begin{enumerate}
    \def\labelenumiii{\arabic{enumiii}.}
    \item
      Where would each type be found?
    \item
      Which is the most abundant?
    \end{enumerate}
  \item
    What are the major types of neuroglia in the CNS?
  \item
    What are the major types of neuroglia in the PNS?
  \end{enumerate}
\end{enumerate}

\includegraphics[width=6.5in,height=2.44444in]{images/media/image34.png}

\begin{enumerate}
\def\labelenumi{\Roman{enumi}.}
\setcounter{enumi}{5}
\item
  Tissue Injury and Aging

  \begin{enumerate}
  \def\labelenumii{\Alph{enumii}.}
  \item
    \_\_\_\_\_\_\_\_\_\_ is the body's response to injury. Four major
    signs appear:

    \begin{enumerate}
    \def\labelenumiii{\arabic{enumiii}.}
    \item
      Are there any other signs?
    \item
      Will all of the signs appear at once?
    \end{enumerate}
  \item
    Cell death initiates the \_\_\_\_\_\_\_\_\_\_ response in some
    cases. When would this response not occur?
  \item
    Chemicals, \_\_\_\_\_\_\_\_\_\_\_\_\_, are released to initiate the
    inflammatory response.

    \begin{enumerate}
    \def\labelenumiii{\arabic{enumiii}.}
    \item
      Vasodilation is also involved in the parasympathetic response. How
      can your body identify injury versus parasympathetic response?
    \end{enumerate}
  \item
    What are histamines?

    \begin{enumerate}
    \def\labelenumiii{\arabic{enumiii}.}
    \item
      They are present during allergic reactions. How are they helpful
      in both allergic reactions and inflammatory responses?
    \end{enumerate}
  \item
    List the steps in the process of wound healing.
  \end{enumerate}
\end{enumerate}

\includegraphics[width=6.5in,height=3.16667in,alt={This diagram shows the wound healing process in three steps. Each step shows a cross section of wounded skin. The wound extends through the upper layer of skin, labeled the epidermis, about halfway through the dermis, the lower deeper layer of skin. At the base of the cross section, an artery runs horizontally through fatty tissue below the dermis. Several small capillaries branch from the artery and travel into the upper regions of the dermis. In the first step of healing, inflammatory chemicals, symbolized with green dots, are released from the injury site. The chemicals travel through the dermis and enter the horizontal artery. Clotting proteins and plasma proteins also initiate clotting within the wound, forming a scab, which is clearly visible in the second step as a black and brown mass covering the upper regions of the wound. Below the scab, epithelial cells in the epidermis multiply and begin to fill in the wound. In the dermis, three fibrocytes are binding the wound area with white tissue. This tissue is granulation tissue. Laying down granulation tissue restores the vascular supply, as indicated by capillaries growing around the wounded area. In the third step, the scab is gone and the epidermis has grown in and contracted to seal the upper portion of the wound. In the deeper regions, the wound is now completely filled with granulation tissue with is now considered scar tissue.}]{images/media/image35.png}

\begin{enumerate}
\def\labelenumi{\Alph{enumi}.}
\setcounter{enumi}{5}
\item
  What are some of the changes to the human body during aging?

  \begin{enumerate}
  \def\labelenumii{\arabic{enumii}.}
  \item
    How are telomeres affected by aging?
  \item
    Are cancer risks increased with age? Why or why not?
  \item
  \end{enumerate}
\end{enumerate}

\section{}\label{section-3}

\section{Chapter 5}\label{chapter-5}

\begin{enumerate}
\def\labelenumi{\Roman{enumi}.}
\item
  Functional Anatomy of Skin

  \begin{enumerate}
  \def\labelenumii{\Alph{enumii}.}
  \item
    Most accessible but often least appreciated organ system.

    \begin{enumerate}
    \def\labelenumiii{\arabic{enumiii}.}
    \item
      The skin, or simply integument, accounts for approximately 16\% of
      your total body weight.
    \item
      The skin's surface, 1.5 - 2.0 m\textsuperscript{2}, is constantly
      worn away, attacked by micro-organisms, irradiated by sunlight,
      and exposed to environmental chemicals.
    \item
      Skin is composed of two major components:

      \begin{enumerate}
      \def\labelenumiv{\alph{enumiv}.}
      \item
        Cutaneous membrane:

        \begin{enumerate}
        \def\labelenumv{\roman{enumv}.}
        \item
          The \textbf{epidermis} consists of stratified squamous.
        \item
          The \textbf{dermis} consists of a papillary layer of areolar
          tissue and a reticular layer of dense irregular connective
          tissue.
        \end{enumerate}
      \item
        Accessory Structures:

        \begin{enumerate}
        \def\labelenumv{\roman{enumv}.}
        \item
          Nerve fibers and corpuscles
        \item
          Hair follicles, hair shafts, and arrector pili muscles
        \item
          Oil glands and sweat glands
        \item
          Arteries, veins, and lymph vessels forming the
          \textbf{cutaneous network}
        \end{enumerate}
      \end{enumerate}
    \item
      The \textbf{hypodermis} (also known as the subcutaneous layer or
      superficial fascia) separates the integument from the fascia
      around the deeper organs. Note this layer is NOT part of the
      integument.
    \end{enumerate}
  \end{enumerate}

  \begin{enumerate}
  \def\labelenumii{\arabic{enumii}.}
  \item
    The Layers of the Skin
  \end{enumerate}
\end{enumerate}

\begin{enumerate}
\def\labelenumi{\arabic{enumi}.}
\item
  The \textbf{EPIDERMIS} is composed of layers with various functions.
\end{enumerate}

\begin{enumerate}
\def\labelenumi{\Alph{enumi}.}
\item
  The epidermis is dominated by \textbf{\ul{keratinocytes}}, the body's
  most abundant epithelial cells. These cells form several layers called
  \textbf{strata}.
\end{enumerate}

\begin{enumerate}
\def\labelenumi{\arabic{enumi}.}
\item
  Thin skin, which covers most of the body surface, contains four strata
  and is about as thick as the wall of a plastic sandwich bag (roughly
  0.08 mm).
\item
  Thick skin, which occurs on the palms of the hands and soles of the
  feet, possesses five strata. It is about as thick as a standard paper
  towel (roughly 0.50 mm).
\item
  Note that the terms ``thick'' and ``thin'' refer to the relative
  thickness of the epidermis, not the integument as a whole.
\end{enumerate}

\begin{enumerate}
\def\labelenumi{\Alph{enumi}.}
\setcounter{enumi}{1}
\item
  Strata of the Epidermis
\end{enumerate}

\begin{enumerate}
\def\labelenumi{\arabic{enumi}.}
\item
  Stratum Basale

  \begin{enumerate}
  \def\labelenumii{\alph{enumii}.}
  \item
    The deepest epidermal layer consisting of a single row of
    \textbf{basal cells}, or germinative cells, that are undergoing
    rapid mitotic divisions. These cells are sometimes called stem cells
    because their mitotic divisions replace the more superficial
    keratinocytes that are lost or shed at the surface.
  \item
    Hemidesmosomes attach the cells of this layer to the basal lamina
    that separates the epidermis from the areolar tissue of the adjacent
    papillary layer of the dermis.
  \item
    Approximately 10 -- 25\% of cells in this layer are melanocytes
    which produce \textbf{melanin}, a brown, yellowish-brown, or black
    skin pigment.
  \item
    In hairless skin, specialized cells called \textbf{merkel cells}
    exist in small numbers. These cells are sensitive to touch and when
    compressed, they release chemicals that stimulate sensory nerve
    endings.
  \end{enumerate}
\item
  Stratum Spinosum
\end{enumerate}

\begin{enumerate}
\def\labelenumi{\alph{enumi}.}
\item
  Consists of approximately 8 -- 10 layers of keratinocytes bound
  together by desmosomes and microfilaments of pre-keratin.
\item
  The name \emph{stratum spinosum}, which means ``spiny layer'', refers
  to the fact that the cells look like miniature pincushions in standard
  histological sections.
\item
  Large numbers of \textbf{dendritic cells} are found in this layer.
  These are specialized cells that participate in the immune response by
  stimulating a defense mechanism against 1) microorganisms that manage
  to penetrate the superficial layers of the epidermis and 2)
  superficial skin cancers.
\end{enumerate}

\begin{enumerate}
\def\labelenumi{\arabic{enumi}.}
\setcounter{enumi}{2}
\item
  Stratum Granulosum
\end{enumerate}

\begin{enumerate}
\def\labelenumi{\alph{enumi}.}
\item
  Consists of 3 -- 5 cell layers where the keratinocytes appearance
  begins to change. The name \emph{stratum granulosum} means ``grainy
  layer''.
\item
  These cells become flattened, the plasma membrane becomes less
  permeable, and the organelles deteriorate.
\item
  By the time the cells reach this layer, most have stopped dividing and
  have started making large amounts of \textbf{keratin} and
  \textbf{keratinohyalin} stored in numerous visible granules.
\item
  Beyond this layer, there is no nutrient availability.
\end{enumerate}

\begin{enumerate}
\def\labelenumi{\arabic{enumi}.}
\setcounter{enumi}{3}
\item
  Stratum Lucidum
\end{enumerate}

\begin{enumerate}
\def\labelenumi{\alph{enumi}.}
\item
  In the thick skin of the palms and soles, a stratum lucidum separates
  the stratum corneum from deeper layers.
\item
  The cells of this layer are flattened, densely packed, largely devoid
  of organelles, and filled with the proteins keratin and keratohyalin.
\item
  By the time they reach the stratum lucidum, the cells are dead and
  undergoing dehydration.
\end{enumerate}

\begin{enumerate}
\def\labelenumi{\arabic{enumi}.}
\setcounter{enumi}{4}
\item
  Stratum Corneum
\end{enumerate}

\begin{enumerate}
\def\labelenumi{\alph{enumi}.}
\item
  Outermost layer of keratinocytes (sometimes called the ``horny
  layer'').
\item
  A broad zone of 15 -- 30 layers of keratinized cells that accounts for
  up to three-quarters of the epidermal thickness.
\item
  Keratinization is the formation of protective, superficial layers of
  cells filled with keratin.
\item
  The dead cells in each layer of the stratum corneum remain tightly
  interconnected by desmosomes. It takes 7 to 10 days for a cell to move
  from the stratum basale to the stratum corneum. The dead cells
  generally remain in the exposed stratum corneum for an additional two
  weeks before they are shed or washed away.
\item
  Glycolipids in this layer provide a waterproofing quality to the
  epidermis.
\end{enumerate}

\begin{enumerate}
\def\labelenumi{\Alph{enumi}.}
\setcounter{enumi}{2}
\item
  The deeper layers of the epidermis form \textbf{\ul{epidermal ridges}}
  which extend into the dermis and are adjacent to the dermal
  projections called \textbf{dermal papillae} that project upward to the
  epidermis. These ridges and papillae are significant because they
  greatly increase the surface area for attachment, firmly binding the
  epidermis to the dermis.
\item
  The ridge patterns in the thick skin on the surface of the fingertips
  produce fingerprints, which have been used to identify individuals in
  criminal investigations for more than a century.
\item
  Like all other epithelia, the epidermis lacks local blood vessels.
  Epidermal cells rely of the diffusion of nutrients and oxygen from
  capillaries within the dermis. As a result the cells with the highest
  metabolic demand are closest to the underlying dermis
\end{enumerate}

Name the types of tissues associated with the epidermis, dermis, and
hypodermis.

List the 5 major layers of epidermis and describe the functions and
characteristics of each.

\begin{enumerate}
\def\labelenumi{\arabic{enumi}.}
\setcounter{enumi}{1}
\item
  The \textbf{DERMIS} supports the epidermis, and the hypodermis
  connects the dermis to the rest of the body.

  \begin{enumerate}
  \def\labelenumii{\Alph{enumii}.}
  \item
    The \textbf{dermis} lies between the epidermis and hypodermis. The
    dermis consists of two layers:

    \begin{enumerate}
    \def\labelenumiii{\arabic{enumiii}.}
    \item
      \textbf{\ul{Papillary layer}} = consists of a highly vascularized
      areolar tissue with all of the typical cell types within it.

      \begin{enumerate}
      \def\labelenumiv{\alph{enumiv}.}
      \item
        This layer also contains the capillaries, lymphatic vessels, and
        sensory neurons that supply the surface of the skin.
      \item
        The papillary layer gets its name from the dermal papillae that
        project between the epidermal ridges.
      \item
        This layer nourishes and supports epidermis.
      \end{enumerate}
    \item
      \textbf{\ul{Reticular layer}} = consists of an interwoven meshwork
      of dense irregular connective tissue containing both collagen and
      elastic fibers.

      \begin{enumerate}
      \def\labelenumiv{\alph{enumiv}.}
      \item
        Bundles of collagen fibers extend superficially to blend into
        those of the papillary layer and deeply to blend with the
        hypodermis.
      \item
        The collagen fibers provide strength while the elastic fibers
        provide flexibility.
      \item
        This layer restricts the spread of pathogens, stores lipid
        reserves, attaches skin to deeper tissues, possesses sensory
        receptors, and contains blood vessels for temperature
        regulation.
      \end{enumerate}
    \end{enumerate}
  \item
    \textbf{\ul{Cleavage Lines}} = within the dermis, the collagen and
    elastin fibers are arranged in parallel bundles oriented to resist
    the forces applied to the skin during normal movements. The
    resulting pattern of fiber bundles establishes the lines of
    cleavage. These lines are clinically significant: a cut parallel to
    a cleavage line will usually remain closed and heal with little
    scarring whereas a cut at a right angle to a cleavage line will be
    pulled open as movement occurs and result in greater scarring.
  \end{enumerate}
\end{enumerate}

List the 2 major areas of the dermis and describe the characteristics of
each.

\begin{enumerate}
\def\labelenumi{\arabic{enumi}.}
\setcounter{enumi}{2}
\item
  The \textbf{hypodermis} separates the skin from deeper structures.

  \begin{enumerate}
  \def\labelenumii{\arabic{enumii}.}
  \item
    It stabilizes the position of skin in relation to underlying tissues
    (such as skeletal muscles or other organs) while permitting
    independent movement.
  \item
    Because it is often dominated by adipose tissue, the hypodermis also
    represents an important site for 1) insulation, 2) cushioning, and
    3) the storage of energy reserves.
  \item
    At puberty men accumulate subcutaneous fat at the neck, on the arms,
    along the lower back, and over the buttock. In contrast, women
    accumulate subcutaneous fat at the breasts, buttocks, hips, and
    thighs. In both genders, there are almost no fat cells on the back
    of the hands and feet but distressingly large numbers in the
    abdominal regions (resulting in the ``potbelly'').
  \end{enumerate}
\item
  Factors influencing \textbf{skin color} include epidermal pigmentation
  and dermal circulation.

  \begin{enumerate}
  \def\labelenumii{\Alph{enumii}.}
  \item
    The color of one's skin is genetically programmed. However,
    increased pigmentation, or tanning, can result in response to
    ultraviolet radiation.
  \item
    Skin color is influenced by the presence of pigments in the
    epidermis:

    \begin{enumerate}
    \def\labelenumiii{\arabic{enumiii}.}
    \item
      \textbf{\ul{Melanin}} = a brown, yellowish-brown, or black pigment
      produced by melanocytes.

      \begin{enumerate}
      \def\labelenumiv{\alph{enumiv}.}
      \item
        \textbf{\ul{Melanocytes}} are located within the stratum basale,
        squeezed between or deep to the keratinocytes. Melanocytes
        manufacture melanin from the amino acid \textbf{tyrosine}, and
        package it in intracellular vesicles called
        \textbf{\ul{melanosomes}}.
      \item
        Melanosomes travel within the processes of melanocytes and are
        transferred intact to keratinocytes. The transfer of
        pigmentation colors the keratinocyte temporarily, until the
        melanosomes are destroyed by fusion with lysosomes.
      \item
        In individuals with pale skin, this transfer occurs in the
        stratum basale and stratum spinosum, and the cells of more
        superficial layers lose their pigmentation. In dark-skinned
        individuals, the melanosomes are larger, and the transfer may
        occur in the stratum granulosum as well; thus skin pigmentation
        is darker and more persistent.
      \item
        The skin covering most areas of the body has about 1000
        melanocytes per square millimeter. Differences in skin
        pigmentation among individuals do not reflect different numbers
        of melanocytes but instead different levels of melanin
        production.
      \end{enumerate}
    \item
      \textbf{\ul{Carotene}} = an orange-yellow pigment that normally
      accumulates in epidermal cells. It is most apparent in cells of
      the stratum corneum of light-skinned individuals, but it also
      accumulates in fatty tissues in the deep dermis and hypodermis.
      Carotene is found in a variety of orange and yellow vegetables
      (sweet potatoes, carrots, squash).
    \end{enumerate}
  \item
    The blood supply affects skin color because blood contains red blood
    cells filled with the red pigment \textbf{hemoglobin}.

    \begin{enumerate}
    \def\labelenumiii{\arabic{enumiii}.}
    \item
      When bound to oxygen, hemoglobin is bright red, giving capillaries
      in the dermis a reddish tint that is most apparent in
      light-skinned individuals.
    \item
      If those vessels are dilated, the red tones become much more
      pronounced. For example, your skin becomes \textbf{\ul{flushed}}
      and red when your body temperature rises because the superficial
      blood vessels dilate so that the skin can act like a radiator and
      lose heat.
    \item
      When the blood flow decreases, oxygen levels in the tissues
      decline, and under these conditions hemoglobin releases oxygen and
      turns a much darker red. Seen from the surface the skin takes on a
      bluish color. This coloration is called \textbf{\ul{cyanosis}}. In
      individuals of any skin color, cyanosis is most obvious in areas
      of very thin skin (lips and under the fingernails).
    \end{enumerate}
  \end{enumerate}
\end{enumerate}

Describe the factors that normally contribute to skin color.

\begin{enumerate}
\def\labelenumi{\arabic{enumi}.}
\item
  Accessory Organs of the Skin
\end{enumerate}

\begin{enumerate}
\def\labelenumi{\Alph{enumi}.}
\item
  Hair and its associated structures:

  \begin{enumerate}
  \def\labelenumii{\arabic{enumii}.}
  \item
    \textbf{\ul{Hair follicles}} are a complex structure composed of
    epithelial cells and connective tissues that are responsible for the
    formation of a single hair. The hair follicle has three regions (the
    internal root sheath, the external root sheath, and glassy
    membrane).
  \item
    Hair production begins at the base of the hair follicle. Here a mass
    of epithelial cells forms a cap, called the \textbf{\ul{hair bulb}}
    that surrounds a smaller \textbf{\ul{hair papilla}}, a peg of
    connective tissue containing capillaries and nerves.
  \item
    \textbf{\ul{Root hair plexus}} are sensory nerves that surround the
    hair bulb and give hair the ability to detect touch.
  \item
    Associated with each hair follicle is a bundle of smooth muscle
    cells called an \textbf{\ul{arrector pili muscle}}. These muscles
    will contract and cause hairs to stand up or become erect.
  \item
    The human body has about 2.5 million hairs and 75\% of them are on
    the general body surface and not on the head. Hairs are non-living
    structures composed of keratinocytes.
  \item
    Parts of a Hair:

    \begin{enumerate}
    \def\labelenumiii{\alph{enumiii}.}
    \item
      \textbf{\ul{Hair shaft}} is the portion of the hair that extends
      through the follicle and protrudes above the skin line.
    \item
      \textbf{\ul{Hair root}} the portion that anchors the hair into the
      skin
    \item
      \textbf{\ul{Cuticle}} forms the surface of the hair. Composed of
      \textbf{hard keratin}.
    \item
      \textbf{\ul{Cortex}} an intermediate layer of cells deep to the
      cuticle. Contains thick layers of \textbf{hard keratin}, which
      give hairs their stiffness.
    \item
      \textbf{\ul{Medulla}}, or core, consists of cells at the center of
      the hair matrix filled with \textbf{soft keratin}.
    \item
      \textbf{\ul{Hair matrix}} consists of superficial cells of the
      hair bulb. These germinative cells in the hair matrix produce the
      hair.
    \end{enumerate}
  \item
    Variations in hair color reflect differences in hair structure and
    variation in the pigment produced by melanocytes at the hair
    papilla. Different forms of melanin give a dark brown, yellow-brown,
    or red color to the hair. As pigment production decreases with age,
    hair color lightens. White hair results from the combination of a
    lack of pigment and the presence of air bubbles in the medulla of
    the hair shaft.
  \end{enumerate}
\item
  \textbf{\ul{Nails}} = thick sheets of keratinized epidermal cells.

  \begin{enumerate}
  \def\labelenumii{\arabic{enumii}.}
  \item
    \textbf{Nails} protect the exposed dorsal surfaces of the tips of
    the fingers and toes. They also help limit distortion of the digits
    whey they are subjected to mechanical stress.
  \item
    The cells producing the nails can be affected by conditions that
    alter body metabolism, so changes in the shape, structure, or
    appearance of the nails can provide useful diagnostic information.
  \item
    Parts of a nail:

    \begin{enumerate}
    \def\labelenumiii{\alph{enumiii}.}
    \item
      \textbf{\ul{Nail body}} = consists of dead, tightly compressed
      keratinocytes packed with keratin. The nail body is the portion of
      the nail to which polish might be applied.
    \item
      \textbf{\ul{Nail bed}} = the nail body covers an area of the
      epidermis that contains rapidly dividing cells that divide to
      replace the cells that are lost.
    \item
      \textbf{\ul{Nail root}} = the epidermal fold not visible from the
      surface and anchors the nail body into the underlying tissues; the
      deepest portion of the nail root lies very close to the bone of
      the fingertip.
    \item
      \textbf{\ul{Hyponychium}} = the free edge of the nail composed of
      a thickened stratum corneum; the distal portion that continues
      past the nail bed.
    \item
      \textbf{\ul{Eponychium}} = a portion of stratum corneum of the
      nail root that extends over the exposed nail; more commonly called
      the cuticle.
    \item
      \textbf{\ul{Lunula}} = a pale crescent shaped area near the root
      where the dermal blood vessels are obscured; may not be present in
      all nails
    \end{enumerate}
  \end{enumerate}
\end{enumerate}

Describe the structure and function of nails.

\begin{enumerate}
\def\labelenumi{\Alph{enumi}.}
\setcounter{enumi}{2}
\item
  Glands
\end{enumerate}

\begin{enumerate}
\def\labelenumi{\arabic{enumi}.}
\item
  Sebaceous glands = oil glands

  \begin{enumerate}
  \def\labelenumii{\alph{enumii}.}
  \item
    Simple alveolar glands that are found all over the body except on
    the palms and the soles. Sebaceous follicles secrete onto skin
    surfaces located on the face, back, chest, nipples, and external
    genitalia.
  \item
    Contractions of the arrector pili muscles squeeze the sebaceous
    gland and force the \textbf{\ul{sebum}} (a mixture of triglycerides,
    cholesterol, proteins, and electrolytes) into the hair follicle and
    onto the surface of the skin.
  \item
    These glands are the \textbf{\ul{holocrine type}}, the cells fill up
    with oil then bust.
  \item
    Sebum is secreted into a \textbf{\ul{hair follicle}}, or
    occasionally a pore, or follicle, on the skin surface.
  \item
    Sebum softens and lubricates hair and surrounding skin and also has
    anti-bacterial properties.
  \end{enumerate}
\item
  Sweat glands
\end{enumerate}

\begin{enumerate}
\def\labelenumi{\alph{enumi}.}
\item
  Distributed all over the surface of the body except the nipple, parts
  of the external genitalia, and the lips.
\item
  \textbf{\ul{Eccrine (merocrine) sweat glands}} are very numerous in
  the palms, soles of the feet and forehead.

  \begin{enumerate}
  \def\labelenumii{\roman{enumii}.}
  \item
    Eccrine gland secretions, commonly called \textbf{\ul{sweat}}, are a
    hypotonic filtrate of the blood that passes through secretory cells
    of the sweat gland and is release by exocytosis.
  \item
    Once released, the sweat travels via a duct to the surface of the
    skin where it opens into a funnel-shaped pore.
  \item
    Normal pH of sweat is between 4 and 6.
  \end{enumerate}
\item
  \textbf{\ul{Apocrine sweat glands}}* are largely confined to the
  axillary and anogenital areas.
\end{enumerate}

\begin{enumerate}
\def\labelenumi{\roman{enumi}.}
\item
  Larger than eccrine sweat glands and release their secretions into
  hair follicles.
\item
  The secretions produced are similar to sweat but they also contain
  fatty substances and proteins.
\item
  Apocrine glands begin functioning at puberty.
\end{enumerate}

\begin{enumerate}
\def\labelenumi{\alph{enumi}.}
\setcounter{enumi}{3}
\item
  Ceruminous glands
\end{enumerate}

\begin{enumerate}
\def\labelenumi{\roman{enumi}.}
\item
  Modified Apocrine glands that line the external ear canal and secrete
  a sticky, bitter substance called cerumen.
\item
  \textbf{\ul{Cerumen}}=earwax.
\end{enumerate}

\begin{enumerate}
\def\labelenumi{\alph{enumi}.}
\setcounter{enumi}{4}
\item
  Mammary glands
\end{enumerate}

\begin{enumerate}
\def\labelenumi{\roman{enumi}.}
\item
  Specialized Apocrine sweat gland that secretes \textbf{\ul{milk}}.
\end{enumerate}

Compare and contrast the modified sweat glands including: eccrine,
apocrine, ceruminous, and mammary glands.

Compare the structure, location, and product of sweat glands versus oil
glands.

\begin{enumerate}
\def\labelenumi{\arabic{enumi}.}
\item
  Functions of the integumentary system
\end{enumerate}

\begin{enumerate}
\def\labelenumi{\Alph{enumi}.}
\item
  Protection
\item
  Sensory: The integument contains many sensory receptors:
\end{enumerate}

\begin{enumerate}
\def\labelenumi{\arabic{enumi}.}
\item
  \textbf{\ul{Free nerve endings}} = numerous unencapsulated nerve
  endings for pain and temperature detection
\item
  \textbf{\ul{Tactile discs}} = extend from the dermis into the
  epidermis where they connect to Merkel cells and monitor the chemical
  secretions from these cells which produce tactile stimuli.
\item
  \textbf{\ul{Tactile corpuscles (Meissner's corpuscles)}} = receptors
  located in the dermal papillae; responsible for the detection of light
  touch
\item
  \textbf{\ul{Lamellated corpuscles (Pacinian corpuscles)}} = receptors
  located in the reticular layer of the dermis; responsible for
  detection of deep pressure and vibration
\end{enumerate}

\begin{enumerate}
\def\labelenumi{\Alph{enumi}.}
\setcounter{enumi}{2}
\item
  Thermoregulation: your skin can help regulate your body temperature
  via vasodilation and vasoconstriction.
\item
  Vitamin D synthesis: \textbf{\ul{Hormonal Vitamin D}} = also known as
  calcitriol. When exposed to ultraviolet light, epidermal cells in the
  stratum spinosum and stratum basale converts a cholesterol-related
  steroid into \textbf{cholecalciferol}. Although cholecalciferol can be
  obtained from the diet, few foods contain it. In fact most foods that
  contain cholecalciferol have been fortified with it. The liver then
  converts cholecalciferol into an intermediary product used by the
  kidneys to synthesize the hormone \textbf{\ul{calcitriol}}. Calcitriol
  is required for stimulating normal absorption of calcium and
  phosphorus in the small intestine. An inadequate supply of calcitriol
  leads to impaired bone growth and maintenance such as typical of
  \textbf{\ul{rickets}}.

  \begin{enumerate}
  \def\labelenumii{\arabic{enumii}.}
  \item
    Diseases, disorders and injuries of the immune system
  \end{enumerate}
\end{enumerate}

\begin{enumerate}
\def\labelenumi{\Alph{enumi}.}
\item
  \textbf{Skin cancers} are the most common types of cancer.

  \begin{enumerate}
  \def\labelenumii{\arabic{enumii}.}
  \item
    The most common form of skin cancer is \textbf{\ul{basal cell
    carcinoma}}. This is a cancer that originates in keratinocytes of
    the stratum basale, due to mutations caused by overexposure to the
    UV light. Metastasis virtually never occurs in basal cell
    carcinomas, and most people survive these cancers.
  \item
    In contrast, \textbf{\ul{melanoma}} is the least common form of skin
    cancer but is extremely dangerous. In this condition cancerous
    melanocytes within the stratum basale grow rapidly and metastasize
    through the lymphatic system. The outlook for long-term survival is
    in many cases determined by how early the condition is diagnosed. If
    the cancer is detected early, while it is still localized, the
    affected area can be surgically removed, and the 5-year survival
    rate is 99 percent. If the condition is not detected until extensive
    metastasis has occurred, the 5-year survival rate drops to 14\%.
  \item
    \textbf{\ul{Squamous cell carcinoma}} originates in the stratum
    spinosum layer and like basal cell carcinoma, it rarely
    metastasizes.
  \end{enumerate}
\item
  \textbf{Eczema} is an allergic reaction that manifests as dry, itchy
  patches of skin that look like a rash. It may swell, flake, crack and
  bleed and can be treated with corticosteroids and immunosuppressants.
\item
  \textbf{Acne} occurs from an overproductive, blocked sebaceous gland.
\item
  \textbf{Injuries and Burns: First degree burn} effects only the
  epidermis. \textbf{Second degree burn} goes deeper and effects both
  epidermis and dermis. \textbf{Third degree burn} extends through the
  epidermis and dermis to damage underlying tissue and nerve endings.
  \textbf{Fourth degree burn} includes damage to all of the above as
  well as muscle and bone. Full thickness burns can NOT be repaired by
  the body and require a skin graft.
\item
  \textbf{Scars} are collagen-rich skin formed after the process of
  wound healing that differs from normal skin. \textbf{Keloids} are
  raised scars
\item
  \textbf{Bedsores} happen in areas exposed to prolong pressure
  resulting in loss of blood flow and necrosis of the tissues.
\item
  \textbf{Stretch Marks} result from the skin is stretched beyond its
  normal capacity.
\item
  \textbf{Calluses} and \textbf{Corns} form from areas of constant
  abrasion.

  \begin{enumerate}
  \def\labelenumii{\arabic{enumii}.}
  \item
    Age-related changes alter the appearance of structure of the
    integument.

    \begin{enumerate}
    \def\labelenumiii{\Alph{enumiii}.}
    \item
      Melanocyte activity declines, and in light skinned individuals,
      the skin becomes pale. With less melanin in the skin, people
      become more sensitive to sun exposure and more likely to
      experience sunburn.
    \item
      Sebaceous gland secretions decreases with age and the skin becomes
      dry and often scaly.
    \item
      The epidermis thins as germinative cell activity declines, and the
      connections between the epidermis and dermis weakens, making older
      people more prone to injury, skin tears, and skin infections.
    \item
      The metabolic activity in the skin decreases as well. Synthesis of
      calcitriol (vitamin D\textsubscript{3}) decreases leading to
      muscle weakness and brittle bones.
    \item
      The number of dendritic cells decreases to about half the levels
      seen at maturity. This reduction in cells may decrease sensitivity
      of the immune response and further encourage skin damage and
      infection.
    \item
      The dermis becomes thinner and has fewer elastic fibers, making
      the integument weaker and less resilient. The results -- sagging
      and wrinkling -- are most pronounced in body regions with the most
      sun exposure.
    \item
      Merocrine sweat glands become less active and with impaired
      perspiration processes, older people cannot lose hear at fast as
      younger people. Thus the elder are at greater risk of overheating
      in warm environments.
    \item
      A reduction in dermal blood supply cools the skin, which can
      stimulate thermoreceptors and make a person feel cold even in a
      warm room. Reduced circulation and sweat gland function lessens
      their ability to lose body heat, which can cause their body
      temperature to soar dangerously high.
    \item
      With declining levels of sex hormones, differences in secondary
      sexual characteristics with respect to hair distribution and
      body-fat distribution begin to fade. As a consequence, people age
      90 -- 100 of both sexes tend to look alike.
    \item
      Hair follicles stop functioning or produce thinner, finer hairs.
      With decreased melanocyte activity, these hairs are gray or white.
    \end{enumerate}
  \end{enumerate}
\end{enumerate}

Describe the various skin disorders discussed in class as well as the
age related changes that occur in skin.

Define the following terms: strata, keratin, cyanosis, epidermal ridges,
dermal papillae, cleavage lines, striae, cutaneous network, and
melanosomes.

\section{}\label{section-4}

\section{\texorpdfstring{Chapter 6 }{Chapter 6 }}\label{chapter-6}

\begin{enumerate}
\def\labelenumi{\arabic{enumi}.}
\item
  The Functions of the Skeletal System

  \begin{enumerate}
  \def\labelenumii{\arabic{enumii}.}
  \item
    \textbf{\ul{Support}} = provides structural support for the entire
    body. Individual bones or groups of bones provide a framework for
    the attachment of soft tissues and or organs.
  \item
    \textbf{\ul{Protection}} = delicate tissues and organs are often
    surrounded by skeletal elements. The ribs protect the heart and
    lungs, the skull encloses the brain, the vertebrae shield the spinal
    cord, and the pelvis cradles delicate urinary and reproductive
    organs.
  \item
    \textbf{\ul{Leverage}} = many bones of the skeleton function as
    levers that can change the magnitude and direction of the forces
    generated by skeletal muscles. The movements produced range from the
    delicate motions of a fingertip to powerful changes in the position
    of the entire body.
  \item
    \textbf{\ul{Storage of minerals}} = the calcium salts of bone
    represents a valuable mineral reserve that maintains normal
    concentrations of calcium and phosphate ions in the body fluids.
    Calcium is the most abundant mineral in the human body. A typical
    human body contains 1-2 kg of calcium, with more than 98 \% of it
    deposited in the bones of the skeleton.
  \item
    \textbf{\ul{Blood cell production}} = also known as hematopoiesis;
    red blood cells, white blood cells, and platelets are produced in
    the red bone marrow, which fills the internal cavities of many
    bones.
  \end{enumerate}
\end{enumerate}

6.2 Bone Classification:

A. The adult skeleton system includes approximately 206 separate bones
and a number of associated cartilages. This body system is divided into
the axial skeleton and appendicular skeleton.

\begin{enumerate}
\def\labelenumi{\arabic{enumi}.}
\item
  \textbf{\ul{Axial skeleton}} = (80 bones) consists of the bones of the
  skull, hyoid, sternum, rib cage, vertebral column, sacrum, and coccyx.
\item
  \textbf{\ul{Appendicular skeleton}} = (126 bones) includes bones of
  the limbs and the pectoral and pelvic girdles that attach the limbs to
  the axial skeleton.

  \begin{enumerate}
  \def\labelenumii{\Alph{enumii}.}
  \item
    Bones are classified according to shape and structure and also their
    surface features.

    \begin{enumerate}
    \def\labelenumiii{\arabic{enumiii}.}
    \item
      \textbf{\ul{Flat bones}} = thin, roughly parallel surfaces. Flat
      bones form the roof of the skull, sternum, the ribs, and the
      scapulae. They provide protection from underlying soft tissues and
      offer an extensive surface for the attachment of skeletal muscles.
    \item
      \textbf{\ul{Sutural bones}} = also known as Wormian bones; are
      small, flat, irregularly shaped bones between the flat bones of
      the skull. There are individual variations in the number, shape,
      and position of sutural bones. Their borders are like pieces of a
      jigsaw puzzle, and they range in size from a grain of sand to a
      quarter.
    \item
      \textbf{\ul{Long bones}} = are relatively long and slender. They
      are located in the arm, forearm, thigh, lower leg, palms, soles,
      fingers and toes. The femur, the long bone of the thigh, is the
      largest and heaviest bone in the body.
    \item
      \textbf{\ul{Irregular bones}} = have complex shapes with short,
      flat, notched, or ridged surfaces. The spinal vertebrae, the bones
      of the pelvis, and several of the skull bones (mandible for
      example) are irregular bones.
    \item
      \textbf{\ul{Sesamoid bones}} = are generally small, flat, and
      shaped somewhat like a sesame seed. They develop inside of tendons
      and are most commonly located near joints at the knees, the hands,
      and the feet. Everyone has sesamoid patellae, or kneecaps, but
      individuals vary in the location and abundance of other sesamoid
      bones. This variation, among others, accounts for disparities in
      the total number of bones in the skeleton.
    \item
      \textbf{\ul{Short bones}} = small and boxy. Examples of short
      bones include bones of the wrist (carpals) and bones of the ankles
      (tarsals).
    \end{enumerate}
  \end{enumerate}
\end{enumerate}

6.3 Bone Structure

\begin{enumerate}
\def\labelenumi{\Alph{enumi}.}
\item
  Gross Anatomy of long bones: Long bones are designed to transmit
  forces along the shaft and have a rich blood supply.
\end{enumerate}

\begin{enumerate}
\def\labelenumi{\arabic{enumi}.}
\item
  \textbf{\ul{Diaphysis}} = long tubular shaft that forms the axis of a
  typical long bone; the walls of the shaft are made primarily of
  \textbf{compact bone}.
\item
  \textbf{\ul{Epiphyses}} = ends of the bones composed primarily of
  \textbf{spongy bone}, also called \textbf{trabecular bone}. Spongy
  bone consists of an open network of struts and plates (called
  trabeculae) that resemble a latticework with red bone marrow filling
  in the spaces between. The spongy bone is then covered by a thin layer
  of compact bone and articular cartilage.
\end{enumerate}

\begin{enumerate}
\def\labelenumi{\alph{enumi}.}
\item
  \textbf{\ul{Proximal epiphyses}}=end closest to the origin of
  attachment.
\item
  \textbf{\ul{Distal epiphyses}}=end furthers from the origin of
  attachment.
\end{enumerate}

\begin{enumerate}
\def\labelenumi{\arabic{enumi}.}
\setcounter{enumi}{2}
\item
  \textbf{\ul{Metaphysis}} = a narrow zone that connects the diaphysis
  to the epiphyses. The \textbf{epiphyseal plate}, a thin layer of
  hyaline cartilage more commonly called the growth plate, is important
  for growth in the length of bones.
\item
  \textbf{\ul{Medullary cavity}} = within the shaft of a long bone is a
  cavity where bone marrow is located. In childhood, the medullary
  cavity is filled with \textbf{red bone marrow} but as we age, fat
  accumulates within the red marrow transforming it to \textbf{yellow
  bone marrow}. Red bone marrow is important for hematopoiesis but
  yellow bone marrow is no longer hematopoietic and instead stores fat
  as an important energy source.
\item
  Membranes associated with bone:
\end{enumerate}

\begin{enumerate}
\def\labelenumi{\alph{enumi}.}
\item
  \textbf{\ul{Periosteum}}=outermost covering of bone made primarily of
  dense irregular tissue and held on by Sharpey's fibers (collagen).
\item
  \textbf{\ul{Endosteum}}=internal membrane of bone made of connective
  tissue. Also lines the many canals that pass through bone to supply
  blood and nerves to the bone.
\end{enumerate}

\begin{enumerate}
\def\labelenumi{\arabic{enumi}.}
\setcounter{enumi}{5}
\item
  \textbf{\ul{Nutrient foramen}} = in order for bones to grow and be
  maintained, they require an extensive blood supply. The nutrient
  foramen is a tunnel that penetrates the diaphysis and provides access
  for the blood vessels into the shaft of the bone.
\end{enumerate}

\begin{enumerate}
\def\labelenumi{\alph{enumi}.}
\item
  \textbf{\ul{Nutrient artery}} = transports oxygenated, nutrient-rich
  blood to the bone.
\item
  \textbf{\ul{Nutrient vein}} = transports deoxygenated, waste-laden
  blood from the bone.
\end{enumerate}

\begin{enumerate}
\def\labelenumi{\arabic{enumi}.}
\setcounter{enumi}{6}
\item
  \textbf{\ul{Metaphyseal artery and metaphyseal vein}} = carry blood to
  and from the area of the metaphysis and to the epiphysis.
\item
  \textbf{\ul{Articular cartilage}} = covers portions of the epiphysis
  that articulate with other bones. The cartilage is avascular,
  \textbf{hyaline cartilage}. It relies primarily on diffusion from the
  synovial fluid to obtain oxygen and nutrients and to eliminate wastes.
\end{enumerate}

\begin{enumerate}
\def\labelenumi{\Alph{enumi}.}
\setcounter{enumi}{1}
\item
  Bone Markings: also known as surface features
\end{enumerate}

\begin{enumerate}
\def\labelenumi{\arabic{enumi}.}
\item
  Depressions and openings allowing blood vessels and nerves to pass

  \begin{enumerate}
  \def\labelenumii{\roman{enumii}.}
  \item
    \textbf{\ul{Fossa}} = a shallow depression or recess in the surface
    of a bone
  \item
    \textbf{\ul{Fissure}} = a narrow, slit-like opening or an elongated
    cleft or gap
  \item
    \textbf{\ul{Foramen}} = round or oval opening through the bone
  \item
    \textbf{\ul{Canal or meatus}} = a large passageway through the a
    bone
  \item
    \textbf{\ul{Sulcus or groove}} = a furrow or narrow trough in a bone
  \item
    \textbf{\ul{Sinus}} = a chamber within a bone filled with air and
    lined with a mucous membrane.
  \end{enumerate}
\item
  Projections that are sites for muscle and ligament attachment
\end{enumerate}

\begin{enumerate}
\def\labelenumi{\alph{enumi}.}
\item
  \textbf{\ul{Tuberosity}}= large, round or rough projection that may
  cover a broad area
\item
  \textbf{\ul{Crest}}=narrow ridge of bone; usually prominent
\item
  \textbf{\ul{Trochanter}}=very large, irregularly shaped projection
\item
  \textbf{\ul{Line}}=narrow ridges of bone; less prominent than a crest
\item
  \textbf{\ul{Tubercle}}=small, rounded projection
\item
  \textbf{\ul{Epicondyle}}=raised area above a condyle
\item
  \textbf{\ul{Spine}}=sharp, slender, and often pointed process
\end{enumerate}

\begin{enumerate}
\def\labelenumi{\arabic{enumi}.}
\setcounter{enumi}{2}
\item
  Projections that form joints
\end{enumerate}

\begin{enumerate}
\def\labelenumi{\alph{enumi}.}
\item
  \textbf{\ul{Head}}=expanded proximal end of a bone carried on a narrow
  neck
\item
  \textbf{\ul{Facet}}=smooth, flat articular surface
\item
  \textbf{\ul{Condyle}}=smooth, rounded articular surface
\item
  \textbf{\ul{Ramus}}=arm-like bar of a bone

  \begin{enumerate}
  \def\labelenumii{\Alph{enumii}.}
  \item
    Microscopic anatomy of compact bone cells and tissues.

    \begin{enumerate}
    \def\labelenumiii{\arabic{enumiii}.}
    \item
      \textbf{\ul{Osteon}} = the basic structural and functional unit of
      bone consisting of bone cells organized around a central canal and
      separated by concentric lamellae.
    \item
      \textbf{\ul{Central canal}} = also known as the \textbf{Haversian
      canal}, runs parallel to the axis of bone and are located in the
      middle of each osteon. Each central canal possesses an artery and
      vein, lymph vessel, and nerve.
    \item
      \textbf{\ul{Perforating canals}} = passageways that extend
      perpendicular to the axis of the bone and connect the central
      canals of adjacent osteons.
    \item
      \textbf{\ul{Lamellae}} = nested, concentric rings of matrix
      surrounding the central canal.

      \begin{enumerate}
      \def\labelenumiv{\arabic{enumiv}.}
      \item
        \textbf{\ul{Circumferential lamellae}} = specialized lamellae
        found at the outer and inner surfaces of bone, where they are
        covered by the periosteum and endosteum, respectively. These
        lamellae are produced during the growth and maintenance of bone.
      \item
        \textbf{\ul{Interstitial lamellae}} = fill in the spaces between
        adjacent osteons of compact bone. These lamellae are remnants of
        osteons whose matrix components have been almost completely
        recycled by the action of bone digesting cells.
      \end{enumerate}
    \item
      \textbf{\ul{Lacunae}} = mature bones cells, called osteocytes, are
      trapped within an open space called a lacuna. Osteocytes cannot
      divide and therefore each lacuna contains only one osteocyte.
    \item
      \textbf{\ul{Canaliculi}} = processes of the osteocytes extend into
      narrow crevices, called canaliculi, that penetrate the lamellae
      and connect the lacunae to the central canal.
    \end{enumerate}
  \item
    Bone is associated with four cells that account for approximately
    2\% of the bones weight.

    \begin{enumerate}
    \def\labelenumiii{\arabic{enumiii}.}
    \item
      \textbf{\ul{Osteocytes}} = mature bone cells that maintain the
      protein and mineral content of the surrounding matrix through the
      turnover of matrix components. Osteocytes secrete chemicals that
      dissolve the adjacent matrix, and the release minerals enter the
      circulation. The osteocytes then rebuild the matrix, stimulating
      the deposition of mineral crystals. Osteocytes also participate in
      the repair of damaged bones.
    \item
      \textbf{\ul{Osteoblasts}} = immature bone cells located on the
      surface of bone; produce new bone matrix in a process called
      \textbf{osteogenesis}, or \textbf{ossification}. Osteoblasts make
      and release the proteins and other organic components of the
      matrix. Before calcium salts are deposited, this organic matrix is
      called \textbf{\ul{osteoid}}. Osteocytes develop from osteoblasts
      that have become completely surrounded by bone matrix and trapped
      within a lacuna.
    \item
      \textbf{\ul{Osteoprogenitor cells}} = mesenchymal cells located
      with the periosteum and endosteum. These stem cells divide to
      produce daughter cells that differentiate into osteoblasts, and
      they are important in the formation of osteocytes.
    \item
      \textbf{\ul{Osteoclasts}} = bone digesting cells that remove and
      recycle bone matrix. These are giant cells with 50 or more nuclei.
      Osteoclasts are not related to osteoprogenitor cells or their
      descendants. Instead, they are derived from the same stem cells
      that produce phagocytic white blood cells, called monocytes. Acids
      and proteolytic enzymes secreted by osteoclasts dissolve the
      matrix and release stored minerals. This process, called
      \textbf{\ul{osteolysis}}, or resorption, is important in bone
      remodeling.
    \end{enumerate}
  \item
    Chemical composition of bone:

    \begin{enumerate}
    \def\labelenumiii{\arabic{enumiii}.}
    \item
      Organic \textbf{\ul{Osteoid}} = roughly 1/3 of the weight of bone
      is contributed by collagen fibers. Collagen fibers are strong and
      flexible, but if they are compressed, they bend.
    \item
      Inorganic \textbf{\ul{Hydroxyapatites}} = mineral salts account
      for almost 2/3 of the weight of bone. Calcium phosphate interacts
      with calcium hydroxide to form crystals of hydroxyapatite. As they
      form, these crystals incorporate other calcium salts, such as
      calcium carbonate, and ions such as sodium, magnesium, and
      fluoride. By combining the hydroxyapatite with the collagen
      fibers, a strong, somewhat flexible, material is produced.
      Furthermore, this protein-crystal combination is highly resistant
      to shattering. In fact, bone is far superior to concrete and is
      more in par with steel-reinforced concrete.
    \end{enumerate}
  \end{enumerate}

  \begin{enumerate}
  \def\labelenumii{\arabic{enumii}.}
  \setcounter{enumii}{3}
  \item
    Bone Formation and Development
  \end{enumerate}
\end{enumerate}

\begin{enumerate}
\def\labelenumi{\Alph{enumi}.}
\item
  The formation of bone, osteogenesis or ossification, begins during
  embryonic development. Two types of osteogenesis occur in the embryo:
\end{enumerate}

\begin{enumerate}
\def\labelenumi{\arabic{enumi}.}
\item
  Endochondral Ossification

  \begin{enumerate}
  \def\labelenumii{\arabic{enumii}.}
  \item
    Formation of most bones using a hyaline cartilage model. Begins
    approximately 6 weeks after fertilization.
  \item
    Hyaline cartilage does not turn into bone instead it is broken down
    as ossification occurs.
  \item
    Steps of endochondral ossification:

    \begin{enumerate}
    \def\labelenumiii{\roman{enumiii}.}
    \item
      \textbf{\ul{Cavitation of hyaline shaft}}: (picture \#1 and \#2 in
      the diagram)

      \begin{enumerate}
      \def\labelenumiv{\alph{enumiv})}
      \item
        Chondrocytes within the shaft hypertrophy (enlarge) and the
        surrounding matrix begins to calcify.
      \item
        The impermeable matrix causes chondrocytes to die from lack of
        nutrients leaving the matrix that starts to deteriorate
        (cavitate).
      \item
        Blood vessels grow around the edges of the cartilage.
      \item
        The cells of the perichondrium convert to osteoblasts producing
        a superficial layer of bone sometimes called the bony collar.
      \end{enumerate}
    \item
      \textbf{\ul{Invasion of internal cavities}}: (picture \#3 in the
      diagram)

      \begin{enumerate}
      \def\labelenumiv{\alph{enumiv})}
      \item
        Blood vessels penetrate the cartilage and invade the central
        region. This area within the shaft of hyaline cartilage is
        called the \textbf{primary ossification center}.
      \item
        Migrating with the blood vessels are fibroblasts (which
        differentiate into osteoblasts), lymph vessels, nerve fibers,
        red marrow elements. Collectively, these are called the
        \textbf{periosteal bud}.
      \item
        The osteoblasts secrete osteoid around remaining fragments of
        hyaline, forming trabeculae, or spongy bone.
      \end{enumerate}
    \item
      Formation of the Medullary cavity: (picture \#4 in the diagram)

      \begin{enumerate}
      \def\labelenumiv{\alph{enumiv})}
      \item
        As the primary ossification center enlarges, osteoclasts break
        down newly formed spongy bone and opens up a medullary cavity in
        the center of the diaphysis.
      \item
        The osseous tissue of the outer shaft becomes thicker forming
        compact bone.
      \end{enumerate}
    \item
      \textbf{\ul{Formation of epiphyses}}: (picture \#5 in the diagram)

      \begin{enumerate}
      \def\labelenumiv{\alph{enumiv})}
      \item
        \textbf{Secondary ossification centers} appear in the area at
        the opposite ends of the bone. The cartilage in the epiphyses
        calcifies and deteriorates, forming cavities that allow entry of
        a periosteal bud.
      \item
        Soon the epiphyses are filled with spongy bone. The spongy bone
        is NOT broken down during the remodeling process.
      \end{enumerate}
    \end{enumerate}
  \end{enumerate}
\item
  Intramembranous Ossification
\end{enumerate}

\begin{enumerate}
\def\labelenumi{\alph{enumi}.}
\item
  Formation of bones without a cartilage model. Typical in flat bones,
  mandible, clavicles, and patella. Begins approximately 8 weeks after
  fertilization.
\item
  Mesenchyme cells differentiate into osteoblasts within fibrous
  connective tissues. This type of ossification normally occurs in the
  deeper layers of the dermis or in the connective tissues of tendons.
\item
  Steps of intramembranous ossification:
\end{enumerate}

\begin{enumerate}
\def\labelenumi{\alph{enumi}.}
\item
  Formation of bone matrix within fibrous membrane:
\end{enumerate}

\begin{enumerate}
\def\labelenumi{\alph{enumi})}
\item
  Mesenchymal cells cluster and secrete organic components of the
  matrix. The location of this activity is the \textbf{ossification
  center}.
\item
  The resulting osteoid mineralizes and the mesenchymal cells
  differentiate into osteoblasts.
\item
  As ossification proceeds, the osteoblasts get trapped within lacunae
  and differentiate into osteocytes.
\end{enumerate}

\begin{enumerate}
\def\labelenumi{\alph{enumi}.}
\setcounter{enumi}{1}
\item
  Formation of woven bone and periosteum:
\end{enumerate}

\begin{enumerate}
\def\labelenumi{\alph{enumi})}
\item
  Osteoid accumulates, fuses together forming struts called
  \textbf{trabeculae}, or spicules, around blood vessels.
\item
  The overall structure is similar to spongy bone.
\end{enumerate}

\begin{enumerate}
\def\labelenumi{\alph{enumi}.}
\setcounter{enumi}{2}
\item
  Formation of compact bone plate:
\end{enumerate}

\begin{enumerate}
\def\labelenumi{\alph{enumi})}
\item
  Initially, the intramembranous bone consists of spongy bone only.
\item
  Subsequent remodeling around trapped blood vessels can produce osteons
  typical of compact bone.
\item
  As the rate of growth slows at the surface, the connective tissue
  around the bone becomes organized into the fibrous layer of the
  periosteum.
\end{enumerate}

\begin{enumerate}
\def\labelenumi{\Alph{enumi}.}
\setcounter{enumi}{1}
\item
  The growth of bone occurs by two primary processes:
\end{enumerate}

\begin{enumerate}
\def\labelenumi{\Alph{enumi}.}
\item
  Longitudinal Growth (length)
\end{enumerate}

\begin{enumerate}
\def\labelenumi{\alph{enumi}.}
\item
  Hyaline cartilage cells form tall columns at the \textbf{epiphyseal
  plate} (or growth plate) and within the articular cartilage.
\item
  The cells at the top of the stack divide quickly, increasing the
  thickness of the epiphyseal plates and causing the entire long bone to
  lengthen.
\item
  Older chondrocytes closer to the shaft enlarge, die, and the
  surrounding cartilage matrix deteriorates.
\item
  The deterioration leaves spicules of calcified cartilage.
\item
  Osteoblasts in the medullary cavity then ossify the cartilage
  spicules, forming spongy bone.
\item
  The hyaline cartilage at the epiphyseal plate is eventually replaced
  entirely by bone. Once completely replaced with bone, the epiphyseal
  plate is now called the \textbf{epiphyseal line}. This typically
  occurs in the person's early twenties and as a result the person stops
  growing in height.
\end{enumerate}

\begin{enumerate}
\def\labelenumi{\Alph{enumi}.}
\setcounter{enumi}{1}
\item
  Appositional Growth (width)
\end{enumerate}

\begin{enumerate}
\def\labelenumi{\alph{enumi}.}
\item
  Osteoprogenitor cells beneath the periosteum differentiate into
  osteoblasts and form new osteons on the external bone surface.
\item
  While bone is being added to the outer surface through appositional
  growth, osteoclasts are removing and recycling lamellae at the inner
  surface. As a result, the medullary cavity gradually enlarges as the
  bone increases in diameter.
\item
  Appositional growth is important in increasing the diameter of
  existing bones but it does not form the original bone.

  \begin{enumerate}
  \def\labelenumii{\arabic{enumii}.}
  \setcounter{enumii}{3}
  \item
    Fractures: Repair of cracked or broken bones:
  \end{enumerate}
\end{enumerate}

\begin{enumerate}
\def\labelenumi{\Alph{enumi}.}
\item
  Hematoma formation
\end{enumerate}

\begin{enumerate}
\def\labelenumi{\arabic{enumi}.}
\item
  Blood vessels in bone tear and hemorrhage occurs.
\item
  Over a period of several hours, a large blood clot, or
  \textbf{hematoma}, develops.
\end{enumerate}

\begin{enumerate}
\def\labelenumi{\Alph{enumi}.}
\setcounter{enumi}{1}
\item
  Fibrocartilage callus formation
\end{enumerate}

\begin{enumerate}
\def\labelenumi{\arabic{enumi}.}
\item
  Capillaries grow into the hematoma and phagocytic cells invade the
  area.
\item
  Fibroblasts and osteoblasts migrate to the fracture.
\item
  Fibroblasts secrete collagen fibers and/or differentiate into
  chondroblasts that secrete a cartilage matrix.
\item
  Osteoblasts form spongy bone.
\item
  The mass of repair tissue is referred to as a
  \textbf{\ul{fibrocartilage callus}}.
\item
  An \textbf{internal callus} connects bone ends and an \textbf{external
  callus} protrudes from the outer bone surface.
\end{enumerate}

\begin{enumerate}
\def\labelenumi{\Alph{enumi}.}
\setcounter{enumi}{2}
\item
  Bony callus formation
\end{enumerate}

\begin{enumerate}
\def\labelenumi{\arabic{enumi}.}
\item
  Osteoblasts and osteoclasts continue to migrate inward and multiply
  rapidly in the fibrocartilaginous callus.
\item
  As the material calcifies, the tissue becomes a \textbf{\ul{bony
  callus}}.
\end{enumerate}

\begin{enumerate}
\def\labelenumi{\Alph{enumi}.}
\setcounter{enumi}{3}
\item
  Fractures are classified on the basis of:

  \begin{enumerate}
  \def\labelenumii{\arabic{enumii}.}
  \item
    Whether the bone penetrates the skin.

    \begin{enumerate}
    \def\labelenumiii{\alph{enumiii}.}
    \item
      \textbf{\ul{Simple}} (closed) =bone breaks cleanly, but does not
      penetrate the skin.
    \item
      \textbf{\ul{Compound}} (open) =broken ends of bone protrude
      through the tissue and skin.
    \end{enumerate}
  \item
    Orientation of the break.

    \begin{enumerate}
    \def\labelenumiii{\alph{enumiii}.}
    \item
      \textbf{\ul{Transverse}}=break occurs perpendicular to the long
      axis of a bone.
    \item
      \textbf{\ul{Linear}}=breaks parallel to the long axis of the bone.
    \end{enumerate}
  \item
    Position of the bone ends after the fracture.

    \begin{enumerate}
    \def\labelenumiii{\alph{enumiii}.}
    \item
      \textbf{\ul{Non-displaced}}=the bone ends retain their position.
    \item
      \textbf{\ul{Displaced}}=the bone end are out of normal alignment.
    \end{enumerate}
  \end{enumerate}
\item
  Types of Fractures
\end{enumerate}

\begin{enumerate}
\def\labelenumi{\arabic{enumi}.}
\item
  \textbf{\ul{Comminuted}} = bone fragments into many pieces.
\item
  \textbf{\ul{Compression}} = bone is crushed from upward and downward
  forces
\item
  \textbf{\ul{Depressed}} = broken bone is pressed inward (skull)
\item
  \textbf{\ul{Spiral}} = raged break as a result of excessive twisting
  of the bone.
\item
  \textbf{\ul{Epiphyseal}} = break occurring along the epiphyseal plate
\item
  \textbf{\ul{Greenstick}} = bone breaks incompletely
\item
  \textbf{\ul{Colle's}} = distal part of the radius breaks
\item
  \textbf{\ul{Pott's}} = malleolus of tibia and fibula break
\end{enumerate}

6.6 Exercise, Nutrition and Hormones and Bone Tissue

A. Exercise -- lack of exercise and stress on bone can lead to loss of
bone mass.

B. Nutrition

1. \textbf{Calcium and Vitamin D:} Since the body cannot make calcium,
it must be obtained from the diet. However, calcium cannot be absorbed
from the small intestine without vitamin D. Therefore, intake of vitamin
D is also critical to bone health. Dairy as well as leafy vegetables are
a source of calcium.

2. \textbf{Vitamin K} also supports bone mineralization and may have a
synergistic role with vitamin D in the regulation of bone growth. Green
leafy vegetables are a good source of vitamin K.

C. Hormones:

1. \textbf{Growth Hormone}: synthesized in the pituitary controls bone
growth in multiple ways. It It triggers chondrocyte proliferation in
epiphyseal plates, resulting in the increasing length of long bones. GH
also increases calcium retention, which enhances mineralization, and
stimulates osteoblastic activity, which improves bone density.

2. \textbf{Thyroxine}: secreted by the thyroid gland promotes
osteoblastic activity and the synthesis of bone matrix.

3. Sex Hormones: \textbf{Estrogen} and \textbf{Testosterone} promote
osteoblastic activity and production of bone matrix, and in addition,
are responsible for the growth spurt that often occurs during
adolescence. They also promote the conversion of the epiphyseal plate to
the epiphyseal line (i.e., cartilage to its bony remnant), thus bringing
an end to the longitudinal growth of bones.

4. \textbf{Calcitriol} = the active form of vitamin D, is produced by
the kidneys and stimulates the absorption of calcium and phosphate from
the digestive tract.

6.7 Calcium Homeostasis: Interactions of the Skeletal System and Other
Organ Systems

\begin{enumerate}
\def\labelenumi{\Alph{enumi}.}
\item
  Bone is constantly undergoing deposition and resorption in a process
  known as \textbf{remodeling}.
\item
  Coordinated activity by osteoblasts and osteoclasts regulates both
  processes.
\end{enumerate}

\begin{enumerate}
\def\labelenumi{\arabic{enumi}.}
\item
  Bone deposition occurs where bone is injured or added bone strength is
  needed and is accomplished by osteoblasts. Bands of new matrix
  deposited in the area are referred to as an osteoid seam.
\item
  Bone reabsorption is accomplished by osteoclasts. Osteoclasts secrete
  lysosomal enzymes that digest the organic matrix and then secrete
  metabolic acids that convert calcium salts into soluble forms.
\end{enumerate}

\begin{enumerate}
\def\labelenumi{\Alph{enumi}.}
\setcounter{enumi}{2}
\item
  Remodeling is under negative feedback hormonal control.
\end{enumerate}

\begin{enumerate}
\def\labelenumi{\arabic{enumi}.}
\item
  Changes in the levels of blood calcium will trigger the release of
  either parathyroid hormone (PTH) or calcitonin.
\item
  When blood calcium levels are too low: the \textbf{parathyroid gland}
  secretes \textbf{\ul{parathyroid hormone}}, or PTH. PTH has three
  effects all leading to a drop in blood calcium levels:
\end{enumerate}

\begin{enumerate}
\def\labelenumi{\alph{enumi}.}
\item
  Bone response = stimulates osteoclasts so they accelerate the erosion
  of bone matrix which leads to the release of stored calcium ions into
  the blood.
\item
  Intestinal response = PTH enhances the calcium-absorbing effects of
  calcitriol on the intestine. As a result the rate of intestinal
  calcium absorption increases.
\item
  Kidney response = PTH increases the production of the hormone
  calcitriol, which is continuously secreted by the kidneys at low
  levels. This hormone in turn stimulates calcium reabsorption at the
  kidney tubules.
\end{enumerate}

\begin{enumerate}
\def\labelenumi{\arabic{enumi}.}
\setcounter{enumi}{2}
\item
  When blood calcium levels are too high: the \textbf{C cells of the}
  \textbf{thyroid gland} secrete \textbf{\ul{calcitonin}}. Calcitonin
  has three effects all leading to a drop in blood calcium levels:
\end{enumerate}

\begin{enumerate}
\def\labelenumi{\alph{enumi}.}
\item
  Bone response = calcitonin inhibits osteoclasts but does not affect
  osteoblasts so that they continue to deposit calcium ions into the
  matrix of bone.
\item
  Intestinal response = calcitonin decreases the rate of calcium
  absorption from foods in the digestive tract.
\item
  Kidney response = calcitonin inhibits the absorption of calcium in
  urine so that more calcium is excreted from the body.
\end{enumerate}

\begin{enumerate}
\def\labelenumi{\Alph{enumi}.}
\setcounter{enumi}{3}
\item
  By providing a calcium reserve, the skeleton plays the primary role in
  the homeostatic maintenance of normal calcium ion concentration of
  body fluids. The skeleton is also important in the homeostatic balance
  of other ions as well.

  \begin{enumerate}
  \def\labelenumii{\arabic{enumii}.}
  \setcounter{enumii}{7}
  \item
    Bone Disorders
  \end{enumerate}
\end{enumerate}

\begin{enumerate}
\def\labelenumi{\Alph{enumi}.}
\item
  \textbf{\ul{Pituitary growth failure}} = or dwarfism, results from
  inadequate production of growth hormone which leads to reduced
  epiphyseal cartilage activity and abnormally short bones. Rare in the
  United States because children can be treated with synthetic growth
  hormone.
\item
  \textbf{\ul{Achondroplasia}} = results from abnormal hyaline cartilage
  development. Because hyaline cartilage forms the model for long bone
  formation, the individual will have short, stocky limbs but the torso
  and head are of normal size.
\item
  \textbf{\ul{Marfan syndrome}} = very tall with long, slender limbs due
  to excessive cartilage formation at the epiphyseal plates. Other
  defects in the structure of connective tissues commonly cause
  life-threatening cardiovascular problems.
\item
  \textbf{\ul{Gigantism}} = results from an overproduction of growth
  hormone before puberty. Puberty is often delayed. The most common
  cause is a pituitary tumor which may be treated by surgery, radiation,
  or drugs that suppress the release of growth hormone.
\item
  \textbf{\ul{Acromegaly}} = result from too much growth hormone after
  the epiphyseal plates close so that the bones do not grow longer but
  instead get thicker (especially the bones of the face, hands, and
  jaw). This leads to changes in their physical appearance.
\item
  \textbf{\ul{Fibrodysplasia ossificans progressiva}} (FOP) = a rare
  gene mutation that causes the deposition of bone around skeletal
  muscles and the normally soft tissues of the body. There is no
  effective treatment for this painful and debilitating condition, and
  patients seldom survive into their 40's.
\item
  \textbf{\ul{Paget's Disease}} = overactive osteoclasts cause pores and
  weakening of the long bones leading to bending/bowing. Osteoblasts try
  to compensate for the overactive osteoclasts, but the bone laid down
  is weak and brittle and prone to fractures.
\end{enumerate}

\section{}\label{section-5}

\section{\texorpdfstring{Chapter 7 }{Chapter 7 }}\label{chapter-7}

\begin{enumerate}
\def\labelenumi{\Roman{enumi}.}
\item
  Division of the skeletal system

  \begin{enumerate}
  \def\labelenumii{\arabic{enumii}.}
  \item
    Function of the skeleton:
  \item
    How is movement produced?
  \item
    \_\_\_\_\_\_\_\_\_\_\_\_ are found in an average adult human body.
  \item
    What is the function of the lower skeleton?
  \item
    What is the function of the upper skeleton?
  \end{enumerate}

  \begin{enumerate}
  \def\labelenumii{\Alph{enumii}.}
  \item
    Axial skeleton

    \begin{enumerate}
    \def\labelenumiii{\arabic{enumiii}.}
    \item
      Includes skull,
      \_\_\_\_\_\_\_\_\_\_\_\_\_\_\_\_\_\_\_\_\_\_\_\_\_\_ and
      \_\_\_\_\_\_\_\_\_\_\_\_\_\_\_.

      \begin{enumerate}
      \def\labelenumiv{\alph{enumiv})}
      \item
        There are \_\_\_\_\_\_\_\_ bones in the axial skeleton.
      \end{enumerate}
    \item
      The only bone not connected to another bone in the body is the
      \_\_\_\_\_\_\_\_\_\_\_\_.
    \item
      The middle ear consists of \_\_\_\_\_\_\_\_
      \_\_\_\_\_\_\_\_\_\_\_\_\_.
    \item
      The skull consist of \_\_\_\_\_\_\_\_ bones
    \item
      The vertebral column contains \_\_\_\_\_\_\_\_\_\_\_\_\_\_
      vertebra, including the \_\_\_\_\_\_\_\_\_\_ and the
      \_\_\_\_\_\_\_\_\_\_\_\_\_.
    \item
      The thoracic cage consists of the \_\_\_\_\_\_\_\_\_\_\_\_\_, and
      the sternum.
    \end{enumerate}
  \item
    Appendicular skeleton

    \begin{enumerate}
    \def\labelenumiii{\arabic{enumiii}.}
    \item
      \_\_\_\_\_\_\_\_\_\_\_\_ bones make up the appendicular skeleton.
    \end{enumerate}
  \end{enumerate}
\end{enumerate}

\includegraphics[width=5.95139in,height=4.12252in,alt={This diagram shows the human skeleton and identifies the major bones. The left panel shows the anterior view (from the front) and the right panel shows the posterior view (from the back).}]{images/media/image36.png}

\begin{enumerate}
\def\labelenumi{\Roman{enumi}.}
\setcounter{enumi}{1}
\item
  The skull=\_\_\_\_\_\_\_\_\_\_\_\_

  \begin{enumerate}
  \def\labelenumii{\Alph{enumii}.}
  \item
    Two major divisions:

    \begin{enumerate}
    \def\labelenumiii{\arabic{enumiii}.}
    \item
      Facial bones
    \item
      Brain case or cranial vault
    \end{enumerate}
  \item
    The only mobile bone of the skull is the
    \_\_\_\_\_\_\_\_\_\_\_\_\_\_\_.
  \item
    Anterior view of the skull

    \begin{enumerate}
    \def\labelenumiii{\arabic{enumiii}.}
    \item
      Eyeballs are contained in the \_\_\_\_\_\_\_\_\_\_ of the skull.
    \item
      The nasal cavity is divided by the \_\_\_\_\_\_\_\_\_\_
      \_\_\_\_\_\_\_\_\_. The cavity contains the \_\_\_\_\_\_\_\_\_\_\_
      \_\_\_\_\_\_\_\_\_\_\_ of the ethmoid bone and the lower
      \_\_\_\_\_\_\_\_\_. Inside the nasal cavity two sets of bony
      projections appear on the lateral wall; the largest set are the
      \_\_\_\_\_\_\_\_\_\_\_\_\_ \_\_\_\_\_\_\_\_\_\_\_\_\_\_\_ and the
      set above those are the \_\_\_\_\_\_\_\_\_\_
      \_\_\_\_\_\_\_\_\_\_\_.
    \end{enumerate}
  \end{enumerate}
\end{enumerate}

\includegraphics[width=5.45022in,height=4.61806in,alt={This image shows the anterior view (from the front) of the human skull. The major bones on the skull are labeled.}]{images/media/image37.png}

\begin{enumerate}
\def\labelenumi{\Alph{enumi}.}
\setcounter{enumi}{3}
\item
  Lateral view of the skull

  \begin{enumerate}
  \def\labelenumii{\arabic{enumii}.}
  \item
    The zygomatic arch consist of two bones,
    \_\_\_\_\_\_\_\_\_\_\_\_\_\_\_ or the apple of the cheek and
    \_\_\_\_\_\_\_\_\_\_\_\_ and two processes
    \_\_\_\_\_\_\_\_\_\_\_\_\_\_\_\_\_\_\_\_\_\_\_\_\_\_\_\_\_\_\_\_\_
    and \_\_\_\_\_\_\_\_\_\_\_\_\_\_\_\_\_\_\_\_\_\_\_\_\_\_\_\_\_\_\_.
  \item
    The mandible articulates and connects to the skull via the
    \_\_\_\_\_\_\_\_\_\_\_\_ fossa and the
    \_\_\_\_\_\_\_\_\_\_\_\_\_\_\_ fossa of the skull. This allows for
    chewing to occur.
  \end{enumerate}
\end{enumerate}

\includegraphics[width=6.5in,height=4.45833in,alt={This image shows the lateral view of the human skull and identifies the major parts.}]{images/media/image38.png}

\begin{enumerate}
\def\labelenumi{\arabic{enumi}.}
\setcounter{enumi}{2}
\item
  Skull cap= \_\_\_\_\_\_\_\_\_\_\_\_\_
\item
  The plates of the skull consist of:

  \begin{enumerate}
  \def\labelenumii{\alph{enumii})}
  \item
    \_\_\_\_\_\_\_\_\_\_\_ plate which is a pair of bones on the upper
    lateral side of the skull
  \item
    \_\_\_\_\_\_\_\_\_\_\_\_ plate which is paired and either side of
    the skull. The plates were named because of thie graying of the hair
    in this area occurs first.

    \begin{enumerate}
    \def\labelenumiii{(\arabic{enumiii})}
    \item
      The \_\_\_\_\_\_\_\_\_\_\_ \_\_\_\_\_\_\_\_\_\_\_ can be felt on
      the side of the head just behind the earlobe.
    \end{enumerate}
  \item
    \_\_\_\_\_\_\_\_\_\_\_\_\_ bone which is a single bone located
    anterior and contains the forehead.

    \begin{enumerate}
    \def\labelenumiii{(\arabic{enumiii})}
    \item
      The slight depression between eyebrows is the
      \_\_\_\_\_\_\_\_\_\_\_\_\_\_\_\_.
    \end{enumerate}
  \item
    \_\_\_\_\_\_\_\_\_\_\_\_\_ bone is also a single bone in the
    posterior skull. The area that forms the nape of the neck is
    actually the \_\_\_\_\_\_\_\_ \_\_\_\_\_\_\_\_\_\_\_
    \_\_\_\_\_\_\_\_. The brainstem becomes the spinal cord once it
    passes through the \_\_\_\_\_\_\_\_\_\_\_\_\_ magnum.
  \end{enumerate}
\end{enumerate}

\begin{quote}
\includegraphics[width=6.5in,height=8.81944in]{images/media/image39.png}
\end{quote}

\begin{enumerate}
\def\labelenumi{\alph{enumi})}
\setcounter{enumi}{4}
\item
  The sutures connecting all plates are :
\end{enumerate}

\begin{enumerate}
\def\labelenumi{\Alph{enumi}.}
\setcounter{enumi}{4}
\item
  The \_\_\_\_\_\_\_\_\_\_ bone contains the \_\_\_\_\_\_\_\_\_\_
  \_\_\_\_\_\_\_\_\_\_\_ or the home for the pituitary gland in the
  \_\_\_\_\_\_\_\_\_\_\_\_\_ fossa, when the brain is in the skull.
\item
  30\% of injury related deaths related to \_\_\_\_\_\_\_\_\_\_\_\_\_
  injuries. Which ages are most likely affected?

  \begin{enumerate}
  \def\labelenumii{\arabic{enumii}.}
  \item
    What are some causes of traumatic brain injury?
  \item
    The upper lip or upper jaw = \_\_\_\_\_\_\_\_\_\_\_\_ bone
  \end{enumerate}
\item
  The \_\_\_\_\_\_\_\_ bone is a pair of bones that contain the
  \_\_\_\_\_\_\_\_\_\_\_\_ plate or the roof of the mouth.
\item
  Cleft lip and palate

  \begin{enumerate}
  \def\labelenumii{\arabic{enumii}.}
  \item
    What is the difference between the cleft lip versus the cleft
    palate?
  \item
    How often can each occur?
  \end{enumerate}
\item
  The \_\_\_\_\_\_\_\_ sinuses occupy air-filled spaces that can contain
  \_\_\_\_\_\_\_\_ mucosa.
\end{enumerate}

\includegraphics[width=6.5in,height=4.38889in]{images/media/image40.png}

\begin{enumerate}
\def\labelenumi{\Roman{enumi}.}
\setcounter{enumi}{2}
\item
  The vertebral column

  \begin{enumerate}
  \def\labelenumii{\Alph{enumii}.}
  \item
    Vertebra are connected via \_\_\_\_\_\_\_\_\_\_\_\_\_\_\_
    \_\_\_\_\_\_\_\_\_, which is made of \_\_\_\_\_\_\_\_\_\_\_\_\_\_
    connective tissue.

    \begin{enumerate}
    \def\labelenumiii{\arabic{enumiii}.}
    \item
      The vertebral column is broken down into 3 major categories:

      \begin{enumerate}
      \def\labelenumiv{\alph{enumiv})}
      \item
        Cervical

        \begin{enumerate}
        \def\labelenumv{(\arabic{enumv})}
        \item
          C1 or \_\_\_\_\_\_ articulates with the skull
        \item
          C2 or \_\_\_\_\_\_ articulates superiorly with C1
        \end{enumerate}
      \item
        \_\_\_\_\_\_\_\_\_\_\_\_
      \item
        \_\_\_\_\_\_\_\_\_\_\_\_
      \end{enumerate}
    \item
      The number of vertebrae in each category corresponds with times of
      meals

      \begin{enumerate}
      \def\labelenumiv{\alph{enumiv})}
      \item
        Cervical vertebra number \_\_\_\_\_\_
      \item
        \_\_\_\_\_\_\_\_\_ vertebra number \_\_\_\_\_
      \item
        \_\_\_\_\_\_\_\_\_ vertebra number \_\_\_\_\_
      \end{enumerate}
    \item
      The sacrum contains \_\_\_\_\_\_\_\_\_\_ of fused vertebrae, while
      the coccyx contains \_\_\_\_\_\_.
    \item
      Which two regions of the vertebral column retain the fetal
      curvature?
    \item
      What disorders can occur due to improper curvature of the
      vertebral column?
    \end{enumerate}
  \item
    Structure of a vertebra

    \begin{enumerate}
    \def\labelenumiii{\arabic{enumiii}.}
    \item
      The large opening between the vertebral arch and body
      \_\_\_\_\_\_\_\_ \_\_\_\_\_\_\_\_\_, which will contain the spinal
      cord.
    \item
      In contrast, the spinal nerves travel through the
      \_\_\_\_\_\_\_\_\_\_\_\_\_ foramen when the vertebrae are aligned
      together.
    \item
      What you feel when you run your fingers down the middle of
      someone's back is the \_\_\_\_\_\_\_\_\_ process of the vertebrae.
    \item
      Vertebrae articulate with each other via
      \_\_\_\_\_\_\_\_\_\_\_\_\_\_ and \_\_\_\_\_\_\_\_\_\_\_\_\_\_
      articular processes.
    \item
      \_\_\_\_\_\_\_\_\_\_\_\_\_ processes are paired and project
      laterally from each vertebrae. However, only cervical vertebrae
      have transverse \_\_\_\_\_\_\_\_\_.
    \end{enumerate}
  \end{enumerate}
\end{enumerate}

\includegraphics[width=6.5in,height=7.44444in]{images/media/image41.png}

\begin{enumerate}
\def\labelenumi{\arabic{enumi}.}
\setcounter{enumi}{5}
\item
  Vertebrae that have a ``giraffe'' appearance due to the elongated
  downward facing \_\_\_\_\_\_\_\_\_ process are
  \_\_\_\_\_\_\_\_\_\_\_\_\_\_\_\_. There are \_\_\_\_\_\_ of these in
  an adult human.
\item
  \includegraphics[width=6.5in,height=4.875in,alt={This figure shows the structure of the thoracic vertebra. The left panel shows the vertebral column with the thoracic vertebrae highlighted in pink. The right panel shows the detailed structure of a single thoracic vertebra.}]{images/media/image42.png}These
  vertebrae have a ``moose'' appearance due to shorter spinous processes
  and \_\_\_\_\_\_\_\_\_\_\_ centrum or body. These are
  \_\_\_\_\_\_\_\_\_\_ vertebrae.
\end{enumerate}

\includegraphics[width=6.5in,height=5.98611in]{images/media/image43.png}

\begin{enumerate}
\def\labelenumi{\arabic{enumi}.}
\setcounter{enumi}{7}
\item
  The \_\_\_\_\_\_\_ or tailbone is derived from the fusion of
  \_\_\_\_\_\_\_\_\_\_\_ vertebrae during development.
\end{enumerate}

\includegraphics[width=5.07639in,height=2.26973in]{images/media/image44.png}

\begin{enumerate}
\def\labelenumi{\Alph{enumi}.}
\setcounter{enumi}{2}
\item
  What is the major job of a chiropractor?

  \begin{enumerate}
  \def\labelenumii{\arabic{enumii}.}
  \item
    Why do chiropractors use a drug-free approach to healing?
  \item
    Are they considered to be medical doctors?
  \end{enumerate}
\end{enumerate}

\begin{enumerate}
\def\labelenumi{\Roman{enumi}.}
\setcounter{enumi}{3}
\item
  Thoracic cage

  \begin{enumerate}
  \def\labelenumii{\Alph{enumii}.}
  \item
    What is the major function of the thoracic cage?
  \item
    The \_\_\_\_\_\_\_\_\_\_\_ is broken down into 3 major parts. The
    \_\_\_\_\_\_\_\_\_\_ which is wider and the superior portion. The
    central portion or \_\_\_\_\_\_\_\_\_\_, which receives the most
    force during chest compression while administering CPR. Finally, the
    \_\_\_\_\_\_\_\_\_\_ process, which is easily broken if CPR is not
    administered correctly.
  \item
    The rib cage consists of \_\_\_\_\_\_\_\_\_\_\_\_ ribs.

    \begin{enumerate}
    \def\labelenumiii{\arabic{enumiii}.}
    \item
      \_\_\_\_\_\_\_\_\_\_\_\_ pairs of true ribs
    \item
      \_\_\_\_\_\_\_\_\_\_\_\_ pairs of false ribs, \_\_\_\_\_\_\_\_\_\_
      pairs of which are floating ribs
    \item
      \_\_\_\_\_\_\_\_\_\_\_ ribs are connected to the sternum via
      \_\_\_\_\_\_\_\_\_\_\_\_\_ connective tissue.
    \end{enumerate}
  \end{enumerate}
\end{enumerate}

\includegraphics[width=4.75694in,height=2.81257in]{images/media/image45.png}

\begin{enumerate}
\def\labelenumi{\Roman{enumi}.}
\setcounter{enumi}{4}
\item
  Embryonic development of the axial skeleton

  \begin{enumerate}
  \def\labelenumii{\Alph{enumii}.}
  \item
    The brain and spinal cord are formed from a primitive structure
    during embryonic development called the \_\_\_\_\_\_\_\_\_\_\_.

    \begin{enumerate}
    \def\labelenumiii{\arabic{enumiii}.}
    \item
      The \_\_\_\_\_\_\_\_\_ tube is the first differentiation, which
      occurs during the first six weeks of gestation.
    \item
      What is the function of endochondral ossification?
    \end{enumerate}
  \item
    Why is the brain case larger in infants?
  \item
    What is the scientific term for the ``soft spot''?

    \begin{enumerate}
    \def\labelenumiii{\arabic{enumiii}.}
    \item
      What type of tissue is located in this region?
    \end{enumerate}
  \end{enumerate}
\end{enumerate}

\includegraphics[width=6.5in,height=2.19444in,alt={This diagram shows the image of a newborn human skull. The major parts of the skull are labeled. The left panel shows the superior view (from the top) and the right side shows the lateral view (from the side).}]{images/media/image46.png}

\section{}\label{section-6}

\section{Chapter 8}\label{chapter-8}

Divisions of the skeleton:

The skeleton can be divided into two basic parts; the \textbf{axial
skeleton} and the \textbf{appendicular skeleton}. The axial skeleton is
the bones associated with the central portion of the body and include
the bones of the skull, thoracic (chest) cage, and the vertebral column.
The appendicular skeleton is made up of the bones associated with the
limbs and includes the pectoral girdle, the upper limbs, the pelvic
girdle, and the lower limbs.

You are responsible for learning all the bones of the skeleton and all
the markings listed on the next several pages.~

\subsection{APPENDICULAR SKELETON}\label{appendicular-skeleton}

8.1 Pectoral Girdle

\begin{enumerate}
\def\labelenumi{\arabic{enumi}.}
\item
  Scapula~ (right or left) - 2
\end{enumerate}

\begin{enumerate}
\def\labelenumi{\alph{enumi}.}
\item
  Coracoid process
\item
  Acromion process
\item
  Spine
\item
  Supraspinous fossa
\item
  Infraspinous fossa
\item
  Glenoid fossa (cavity)
\item
  Subscapular fossa
\item
  Suprascapular notch
\item
  Vertebral (medial) border
\item
  Axillary (lateral) border
\item
  Inferior angle
\item
  Superior angle
\end{enumerate}

\begin{enumerate}
\def\labelenumi{\arabic{enumi}.}
\item
  Clavicle - 2

  \begin{enumerate}
  \def\labelenumii{\alph{enumii}.}
  \item
    Acromial (lateral) end
  \item
    Sternal (medial) end
  \end{enumerate}
\end{enumerate}

\begin{enumerate}
\def\labelenumi{\arabic{enumi}.}
\setcounter{enumi}{1}
\item
  Bones of upper limb
\end{enumerate}

\begin{enumerate}
\def\labelenumi{\arabic{enumi}.}
\item
  Humerus~ (right or left) - 2

  \begin{enumerate}
  \def\labelenumii{\alph{enumii}.}
  \item
    Head
  \item
    Anatomical neck
  \item
    Surgical neck
  \item
    Greater tubercle
  \item
    Lesser tubercles
  \item
    Intertubercular groove
  \item
    Deltoid tuberosity
  \item
    Medial epicondyle
  \item
    Lateral epicondyle
  \item
    Trochlea
  \item
    Capitulum
  \item
    Olecranon fossa
  \item
    coronoid fossa
  \end{enumerate}
\item
  Radius - 2
\end{enumerate}

\begin{enumerate}
\def\labelenumi{\alph{enumi}.}
\item
  Head
\item
  Neck
\item
  Shaft
\item
  Styloid process
\item
  Ulnar notch
\item
  Radial tuberosity
\end{enumerate}

\begin{enumerate}
\def\labelenumi{\arabic{enumi}.}
\setcounter{enumi}{2}
\item
  \textbf{Ulna} - 2
\end{enumerate}

\begin{enumerate}
\def\labelenumi{\alph{enumi}.}
\item
  Olecranon process
\item
  Trochlear notch
\item
  Radial notch
\item
  Coronoid process
\item
  Head
\item
  Styloid process
\end{enumerate}

\begin{enumerate}
\def\labelenumi{\arabic{enumi}.}
\setcounter{enumi}{3}
\item
  \textbf{Carpals} (8 bones in each wrist)
\item
  \textbf{Metacarpals} (5 bones forming the palm of each hand)
\item
  \textbf{Phalanges} (14 bones forming the fingers of each hand)
\end{enumerate}

\begin{enumerate}
\def\labelenumi{\alph{enumi}.}
\item
  Proximal phalanx
\item
  Middle phalanx (not present in the \textbf{pollex})
\item
  Distal phalanx
\end{enumerate}

8.3 Pelvic girdle (\textbf{os coxae}) - 2

\begin{enumerate}
\def\labelenumi{\arabic{enumi}.}
\item
  Ilium
\end{enumerate}

\begin{enumerate}
\def\labelenumi{\alph{enumi}.}
\item
  Iliac crest
\item
  Anterior superior iliac spine (ASIS)
\item
  Anterior inferior iliac spine (AIIS)
\item
  Posterior superior iliac spine (PSIS)
\item
  Posterior inferior iliac spine (PIIS)
\item
  Iliac fossa
\item
  Sacroiliac joint (only present in articulated skeleton)
\item
  Greater sciatic notch
\end{enumerate}

\begin{enumerate}
\def\labelenumi{\arabic{enumi}.}
\setcounter{enumi}{1}
\item
  Ischium
\end{enumerate}

\begin{enumerate}
\def\labelenumi{\alph{enumi}.}
\item
  Ischial spine
\item
  Lesser sciatic notch
\item
  Ischial tuberosity
\end{enumerate}

\begin{enumerate}
\def\labelenumi{\arabic{enumi}.}
\setcounter{enumi}{2}
\item
  Pubis
\end{enumerate}

\begin{enumerate}
\def\labelenumi{\alph{enumi}.}
\item
  Pubic rami
\item
  Pubic symphysis
\item
  Obturator foramen
\end{enumerate}

\begin{enumerate}
\def\labelenumi{\arabic{enumi}.}
\setcounter{enumi}{3}
\item
  \textbf{Acetabulum} (acetabular fossa) -- hip socket
\end{enumerate}

8.4 Bones of the lower limb

\begin{enumerate}
\def\labelenumi{\arabic{enumi}.}
\item
  Femur~(right or left) - 2
\end{enumerate}

\begin{enumerate}
\def\labelenumi{\alph{enumi}.}
\item
  Head
\item
  Fovea capitis
\item
  Neck
\item
  Greater trochanter
\item
  Lesser trochanter
\item
  Gluteal tuberosity
\item
  Shaft
\item
  Linea aspera
\item
  Medial epicondyle
\item
  Lateral epicondyle
\item
  Medial epicondyle
\item
  Lateral condyle
\item
  Intercondylar fossa
\item
  Patellar surface
\end{enumerate}

\begin{enumerate}
\def\labelenumi{\arabic{enumi}.}
\setcounter{enumi}{1}
\item
  Patella - 2
\item
  Tibia (right or left) - 2
\end{enumerate}

\begin{enumerate}
\def\labelenumi{\alph{enumi}.}
\item
  Medial condyle
\item
  Lateral condyle
\item
  Intercondylar eminence
\item
  Tibial tuberosity
\item
  Shaft
\item
  Anterior border
\item
  Medial malleolus
\end{enumerate}

\begin{enumerate}
\def\labelenumi{\arabic{enumi}.}
\setcounter{enumi}{3}
\item
  Fibula - 2
\end{enumerate}

\begin{enumerate}
\def\labelenumi{\alph{enumi}.}
\item
  Head
\item
  Shaft
\item
  Lateral malleolus
\end{enumerate}

\begin{enumerate}
\def\labelenumi{\arabic{enumi}.}
\setcounter{enumi}{4}
\item
  Tarsals (7 bones in each ankle)
\end{enumerate}

\begin{enumerate}
\def\labelenumi{\alph{enumi}.}
\item
  \textbf{Talus --} bone of the ankle
\item
  \textbf{Calcaneous} bone of the heel
\end{enumerate}

\begin{enumerate}
\def\labelenumi{\arabic{enumi}.}
\setcounter{enumi}{5}
\item
  Metatarsals (5 bones forming the top of each foot)
\item
  \textbf{Phalanges} (14 bones forming the toes of each foot)
\end{enumerate}

\begin{enumerate}
\def\labelenumi{\alph{enumi}.}
\item
  Proximal phalanx
\item
  Middle phalanx (not present in the \textbf{hallux})
\item
  Distal phalanx

  \begin{enumerate}
  \def\labelenumii{\arabic{enumii}.}
  \setcounter{enumii}{4}
  \item
    Development of the appendicular skeleton
  \end{enumerate}
\end{enumerate}

\begin{enumerate}
\def\labelenumi{\Alph{enumi}.}
\item
  Each upper and lower limb initially develops as a small bulge called a
  \textbf{limb bud}, which appears on the lateral side of the early
  embryo. The upper limb bud appears near the end of the fourth week of
  development, with the lower limb bud appearing shortly after.
\item
  During the sixth week of development, the distal ends of the upper and
  lower limb buds expand and flatten into a paddle shape. This region
  will become the hand or foot.
\item
  All of the girdle and limb bones, except for the clavicle, develop by
  the process of \textbf{endochondral ossification}.
\end{enumerate}

\section{\texorpdfstring{Chapter 9 }{Chapter 9 }}\label{chapter-9}

\begin{enumerate}
\def\labelenumi{\Roman{enumi}.}
\item
  Classification of Joints

  \begin{enumerate}
  \def\labelenumii{\Alph{enumii}.}
  \item
    Bones connect to each other at articulations or
    \_\_\_\_\_\_\_\_\_\_\_\_\_.
  \item
    How are joints classified?
  \item
    Structural classification of joints include

    \begin{enumerate}
    \def\labelenumiii{\arabic{enumiii}.}
    \item
      Fibrous joints:
    \item
      Synovial joints:
    \item
      Cartilaginous joints:
    \end{enumerate}
  \item
    Functional classification includes

    \begin{enumerate}
    \def\labelenumiii{\arabic{enumiii}.}
    \item
      Synarthrosis:
    \end{enumerate}
  \end{enumerate}
\end{enumerate}

\includegraphics[width=4.54948in,height=3.88601in,alt={This image shows the lateral view of the human skeleton. The lambdoid, coronal, and squamous sutures are labeled."}]{images/media/image47.png}

\begin{enumerate}
\def\labelenumi{\arabic{enumi}.}
\setcounter{enumi}{1}
\item
  Amphiarthrosis:
\end{enumerate}

\includegraphics[width=3.13281in,height=2.263in]{images/media/image48.png}

\begin{enumerate}
\def\labelenumi{\arabic{enumi}.}
\setcounter{enumi}{2}
\item
  diarthrosis:
\end{enumerate}

\includegraphics[width=4.33073in,height=2.58872in]{images/media/image49.png}

\begin{enumerate}
\def\labelenumi{\Roman{enumi}.}
\setcounter{enumi}{1}
\item
  Fibrous joints

  \begin{enumerate}
  \def\labelenumii{\Alph{enumii}.}
  \item
    Suture:

    \begin{enumerate}
    \def\labelenumiii{\arabic{enumiii}.}
    \item
      Location of joint:
    \item
      Newborns and infants have wider areas between the bones containing
      connective tissue called \_\_\_\_\_\_\_\_\_\_\_\_\_\_.

      \begin{enumerate}
      \def\labelenumiv{\alph{enumiv})}
      \item
        How do they aid in delivery?
      \item
        Fusion of bones or \_\_\_\_\_\_\_\_\_\_\_\_\_\_\_.
      \end{enumerate}
    \end{enumerate}
  \end{enumerate}
\end{enumerate}

\includegraphics[width=3.89323in,height=2.60469in,alt={This figure shows the lateral view of the newborn skull with the major parts labeled.}]{images/media/image50.png}

\begin{enumerate}
\def\labelenumi{\Alph{enumi}.}
\setcounter{enumi}{1}
\item
  Syndesmosis:

  \begin{enumerate}
  \def\labelenumii{\arabic{enumii}.}
  \item
    Location of joint:
  \item
    \_\_\_\_\_\_\_\_\_\_\_ connect bone to bone.
  \item
    Interosseous membrane:
  \end{enumerate}
\item
  Gomphosis:

  \begin{enumerate}
  \def\labelenumii{\arabic{enumii}.}
  \item
    Also known as
    \_\_\_\_\_\_\_\_\_\_\_\_\_\_\_\_\_\_\_\_\_\_\_\_\_\_\_\_\_\_
  \item
    Location of joint:
  \item
    \_\_\_\_\_\_\_\_\_\_\_\_\_ because they are immobile.
  \end{enumerate}
\end{enumerate}

\begin{enumerate}
\def\labelenumi{\Roman{enumi}.}
\setcounter{enumi}{2}
\item
  Cartilaginous joints

  \begin{enumerate}
  \def\labelenumii{\Alph{enumii}.}
  \item
    Synchondrosis:

    \begin{enumerate}
    \def\labelenumiii{\arabic{enumiii}.}
    \item
      Location of joint:
    \item
      When would a synchondrosis joint be temporary or permanent?

      \begin{enumerate}
      \def\labelenumiv{\alph{enumiv})}
      \item
        Example of a temporary synchondrosis joint:
      \item
        Example of a permanent synchondrosis joint:
      \end{enumerate}
    \end{enumerate}
  \end{enumerate}
\end{enumerate}

\includegraphics[width=4.29948in,height=2.0395in]{images/media/image51.png}

\begin{enumerate}
\def\labelenumi{\Alph{enumi}.}
\setcounter{enumi}{1}
\item
  Symphysis:

  \begin{enumerate}
  \def\labelenumii{\arabic{enumii}.}
  \item
    Location of joint:
  \item
    \_\_\_\_\_\_\_\_\_\_\_\_\_ connects bones
  \end{enumerate}
\end{enumerate}

\begin{enumerate}
\def\labelenumi{\Roman{enumi}.}
\setcounter{enumi}{3}
\item
  Synovial joint
\end{enumerate}

\includegraphics[width=3.30879in,height=3.66406in,alt={This figure shows a synovial joint. The cavity between two bones contains the synovial fluid which lubricates the two joints.}]{images/media/image52.png}

\begin{enumerate}
\def\labelenumi{\Alph{enumi}.}
\item
  Where would you find an articular capsule?

  \begin{enumerate}
  \def\labelenumii{\arabic{enumii}.}
  \item
    Function:
  \end{enumerate}
\item
  Each bone is covered by a thin layer of hyaline cartilage called the
  \_\_\_\_\_\_\_\_\_\_\_\_ cartilage.
\item
  Lining each articular capsule is a \_\_\_\_\_\_\_\_\_ membrane which
  secrete \_\_\_\_\_\_\_\_\_\_\_ \_\_\_\_\_\_\_\_\_.
\item
  Compare and contrast ligaments and tendons.

  \begin{enumerate}
  \def\labelenumii{\arabic{enumii}.}
  \item
    Extrinsic ligament:
  \item
    Intrinsic ligament:
  \item
    Intracapsular ligament:
  \end{enumerate}
\item
  Bursa:

  \begin{enumerate}
  \def\labelenumii{\arabic{enumii}.}
  \item
    Located between the skin and underlying bone,
    \_\_\_\_\_\_\_\_\_\_\_\_\_\_\_ bursa.

    \begin{enumerate}
    \def\labelenumiii{\alph{enumiii})}
    \item
      Example:
    \end{enumerate}
  \item
    Found between the muscle and underlying bone,
    \_\_\_\_\_\_\_\_\_\_\_\_ bursa.

    \begin{enumerate}
    \def\labelenumiii{\alph{enumiii})}
    \item
      Example:
    \end{enumerate}
  \item
    Found between a tendon and bone, \_\_\_\_\_\_\_\_\_\_\_\_ bursa.

    \begin{enumerate}
    \def\labelenumiii{\alph{enumiii})}
    \item
      Example:
    \end{enumerate}
  \end{enumerate}
\item
  Inflammation of a bursa near a joint, \_\_\_\_\_\_\_\_\_\_\_.

  \begin{enumerate}
  \def\labelenumii{\arabic{enumii}.}
  \item
    Symptoms:
  \item
    Common areas of inflammation:
  \item
    Treatment:
  \end{enumerate}
\item
  Types of synovial joints:
\end{enumerate}

{\def\LTcaptype{none} % do not increment counter
\begin{longtable}[]{@{}
  >{\raggedright\arraybackslash}p{(\linewidth - 4\tabcolsep) * \real{0.3333}}
  >{\raggedright\arraybackslash}p{(\linewidth - 4\tabcolsep) * \real{0.3333}}
  >{\raggedright\arraybackslash}p{(\linewidth - 4\tabcolsep) * \real{0.3333}}@{}}
\toprule\noalign{}
\endhead
\bottomrule\noalign{}
\endlastfoot
Joint type & Location & Description \\
Pivot & & \\
Hinge & & \\
Condyloid & & \\
Saddle & & \\
Plane & & \\
Ball and Socket & & \\
\end{longtable}
}

\includegraphics[width=5.30625in,height=4.48889in]{images/media/image53.png}

\begin{enumerate}
\def\labelenumi{\Alph{enumi}.}
\setcounter{enumi}{7}
\item
  How is arthritis different from bursitis?

  \begin{enumerate}
  \def\labelenumii{\arabic{enumii}.}
  \item
    What bacterial or viral infections can lead to arthritis?
  \item
    Which type of arthritis is most common?
  \item
    Treatment for arthritis:
  \end{enumerate}
\end{enumerate}

\begin{enumerate}
\def\labelenumi{\Roman{enumi}.}
\setcounter{enumi}{4}
\item
  Types of body movement

  \begin{enumerate}
  \def\labelenumii{\Alph{enumii}.}
  \item
    Which joints aid in the body's ability to achieve range of motion?
  \item
    Compare and contrast flexion and extension.
  \item
    \_\_\_\_\_\_\_\_ is excessive extension of a joint beyond its normal
    range of motion, resulting in injury.
  \item
    Alternatively, \_\_\_\_\_\_\_\_\_\_\_\_ excessive flexion at a
    joint.
  \item
    Medial and lateral motions of limbs in the coronal plane is
    \_\_\_\_\_\_\_\_\_\_\_\_\_\_\_\_\_ and \_\_\_\_\_\_\_\_\_\_\_\_\_\_.

    \begin{enumerate}
    \def\labelenumiii{\arabic{enumiii}.}
    \item
      \_\_\_\_\_\_\_\_\_\_\_ lateral movement of a limb away from the
      midline of the body.

      \begin{enumerate}
      \def\labelenumiv{\alph{enumiv})}
      \item
        Give an example.
      \end{enumerate}
    \item
      \_\_\_\_\_\_\_\_\_\_\_ is movement of a limb toward the midline of
      the body.

      \begin{enumerate}
      \def\labelenumiv{\alph{enumiv})}
      \item
        Give an example.
      \end{enumerate}
    \end{enumerate}
  \item
    Would hula hooping be an example of circumduction? Explain.
  \item
    How is circumduction different from rotation?

    \begin{enumerate}
    \def\labelenumiii{\arabic{enumiii}.}
    \item
      Which joints are involved in rotation?
    \item
      How is medial rotation different from lateral rotation?
    \end{enumerate}
  \item
    Athletic shoe stores often help people find the best sneakers based
    on their supination or pronation. Describe the foot and leg
    alignment of a person that supinates versus pronates.
  \item
    Pointing of the toes is an example of
    \_\_\_\_\_\_\_\_\_\_\_\_\_\_\_\_\_\_.
  \item
    \_\_\_\_\_\_\_\_\_\_\_\_ is turning of the foot toward the midline.
  \item
    \_\_\_\_\_\_\_\_\_\_\_\_ is turning of the foot away from the
    midline of the body.
  \item
    Slouching in a chair would be an example of which type of body
    movement?
  \item
    Sitting erect with great posture in a chair would be an example
    of\_\_\_\_\_\_\_\_\_\_\_\_\_\_\_\_\_\_\_\_.
  \end{enumerate}
\item
  Anatomy of selected synovial joints

  \begin{enumerate}
  \def\labelenumii{\Alph{enumii}.}
  \item
    Adjacent vertebrae articulate with each other at
    \_\_\_\_\_\_\_\_\_\_\_\_\_\_\_\_ joints.

    \begin{enumerate}
    \def\labelenumiii{\arabic{enumiii}.}
    \item
      What types of joints are these?
    \end{enumerate}
  \item
    When the cervical vertebrae articulates with the occipital condyles
    of the skull the joint formed is the
    \_\_\_\_\_\_\_\_\_\_\_\_\_\_\_\_\_\_\_\_\_.
  \item
    This allows for the movement of the head for nonverbal
    \_\_\_\_\_\_\_\_\_.
  \item
    While the articulation of C1 and C2 vertebrae to allow nonverbal
    \_\_\_\_\_\_\_\_ is a result of the \_\_\_\_\_\_\_\_\_\_\_\_\_\_\_\_
    joint.
  \item
    \_\_\_\_\_\_\_\_\_\_\_\_\_\_\_\_\_\_\_\_\_ joint allows for the
    opening and closing of the mouth via mandibular depression and
    mandibular elevation.

    \begin{enumerate}
    \def\labelenumiii{\arabic{enumiii}.}
    \item
      This joint is formed by the articulation of which bones and parts
      of the bones?
    \item
      What causes the dislocation of this joint?
    \item
      Which demographic is mostly affected by disease associated with
      this joint?
    \item
      How is this treated?
    \end{enumerate}
  \end{enumerate}
\end{enumerate}

\includegraphics[width=4.20573in,height=3.96983in]{images/media/image54.png}

\begin{enumerate}
\def\labelenumi{\Alph{enumi}.}
\setcounter{enumi}{5}
\item
  Another name for the shoulder joint is the
  \_\_\_\_\_\_\_\_\_\_\_\_\_\_\_\_.
\item
  Describe the rotator cuff. What causes injury to the rotator cuff?

  \begin{enumerate}
  \def\labelenumii{\arabic{enumii}.}
  \item
    Another name for a frozen shoulder is a
    \_\_\_\_\_\_\_\_\_\_\_\_\_\_\_\_\_\_\_\_\_\_.

    \begin{enumerate}
    \def\labelenumiii{\alph{enumiii})}
    \item
      What causes this to occur?
    \end{enumerate}
  \end{enumerate}
\end{enumerate}

\includegraphics[width=3.50781in,height=2.5634in]{images/media/image55.png}

\begin{enumerate}
\def\labelenumi{\Alph{enumi}.}
\setcounter{enumi}{7}
\item
  The uniaxial hinge joint that makes up the elbow is the
  \_\_\_\_\_\_\_\_\_\_\_\_\_\_\_ joint.

  \begin{enumerate}
  \def\labelenumii{\arabic{enumii}.}
  \item
    Which bones and parts articulate to form this joint?
  \item
    How is hyperextension prevented at the elbow joint?
  \item
    The \_\_\_\_\_\_\_\_\_\_\_\_ ligament is on the medial side of the
    joint, while the \_\_\_\_\_\_\_\_\_\_\_\_\_\_ ligament supports the
    lateral side of the joint.
  \item
    The \_\_\_\_\_\_\_\_\_\_\_\_\_ ligaments encircle the radius head.
  \end{enumerate}
\item
  The hip joint is a \_\_\_\_\_\_\_\_\_\_\_\_ ball-and-socket joint
  between the \_\_\_\_\_\_\_\_\_ and \_\_\_\_\_\_\_\_.

  \begin{enumerate}
  \def\labelenumii{\arabic{enumii}.}
  \item
    The socket portion of the hip joint is the \_\_\_\_\_\_\_\_\_\_.
  \item
    When in the upright standing position which ligaments pull the head
    of the femur deeply into the acetabulum?
  \item
    Why is the hip prone to osteoarthritis?
  \end{enumerate}
\item
  The largest joint in the body is the \_\_\_\_\_\_\_\_\_\_\_\_\_\_\_.

  \begin{enumerate}
  \def\labelenumii{\arabic{enumii}.}
  \item
    Why is this joint so large?
  \item
    The patella serves to protect the \_\_\_\_\_\_\_\_\_\_\_\_\_ from
    friction against the \_\_\_\_\_\_\_\_\_\_\_\_\_\_\_\_.
  \item
    Describe the dynamic ligament.
  \item
    When a patient has a torn meniscus, what does this mean? How is it
    treated?

    \begin{enumerate}
    \def\labelenumiii{\alph{enumiii})}
    \item
      How does this affect their ability to walk?
    \end{enumerate}
  \item
    When would a person need a knee replacement?
  \end{enumerate}
\item
  The ankle is formed by the \_\_\_\_\_\_\_\_\_\_\_\_ joint.

  \begin{enumerate}
  \def\labelenumii{\arabic{enumii}.}
  \item
    What occurs during an ankle sprain?
  \item
    How is it treated?
  \item
    How does this affect mobility?
  \end{enumerate}
\end{enumerate}

\begin{enumerate}
\def\labelenumi{\Roman{enumi}.}
\setcounter{enumi}{6}
\item
  Development of joints

  \begin{enumerate}
  \def\labelenumii{\Alph{enumii}.}
  \item
    When are joints formed?

    \begin{enumerate}
    \def\labelenumiii{\arabic{enumiii}.}
    \item
      How?
    \end{enumerate}
  \item
    Where do synovial joints form?
  \end{enumerate}
\end{enumerate}

\section{\texorpdfstring{Chapter 10 }{Chapter 10 }}\label{chapter-10}

\begin{enumerate}
\def\labelenumi{\Roman{enumi}.}
\item
  Muscle tissue

  \begin{enumerate}
  \def\labelenumii{\Alph{enumii}.}
  \item
    All three types of muscle tissue exhibit \_\_\_\_\_\_\_\_\_\_\_,
    which means
    \_\_\_\_\_\_\_\_\_\_\_\_\_\_\_\_\_\_\_\_\_\_\_\_\_\_\_\_\_\_\_\_\_\_\_\_\_\_\_\_\_\_\_\_\_\_\_\_\_\_\_\_\_\_.
  \item
    When a muscle contracts, the muscle fibers \_\_\_\_\_\_\_\_\_\_\_\_.

    \begin{enumerate}
    \def\labelenumiii{\arabic{enumiii}.}
    \item
      This occurs when \_\_\_\_\_\_\_\_\_ protein pulls on
      \_\_\_\_\_\_\_\_\_\_ protein.
    \item
      What happens to expose the actin binding sites?
    \item
      As a person ages what happens to the muscle elasticity?
    \end{enumerate}
  \item
    Which two types of muscle have striations?

    \begin{enumerate}
    \def\labelenumiii{\arabic{enumiii}.}
    \item
      How are they morphologically different?
    \end{enumerate}
  \item
    Why is smooth muscle not striated?

    \begin{enumerate}
    \def\labelenumiii{\arabic{enumiii}.}
    \item
      Does this affect its contractibility?
    \end{enumerate}
  \end{enumerate}
\item
  Skeletal muscle

  \begin{enumerate}
  \def\labelenumii{\Alph{enumii}.}
  \item
    What is the function of skeletal muscle?
  \item
    How is heat generated by the skeletal muscle?
  \item
    Skeletal muscle consists of three connective tissue layers or
    \_\_\_\_\_\_\_\_.

    \begin{enumerate}
    \def\labelenumiii{\arabic{enumiii}.}
    \item
      List the three layers and their functions

      \begin{enumerate}
      \def\labelenumiv{\alph{enumiv})}
      \item
        \_\_\_\_\_\_\_\_\_\_\_\_\_ is the outermost layer of connective
        tissue around the muscle.

        \begin{enumerate}
        \def\labelenumv{(\arabic{enumv})}
        \item
          It consists of \_\_\_\_\_\_\_\_\_\_\_ connective tissue which
          allows muscle contraction and movement while maintaining
          structural integrity.
        \end{enumerate}
      \item
        The middle layer of connective tissue or
        \_\_\_\_\_\_\_\_\_\_\_\_.

        \begin{enumerate}
        \def\labelenumv{(\arabic{enumv})}
        \item
          This layers helps to organize bundles of muscle into
          \_\_\_\_\_\_\_\_\_\_\_\_.
        \end{enumerate}
      \item
        Each muscle fiber is encased in a thin layer of
        \_\_\_\_\_\_\_\_\_\_\_\_ and \_\_\_\_\_\_\_\_\_\_ fibers in the
        \_\_\_\_\_\_\_\_\_\_\_\_ layer of connective tissue.
      \end{enumerate}
    \end{enumerate}
  \end{enumerate}
\end{enumerate}

\includegraphics[width=4.10156in,height=4.23302in,alt={This figure shows the structure of muscle fibers. The top panel shows a skeleton muscle fiber, and a magnified view of the muscle fascicles are shown. The middle panel shows a magnified view of the muscle fascicles with the muscle fibers, perimysium and the endomysium. The bottom panel shows the structure of the muscle fiber with the sarcolemma highlighted."\textgreater{} \textless img src="/resources/7990f132356dc4b9ee2949417ac064b52b47b06f" data-media-type="image/jpg" alt="This figure shows the structure of muscle fibers. The top panel shows a skeleton muscle fiber, and a magnified view of the muscle fascicles are shown. The middle panel shows a magnified view of the muscle fascicles with the muscle fibers, perimysium and the endomysium. The bottom panel shows the structure of the muscle fiber with the sarcolemma highlighted.}]{images/media/image56.png}

\begin{enumerate}
\def\labelenumi{\Alph{enumi}.}
\setcounter{enumi}{3}
\item
  Explain how tendons and skeletal muscle works to pull on the bone.

  \begin{enumerate}
  \def\labelenumii{\arabic{enumii}.}
  \item
    Tendon-like sheets or \_\_\_\_\_\_\_\_\_\_\_\_\_\_ connective tissue
    between skin and bones.
  \end{enumerate}
\item
  Skeletal muscle fibers can have a diameter of up to \_\_\_\_\_\_ and a
  length of up to \_\_\_\_\_\_\_\_ in the upper leg.

  \begin{enumerate}
  \def\labelenumii{\arabic{enumii}.}
  \item
    Define

    \begin{enumerate}
    \def\labelenumiii{\alph{enumiii})}
    \item
      Sacrcolemma:
    \item
      Sarcoplasm:
    \item
      Sarcoplasmic reticulum:
    \item
      Sarcomere:
    \item
      Tropomyosin:
    \item
      Troponin:
    \item
      myofibril:
    \end{enumerate}
  \end{enumerate}
\end{enumerate}

\includegraphics[width=4.35156in,height=3.48683in,alt={This figure shows the structure of the muscle fibers. In the top panel, a sarcolemma is shown with the major parts labeled. In the bottom panel, a magnified view of a single myofibril is shown and the major parts are labeled.}]{images/media/image57.png}

\begin{enumerate}
\def\labelenumi{\arabic{enumi}.}
\setcounter{enumi}{1}
\item
  The thin filament of a sarcomere is the \_\_\_\_\_\_\_\_\_\_.
\item
  The thick filament of a sarcomere is the \_\_\_\_\_\_\_\_\_\_\_.
\item
  The \_\_\_\_\_\_\_\_\_\_\_\_\_\_\_\_\_ junction is where the neuron
  terminal meets the muscle fiber.
\item
  Describe a membrane potential.

  \begin{enumerate}
  \def\labelenumii{\alph{enumii})}
  \item
    What causes the electrical charge of the cell?
  \item
    Which cations are more prevalent inside of the cell at resting?
  \item
    Which cations are more prevalent inside of the cell during
    depolarization?
  \item
    Describe an action potential generation and the different parts of
    the wave.
  \item
    \_\_\_\_\_\_\_\_\_\_\_-\_\_\_\_\_\_\_\_\_\_\_\_ coupling is used to
    explain the idea that skeletal muscle contraction is relied upon
    \_\_\_\_\_\_\_\_\_\_\_\_ or stimulation.

    \begin{enumerate}
    \def\labelenumiii{(\arabic{enumiii})}
    \item
      Release of \_\_\_\_\_\_ ions from the sarcoplasmic reticulum
      allows interaction with shielding proteins causing the
      actin-binding sites to become exposed for myosin binding.
    \item
      What is the motor end-plate?

      \begin{enumerate}
      \def\labelenumiv{(\alph{enumiv})}
      \item
        Where is it located?
      \end{enumerate}
    \end{enumerate}
  \item
    At the neuromuscular junction, which neurotransmitter is released
    from the neurons?

    \begin{enumerate}
    \def\labelenumiii{(\arabic{enumiii})}
    \item
      What is the function of the neuron?
    \end{enumerate}
  \end{enumerate}
\end{enumerate}

\begin{enumerate}
\def\labelenumi{\Roman{enumi}.}
\setcounter{enumi}{2}
\item
  Muscle fiber contraction and relaxation

  \begin{enumerate}
  \def\labelenumii{\Alph{enumii}.}
  \item
    What is the first step of muscle contraction?
  \item
    This will lead to the depolarization of muscle membranes once
    \_\_\_\_\_\_\_\_\_\_\_\_\_\_\_\_\_\_\_\_\_\_\_\_\_.
  \item
    Next, \_\_\_\_\_\_\_\_\_\_\_ ions are released from the
    \_\_\_\_\_\_\_\_\_\_\_\_\_\_\_\_\_\_\_\_.

    \begin{enumerate}
    \def\labelenumiii{\arabic{enumiii}.}
    \item
      This causes the initial \_\_\_\_\_\_\_\_\_\_\_\_\_\_.
    \end{enumerate}
  \item
    Calcium then binds to \_\_\_\_\_\_\_\_\_\_\_\_ once the myosin
    binding site is exposed by the morphological change of
    \_\_\_\_\_\_\_\_\_\_\_\_\_\_\_\_.
  \item
    \_\_\_\_\_\_\_\_ binds to actin when ATP is available, pulling on
    actin and shortening the \_\_\_\_\_\_\_\_\_\_\_\_\_.
  \item
    When does muscle contraction stop?
  \item
    How are cross bridge formation and the sliding filament model
    different?
  \item
    How does creatine phosphate aid in muscle contraction?
  \item
    What is the role of potassium in muscle contraction?
  \item
    Define oxygen debt:
  \item
    Muscular dystrophy:

    \begin{enumerate}
    \def\labelenumiii{\arabic{enumiii}.}
    \item
      What is the function of dystrophin in muscle contraction?
    \item
      Duchenne muscular dystrophy mostly affects males, why?
    \item
      Why would the use of myoblast be thought to aid in treatment of
      this disease?

      \begin{enumerate}
      \def\labelenumiv{\alph{enumiv})}
      \item
        Why has this treatment not worked?
      \end{enumerate}
    \end{enumerate}
  \end{enumerate}
\item
  Nervous system control of muscle tension

  \begin{enumerate}
  \def\labelenumii{\Alph{enumii}.}
  \item
    Muscle tension:

    \begin{enumerate}
    \def\labelenumiii{\arabic{enumiii}.}
    \item
      \_\_\_\_\_\_\_\_\_\_\_\_\_\_ contractions is when the tension in
      the muscle stays constant and a load is moved as the muscle length
      changes.

      \begin{enumerate}
      \def\labelenumiv{\alph{enumiv})}
      \item
        \_\_\_\_\_\_\_\_\_\_\_\_\_ contraction involves the shortening
        of muscle to move a load.

        \begin{enumerate}
        \def\labelenumv{(\arabic{enumv})}
        \item
          Example:
        \end{enumerate}
      \item
        \_\_\_\_\_\_\_\_\_\_\_\_\_ contraction occurs as tension
        diminishes and muscle lengthens.

        \begin{enumerate}
        \def\labelenumv{(\arabic{enumv})}
        \item
          Example:
        \end{enumerate}
      \end{enumerate}
    \item
      \_\_\_\_\_\_\_\_\_\_\_\_\_\_\_ contraction is when tension is
      produced without changing the angle of the skeletal joint.

      \begin{enumerate}
      \def\labelenumiv{\alph{enumiv})}
      \item
        Why will this contraction result in no movement of a load?

        \begin{enumerate}
        \def\labelenumv{(\arabic{enumv})}
        \item
          Example:
        \end{enumerate}
      \end{enumerate}
    \end{enumerate}
  \end{enumerate}
\end{enumerate}

\includegraphics[width=5.70833in,height=3.75781in,alt={This figure shows the different types of muscle contraction and the associated body movements. The top panel shows concentric contraction, the middle panel shows eccentric contraction, and the bottom panel shows isometric contraction." }]{images/media/image58.png}

\begin{enumerate}
\def\labelenumi{\Alph{enumi}.}
\setcounter{enumi}{1}
\item
  Motor unit:
\end{enumerate}

\includegraphics[width=4.59635in,height=3.47673in]{images/media/image59.png}

\begin{enumerate}
\def\labelenumi{\arabic{enumi}.}
\item
  A \_\_\_\_\_\_\_\_\_ motor unit allows for very fine motor control of
  muscle because a single motor neuron supplies a small number of muscle
  fibers in a muscle.

  \begin{enumerate}
  \def\labelenumii{\alph{enumii})}
  \item
    Example:
  \end{enumerate}
\item
  A \_\_\_\_\_\_\_\_\_\_ motor unit allows for gross movement because a
  single motor neuron supplies a large number of muscle fibers in a
  muscle.

  \begin{enumerate}
  \def\labelenumii{\alph{enumii})}
  \item
    Example:
  \end{enumerate}
\item
  Which of the motor units is most excitable?
\item
  How are motor units employed to lift a piano buy two men up to the
  third floor of a building?
\end{enumerate}

\begin{enumerate}
\def\labelenumi{\Alph{enumi}.}
\setcounter{enumi}{2}
\item
  Describe the length-tension relationship.

  \begin{enumerate}
  \def\labelenumii{\arabic{enumii}.}
  \item
    What is the ideal length of a sarcomere to produce maximal tension?
  \end{enumerate}
\item
  Twitch:
\item
  Treppe:
\end{enumerate}

\includegraphics[width=3.86458in,height=1.70573in]{images/media/image60.png}

\begin{enumerate}
\def\labelenumi{\Alph{enumi}.}
\setcounter{enumi}{5}
\item
  Muscle tone:
\item
  Hypotonia:
\item
  Hypertonia:
\end{enumerate}

\begin{enumerate}
\def\labelenumi{\Roman{enumi}.}
\setcounter{enumi}{4}
\item
  Types of muscle fibers

  \begin{enumerate}
  \def\labelenumii{\Alph{enumii}.}
  \item
    What two criteria are important for classifying muscle fibers?
  \item
    The three main types of skeletal muscle fibers are:

    \begin{enumerate}
    \def\labelenumiii{\arabic{enumiii}.}
    \item
      Slow oxidative fibers which
      \_\_\_\_\_\_\_\_\_\_\_\_\_\_\_\_\_\_\_\_\_\_\_\_\_\_\_\_\_\_\_\_.
    \item
      Fast oxidative fibers which
      \_\_\_\_\_\_\_\_\_\_\_\_\_\_\_\_\_\_\_\_\_\_\_\_\_\_\_\_\_\_\_\_\_\_.
    \item
      Fast glycolytic fibers which
      \_\_\_\_\_\_\_\_\_\_\_\_\_\_\_\_\_\_\_\_\_\_\_\_\_\_\_\_\_\_\_\_\_\_\_.
    \end{enumerate}
  \item
    Contraction speed is dependent on the \_\_\_\_\_\_\_\_\_ATPase
    hydrolysis of ATP.
  \item
    Primary metabolic pathway used by a muscle fiber determines if the
    fiber is \_\_\_\_\_\_\_\_\_\_\_\_ or \_\_\_\_\_\_\_\_\_\_\_.

    \begin{enumerate}
    \def\labelenumiii{\arabic{enumiii}.}
    \item
      Compare and contrast the two types of pathways.
    \end{enumerate}
  \item
    Slow oxidative fibers are able to function for long periods of time
    with fatigue. How does this aid doctors during long surgeries?
  \item
    Fast glycolytic fibers contain high amounts of
    \_\_\_\_\_\_\_\_\_\_\_\_ which allows for the production of ATP via
    \_\_\_\_\_\_\_\_\_\_\_\_\_. These fibers produce high levels of
    \_\_\_\_\_\_\_\_\_\_.

    \begin{enumerate}
    \def\labelenumiii{\arabic{enumiii}.}
    \item
      Are these muscle fibers slow or fast to fatigue? Explain.
    \end{enumerate}
  \item
    What type(s) of fiber(s) would you expect to find in the lower back
    of someone working from home during quarantine?
  \end{enumerate}
\item
  Exercise and muscle performance

  \begin{enumerate}
  \def\labelenumii{\Alph{enumii}.}
  \item
    When bodybuilders begin their weight lifting journey, are they
    making new muscle cells? Explain.

    \begin{enumerate}
    \def\labelenumiii{\arabic{enumiii}.}
    \item
      Hypertrophy:
    \item
      Atrophy:
    \end{enumerate}
  \item
    Compare and contrast slow twitch muscle and fast twitch muscle.
  \item
    Explain a time when a distance runner would utilize fast twitch?
  \item
    Performance enhancing drugs are used to
    \_\_\_\_\_\_\_\_\_\_\_\_\_\_\_\_\_\_\_\_\_\_\_\_\_\_\_\_\_\_\_\_\_\_\_\_\_\_\_\_\_\_\_\_\_\_\_\_\_\_\_.

    \begin{enumerate}
    \def\labelenumiii{\arabic{enumiii}.}
    \item
      \_\_\_\_\_\_\_\_\_\_\_\_\_ steroids are used to boost muscle mass
      and power output. These steroids are forms of
      \_\_\_\_\_\_\_\_\_\_\_\_\_\_.
    \item
      Are performance enhancing drugs legal?
    \end{enumerate}
  \item
    Describe how muscle tissue is affected by aging?
  \item
    How would muscle tissue be affected in persons that are
    quadriplegic?
  \end{enumerate}
\item
  Cardiac muscle tissue

  \begin{enumerate}
  \def\labelenumii{\Alph{enumii}.}
  \item
    Contraction of cardiac muscle allows pumping of \_\_\_\_\_\_\_\_\_\_
    into the vessels of the \_\_\_\_\_\_\_\_\_\_ system.
  \item
    Unlike skeletal muscle fibers, cardiac muscle contains many
    \_\_\_\_\_\_\_\_\_\_\_\_\_\_\_\_ and \_\_\_\_\_\_\_\_\_\_ for ATP
    production.
  \item
    Cardiac muscle contracts in waves to allow the heart to pump via
    \_\_\_\_\_\_\_\_\_\_\_\_\_\_\_\_\_\_\_\_.

    \begin{enumerate}
    \def\labelenumiii{\arabic{enumiii}.}
    \item
      Two important structures important for muscle contraction are
      \_\_\_\_\_\_\_\_\_\_\_\_\_ and
      \_\_\_\_\_\_\_\_\_\_\_\_\_\_\_\_\_\_\_.
    \item
      Explain how each will aid in muscle contraction.
    \end{enumerate}
  \item
    Describe the role of the pacemaker in muscle contraction.

    \begin{enumerate}
    \def\labelenumiii{\arabic{enumiii}.}
    \item
      Define functional syncytium:
    \end{enumerate}
  \item
    Why are longer action potentials important for cardiac muscle?
  \end{enumerate}
\item
  Smooth muscle

  \begin{enumerate}
  \def\labelenumii{\Alph{enumii}.}
  \item
    Location:
  \item
    How does smooth muscle aid in the eye?
  \item
    Smooth muscle fibers are football shaped with \_\_\_\_\_\_ nucleus,
    ranging from \_\_\_\_\_\_\_\_\_\_\_\_\_\_\_\_\_\_ in length.
  \item
    Will you find sarcomeres in smooth muscle? Why or why not?
  \item
    \_\_\_\_\_\_\_\_\_\_\_\_ body is analogous to the Z-discs of
    skeletal and cardiac muscle and fastened to the
    \_\_\_\_\_\_\_\_\_\_\_\_\_\_.
  \item
    Define calveoli:
  \item
    Smooth muscle cells contain \_\_\_\_\_\_\_\_\_\_\_ instead of the
    troponin-tropomyosin complex.
  \item
    Calcium ions pass through the opened calcium channels in the
    \_\_\_\_\_\_\_\_\_\_\_\_, and additional calcium is released from
    the \_\_\_\_\_\_\_\_\_, binds to \_\_\_\_\_\_\_\_\_\_.
  \item
    Myosin kinase is then activated by the
    \_\_\_\_\_\_\_\_\_\_\_\_\_\_\_\_\_\_\_\_ complex; this will then
    activate the \_\_\_\_\_\_\_\_\_\_\_\_\_\_\_.

    \begin{enumerate}
    \def\labelenumiii{\arabic{enumiii}.}
    \item
      The myosin head is then \_\_\_\_\_\_\_\_\_\_\_\_\_ which causes
      the activation of ATP by converting
      \_\_\_\_\_\_\_\_\_\_\_\_\_\_\_\_\_\_\_\_\_\_\_.
    \item
      The \_\_\_\_\_\_\_ filaments are anchored to the \_\_\_\_\_\_\_\_;
      the structures invested in the inner membrane of the
      \_\_\_\_\_\_\_\_\_\_ at \_\_\_\_\_\_\_\_\_ junctions, have
      cord-like intermediate filaments attached to them.
    \item
      What happens when the thin filaments slide past the thick
      filaments?
    \end{enumerate}
  \end{enumerate}
\end{enumerate}

\includegraphics[width=6.5in,height=2.79167in]{images/media/image61.png}

\begin{enumerate}
\def\labelenumi{\Alph{enumi}.}
\setcounter{enumi}{9}
\item
  When does muscle contraction stop?
\item
  Describe the latch-bridge and smooth muscle contraction.
\item
  The division of smooth muscle to produce more cells is
  \_\_\_\_\_\_\_\_\_\_\_\_\_\_.
\end{enumerate}

\begin{enumerate}
\def\labelenumi{\Roman{enumi}.}
\setcounter{enumi}{8}
\item
  Development and regeneration of muscle tissue

  \begin{enumerate}
  \def\labelenumii{\Alph{enumii}.}
  \item
    Which embryonic tissue will muscle arise from?

    \begin{enumerate}
    \def\labelenumiii{\arabic{enumiii}.}
    \item
      Skeletal muscle of head and limbs develop from?
    \item
      Other skeletal muscles arise from
      \_\_\_\_\_\_\_\_\_\_\_\_\_\_\_\_.
    \item
      Define myoblast:

      \begin{enumerate}
      \def\labelenumiv{\alph{enumiv})}
      \item
        Myotube is made from the fusion of several different myoblast
        cells leading to the \_\_\_\_\_\_\_\_\_\_\_\_\_.
      \end{enumerate}
    \end{enumerate}
  \item
    Unlike the skeletal muscle, \_\_\_\_\_\_\_\_\_ junctions develop in
    cardiac and single-unit smooth muscle during early development.
  \item
    ACh receptors are initially present along most of the surface of the
    \_\_\_\_\_\_\_\_\_\_\_.

    \begin{enumerate}
    \def\labelenumiii{\arabic{enumiii}.}
    \item
      Spinal nerve innervation causes the release of growth factors that
      stimulate the formation of \_\_\_\_\_\_\_\_\_\_\_\_\_\_.
    \end{enumerate}
  \item
    How are satellite cells and myoblast different?
  \item
    Stem cells for smooth muscle cells are \_\_\_\_\_\_\_\_\_\_\_.

    \begin{enumerate}
    \def\labelenumiii{\arabic{enumiii}.}
    \item
      Why are smooth muscle cells able to regenerate more readily than
      cardiac and skeletal muscle?
    \end{enumerate}
  \item
    What happens when a muscle cell dies?
  \item
    \_\_\_\_\_\_\_\_\_\_ work with patients to maintain muscles and
    target muscles susceptible to \_\_\_\_\_\_\_\_\_\_\_\_.

    \begin{enumerate}
    \def\labelenumiii{\arabic{enumiii}.}
    \item
      How can they help prevent atrophy?
    \item
      How can physical therapy aid patients that have been in a coma for
      some time?
    \end{enumerate}
  \end{enumerate}
\end{enumerate}

\section{}\label{section-7}

\section{Chapter 11}\label{chapter-11}

\begin{enumerate}
\def\labelenumi{\arabic{enumi}.}
\item
  Interactions of Skeletal Muscles, Their Fascicle Arrangement, and
  Their Lever Systems
\end{enumerate}

\begin{enumerate}
\def\labelenumi{\Alph{enumi}.}
\item
  The skeletal muscles of the muscular system account for almost half of
  the weight of our bodies.
\end{enumerate}

\begin{enumerate}
\def\labelenumi{\arabic{enumi}.}
\item
  The human body contains approximately 700 skeletal muscles that differ
  widely in size, shape, and function.
\item
  Although the individual skeletal muscle fibers contract the same way
  and to the same degree, the performance of a skeletal muscle varies
  depending on the way the muscle fibers are organized and how the
  muscles attach to the skeleton.
\end{enumerate}

\begin{enumerate}
\def\labelenumi{\Alph{enumi}.}
\setcounter{enumi}{1}
\item
  All muscles have at least two points of attachment:

  \begin{enumerate}
  \def\labelenumii{\alph{enumii}.}
  \item
    \textbf{\ul{Origin}} = the fixed attachment point
  \item
    \textbf{\ul{Insertion}} = the moveable attachment point
  \item
    The origin is typically proximal to the insertion when the body is
    in anatomical position.
  \end{enumerate}
\item
  When complex movements occur, muscles commonly work in groups rather
  than individually. Their cooperation improves the efficiency of a
  particular movement. For example, large muscles of the limbs produce
  flexion or extension over an extended range of motion.

  \begin{enumerate}
  \def\labelenumii{\alph{enumii}.}
  \item
    \textbf{\ul{Agonist}}=a muscle that provides the major force for
    producing a specific movement.~ Also known as the ``prime mover''.
  \item
    \textbf{\ul{Antagonist}}=muscles that oppose, or reverse, a
    particular movement.
  \item
    \textbf{\ul{Synergists}}=muscles that help prime movers by: Adding a
    little extra force to the same movement or undesirable or
    unnecessary movements that might occur as the prime movers contract.
  \item
    \textbf{\ul{Fixators}}=when synergists immobilize a bone, or a
    muscle's origin.
  \end{enumerate}
\item
  Fascicle Organization

  \begin{enumerate}
  \def\labelenumii{\arabic{enumii}.}
  \item
    \textbf{\ul{Circular}} = also called a sphincter muscle; when the
    fascicles are arranged in concentric rings (example:~ orbicularis
    oris)
  \item
    \textbf{\ul{Convergent}}=when the muscle has a broad origin and the
    fascicles converge toward a single tendon or insertion (example:~
    pectoralis major)
  \item
    \textbf{\ul{Parallel}}=the long axes of the fascicles run parallel
    to the long axis of the muscle.~ Strap-like muscles (example:~
    sartorius and biceps brachii)
  \item
    \textbf{\ul{Pennate}}=the fascicles are short and they attach
    obliquely to a central tendon that runs the length of the muscle.

    \begin{enumerate}
    \def\labelenumiii{\alph{enumiii}.}
    \item
      \textbf{\ul{Unipennate}}=the fascicles insert into only one side
      of the tendon (example:~ extensor digitorum longus)
    \item
      \textbf{\ul{Bipennate}}=fascicles insert into the tendon from
      opposite sides so that the muscle's ``grain'' resembles a feather
      (example:~ rectus femoris)
    \item
      \textbf{\ul{Multipennate}}=arrangement looks like many feathers
      situated side by side (example:~ deltoid)
    \end{enumerate}
  \end{enumerate}
\item
  Muscle Mechanics: Lever Systems

  \begin{enumerate}
  \def\labelenumii{\arabic{enumii}.}
  \item
    The operation of most skeletal muscles involves the use of leverage
    and a lever system.
  \item
    A \textbf{\ul{lever}} is a rigid bar -- such as a board, a crowbar,
    or a bone -- that moves on a fixed point, or \textbf{\ul{fulcrum}},
    when a force is applied to it. The applied force, or
    \textbf{\ul{effort}}, is used to move a resistance, or
    \textbf{\ul{load}}.
  \item
    In the human body, joints are the fulcrums, your bones act as
    levers, and your muscle provide the effort.
  \item
    Levers can operate in one of two ways:

    \begin{enumerate}
    \def\labelenumiii{\alph{enumiii}.}
    \item
      \textbf{\ul{Mechanical advantage}}: the load is close to the
      fulcrum and the effort is applied far from the fulcrum. This
      situation requires minimal effort to move a large load and is
      therefore designed for power. \emph{Power lever}
    \item
      \textbf{\ul{Mechanical disadvantage}}: the load is far from the
      fulcrum and the effort is applied near the fulcrum. This situation
      requires the force to be greater than the load to be moved but
      although it cannot move a large load, it can move loads farther
      and faster. \emph{Speed lever}
    \end{enumerate}
  \item
    Depending on the relative positions of the three elements (effort,
    fulcrum, and load), a lever belongs to one of three classes:

    \begin{enumerate}
    \def\labelenumiii{\alph{enumiii}.}
    \item
      \textbf{\ul{First-class lever}} = the fulcrum (F) lies between the
      applied force (AF) and the load (L). An example: scissors or the
      capitis muscles.
    \item
      \textbf{\ul{Second-class lever}} = the load (L) lies between the
      applied force (AF) and the fulcrum (F). An example: wheelbarrow or
      the gastrocnemius.
    \item
      \textbf{\ul{Third-class lever}} = the most common of levers in the
      body, the applied force (AF) is located between the load (L) and
      the fulcrum (F). An example: tweezers and biceps brachii.
    \end{enumerate}
  \end{enumerate}
\end{enumerate}

11.2 Muscle Names

\begin{enumerate}
\def\labelenumi{\Alph{enumi}.}
\item
  Common terms are used in the names of muscles (see the table on page
  430)

  \begin{enumerate}
  \def\labelenumii{\arabic{enumii}.}
  \item
    Terms indicating specific regions of the body

    \begin{enumerate}
    \def\labelenumiii{\alph{enumiii}.}
    \item
      Abdominis (abdomen)
    \item
      Brachialis (brachium)
    \item
      Capitis (head)
    \item
      Carpi (wrist)
    \item
      Cervicis (neck)
    \item
      Cleido- or clavius (clavicle)
    \item
      Coccygeus (coccyx)
    \item
      Costalis (rib)
    \item
      Femoris (femur)
    \item
      Glosso- or glossal (tongue)
    \item
      Hallucis (great toe)
    \item
      Ilio- (ilium)
    \item
      Inguinal (groin)
    \item
      Nasalis (nose)
    \item
      Oculo (eye)
    \item
      Oris (mouth)
    \item
      Palpebrae (eyelid)
    \item
      Pollicis (thumb)
    \item
      Popliteus (back of knee)
    \item
      Psoas (loin)
    \item
      Radialis (radius)
    \item
      Scapularis (scapula)
    \item
      Temporalis (temporal)
    \item
      Tibialis (tibia)
    \item
      Ulnaris (ulna)
    \item
      Uro- (urinary)
    \end{enumerate}
  \item
    Terms indicating position, direction, or fascicle organization

    \begin{enumerate}
    \def\labelenumiii{\alph{enumiii}.}
    \item
      Anterior (front)
    \item
      Externus (superficial)
    \item
      Extrinsic (outside)
    \item
      Inferioris (inferior)
    \item
      Internus (deep, internal)
    \item
      Lateralis (side)
    \item
      Mediallis and medius (in the middle).
    \item
      Oblique (diagonal)
    \item
      Posterior (back)
    \item
      Profundus (neck)
    \item
      Rectus (straight or paralleled)
    \item
      Superficialis (superficial)
    \item
      Superiorus (superior)
    \item
      Transversus (transverse)
    \end{enumerate}
  \item
    Terms indicating structural characteristics

    \begin{enumerate}
    \def\labelenumiii{\alph{enumiii}.}
    \item
      Number of origins

      \begin{enumerate}
      \def\labelenumiv{\roman{enumiv}.}
      \item
        Biceps (two origins)
      \item
        Triceps (three origins)
      \item
        Quadriceps (four origins)
      \end{enumerate}
    \item
      Shape

      \begin{enumerate}
      \def\labelenumiv{\roman{enumiv}.}
      \item
        Deltoid (triangle)
      \item
        Orbicularis (circle)
      \item
        Pectinate (comblike)
      \item
        Platy (flat)
      \item
        Rhomboid (rhombus)
      \item
        Serratus (serrated)
      \item
        Spleinus (bandage)
      \item
        Teres (long and round)
      \item
        Trapezius (trapezoid)
      \end{enumerate}
    \item
      Other striking features

      \begin{enumerate}
      \def\labelenumiv{\roman{enumiv}.}
      \item
        Alba (white)
      \item
        Brevis (short)
      \item
        Gracilis (slender)
      \item
        Lata (wide)
      \item
        Latissimus (widest)
      \item
        Longissimus (longest)
      \item
        Longus (long)
      \item
        Magnus (large)
      \item
        Major (larger)
      \item
        Maximus (largest)
      \item
        Minimus (smallest)
      \item
        Minor (smaller)
      \item
        Vastus (great)
      \end{enumerate}
    \end{enumerate}
  \item
    Terms indicating actions

    \begin{enumerate}
    \def\labelenumiii{\alph{enumiii}.}
    \item
      General

      \begin{enumerate}
      \def\labelenumiv{\roman{enumiv}.}
      \item
        Abductor
      \item
        Adductor
      \item
        Depressor
      \item
        Extensor
      \item
        Flexor
      \item
        Levator
      \item
        Pronator
      \item
        Rotator
      \item
        Supinator
      \item
        Tensor
      \end{enumerate}
    \item
      Specific

      \begin{enumerate}
      \def\labelenumiv{\roman{enumiv}.}
      \item
        Buccinator (trumpeter)
      \item
        Risorius (laughter)
      \item
        Sartorius (like a tailor)
      \end{enumerate}
    \end{enumerate}
  \end{enumerate}
\end{enumerate}

11.3 Muscles of facial expression

A. \textbf{Frontalis}: wrinkle skin of forehead, raise eyebrows

B. Orbicularis oculi: close eyelids

C. \textbf{Orbicularis oris}: close and purse lips

D. \textbf{Risorius}: pulls corners of lips laterally, grimace

E. \textbf{Zygomaticus} \textbf{major} and \textbf{zygomaticus}
\textbf{minor}: pulls corners of lips up, smile

F. \textbf{Buccinator}: tone in cheek, sucking, whistling

G. \textbf{Levator} \textbf{palpebrae}: opens the eye

H. \textbf{Platysma}: depresses mandible, corners of lips down (not
visible on any model in Cartersville)

\textbf{Muscles of mastication}A. \textbf{Masseter}: strongly elevate
mandible, close jaws

B. \textbf{Temporalis}: strongly elevate mandible, close jaws

\subsection{Muscles moving the head and
spine}\label{muscles-moving-the-head-and-spine}

A. \textbf{Sternocleidomastoid}: together, flex forward; singly, tilt
and rotate

B. \textbf{Sternohyoid}: depresses larynx and hyoid bone

C. \textbf{Occipitalis}: moves the scalp posteriorly

D. \textbf{Splenius} \textbf{capitis}: turn the head side to side to say
``no''

E. \textbf{Erector} \textbf{spinae}: extends the vertebral column

\subsection{\texorpdfstring{Extrinsic muscles of the eye: move the
eyeball
}{Extrinsic muscles of the eye: move the eyeball }}\label{extrinsic-muscles-of-the-eye-move-the-eyeball}

A. Superior rectus: eyeball up

B. \textbf{Medial rectus}: eyeball medial

C. \textbf{Inferior rectus}: eyeball down

D. \textbf{Lateral rectus}: eyeball lateral

E. \textbf{Superior oblique:} eyeball downward and outward rotation

F. \textbf{Inferior oblique:} eyeball upward and outward rotation

\subsection{Muscles of the abdominal
wall}\label{muscles-of-the-abdominal-wall}

A. External abdominal oblique

B. Internal abdominal oblique

C. Transversus abdominis

D. Rectus abdominis

E. Linea Alba

\subsection{Muscles for breathing}\label{muscles-for-breathing}

A. \textbf{Diaphragm}: pushes abdominal contents down (inspiration)

B. \textbf{External intercostal muscles}: raise and spread ribs
(inspiration)

C. Internal intercostal muscles: forced expiration only

\subsection{Muscles of the pelvic
floor}\label{muscles-of-the-pelvic-floor}

A. \textbf{Coccygeus} \textbf{group}: tone supports pelvic organs

B. \textbf{Levator ani}: controls bowel elimination

C. \textbf{External anal sphincter}: controls bowel elimination

\subsection{Muscles acting on the
scapula}\label{muscles-acting-on-the-scapula}

A. \textbf{Trapezius}: upper portion elevates; lower, depresses

B. Serratus anterior: rotates

C. \textbf{Pectoralis minor:} pulls anteriorly

D. Levator scapulae: elevates

E. \textbf{Rhomboideus major}: elevate and adduct

F. \textbf{Rhomboideus minor:} elevate and adduct

\subsection{Muscles acting on the humerus (at shoulder
joint)}\label{muscles-acting-on-the-humerus-at-shoulder-joint}

A. \textbf{Pectoralis} \textbf{major}: flexes, adducts, and medially
rotates

B. \textbf{Latissimus} \textbf{dorsi}: extends, adducts, and medially
rotates

C. \textbf{Deltoid}: abducts

D. Supraspinatus: abducts

E. \textbf{Infraspinatus}: laterally rotates

F. \textbf{Subscapularis:} medially rotates

G. \textbf{Teres major:} extends, adducts, and medially rotates

H. \textbf{Teres minor:} laterally rotates

I. \textbf{Coracobrachialis}: flexes and adducts

\subsection{\texorpdfstring{Muscles acting on the forearm (at elbow
joint)
}{Muscles acting on the forearm (at elbow joint) }}\label{muscles-acting-on-the-forearm-at-elbow-joint}

A. \textbf{Brachialis}: flexesB. \textbf{Brachioradialis}: flexesC.
\textbf{Biceps} \textbf{brachii}: flexes and supinates

D. Triceps brachii: extends

E. \textbf{Supinator}: supinates forearm

F. \textbf{Pronator teres}: pronates forearm

\subsubsection{Muscles acting at the wrist
joint}\label{muscles-acting-at-the-wrist-joint}

A. \textbf{Flexor carpi radialis:} flexes and abducts hand at wrist

B. \textbf{Flexor carpi ulnaris}: flexes and adducts hand at wrist

C. \textbf{Extensor carpi radialis}: extends, abducts hand at wrist

D. \textbf{Extensor} \textbf{carpi} \textbf{ulnaris}: extends and
adducts hand at wrist

E. \textbf{Palmaris} \textbf{longus}: flexes wrist

\subsection{Muscles acting on the
fingers}\label{muscles-acting-on-the-fingers}

A. Flexor digitorum superficialis: flexes fingers

B. Flexor digitorum profundus: flexes fingers

C. \textbf{Extensors} \textbf{digitorum}: extends fingers

D. \textbf{Interosseous}: intrinsic, abduct fingers at
metacarpo-phalangeal joint

E. \textbf{Lumbricals}: intrinsic, adduct fingers at
metacarpo-phalangeal joint

\subsection{Muscles acting on the femur at the hip
joint}\label{muscles-acting-on-the-femur-at-the-hip-joint}

\begin{enumerate}
\def\labelenumi{\Alph{enumi}.}
\item
  Iliacusandpsoas major:flexes
\item
  \textbf{Iliopsoas}: formed from the merger of the iliacus and psoas
  major
\item
  \textbf{Gluteus} \textbf{maximus}: extend, lateral rotation
\item
  \textbf{Gluteus} \textbf{medius}: abduct, medial rotation
\item
  \textbf{Gluteus} \textbf{minimus}: abduct, lateral rotation (not
  visible on any model in Cartersville)
\item
  \textbf{Adductor} \textbf{longus} and adductor magnus: adduct and
  flexes
\item
  \textbf{Pectineus}: adducts, flexes, and medially rotates thigh
\item
  \textbf{Tensor} \textbf{fasciae} \textbf{latae}: abducts, and medially
  rotates thigh, steadies trunk
\end{enumerate}

\subsection{Long muscles of the thigh, cross two
joints}\label{long-muscles-of-the-thigh-cross-two-joints}

A. \textbf{Gracilis}: flexes knee and adducts femur

B. \textbf{Sartorius}: flexes knee and femur, laterally rotates femur

\subsection{Muscles acting on the leg at knee
joint}\label{muscles-acting-on-the-leg-at-knee-joint}

A. Quadriceps femoris: extends leg at knee

1. Rectus femoris

2. Vastus lateralis

3. Vastus intermedius

4. Vastus medialis

B. ``Hamstrings'': flex leg and extend thigh (cross two joints)

1. \textbf{Semitendinosus} (medial and superficial)

2. \textbf{Semimembranosus} (medial and deep)

3. \textbf{Biceps} \textbf{femoris} (lateral)

\subsection{Muscles acting on foot at ankle
joint}\label{muscles-acting-on-foot-at-ankle-joint}

A. \textbf{Gastrocnemius:} plantar flexes foot

B. \textbf{Soleus:} plantar flexes foot

C. \textbf{Tibialis anterior}: dorsiflexes foot

D. Tibialis posterior: inverts foot

E. Fibularis longus: everts foot

F. Fibularis brevis: everts foot

G. \textbf{Extensor digitorum longus:} prime mover of toe extension and
dorsiflexes the foot

H. \textbf{Flexor digitorum longus:} plantar flexes and inverts foot,
flexes the toes and helps foot grip the

group

\section{}\label{section-8}

\section{\texorpdfstring{Chapter 12 }{Chapter 12 }}\label{chapter-12}

\begin{enumerate}
\def\labelenumi{\arabic{enumi}.}
\item
  Anatomical Division of the Nervous System

  \begin{enumerate}
  \def\labelenumii{\alph{enumii}.}
  \item
    Central Nervous System has two parts:

    \begin{enumerate}
    \def\labelenumiii{\roman{enumiii}.}
    \item
      Brain- Define these terms:

      \begin{enumerate}
      \def\labelenumiv{\arabic{enumiv}.}
      \item
        Neurons
      \item
        Glial cells
      \item
        Soma
      \item
        Axon
      \item
        Dendrite
      \item
        Gray matter
      \item
        White matter
      \end{enumerate}
    \item
      Spinal cord- Define these terms:

      \begin{enumerate}
      \def\labelenumiv{\arabic{enumiv}.}
      \item
        Ganglion vs. nucleus
      \item
        Tract
      \end{enumerate}
    \end{enumerate}
  \item
    Peripheral Nervous System

    \begin{enumerate}
    \def\labelenumiii{\roman{enumiii}.}
    \item
      Nerves that have left the spinal cord

      \begin{enumerate}
      \def\labelenumiv{\arabic{enumiv}.}
      \item
        Define Afferent neurons
      \item
        Define Efferent neurons
      \end{enumerate}
    \end{enumerate}
  \end{enumerate}
\item
  Functional Divisions of the Nervous System

  \begin{enumerate}
  \def\labelenumii{\alph{enumii}.}
  \item
    Sensation, Response, and Integration

    \begin{enumerate}
    \def\labelenumiii{\roman{enumiii}.}
    \item
      Sensory functions
    \item
      Motor functions
    \item
      Integration
    \end{enumerate}
  \item
    Somatic Nervous System

    \begin{enumerate}
    \def\labelenumiii{\roman{enumiii}.}
    \item
      Conscious Perception
    \item
      Voluntary response of muscles
    \item
      Reflexes
    \end{enumerate}
  \item
    Autonomic Nervous System: What is the role of the ANS?

    \begin{enumerate}
    \def\labelenumiii{\roman{enumiii}.}
    \item
      Smooth and Cardiac muscle
    \item
      Glands
    \item
      Enteric Nervous System

      \begin{enumerate}
      \def\labelenumiv{\arabic{enumiv}.}
      \item
        Smooth muscle and glands of digestive tract
      \end{enumerate}
    \end{enumerate}
  \end{enumerate}
\item
  Nervous Tissue

  \begin{enumerate}
  \def\labelenumii{\alph{enumii}.}
  \item
    Define these terms

    \begin{enumerate}
    \def\labelenumiii{\roman{enumiii}.}
    \item
      Synapse
    \item
      Axon hillock
    \item
      Node of Ranvier
    \item
      Axon terminal
    \item
      Synaptic end bulb
    \end{enumerate}
  \end{enumerate}
\end{enumerate}

Label the Diagram

Label the Diagram

\begin{enumerate}
\def\labelenumi{\arabic{enumi}.}
\setcounter{enumi}{3}
\item
  Types of Neurons, according to axon shape

  \begin{enumerate}
  \def\labelenumii{\alph{enumii}.}
  \item
    Unipolar
  \item
    Bipolar
  \item
    Multipolar
  \end{enumerate}
\end{enumerate}

Label the Diagram

\begin{enumerate}
\def\labelenumi{\arabic{enumi}.}
\setcounter{enumi}{4}
\item
  Types of Neurons, according to function

  \begin{enumerate}
  \def\labelenumii{\alph{enumii}.}
  \item
    Glial cells

    \begin{enumerate}
    \def\labelenumiii{\roman{enumiii}.}
    \item
      Astrocyte

      \begin{enumerate}
      \def\labelenumiv{\arabic{enumiv}.}
      \item
        Blood Brain Barrier (BBB)
      \end{enumerate}
    \item
      Oligodendrocyte: Myelin in CNS
    \item
      Microglia
    \item
      Ependymal cell

      \begin{enumerate}
      \def\labelenumiv{\arabic{enumiv}.}
      \item
        Cerebrospinal fluid (CFS)
      \item
        Ventricle
      \item
        Choroid plexus
      \end{enumerate}
    \item
      Glial cells of PNS

      \begin{enumerate}
      \def\labelenumiv{\arabic{enumiv}.}
      \item
        Satellite cell
      \item
        Schwann cell: Myelin in PNS
      \end{enumerate}
    \end{enumerate}
  \end{enumerate}
\item
  Function of Nervous System

  \begin{enumerate}
  \def\labelenumii{\alph{enumii}.}
  \item
    Thermoreceptor
  \item
    Graded Potential
  \item
    Threshold
  \item
    Action Potential
  \item
    Neurotransmitter
  \item
    Thalamus
  \item
    Cerebral cortex
  \item
    Upper Motor Neuron
  \item
    Precentral Gyrus of the Frontal Cortex
  \item
    Lower Motor Neuron
  \end{enumerate}
\item
  Cell Membranes

  \begin{enumerate}
  \def\labelenumii{\alph{enumii}.}
  \item
    Non-specific channel
  \item
    Gated channel

    \begin{enumerate}
    \def\labelenumiii{\roman{enumiii}.}
    \item
      Ligand-gated channel
    \item
      Ionotropic receptor
    \item
      Mechanically gated channel
    \item
      Voltage-gated channel
    \item
      Leakage channel
    \end{enumerate}
  \end{enumerate}
\item
  Membrane Potential

  \begin{enumerate}
  \def\labelenumii{\alph{enumii}.}
  \item
    Resting Membrane Potential
  \item
    Action Potential

    \begin{enumerate}
    \def\labelenumiii{\roman{enumiii}.}
    \item
      Define: Depolarization
    \item
      Define: Repolarization
    \end{enumerate}
  \item
    Activation Gate
  \item
    Inactivation Gate
  \item
    Refractory Period

    \begin{enumerate}
    \def\labelenumiii{\roman{enumiii}.}
    \item
      Define: Absolute refractory period
    \item
      Define: Relative refractory period
    \end{enumerate}
  \item
    Propagation of the Action Potential
  \item
    What is the difference between Continuous vs. Saltatory Conduction?
  \end{enumerate}
\item
  Communication Between Neurons

  \begin{enumerate}
  \def\labelenumii{\alph{enumii}.}
  \item
    Graded potentials
  \item
    Excitatory Postsynaptic Potential (EPSP)
  \item
    Inhibitory Postsynaptic Potential (IPSP)
  \item
    Summation
  \end{enumerate}
\item
  Neurotransmitter Systems

  \begin{enumerate}
  \def\labelenumii{\alph{enumii}.}
  \item
    Cholinergic system
  \item
    Nicotinic receptor
  \item
    Muscarinic receptor
  \end{enumerate}
\item
  What receptors do these use: Glutamate, Glycine, and GABA
\item
  Metabotropic receptor

  \begin{enumerate}
  \def\labelenumii{\alph{enumii}.}
  \item
    Define: G protein
  \end{enumerate}
\item
  Disorders of the Nervous System

  \begin{enumerate}
  \def\labelenumii{\alph{enumii}.}
  \item
    What are some features of Alzheimer's and Parkinson's diseases?
  \end{enumerate}
\end{enumerate}

\subsection{Check your understanding:}\label{check-your-understanding}

\begin{enumerate}
\def\labelenumi{\arabic{enumi}.}
\item
  The \_\_\_\_\_\_\_\_\_\_\_\_\_ is the glial cell that forms the BBB.
\item
  The \_\_\_\_\_\_\_\_\_\_\_\_\_ is the glial cell that forms the myelin
  in the PNS.
\item
  The \_\_\_\_\_\_\_\_\_\_\_\_\_ is the glial cell that forms the myelin
  in the CNS.
\item
  The \_\_\_\_\_\_\_\_\_\_\_\_\_ is the glial cell that forms CSF.
\item
  The parts of the neuron that receives impulses are the
  \_\_\_\_\_\_\_\_\_\_\_\_\_\_\_\_
\item
  The action potential originates at the
  \_\_\_\_\_\_\_\_\_\_\_\_\_\_\_\_\_ and travels down the
  \_\_\_\_\_\_\_\_\_\_\_ of the neuron, until it reaches the
  \_\_\_\_\_\_\_\_\_\_\_\_\_\_\_\_\_\_\_\_\_\_\_, which contains
  vesicles filled with \_\_\_\_\_\_\_\_\_\_\_\_\_\_\_\_\_, which are
  then released.
\item
  The space where two neurons communicate is the
  \_\_\_\_\_\_\_\_\_\_\_\_\_\_\_\_\_\_\_
\item
  Which neuron makes skeletal muscles contract: Upper Motor Neuron or
  Lower Motor Neuron?
\item
  If a channel excludes certain ions, but allows others to pass, what
  kind of channel is it?
\item
  What type of channel opens because a signaling molecule binds to it?
\item
  What type of channel responds to changes in the electrical properties
  of the membrane?
\item
  What type of channel is shown in the below diagram as a ``pore''?
\end{enumerate}

\includegraphics[width=2.56296in,height=1.4214in]{images/media/image65.png}

\begin{enumerate}
\def\labelenumi{\arabic{enumi}.}
\setcounter{enumi}{12}
\item
  What is an action potential?
\item
  What is the difference between Saltatory conduction and continuous
  conduction?
\item
  What are the two types of receptors that acetylcholine can bind to?
\item
  Which one uses a G protein: ionotropic or metabotropic receptors?
\end{enumerate}

\section{}\label{section-9}

\section{\texorpdfstring{Chapter 13 }{Chapter 13 }}\label{chapter-13}

\begin{enumerate}
\def\labelenumi{\arabic{enumi}.}
\setcounter{enumi}{13}
\item
  Embryological Development

  \begin{enumerate}
  \def\labelenumii{\alph{enumii}.}
  \item
    The Neural Tube

    \begin{enumerate}
    \def\labelenumiii{\roman{enumiii}.}
    \item
      Endoderm develops into \_\_\_\_\_\_\_\_\_\_\_\_\_
    \item
      Mesoderm develops into \_\_\_\_\_\_\_\_\_\_\_\_\_
    \item
      Ectoderm develops into \_\_\_\_\_\_\_\_\_\_\_\_\_

      \begin{enumerate}
      \def\labelenumiv{\arabic{enumiv}.}
      \item
        Neural plate
      \item
        Neural groove
      \end{enumerate}
    \end{enumerate}
  \end{enumerate}
\end{enumerate}

Label the Diagram

\begin{enumerate}
\def\labelenumi{\arabic{enumi}.}
\setcounter{enumi}{2}
\item
  Neural tube
\end{enumerate}

\begin{enumerate}
\def\labelenumi{\alph{enumi}.}
\setcounter{enumi}{1}
\item
  Primary and Secondary Vesicles

  \begin{enumerate}
  \def\labelenumii{\roman{enumii}.}
  \item
    Prosencephalon develops into \_\_\_\_\_\_\_\_\_\_\_\_\_

    \begin{enumerate}
    \def\labelenumiii{\arabic{enumiii}.}
    \item
      Telencephalon develops into \_\_\_\_\_\_\_\_\_\_\_\_\_
    \item
      Diencephalon develops into \_\_\_\_\_\_\_\_\_\_\_\_\_
    \end{enumerate}
  \item
    Mesencephalon develops into \_\_\_\_\_\_\_\_\_\_\_\_\_
  \item
    Rhombencephalon develops into \_\_\_\_\_\_\_\_\_\_\_\_\_

    \begin{enumerate}
    \def\labelenumiii{\arabic{enumiii}.}
    \item
      Metencephalon develops into \_\_\_\_\_\_\_\_\_\_\_\_\_
    \item
      Myelencephalon develops into \_\_\_\_\_\_\_\_\_\_\_\_\_
    \end{enumerate}
  \end{enumerate}
\item
  Spinal Cord
\end{enumerate}

\begin{enumerate}
\def\labelenumi{\arabic{enumi}.}
\setcounter{enumi}{14}
\item
  The Central Nervous System

  \begin{enumerate}
  \def\labelenumii{\alph{enumii}.}
  \item
    Cerebrum

    \begin{enumerate}
    \def\labelenumiii{\roman{enumiii}.}
    \item
      Cerebral cortex
    \item
      Longitudinal fissure
    \item
      Cerebral hemisphere
    \item
      Corpus callosum: What is its function?
    \item
      Basal nuclei: What is its function?
    \item
      Basal forebrain: What is its function?
    \item
      Limbic system: What is its function?
    \item
      Gyrus and sulcus

      \begin{enumerate}
      \def\labelenumiv{\arabic{enumiv}.}
      \item
        Lateral sulcus: Divides which lobe does it separate?
      \item
        Central sulcus: Divides which lobes?
      \item
        Parieto-occipital sulcus
      \end{enumerate}
    \item
      Brodmann's areas 41 and 42
    \item
      Precentral gyrus
    \item
      Broca's area
    \item
      Prefrontal lobe
    \item
      Subcortical structures

      \begin{enumerate}
      \def\labelenumiv{\arabic{enumiv}.}
      \item
        Hippocampus
      \item
        Amygdala
      \item
        Basal nuclei
      \item
        Substantia nigra: releases which neurotransmitter?
      \end{enumerate}
    \end{enumerate}
  \item
    Diencephalon

    \begin{enumerate}
    \def\labelenumiii{\roman{enumiii}.}
    \item
      Thalamus: What is its function?
    \item
      Hypothalamus: What is its function?
    \end{enumerate}
  \item
    Brain Stem

    \begin{enumerate}
    \def\labelenumiii{\roman{enumiii}.}
    \item
      Midbrain: What is its function?
    \item
      Pons: What is its function?
    \item
      Medulla: What is its function?
    \end{enumerate}
  \item
    Cerebellum: What is its function?
  \item
    Spinal Cord

    \begin{enumerate}
    \def\labelenumiii{\roman{enumiii}.}
    \item
      Anterior median fissure
    \item
      Posterior median sulcus
    \item
      Dorsal nerve root
    \item
      Ventral nerve root
    \item
      Gray horns

      \begin{enumerate}
      \def\labelenumiv{\arabic{enumiv}.}
      \item
        Posterior horn
      \item
        Anterior horn
      \item
        Lateral horn
      \end{enumerate}
    \item
      White columns

      \begin{enumerate}
      \def\labelenumiv{\arabic{enumiv}.}
      \item
        Ascending tracts
      \item
        Descending tracts
      \item
        Posterior columns
      \item
        Anterior columns
      \item
        Lateral columns
      \end{enumerate}
    \end{enumerate}
  \end{enumerate}
\end{enumerate}

Label the Diagrams

\begin{enumerate}
\def\labelenumi{\arabic{enumi}.}
\setcounter{enumi}{15}
\item
  Circulation and the CNS

  \begin{enumerate}
  \def\labelenumii{\alph{enumii}.}
  \item
    Arterial supply

    \begin{enumerate}
    \def\labelenumiii{\roman{enumiii}.}
    \item
      Common carotid arteries

      \begin{enumerate}
      \def\labelenumiv{\arabic{enumiv}.}
      \item
        External carotid: supplies what area?

        \begin{enumerate}
        \def\labelenumv{\alph{enumv}.}
        \item
          Orthostatic reflex
        \end{enumerate}
      \item
        Internal carotid
      \item
        Vertebral arteries
      \item
        Basilar artery
      \item
        Circle of Willis: What is its purpose?
      \end{enumerate}
    \end{enumerate}
  \item
    Venous Return

    \begin{enumerate}
    \def\labelenumiii{\roman{enumiii}.}
    \item
      Superior sagittal sinus
    \item
      Occipital sinuses
    \item
      Straight sinus
    \item
      Transverse sinuses
    \item
      Sigmoid sinuses
    \item
      Jugular vein
    \end{enumerate}
  \item
    Protective Coverings

    \begin{enumerate}
    \def\labelenumiii{\roman{enumiii}.}
    \item
      Dura mater: Describe
    \item
      Arachnoid mater: Describe

      \begin{enumerate}
      \def\labelenumiv{\arabic{enumiv}.}
      \item
        Subarachnoid space: What is it filled with?
      \item
        Arachnoid granulations: What is its function?
      \end{enumerate}
    \item
      Pia mater: Describe
    \end{enumerate}
  \item
    Ventricular System

    \begin{enumerate}
    \def\labelenumiii{\roman{enumiii}.}
    \item
      Central canal
    \item
      Lateral ventricles
    \item
      Third ventricle
    \item
      Fourth ventricle
    \item
      Cerebral aqueduct
    \item
      Choroid plexus
    \item
      Cerebral Spinal Fluid (CSF)
    \end{enumerate}
  \end{enumerate}
\item
  The Peripheral Nervous System

  \begin{enumerate}
  \def\labelenumii{\alph{enumii}.}
  \item
    Ganglia: Define

    \begin{enumerate}
    \def\labelenumiii{\roman{enumiii}.}
    \item
      Dorsal root ganglion: Contains what types of cell bodies?
    \item
      Sympathetic chain ganglia
    \item
      Plexus: Define
    \end{enumerate}
  \item
    Nerves

    \begin{enumerate}
    \def\labelenumiii{\roman{enumiii}.}
    \item
      Structure

      \begin{enumerate}
      \def\labelenumiv{\arabic{enumiv}.}
      \item
        Epineurium
      \item
        Perineurium
      \item
        Endoneurium
      \item
        Fascicles
      \end{enumerate}
    \item
      Cranial nerves

      \begin{enumerate}
      \def\labelenumiv{\arabic{enumiv}.}
      \item
        CN I Olfactory: What is its function?
      \item
        CN II Optic nerve: What is its function?
      \item
        CN III Oculomotor: What is its function?
      \item
        CN IV Trochlear: What is its function?
      \item
        CN V Trigeminal: What is its function?
      \item
        CN VI Abducens: What is its function?
      \item
        CN VII Facial: What is its function?
      \item
        CN VIII Vestibulocochlear: What is its function?
      \item
        CN IX Glossopharyngeal: What is its function?
      \item
        CN X Vagus: What is its function?
      \item
        CN XI Accessory: What is its function?
      \item
        CN XII Hypoglossal: What is its function?
      \end{enumerate}
    \item
      Spinal nerves

      \begin{enumerate}
      \def\labelenumiv{\arabic{enumiv}.}
      \item
        Cervical plexus

        \begin{enumerate}
        \def\labelenumv{\alph{enumv}.}
        \item
          Phrenic nerve: What does it supply?
        \end{enumerate}
      \item
        Brachial plexus

        \begin{enumerate}
        \def\labelenumv{\alph{enumv}.}
        \item
          Axillary nerve
        \item
          Radial nerve
        \item
          Ulnar nerve
        \item
          Median nerve
        \end{enumerate}
      \item
        Lumbar plexus

        \begin{enumerate}
        \def\labelenumv{\alph{enumv}.}
        \item
          Femoral nerve
        \item
          Saphenous nerve
        \end{enumerate}
      \item
        Sacral plexus

        \begin{enumerate}
        \def\labelenumv{\alph{enumv}.}
        \item
          Sciatic nerve

          \begin{enumerate}
          \def\labelenumvi{\roman{enumvi}.}
          \item
            Tibial nerve
          \item
            Fibular nerve
          \end{enumerate}
        \end{enumerate}
      \end{enumerate}
    \end{enumerate}
  \end{enumerate}
\end{enumerate}

\subsection{Check your understanding:}\label{check-your-understanding-1}

\begin{enumerate}
\def\labelenumi{\arabic{enumi}.}
\setcounter{enumi}{16}
\item
  The ridges in the cerebrum are \_\_\_\_\_\_\_\_\_\_\_\_\_ and the
  grooves are \_\_\_\_\_\_\_\_\_\_\_\_.
\item
  Lower motor neurons original in the
  \_\_\_\_\_\_\_\_\_\_\_\_\_\_\_\_\_\_\_\_\_\_\_\_\_.
\item
  Brodmann's area 41 keeps information about
  \_\_\_\_\_\_\_\_\_\_\_\_\_\_\_\_\_\_\_\_.
\item
  Broca's area is responsible for \_\_\_\_\_\_\_\_\_\_\_\_\_\_\_\_\_\_.
\item
  The prefrontal lobe forms the basis of
  \_\_\_\_\_\_\_\_\_\_\_\_\_\_\_\_\_\_\_\_\_\_\_\_\_\_\_\_\_\_\_\_\_\_\_\_\_\_\_
\item
  What structures are punctured during a lumbar puncture?
\item
  What are the functions of CSF?
\item
  What causes sciatica?
\end{enumerate}

\section{}\label{section-10}

\section{\texorpdfstring{Chapter 14 }{Chapter 14 }}\label{chapter-14}

\begin{enumerate}
\def\labelenumi{\arabic{enumi}.}
\setcounter{enumi}{17}
\item
  Sensory Perception

  \begin{enumerate}
  \def\labelenumii{\alph{enumii}.}
  \item
    Sensory Receptors: How can they be classified functionally?

    \begin{enumerate}
    \def\labelenumiii{\roman{enumiii}.}
    \item
      Structural receptor types

      \begin{enumerate}
      \def\labelenumiv{\arabic{enumiv}.}
      \item
        Free nerve ending
      \item
        Encapsulated ending
      \item
        Specialized receptor cell (give an example)
      \end{enumerate}
    \item
      Functional receptor types

      \begin{enumerate}
      \def\labelenumiv{\arabic{enumiv}.}
      \item
        Chemoreceptor
      \item
        Osmoreceptor
      \item
        Nociceptor
      \item
        Mechanoreceptor
      \item
        thermoreceptor
      \end{enumerate}
    \end{enumerate}
  \item
    Sensory Modalities

    \begin{enumerate}
    \def\labelenumiii{\roman{enumiii}.}
    \item
      General sense (give an example)
    \item
      Proprioception
    \item
      Kinesthesia
    \item
      Visceral senses
    \item
      Somatosensation (list the 7 types)
    \item
      Special senses

      \begin{enumerate}
      \def\labelenumiv{\arabic{enumiv}.}
      \item
        Gustation (taste)

        \begin{enumerate}
        \def\labelenumv{\alph{enumv}.}
        \item
          Papillae
        \item
          Taste buds
        \item
          Gustatory receptor cells
        \end{enumerate}
      \item
        Olfaction (smell)

        \begin{enumerate}
        \def\labelenumv{\alph{enumv}.}
        \item
          Olfactory epithelium
        \item
          Olfactory sensory neurons
        \item
          Odorant molecules
        \item
          Olfactory bulb
        \end{enumerate}
      \item
        Audition (hearing)

        \begin{enumerate}
        \def\labelenumv{\alph{enumv}.}
        \item
          External Ear

          \begin{enumerate}
          \def\labelenumvi{\roman{enumvi}.}
          \item
            Auricle
          \end{enumerate}
        \item
          Middle Ear

          \begin{enumerate}
          \def\labelenumvi{\roman{enumvi}.}
          \item
            Tympanic membrane
          \item
            Eustachian tube
          \item
            Ossicles

            \begin{enumerate}
            \def\labelenumvii{\arabic{enumvii}.}
            \item
              Malleus
            \item
              Incus
            \item
              stapes
            \end{enumerate}
          \end{enumerate}
        \item
          Inner Ear

          \begin{enumerate}
          \def\labelenumvi{\roman{enumvi}.}
          \item
            Cochlea
          \item
            Vestibule
          \item
            Spiral ganglia
          \item
            Oval window
          \item
            Scala vestibuli
          \item
            Cochlear duct
          \item
            Scala tympani
          \item
            Round window
          \item
            Organ of Corti

            \begin{enumerate}
            \def\labelenumvii{\arabic{enumvii}.}
            \item
              Basilar membrane
            \item
              Hair cells
            \item
              Tectorial membrane
            \end{enumerate}
          \end{enumerate}
        \item
          Equilibrium

          \begin{enumerate}
          \def\labelenumvi{\roman{enumvi}.}
          \item
            Utricle and saccule
          \item
            Semicircular canals
          \item
            Vestibular ganglion
          \item
            Otolithic membrane
          \item
            Ampulla
          \item
            Cupula
          \end{enumerate}
        \end{enumerate}
      \item
        Somatosensation (This is listed in the book here, but it is not
        a special sense. How does this differ from a special sense?)

        \begin{enumerate}
        \def\labelenumv{\alph{enumv}.}
        \item
          Touch

          \begin{enumerate}
          \def\labelenumvi{\roman{enumvi}.}
          \item
            Merkel's discs: where are these located?
          \item
            Ruffini's corpuscle: where are these located?
          \item
            Meissner's corpuscle: where are these located?
          \item
            Pacinian corpuscle: where are these located?
          \end{enumerate}
        \item
          Proprioception

          \begin{enumerate}
          \def\labelenumvi{\roman{enumvi}.}
          \item
            Golgi tendon organ
          \end{enumerate}
        \item
          Interoception (do an internet search for more information!)
        \end{enumerate}
      \item
        Vision

        \begin{enumerate}
        \def\labelenumv{\alph{enumv}.}
        \item
          Palpebral conjunctiva
        \item
          Lacrimal gland: What is its function?
        \item
          Lacrimal duct: What is its function?
        \item
          Extraocular eye muscles

          \begin{enumerate}
          \def\labelenumvi{\roman{enumvi}.}
          \item
            Superior rectus
          \item
            Medial rectus
          \item
            Inferior rectus
          \item
            Lateral rectus
          \item
            Superior oblique
          \item
            Inferior oblique
          \end{enumerate}
        \item
          Fibrous Tunic

          \begin{enumerate}
          \def\labelenumvi{\roman{enumvi}.}
          \item
            Sclera
          \item
            Cornea: what is its function?
          \end{enumerate}
        \item
          Vascular Tunic

          \begin{enumerate}
          \def\labelenumvi{\roman{enumvi}.}
          \item
            Choroid: what is its function?
          \item
            Ciliary body
          \item
            Zonule fibers
          \item
            lens
          \item
            iris
          \item
            pupil
          \item
            retina
          \end{enumerate}
        \item
          Cavities

          \begin{enumerate}
          \def\labelenumvi{\roman{enumvi}.}
          \item
            Anterior cavity

            \begin{enumerate}
            \def\labelenumvii{\arabic{enumvii}.}
            \item
              Aqueous humor
            \item
              Anterior chamber
            \item
              Posterior chamber
            \end{enumerate}
          \item
            Posterior cavity

            \begin{enumerate}
            \def\labelenumvii{\arabic{enumvii}.}
            \item
              Vitreous humor
            \end{enumerate}
          \end{enumerate}
        \item
          Optic Nerve

          \begin{enumerate}
          \def\labelenumvi{\roman{enumvi}.}
          \item
            Optic disc
          \item
            Blind spot
          \end{enumerate}
        \item
          Macula

          \begin{enumerate}
          \def\labelenumvi{\roman{enumvi}.}
          \item
            Fovea centralis
          \item
            Visual acuity
          \end{enumerate}
        \item
          Photoreceptors

          \begin{enumerate}
          \def\labelenumvi{\roman{enumvi}.}
          \item
            Rods: contains what pigment?
          \item
            Cones: contains what pigments? What 3 colors are detected?
          \end{enumerate}
        \end{enumerate}
      \end{enumerate}
    \end{enumerate}
  \item
    Sensory Nerves

    \begin{enumerate}
    \def\labelenumiii{\roman{enumiii}.}
    \item
      Spinal nerves

      \begin{enumerate}
      \def\labelenumiv{\arabic{enumiv}.}
      \item
        Dorsal roots: What do they contain?
      \item
        Ventral roots: What do they contain?
      \item
        Contralateral: define this term
      \end{enumerate}
    \item
      Cranial nerves

      \begin{enumerate}
      \def\labelenumiv{\arabic{enumiv}.}
      \item
        Ipsilateral: define this term
      \end{enumerate}
    \end{enumerate}
  \end{enumerate}
\item
  Central Processing

  \begin{enumerate}
  \def\labelenumii{\alph{enumii}.}
  \item
    Sensory Pathways

    \begin{enumerate}
    \def\labelenumiii{\roman{enumiii}.}
    \item
      Spinal Cord and Brain stem ascending pathways

      \begin{enumerate}
      \def\labelenumiv{\arabic{enumiv}.}
      \item
        Dorsal column system: what sensations travel here?

        \begin{enumerate}
        \def\labelenumv{\alph{enumv}.}
        \item
          Fasciculus gracilis
        \item
          Fasciculus cuneatus
        \end{enumerate}
      \item
        Spinothalamic tract: what sensations travel here?
      \item
        Trigeminal pathway: what area of the body is covered by this?
      \item
        Gustation: travels in which two cranial nerves?
      \item
        Audition: travels in which cranial nerve?
      \item
        Optic chiasm: what happens here?
      \end{enumerate}
    \item
      Diencephalon

      \begin{enumerate}
      \def\labelenumiv{\arabic{enumiv}.}
      \item
        Thalamus
      \item
        Hypothalamus
      \end{enumerate}
    \end{enumerate}
  \item
    Cortical Processing

    \begin{enumerate}
    \def\labelenumiii{\roman{enumiii}.}
    \item
      Sensory homunculus: define
    \item
      Primary sensory cortex
    \item
      Association area
    \item
      Multimodal integration area
    \end{enumerate}
  \end{enumerate}
\item
  Motor Responses

  \begin{enumerate}
  \def\labelenumii{\alph{enumii}.}
  \item
    Cortical responses

    \begin{enumerate}
    \def\labelenumiii{\roman{enumiii}.}
    \item
      Executive functions: What are these?
    \item
      Working memory
    \item
      Secondary motor cortices

      \begin{enumerate}
      \def\labelenumiv{\arabic{enumiv}.}
      \item
        Premotor cortex
      \item
        Supplemental motor area
      \item
        Broca's area: a stroke here causes what symptoms?
      \end{enumerate}
    \item
      Primary motor cortex
    \end{enumerate}
  \item
    Descending pathways

    \begin{enumerate}
    \def\labelenumiii{\roman{enumiii}.}
    \item
      Betz cells
    \item
      Corticobulbar tract
    \item
      Corticospinal tract: what are the pyramids?

      \begin{enumerate}
      \def\labelenumiv{\arabic{enumiv}.}
      \item
        Lateral tract
      \item
        Anterior tract
      \end{enumerate}
    \end{enumerate}
  \item
    Extrapyramidal controls

    \begin{enumerate}
    \def\labelenumiii{\roman{enumiii}.}
    \item
      Tectospinal tract
    \item
      Reticulospinal tract
    \item
      Vestibulospinal tract
    \item
      Rubrospinal tract
    \end{enumerate}
  \item
    Ventral Horn output: what fibers are found here?
  \item
    Reflexes

    \begin{enumerate}
    \def\labelenumiii{\roman{enumiii}.}
    \item
      Withdrawal reflex: give an example
    \item
      Stretch reflex: give an example
    \item
      Corneal reflex: describe
    \end{enumerate}
  \end{enumerate}
\end{enumerate}

Label the Diagrams

\subsection{Check your understanding:}\label{check-your-understanding-2}

\begin{enumerate}
\def\labelenumi{\arabic{enumi}.}
\setcounter{enumi}{24}
\item
  The two types of somatosensory signals that are transduced by free
  nerve endings are \_\_\_\_\_\_\_\_\_\_\_\_\_\_\_\_\_\_\_\_\_\_\_ and
  \_\_\_\_\_\_\_\_\_\_\_\_\_\_\_\_\_\_\_\_\_\_.
\item
  The type of receptor that transduces temperature is called a
  \_\_\_\_\_\_\_\_\_\_\_\_\_\_\_\_\_\_.
\item
  The type of receptor that transduces pain is called a
  \_\_\_\_\_\_\_\_\_\_\_\_\_\_\_\_\_\_\_\_\_.
\item
  The sensation of heat in spicey foods involves which molecule in the
  food?
\item
  The area in the brain that serves as the important relay for
  communication between the cerebrum and the rest of the nervous system
  is the \_\_\_\_\_\_\_\_\_\_\_\_\_\_\_\_\_\_.
\item
  Sensory input first goes to the \_\_\_\_\_\_\_\_\_\_\_\_\_\_\_\_\_
  portion of the brain.
\item
  The hypothalamus communicates with the
  \_\_\_\_\_\_\_\_\_\_\_\_\_\_\_\_\_\_\_ system, which controls emotions
  and memory functions.
\end{enumerate}

\section{}\label{section-11}

\section{\texorpdfstring{Chapter 15 }{Chapter 15 }}\label{chapter-15}

\begin{enumerate}
\def\labelenumi{\arabic{enumi}.}
\item
  Autonomic Nervous System

  \begin{enumerate}
  \def\labelenumii{\alph{enumii}.}
  \item
    Sympathetic Division of the Autonomic Nervous System (fight or
    flight)

    \begin{enumerate}
    \def\labelenumiii{\roman{enumiii}.}
    \item
      Sympathetic chain ganglia: Where are they located?

      \begin{enumerate}
      \def\labelenumiv{\arabic{enumiv}.}
      \item
        Target effector: define
      \item
        Central neuron (pre-ganglionic neuron)

        \begin{enumerate}
        \def\labelenumv{\alph{enumv}.}
        \item
          Short and myelinated; most go to a ganglion
        \item
          Some go to the adrenal medulla chromaffin cells
        \end{enumerate}
      \item
        Ganglionic neuron (post-ganglionic neuron)

        \begin{enumerate}
        \def\labelenumv{\alph{enumv}.}
        \item
          Long and unmyelinated
        \item
          Go to target organs
        \end{enumerate}
      \end{enumerate}
    \item
      Collateral ganglia

      \begin{enumerate}
      \def\labelenumiv{\arabic{enumiv}.}
      \item
        Celiac ganglion
      \item
        Superior mesenteric ganglion
      \item
        Inferior mesenteric ganglion
      \end{enumerate}
    \end{enumerate}
  \item
    Parasympathetic Division of the Autonomic Nervous System (rest and
    digest)

    \begin{enumerate}
    \def\labelenumiii{\roman{enumiii}.}
    \item
      What two regions of the vertebral column do they travel from?
    \item
      Where are the terminal ganglia located?
    \item
      Vagus nerve: supplies 90\% of the parasympathetic nervous system
    \end{enumerate}
  \item
    Chemical Signaling in the ANS

    \begin{enumerate}
    \def\labelenumiii{\roman{enumiii}.}
    \item
      Cholinergic synapses (acetylcholine is released)

      \begin{enumerate}
      \def\labelenumiv{\arabic{enumiv}.}
      \item
        Nicotinic receptors

        \begin{enumerate}
        \def\labelenumv{\alph{enumv}.}
        \item
          Ligand-gated channel
        \end{enumerate}
      \item
        Muscarinic receptor

        \begin{enumerate}
        \def\labelenumv{\alph{enumv}.}
        \item
          G-Protein-coupled receptor
        \end{enumerate}
      \end{enumerate}
    \item
      Adrenergic (norepinephrine or epinephrine is released)

      \begin{enumerate}
      \def\labelenumiv{\arabic{enumiv}.}
      \item
        Alpha adrenergic receptors (type 1, 2, and 3)

        \begin{enumerate}
        \def\labelenumv{\alph{enumv}.}
        \item
          G-Protein-coupled receptor
        \end{enumerate}
      \item
        Beta adrenergic receptors (type 1 and 2)

        \begin{enumerate}
        \def\labelenumv{\alph{enumv}.}
        \item
          G-Protein-coupled receptor
        \end{enumerate}
      \end{enumerate}
    \end{enumerate}
  \end{enumerate}
\end{enumerate}

\includegraphics[width=3.35024in,height=2.48333in]{images/media/image75.png}

\begin{enumerate}
\def\labelenumi{\arabic{enumi}.}
\setcounter{enumi}{1}
\item
  Autonomic Reflexes and Homeostasis

  \begin{enumerate}
  \def\labelenumii{\alph{enumii}.}
  \item
    Somatic reflex involves one lower motor neuron
  \item
    Autonomic reflex involves two lower motor neurons

    \begin{enumerate}
    \def\labelenumiii{\roman{enumiii}.}
    \item
      Preganglionic neuron
    \item
      Postganglionic neuron
    \end{enumerate}
  \item
    Vasomotor reflex

    \begin{enumerate}
    \def\labelenumiii{\roman{enumiii}.}
    \item
      Baroreceptors
    \end{enumerate}
  \item
    Referred pain
  \item
    Short vs. Long reflexes: describe
  \item
    Balance in competing autonomic reflex arcs

    \begin{enumerate}
    \def\labelenumiii{\roman{enumiii}.}
    \item
      Competing neurotransmitters
    \item
      Autonomic tone
    \end{enumerate}
  \end{enumerate}
\item
  Central Control

  \begin{enumerate}
  \def\labelenumii{\alph{enumii}.}
  \item
    Forebrain structures

    \begin{enumerate}
    \def\labelenumiii{\roman{enumiii}.}
    \item
      Hypothalamus
    \item
      Amygdala
    \end{enumerate}
  \item
    Medulla

    \begin{enumerate}
    \def\labelenumiii{\roman{enumiii}.}
    \item
      Cardiovascular center

      \begin{enumerate}
      \def\labelenumiv{\arabic{enumiv}.}
      \item
        Cardiac accelerator nerves
      \item
        Vasomotor nerves
      \end{enumerate}
    \end{enumerate}
  \end{enumerate}
\item
  Drugs that Affect the ANS

  \begin{enumerate}
  \def\labelenumii{\alph{enumii}.}
  \item
    Broad Autonomic effects

    \begin{enumerate}
    \def\labelenumiii{\roman{enumiii}.}
    \item
      Nicotine: what effect does it have?
    \end{enumerate}
  \item
    Sympathetic effect

    \begin{enumerate}
    \def\labelenumiii{\roman{enumiii}.}
    \item
      Sympathomimetic drugs (agonists)

      \begin{enumerate}
      \def\labelenumiv{\arabic{enumiv}.}
      \item
        Phenylephrine

        \begin{enumerate}
        \def\labelenumv{\alph{enumv}.}
        \item
          Sinus decongestants
        \item
          Mydriasis: what is the antidote? See p.681
        \end{enumerate}
      \item
        Cocaine
      \item
        Caffeine
      \end{enumerate}
    \item
      Sympatholytic drugs (antagonists)

      \begin{enumerate}
      \def\labelenumiv{\arabic{enumiv}.}
      \item
        Beta blockers for congestive heart failure: Name one drug
      \item
        Antianxiety medicines: Name one drug
      \end{enumerate}
    \end{enumerate}
  \item
    Parasympathetic effects

    \begin{enumerate}
    \def\labelenumiii{\roman{enumiii}.}
    \item
      Parasympathomimetic drugs

      \begin{enumerate}
      \def\labelenumiv{\arabic{enumiv}.}
      \item
        Pilocarpine: what is it used for?
      \end{enumerate}
    \item
      Anticholinergic drugs

      \begin{enumerate}
      \def\labelenumiv{\arabic{enumiv}.}
      \item
        Atropine: What is the antidote?
      \item
        Scopolamine: What is the antidote?
      \end{enumerate}
    \end{enumerate}
  \end{enumerate}
\end{enumerate}

Fill in the ``Example'' Column

\subsection{Check your understanding:}\label{check-your-understanding-3}

\begin{enumerate}
\def\labelenumi{\arabic{enumi}.}
\setcounter{enumi}{31}
\item
  What are the effects of the sympathetic nervous system on heart rate?
  Breathing rate? Blood flow to skeletal muscle? Blood flow to digestive
  system? Sweat gland secretion?
\item
  What are the effects of the parasympathetic nervous system on heart
  rate? Breathing rate? Blood flow to skeletal muscle? Blood flow to
  digestive system? Sweat gland secretion?
\item
  Parasympathetic preganglionic fibers are (long or short) and
  post-ganglionic fibers are (long or short).
\item
  Parasympathetic preganglionic fibers primarily influence the
  \_\_\_\_\_\_\_\_\_\_\_\_, \_\_\_\_\_\_\_\_\_\_\_\_\_\_, and
  \_\_\_\_\_\_\_\_\_\_\_\_\_\_\_\_\_\_ in the thoracic cavity and the
  \_\_\_\_\_\_\_\_\_\_\_\_\_\_\_, \_\_\_\_\_\_\_\_\_\_\_\_\_\_\_,
  \_\_\_\_\_\_\_\_\_\_\_\_\_\_\_\_,
  \_\_\_\_\_\_\_\_\_\_\_\_\_\_\_\_\_\_, and
  \_\_\_\_\_\_\_\_\_\_\_\_\_\_\_\_\_\_\_\_\_\_\_\_\_\_\_\_\_\_\_\_\_\_
  of the abdominal cavity.
\item
  What is an example of a somatic reflex?
\item
  What is an example of an autonomic reflex?
\item
  Where are baroreceptors located? What is their function?
\item
  Describe ``referred pain''.
\item
  Describe orthostatic hypotension.
\item
  What do the sympathetic vasomotor nerves do to blood vessels? What
  will that do to blood pressure?
\item
  What does the parasympathetic Vagus nerve do to blood vessels? What
  will that do to blood pressure?
\item
  Feedback from which receptors tells the body to activate either
  sympathetic or parasympathetic nerves to adjust the blood pressure
  back to normal? (Hint: see \#7)
\item
  Why does nicotine cause cardiovascular disease?
\item
  How do beta blockers affect heart rate and blood vessel constriction
  to help someone who has congestive heart failure?
\end{enumerate}

\section{}\label{section-12}

\section{Chapter 16}\label{chapter-16}

\subsection{16.1 Overview of the Neurological
Exam}\label{overview-of-the-neurological-exam}

\begin{enumerate}
\def\labelenumi{\Alph{enumi}.}
\item
  The~\textbf{neurological exam}~is a clinical assessment tool used to
  rapidly determine which specific parts of the CNS are affected by
  damage or disease. The exam can be broken down into the following
  subsets:
\end{enumerate}

\begin{enumerate}
\def\labelenumi{\arabic{enumi}.}
\item
  \textbf{Mental Status Exam} = assesses the higher cognitive functions
  such as memory, orientation, and language.
\item
  \textbf{Cranial Nerve Exam} = tests the function of the 12 cranial
  nerves and, therefore, the central and peripheral structures
  associated with them.
\item
  \textbf{The Sensory Exam} = tests the sensory functions associated
  with the spinal nerves.
\item
  \textbf{The Motor Exam} = tests the motor functions associated with
  the spinal nerves
\item
  \textbf{The Coordination Exam} = tests the ability to perform complex
  and coordinated movements. \textbf{The Gait Exam} = specifically
  assesses the motor function of walking and can be considered part of
  the coordination exam because walking is a coordinated movement.

  \begin{enumerate}
  \def\labelenumii{\Alph{enumii}.}
  \item
    Causes of Neurological Deficits:

    \begin{enumerate}
    \def\labelenumiii{\arabic{enumiii}.}
    \item
      \textbf{\ul{Stroke}} also called \textbf{CVA} cerebrovascular
      accident= the loss of blood flow to a part of the brain
    \end{enumerate}
  \end{enumerate}
\end{enumerate}

\begin{enumerate}
\def\labelenumi{\alph{enumi}.}
\item
  \textbf{\ul{Ischemic stroke}} = the loss of blood flow to an area
  because vessels are blocked or narrowed. This is often caused by an
  embolus (blood clot or fat deposit), thickening of the vessel wall or
  drop in blood volume = \textbf{hypovolemia}. \textbf{Transient
  ischemic attack (TIA)} is similar to an ischemic stroke, but symptoms
  are resolved within 24 hours.
\item
  \ul{\textbf{Hemorrhagic} \textbf{stroke}} = bleeding into the brain
  because of a damaged blood vessel. Accumulated blood fills in a region
  of the cranial vault and presses against the tissue in the brain. This
  pooling blood causes secondary symptoms such as loss of function,
  pressure on neighboring arteries resulting in a larger damage area,
  potentially compromising the blood-brain barrier resulting in
  additional fluid on brain = \textbf{edema}.

  \begin{enumerate}
  \def\labelenumii{\arabic{enumii}.}
  \item
    Blunt force trauma can also cause neurological deficits.
  \item
    Neurodegenerative diseases, developmental, and other disorders
  \end{enumerate}
\end{enumerate}

\begin{enumerate}
\def\labelenumi{\alph{enumi}.}
\item
  \textbf{\ul{Alzheimer's disease}} = a progressive disorder
  characterized by the loss of higher cerebral functions and is the most
  common cause of senile dementia or senility. Symptoms may appear at 50
  -- 60 years or age. Associated with ACh shortages, shrinkage of the
  gyri, and formation of neural tangles among the CNS neurons and
  Alzheimer plaques within the cerebrum.
\item
  \textbf{\ul{Parkinson's disease}} = neurodegenerative disorder of the
  substantia nigra resulting in decreased production of dopamine. The
  basal nuclei become more active, which raises skeletal muscle tone and
  produces rigidity and stiffness. Individuals, with Parkinson disease
  have difficulty starting voluntary movements, because opposing muscle
  groups do not relax; they must be overpowered. Once a movement is
  underway, every aspect must be voluntarily controlled through intense
  effort and concentration.
\item
  \textbf{\ul{Huntington's disease}} = genetic disorder of the basal
  nuclei result in too much movements. occurs 1 in 20,000 births and is
  the result of a dominant gene located on chromosome 4. Causes a
  progressive neurological degeneration leading to death within 20 years
  from onset of symptoms. Symptoms show in 20s or 30-40 years of age.
\item
  \textbf{\ul{Amyotrophic Lateral Sclerosis (ALS)}} = progressive,
  degenerative disorder that affects the motor neurons in the spinal
  cord, brain stem, and cerebral hemispheres. The degeneration affects
  both upper and lower motor neurons. Because a motor neuron and its
  dependent muscle fibers are so intimately related, the destruction of
  the CNS neurons causes atrophy of the associated skeletal muscles.
\item
  \textbf{\ul{Rabies}} -- A bite from a rabid animal injects the rabies
  virus into the peripheral tissues, where virus particles quickly enter
  the synaptic knobs. Retrograde flow then carries the virus into the
  CNS, with potential fatal consequences. Many toxins (including heavy
  metals), some pathogenic bacteria, and other viruses also bypass CNS
  defenses by exploiting axoplasmic transport.
\item
  \textbf{\ul{Multiple Sclerosis (MS)}} = autoimmune disease causing
  deterioration of the myeline that affects axons in the optic nerve,
  brain, and spinal cord. MS results in paralysis and potentially death.
  The disorder is progressive and functional impairment increases
  following each new incident. Women are 1.5 times more likely to have
  MS than men.
\item
  \textbf{\ul{Cerebral Palsy}} -- refers to a number of disorders that
  affect voluntary motor performance; they appear during infancy or
  childhood and persist throughout the life of affected individuals. The
  cause may be trauma associated with premature or unusually stressful
  birth, maternal exposure to drugs, or a genetic defect that causes the
  improper development of motor pathways.
\item
  \textbf{\ul{Referred Pain}} -- the sensation of pain in a part of the
  body other than its actual source.
\end{enumerate}

\subsection{16.2 The Mental Status Exam}\label{the-mental-status-exam}

A. Functions of the cerebral cortex

The cerebrum is the seat of many of the higher mental functions, such as
memory and learning, language, and conscious perception, which are the
subjects of subtests of the mental status exam. As discussed in Ch. 13
the cerebrum is a thin layer of gray material about 1 mm thick that is
highly folded. Brodmann first described about 50 different regions of
the cerebrum that correspond to their various functions. There are three
types of processing regions:

1. \textbf{Primary} = The primary cortical areas are where sensory
information is initially processed, or where motor commands emerge to go
to the brain stem or spinal cord.

2. \textbf{Association} = Association areas are adjacent to primary
areas and further process the modality-specific input.

3. \textbf{Integration} areas = Multimodal integration areas are found
where the modality-specific regions meet; they can process multiple
modalities together or different modalities on the basis of similar
functions, such as spatial processing in vision or somatosensation.
Example of picking up a glass and based on what is in it determines what
body movements we make.

B. Cognitive Abilities:

\begin{enumerate}
\def\labelenumi{\arabic{enumi}.}
\item
  Orientation and Memory
\end{enumerate}

\begin{enumerate}
\def\labelenumi{\alph{enumi}.}
\item
  \textbf{Orientation} = the patient's awareness of his or her immediate
  circumstances.
\end{enumerate}

\begin{enumerate}
\def\labelenumi{\alph{enumi}.}
\setcounter{enumi}{8}
\item
  \textbf{Awareness of time} = date. ``Do you know what day it is?''
\end{enumerate}

\begin{enumerate}
\def\labelenumi{\roman{enumi}.}
\setcounter{enumi}{1}
\item
  \textbf{Awareness of place} = location of where they are and why as
  well as who they are. ``Do you know where you are?,'' ``What is your
  name?,'' ``Who is the current president?''
\end{enumerate}

\begin{enumerate}
\def\labelenumi{\alph{enumi}.}
\setcounter{enumi}{1}
\item
  \textbf{Memory} = the patient's ability to recall information. Memory
  is largely a function of the temporal lobe, along with structures
  beneath the cerebral cortex such as the hippocampus and the amygdala.
  Short term memory can be assessed using the three-word test. Patients
  are given three words (ex. Book, clock, train) and after a brief time
  period are asked to recall the three words. Amnesia can be defined as
  losing memories of events of the past \textbf{retrograde amnesia} or
  inability to make future memories \textbf{anterograde amnesia.}
\end{enumerate}

\begin{enumerate}
\def\labelenumi{\arabic{enumi}.}
\setcounter{enumi}{1}
\item
  Language and Speech

  \begin{enumerate}
  \def\labelenumii{\alph{enumii}.}
  \item
    \textbf{Language} is at the core of what it means to be self-aware.
    Asking the patient to perform a set of actions can assess the
    ability to understand language. ``Use you right pointer finger to
    touch the tip of your nose and then your left elbow.'' Often,
    language deficits can be determined without specific subtests; if a
    person cannot reply to a question properly, there may be a problem
    with the reception of language. \textbf{Aphasia} is the loss or
    speech or language.
  \item
    Speech
  \end{enumerate}
\end{enumerate}

\begin{enumerate}
\def\labelenumi{\arabic{enumi}.}
\item
  \textbf{Broca's area} = responsible for speech production.
  \textbf{Expressive aphasia} = speech production is compromised leading
  to broken or haulted speech with incorrect grammar usage.
\item
  \textbf{Wernicke's area} = responsible for processing or understanding
  speech. \textbf{Receptive aphasia} = patients do not understand what
  is said to them or what they are saying even when they are talking.
\item
  \textbf{Conduction aphasia} = patient's inability to connect
  understanding of speech to production of speech. Symptoms include
  inability to faithfully repeat spoken language.
\end{enumerate}

\begin{enumerate}
\def\labelenumi{\arabic{enumi}.}
\setcounter{enumi}{2}
\item
  \textbf{Sensorium} = the parts of the brain involved in reception and
  interpretation of sensory stimuli. From the primary cortical areas of
  the somatosensory, visual, auditory, and gustatory senses to the
  association areas that process information in these modalities, the
  cerebral cortex is the seat of conscious sensory perception. Two
  subtests assess specific functions of these cortical areas.
\end{enumerate}

\begin{enumerate}
\def\labelenumi{\alph{enumi}.}
\item
  \textbf{Praxis} = a practical exercise in which the patient performs a
  task completely on the basis of verbal description without any
  demonstration from the examiner. For example, the patient can be told
  to take their left hand and place it palm down on their left thigh,
  then flip it over so the palm is facing up, and then repeat this four
  times.~
\item
  \textbf{Gnosis} -- sensory perception involving two processes.
\end{enumerate}

\begin{enumerate}
\def\labelenumi{\roman{enumi}.}
\item
  \textbf{Stereognosis} = involves the naming of objects strictly on the
  basis of the somatosensory information that comes from manipulating
  them with their eyes closed.
\item
  \textbf{Graphesthesia} = recognize numbers or letters written on the
  palm of the hand with a dull pointer, such as a pen cap.
\end{enumerate}

\begin{enumerate}
\def\labelenumi{\arabic{enumi}.}
\setcounter{enumi}{3}
\item
  \textbf{Judgment and Abstract reasoning} =~Making judgments and
  reasoning in the abstract are necessary to produce movements as part
  of larger responses. ``When your alarm goes off, do you hit the snooze
  button or jump out of bed?'' Is 10 extra minutes in bed worth the
  extra rush to get ready for your day? Will hitting the snooze button
  multiple times lead to feeling more rested or result in a panic as you
  run late? The prefrontal cortex is related to personality
\end{enumerate}

\textbf{16.3 The Cranial Nerve Exam} allows directed tests of forebrain
and brain stem structures

A. Sensory Nerves

\begin{enumerate}
\def\labelenumi{\arabic{enumi}.}
\item
  The olfactory nerve (CN I) receives sense of smell
\item
  Optic nerve (CNII) receives sense of vision. Testing vision relies on
  the tests that are common in an optometry office such as
  the~\textbf{Snellen chart.} Testing the extent of the visual field
  means that the examiner can establish the boundaries of peripheral
  vision as simply as holding their hands out to either side and asking
  the patient when the fingers are no longer visible without moving the
  eyes to track them.~Physical inspection of the optic disk, or where
  the optic nerve emerges from the eye, can be accomplished by looking
  through the pupil with an ophthalmoscope.
\item
  Vestibulocochlear nerves (CN VIII) receives sense of equilibrium and
  hearing. Problems with balance, such as vertigo, and deficits in
  hearing may both point to problems with the inner ear.~Problems with
  hearing can be assessed using a tuning fork to determine types of
  hearing loss:

  \begin{enumerate}
  \def\labelenumii{\alph{enumii}.}
  \item
    \textbf{Conductive hearing}~= relies on vibrations being conducted
    through the ossicles of the middle ear.
  \item
    \textbf{Sensorineural hearing =} relies on the transmission of sound
    stimuli through the neural components of the inner ear and cranial
    nerve.
  \item
    The \textbf{Rinne test} uses a vibrating tuning fork is placed on
    the mastoid process and the patient indicates when the sound
    produced from this is no longer present. Then the fork is
    immediately moved to just next to the ear canal so the sound travels
    through the air. If the sound is not heard through the ear, meaning
    the sound is conducted better through the temporal bone than through
    the ossicles, a conductive hearing deficit is present.
  \item
    The \textbf{Webber} \textbf{test} also uses a tuning fork to
    differentiate between conductive versus sensorineural hearing loss.
    In this test, the tuning fork is placed at the top of the skull, and
    the sound of the tuning fork reaches both inner ears by travelling
    through bone. In a healthy patient, the sound would appear equally
    loud in both ears. With unilateral conductive hearing loss, however,
    the tuning fork sounds louder in the ear with hearing loss. This is
    because the sound of the tuning fork has to compete with background
    noise coming from the outer ear, but in conductive hearing loss, the
    background noise is blocked in the damaged ear, allowing the tuning
    fork to sound relatively louder in that ear. With unilateral
    sensorineural hearing loss, however, damage to the cochlea or
    associated nervous tissue means that the tuning fork sounds quieter
    in that ear.
  \end{enumerate}
\end{enumerate}

\begin{enumerate}
\def\labelenumi{\arabic{enumi}.}
\setcounter{enumi}{3}
\item
  Taste sensation is relayed to the brain stem through fibers of the
  facial (CN VII) and glossopharyngeal nerves (CN IX) and the vagus
  nerve (X).
\item
  The trigeminal nerve (CN V) is a mixed nerve that carries the general
  somatic senses from the head, similar to those coming through spinal
  nerves from the rest of the body. The primary sensory subtest for the
  trigeminal system is sensory discrimination. A cotton-tiped
  applicator, which is cotton attached to the end of a thin wooden
  stick, can be used easily for this. The wood of the applicator can be
  snapped so that a pointed end is opposite the soft cotton-tipped end.
  The cotton end provides a touch stimulus, while the pointed end
  provides a painful, or sharp, stimulus. While the patient's eyes are
  closed, the examiner touches the two ends of the applicator to the
  patient's face, alternating randomly between them. The patient must
  identify whether the stimulus is sharp or dull.
\end{enumerate}

B. Gaze Control

The three nerves that control the extraocular muscles are the:

\begin{enumerate}
\def\labelenumi{\arabic{enumi}.}
\item
  Oculomotor (CN III) - Movement of eyelid and eyeball (via superior
  rectus, inferior rectus, medial rectus, and inferior oblique), shape
  of lens, contracts pupil size
\item
  Trochlear (CN IV) - Movement of eye by the superior oblique
\item
  Abducens (CN VI) - Movement of the eyeball by the lateral rectus
\item
  \textbf{Saccades} = rapid, conjugate movements of the eyes to survey a
  complicated visual stimulus, or to follow a moving visual stimulus.
  Testing eye movement is simply a matter of having the patient track
  the tip of a pen as it is passed through the visual field.~
\item
  \textbf{Diplopia}, or double vision,~as the two eyes are temporarily
  pointed at different stimuli.
\item
  \textbf{Convergence} = when the two eyes move to look at something
  closer to the face, they both adduct. To keep the stimulus in focus,
  the eye also needs to change the shape of the lens, which is
  controlled through the parasympathetic fibers of the oculomotor nerve.
  The change in focal power of the eye is referred to
  as~\textbf{accommodation}. Accommodation ability changes with age;
  focusing on nearer objects, such as the written text of a book or on a
  computer screen, may require corrective lenses later in life.
  Coordination of the skeletal muscles for convergence and coordination
  of the smooth muscles of the ciliary body for accommodation are
  referred to as the~\textbf{accommodation--convergence reflex}.
\item
  A crucial function of the cranial nerves is to keep visual stimuli
  centered on the fovea of the retina. The~\textbf{vestibulo-ocular
  reflex (VOR)}coordinates all of the components, both sensory and
  motor, that make this possible
\end{enumerate}

C. Nerves of the Face and Oral Cavity

The facial (CN VII) and glossopharyngeal (CN IX) nerves convey gustatory
stimulation to the brain. The hypoglossal nerve is the motor nerve that
controls the muscles of the tongue, except for the palatoglossus muscle,
which is controlled by the vagus nerve. There are two sets of muscles of
the tongue. The~\textbf{extrinsic muscles of the tongue}~are connected
to other structures, whereas the~\textbf{intrinsic muscles of the
tongue}~are completely contained within the lingual tissues.~

\begin{enumerate}
\def\labelenumi{\arabic{enumi}.}
\item
  Facial nerve = controls muscles controlling facial expressions,
  secretion of saliva by the submandibular and sublingual glands and
  tears by the lacrimal gland, and sensory function for taste from the
  anterior 2/3 of the tongue
\item
  Glossopharyngeal = controls secretion of saliva by the parotid glands,
  elevation of pharynx during swallowing, and taste.
\item
  \textbf{Aguesia} = loss of taste
\item
  \textbf{Bells Palsy} = characterized by muscle weakness that causes
  one half of the face to droop. Bell's palsy may be a reaction to a
  viral infection and usually resolves on its own within six months.
\item
  These nerves can be tested by sticking out the tongue and saying
  ``ah''
\end{enumerate}

D. Motor Nerves of the Neck

The accessory nerve (CN XI) innervates the sternocleidomastoid (flex
head forward and side to side) and trapezius muscles~(extension and
hyperextension of head as well as shrugging of shoulders).

\subsection{16.4 The Sensory and Motor
Exams}\label{the-sensory-and-motor-exams}

A. Sensory Modalities and Location

\begin{enumerate}
\def\labelenumi{\arabic{enumi}.}
\item
  Somatic senses are incorporated mostly into the skin, muscles, or
  tendons, whereas the visceral senses come from nervous tissue
  incorporated into the majority of organs such as the heart or stomach.
\item
  The somatic senses are those that usually make up the conscious
  perception of the how the body interacts with the environment.
\item
  The visceral senses are most often below the limit of conscious
  perception because they are involved in homeostatic regulation through
  the autonomic nervous system.
\item
  Testing of the senses begins with examining the regions known as
  dermatomes that connect to the cortical region where somatosensation
  is perceived in the postcentral gyrus.
\item
  To test the sensory fields, a simple stimulus of the light touch of
  the soft end of a cotton-tipped applicator is applied at various
  locations on the skin.~
\item
  The \textbf{Romberg test} The patient is asked to stand straight with
  feet together then after achieving balance the patient closes their
  eyes and has to maintain the balance.
\end{enumerate}

B. Muscle Strength and Voluntary Movement

1. The skeletomotor system is largely based on the simple, two-cell
projection from the precentral gyrus of the frontal lobe to the skeletal
muscles. Outputs from the frontal lobe synapse at the ventral horn motor
neurons (Upper Motor Neuron UMN and Lower Motor Neuron LMN) before
projecting to the skeletal muscle.

2. The lack of muscle tone, known as~\textbf{hypotonicity}
or~\textbf{flaccidity}, may indicate that the LMN is not conducting
action potentials that will keep a basal level of acetylcholine in the
neuromuscular junction.

3. If muscle tone is present, muscle strength is tested by having the
patient contract muscles against resistance. The examiner will ask the
patient to lift the arm, for example, while the examiner is pushing down
on it.

4. Diseases that result in UMN lesions include cerebral palsy or MS, or
it may be the result of a stroke. A sign of UMN lesion is a negative
result in the subtest for~\textbf{pronator drift}.

5. The patient is asked to extend both arms in front of the body with
the palms facing up. While keeping the eyes closed, if the patient
unconsciously allows one or the other arm to slowly relax, toward the
pronated position, this could indicate a failure of the motor system to
maintain the supinated position.

C. Reflexes

See Ch. 15 reflex discussion on stretch reflex and superficial reflexes.

\begin{enumerate}
\def\labelenumi{\arabic{enumi}.}
\item
  For the arm, the common reflexes to test are of the biceps,
  brachioradialis, triceps, and flexors for the digits. For the leg, the
  knee-jerk reflex of the quadriceps is common, as is the ankle reflex
  for the gastrocnemius and soleus.~
\end{enumerate}

2. \textbf{Plantar reflex}~that tests for the~\textbf{Babinski sign}~on
the basis of the extension or flexion of the toes at the plantar surface
of the foot. The plantar reflex is commonly tested in newborn infants to
establish the presence of neuromuscular function. To elicit this reflex,
an examiner brushes a stimulus, usually the examiner's fingertip, along
the plantar surface of the infant's foot. An infant would present a
positive Babinski sign, meaning the foot dorsiflexes and the toes extend
and splay out. As a person learns to walk, the plantar reflex changes to
cause curling of the toes and a moderate plantar flexion.~

D. Comparison of Upper and Lower Motor Neuron Damage

1. Many of the tests of motor function can indicate differences that
will address whether damage to the motor system is in the upper or lower
motor neurons. Signs that suggest a UMN lesion include muscle weakness,
strong deep tendon reflexes, decreased control of movement or slowness,
pronator drift, a positive Babinski sign,~\textbf{spasticity}, and
the~\textbf{clasp-knife response}. Spasticity is an excess contraction
in resistance to stretch. It can result in~\textbf{hyperflexia}, which
is when joints are overly flexed. The clasp-knife response occurs when
the patient initially resists movement, but then releases, and the joint
will quickly flex like a pocket knife closing.

2. A lesion on the LMN would result in paralysis, or at least partial
loss of voluntary muscle control, which is known as~\textbf{paresis}.
The paralysis observed in LMN diseases is referred to as~\textbf{flaccid
paralysis}, referring to a complete or partial loss of muscle tone, in
contrast to the loss of control in UMN lesions in which tone is retained
and spasticity is exhibited. Other signs of an LMN lesion
are\textbf{fibrillation},~\textbf{fasciculation}, and compromised or
lost reflexes resulting from the denervation of the muscle fibers

\textbf{16.5 The Coordination and Gait Exams}

\subsubsection{A. Locations and Connections of the
Cerebellum}\label{a.-locations-and-connections-of-the-cerebellum}

\begin{enumerate}
\def\labelenumi{\arabic{enumi}.}
\item
  \textbf{\ul{Cerebellum}} = accounts for 11\% of the brain's mass.
\item
  The cerebellum functions in the coordination and modulation of motor
  command from the cerebral cortex and maintaining balance and
  equilibrium.
\item
  The cerebellum is partially hidden by the cerebral hemispheres and is
  the second largest structure in the brain.
\item
  The cerebellum is separated from the cerebrum by the
  \textbf{\ul{transverse fissure}}.
\item
  The cerebellum also possesses fold-like wrinkles called
  \textbf{\ul{folia}}, is divided into two hemispheres, and further
  subdivided into lobes: the \textbf{anterior lobe and posterior lobe}.
\item
  The two cerebellar hemispheres are separated by the \textbf{vermis}
  while the anterior and posterior lobes are separated by
  \textbf{\ul{the primary fissure}}.
\item
  The white matter of the cerebellum is called the \textbf{\ul{arbor
  vitae}} and is surrounded by gray matter called the
  \textbf{\ul{cerebellar cortex}}.
\end{enumerate}

B. Coordination and Alternating Movement

\begin{enumerate}
\def\labelenumi{\arabic{enumi}.}
\item
  Testing for cerebellar function is the basis of the coordination exam.
  The subtests target appendicular musculature, controlling the limbs,
  and axial musculature for posture and gait. The assessment of
  cerebellar function will depend on the normal functioning of other
  systems addressed in previous sections of the neurological exam. Motor
  control from the cerebrum, as well as sensory input from somatic,
  visual, and vestibular senses, are important to cerebellar function.
\item
  The subtests that address appendicular musculature, and therefore the
  lateral regions of the cerebellum, begin with a check for tremor. The
  patient extends their arms in front of them and holds the position
  while the examiner watches for tremors.
\item
  The~\textbf{check reflex}~depends on cerebellar input to keep
  increased contraction from continuing after the removal of resistance.
  The patient flexes the elbow against resistance from the examiner to
  extend the elbow.~
\end{enumerate}

C. Posture and Gait

\begin{enumerate}
\def\labelenumi{\arabic{enumi}.}
\item
  Gait can either be considered a separate part of the neurological exam
  or a subtest of the coordination exam that addresses walking and
  balance.~
\item
  A subtest called station begins with the patient standing in a normal
  position to check for the placement of the feet and balance. The
  patient is asked to hop on one foot to assess the ability to maintain
  balance and posture during movement.~
\item
  Subtests of walking begin with having the patient walk normally for a
  distance away from the examiner, and then turn and return to the
  starting position. The examiner watches for abnormal placement of the
  feet and the movement of the arms relative to the movement. The
  patient is then asked to walk with a few different variations. Tandem
  gait is when the patient places the heel of one foot against the toe
  of the other foot and walks in a straight line in that manner. Walking
  only on the heels or only on the toes will test additional aspects of
  balance.
\end{enumerate}

D. \textbf{Ataxia} = presents as a loss of coordination in voluntary
movements. Ataxia can also refer to sensory deficits that cause balance
problems, primarily in proprioception and equilibrium. Ataxia is often
the result of exposure to exogenous substances (alcohol, ketamine or
mercury), focal lesions (stroke, trauma, MS, or tumor), or a genetic
disorder.

\section{}\label{section-13}

\section{Chapter 17}\label{chapter-17}

\begin{enumerate}
\def\labelenumi{\arabic{enumi}.}
\item
  \textbf{An Introduction to the Endocrine System and Hormones.} The
  endocrine system works with or in harmony with the nervous system to
  control and coordinate all the activities of the body and to maintain
  homeostasis. \textbf{Endocrinology} - field of medicine that focuses
  on the treatment of endocrine system disorders.

  \begin{enumerate}
  \def\labelenumii{\arabic{enumii}.}
  \item
    The endocrine system and nervous system are similar yet different:

    \begin{enumerate}
    \def\labelenumiii{\arabic{enumiii}.}
    \item
      Both systems rely on the release of chemicals that bind to
      specific receptors on their target cells.
    \item
      Both share many chemical messengers; when released into the
      bloodstream they are called hormones but when released into a
      synapse, they are called neurotransmitters.
    \item
      Both systems are regulated primarily by negative feedback control.
    \item
      Both share a common goal: to preserve homeostasis by coordinating
      and regulating the activities of other cells, tissues, organs, and
      systems.
    \end{enumerate}
  \item
    Mechanisms of Intercellular Communication

    \begin{enumerate}
    \def\labelenumiii{\arabic{enumiii}.}
    \item
      \textbf{Direct communication} -- via \textbf{gap junctions}; use
      ions, small solutes, and other lipid-soluble materials as chemical
      mediators; effects are usually limited to adjacent cells of the
      same type that are interconnected by connexons.
    \item
      \textbf{Paracrine communication} -- via \textbf{extracellular
      fluids}; use paracrine factors as chemical mediators; effects are
      primarily limited to the local area where paracrine factor
      concentrations are relatively high; target cells must have
      appropriate receptors.
    \item
      \textbf{Endocrine communication} -- via the \textbf{bloodstream};
      use hormones as chemical mediators; effects are on target cells
      located in other tissues or organs; target cells must have an
      appropriate receptor.
    \item
      \textbf{Neural communication} -- via \textbf{synaptic clefts}; use
      neurotransmitters as chemical mediators; effects are limited to
      very specific areas; target cells must have appropriate receptors.
    \end{enumerate}
  \item
    Structures of the endocrine system:

    \begin{enumerate}
    \def\labelenumiii{\arabic{enumiii}.}
    \item
      Thalamus
    \item
      Pineal gland
    \item
      Pituitary gland
    \item
      Thyroid
    \item
      Thymus
    \item
      Adrenal
    \item
      Pancreas
    \item
      Ovaries
    \item
      Testes
    \end{enumerate}
  \end{enumerate}
\end{enumerate}

\textbf{17.2 Hormones} are chemical messengers released by endocrine
cells/glands into bloodstream to be transported throughout the body to
regulate the metabolic functions and activities of other cells of the
body. See table at end of outline for list of endocrine glands and their
related hormones.

\begin{enumerate}
\def\labelenumi{\Alph{enumi}.}
\item
  Hormones and paracrine factors of the body can be divided into groups:
\end{enumerate}

\begin{enumerate}
\def\labelenumi{\arabic{enumi}.}
\item
  \textbf{Amino acid derivatives} -- hormones derived from a single
  amino acid. Examples include: the thyroid hormones such as thyroxine
  and triiodothyronine; the \textbf{catecholamines} such as epinephrine,
  norepinephrine, and dopamine; and melatonin
\item
  \textbf{Peptide hormones} -- chains of amino acids. Examples include:

  \begin{enumerate}
  \def\labelenumii{\alph{enumii}.}
  \item
    \textbf{Polypeptides}: antidiuretic hormone (9 amino acids) and
    oxytocin (9 amino acids)
  \item
    \textbf{Small proteins}: insulin (51 amino acids), growth hormone
    (191 amino acids) and prolactin (198 amino acids)
  \item
    \textbf{Glycoproteins}: thyroid-stimulating hormone, luteinizing
    hormone, and follicle-stimulating hormone
  \end{enumerate}
\item
  \textbf{Lipid derivative}s -- consist of carbon rings and side chains
  built either from fatty acids chains or cholesterol.

  \begin{enumerate}
  \def\labelenumii{\alph{enumii}.}
  \item
    \textbf{Eicosanoids} -- built from fatty acid chains and include:
    leukotrienes and prostaglandins.
  \item
    \textbf{Steroid hormones} -- built from cholesterol molecules and
    include: testosterone, estrogen and progesterone, corticosteroids,
    and calcitriol.
  \end{enumerate}
\end{enumerate}

\begin{enumerate}
\def\labelenumi{\Alph{enumi}.}
\item
  Pathways of Hormone Action

  \begin{enumerate}
  \def\labelenumii{\arabic{enumii}.}
  \item
    \textbf{Lipid-soluble} = A steroid hormone directly initiates the
    production of proteins within a target cell. Steroid and thyroid
    hormones easily diffuse through the cell membrane. The steroid
    hormone binds to its receptor in the cytosol, forming a
    receptor--hormone complex. The receptor--hormone complex then enters
    the nucleus and binds to the target gene on the DNA. Thyroid
    hormones bind to receptors already bound to DNA. Transcription of
    the gene creates a messenger RNA that is translated into the desired
    protein within the cytoplasm.
  \item
    \textbf{Water-soluble} = Water-soluble hormones cannot diffuse
    through the cell membrane. These hormones must bind to a surface
    cell-membrane receptor. The receptor then initiates a cell-
    signaling pathway within the cell involving G proteins, adenylyl
    cyclase, the secondary messenger cyclic AMP (cAMP), and protein
    kinases. In the final step, these protein kinases phosphorylate
    proteins in the cytoplasm. This activates proteins in the cell that
    carry out the changes specified by the hormone.
  \end{enumerate}
\item
  Hormones are regulated by feedback mechanisms and hormonal
  interactions

  \begin{enumerate}
  \def\labelenumii{\arabic{enumii}.}
  \item
    Factors Affecting Target Cell Response. Hormones must have receptors
    on their target tissue.

    \begin{enumerate}
    \def\labelenumiii{\alph{enumiii}.}
    \item
      \textbf{Downregulation} - the presence of a significant level of a
      hormone circulating in the bloodstream can cause its target cells
      to decrease their number of receptors for that hormone, allowing
      cells to become less reactive to the excessive hormone levels.
    \item
      \textbf{Upregulation} - cells increase their number of receptors
      due to the level of a hormone becoming chronically reduced.
    \end{enumerate}
  \item
    Interactions between hormones:

    \begin{enumerate}
    \def\labelenumiii{\alph{enumiii}.}
    \item
      \textbf{Antagonistic effects} - one hormone inhibits the response
      of another therefore they generate opposite responses; example:
      insulin lowers blood sugar while glucagon raises blood sugar
    \item
      \textbf{Synergistic effects} - two hormones with similar effects
      produce an amplified response. In some cases, two hormones are
      required for an adequate response. For example, two different
      reproductive hormones---FSH from the pituitary gland and estrogens
      from the ovaries---are required for the maturation of female ova
      (egg cells).
    \item
      \textbf{Permissive effects} - one hormone is needed to activate
      another; example: rennin stimulates the conversion of Angiotensin
      I into Angiotensin II
    \end{enumerate}
  \item
    Feedback mechanisms

    \begin{enumerate}
    \def\labelenumiii{\alph{enumiii}.}
    \item
      \textbf{Negative feedback systems} - physiological response causes
      a decrease in the release of the hormone; most commonly used
    \item
      \textbf{Positive feedback systems} - physiological response causes
      an increase in the release of the hormone; rarely used
    \end{enumerate}
  \end{enumerate}
\item
  Role of endocrine gland stimuli: Hormones can stimulate behavior or
  behavior may stimulate hormones. Elevated hormone levels do not equate
  to elevated behavior.
\end{enumerate}

17.3 Hypothalamus and Pituitary

\begin{enumerate}
\def\labelenumi{\Alph{enumi}.}
\item
  Hypothalmus
\end{enumerate}

\begin{enumerate}
\def\labelenumi{\arabic{enumi}.}
\item
  The hypothalamus provides the highest level of endocrine control. It
  integrates the activities of the nervous system and endocrine system.
\item
  The hypothalamus accomplishes this integration through three
  mechanisms:
\end{enumerate}

\begin{enumerate}
\def\labelenumi{\alph{enumi}.}
\item
  The hypothalamus contains autonomic centers that exert direct neural
  control of the endocrine cells, called chromaffin cells, of the
  adrenal medulla. When the sympathetic division is activated, this
  direct control allows the immediate stimulation of the adrenal gland.
\item
  Hypothalamic neurons synthesize two hormones -- ADH and OXT -- and
  transport them along axons within the infundibulum to the posterior
  lobe of the pituitary for storage and secretion.
\item
  The hypothalamus secretes regulatory hormones that control the
  secretions of the anterior pituitary gland. These regulatory hormones,
  called \textbf{releasing hormones} (RH) and \textbf{inhibiting
  hormones} (IH), flow via a network of fenestrated capillaries called
  the \textbf{hypophyseal portal system}.
\end{enumerate}

\begin{enumerate}
\def\labelenumi{\Alph{enumi}.}
\setcounter{enumi}{1}
\item
  Pituitary Gland -- Hypophysis
\end{enumerate}

\begin{enumerate}
\def\labelenumi{\arabic{enumi}.}
\item
  Also known as the ``master gland'' is located within the \textbf{sella
  turcica} of the sphenoid bone.
\item
  Connected to the hypothalamus via the infundibulum and a network of
  capillaries called the hypophyseal portal system. Divided into two
  lobes: an anterior lobe and a posterior lobe.
\item
  \textbf{Neurohypophysis} -- the posterior lobe of the pituitary gland
  connected to the hypothalamus by the \textbf{infundibulum}; contains
  the axons of the hypothalamic neurons. Stores and secretes hormones
  synthesized in the hypothalamus:
\end{enumerate}

\begin{enumerate}
\def\labelenumi{\alph{enumi}.}
\item
  \textbf{Antidiuretic hormone} (ADH also known as vasopressin) --
  increases water reabsorption within the renal tubules of the kidney.
  This results in a decrease in water loss from urine.
\item
  \textbf{Oxytocin} (OXT) -- stimulates the smooth muscle contractions
  of the uterus which initiates child birth. After delivery, stimulates
  the ejection of milk. In both sexes, known as the ``cuddle hormone''
  as it surges during arousal and orgasm.
\end{enumerate}

\begin{enumerate}
\def\labelenumi{\arabic{enumi}.}
\setcounter{enumi}{3}
\item
  \textbf{Adenohypophysis} -- the anterior lobe of the pituitary gland
  connected to the hypothalamus by the \textbf{hypophyseal portal
  system}. Controlled by regulating hormones, called \textbf{releasing
  hormones} (RH) and \textbf{inhibiting hormones} (IH) from the
  hypothalamus:
\end{enumerate}

\begin{enumerate}
\def\labelenumi{\alph{enumi}.}
\item
  \textbf{Thyroid stimulating hormone} (TSH) -- targets the thyroid
  gland; stimulates the thyroid to grow and increase its secretion of
  the thyroid hormones, T\textsubscript{3} and T\textsubscript{4}.
  Released in response to thyrotropin-releasing hormone (TRH) from
  hypothalamus.
\item
  \textbf{Adrenocorticotropic hormone} (ACTH or corticotropin) --
  stimulates the release of steroid hormones by the adrenal cortex.
  Released in response to corticotropin-releasing hormone (CRH) from the
  hypothalamus.
\item
  \textbf{Follicle stimulating hormone} (FSH or gonadotropin) --
  promotes ovarian follicles to develop in females and, in conjunction
  with luteinizing hormone, stimulates the secretion of estrogens. In
  males FSH promotes the physical maturation in sperm. Released in
  response to gonadotropin-releasing hormone (GnRH) from the
  hypothalamus.
\item
  \textbf{Luteinizing hormone} (LH) -- induces ovulation in females and
  promotes the secretion of estrogen and progesterone. In males it
  stimulates the production of sex hormones called androgens,
  specifically testosterone. Released in response to
  gonadotropin-releasing hormone from the hypothalamus.
\item
  \textbf{Growth hormone} (GH) -- stimulates cell growth and
  reproduction by accelerating the rate of protein synthesis
  particularly in skeletal muscle and bone. Regulated by growth
  hormone-releasing hormone (GH-RH) and growth hormone-inhibiting
  hormone (GH-IH) from the hypothalamus.
\item
  \textbf{Prolactin} (PRL or luteotropic hormone) -- works with other
  hormones to stimulate mammary gland development and the production of
  milk during pregnancy and during nursing. Regulated by several
  prolactin-releasing hormones and prolactin-inhibiting hormone (PIH).
\item
  \textbf{Melanocyte stimulating hormone} (MSH) -- stimulates
  melanocytes of the skin to increase their production of melanin.
  Non-functional in adults.

  \begin{enumerate}
  \def\labelenumii{\arabic{enumii}.}
  \setcounter{enumii}{3}
  \item
    Thyroid Gland
  \end{enumerate}
\end{enumerate}

\begin{enumerate}
\def\labelenumi{\Alph{enumi}.}
\item
  Located in the neck just below the larynx and anterior to the trachea.
  Divided into a right and left lobe connected by a narrow
  \textbf{isthmus}. Regulated by TSH from the pituitary gland.
\item
  The thyroid gland contains large numbers of \textbf{thyroid follicles}
  -- hollow spheres lined by a simple cuboidal epithelium called the
  \textbf{follicle cells}.
\end{enumerate}

\begin{enumerate}
\def\labelenumi{\arabic{enumi}.}
\item
  The follicle cells surround a cavity that holds a viscous
  \textbf{colloid}, a fluid containing a large quantity of dissolved
  proteins.
\item
  The follicle cells synthesize a globular protein called
  \textbf{thyroglobulin} and secrete it into the colloid of the thyroid
  follicle.
\item
  The thyroglobulin molecules contain the amino acid tyrosine. The
  thyroglobulin is combined with \textbf{iodide ions} absorbed from the
  diet to form the thyroid hormones: \textbf{T\textsubscript{3}
  (triiodothyronine)} and \textbf{T\textsubscript{4} (thyroxine)}.
\item
  Thyroid hormones have several effects in the body:
\end{enumerate}

\begin{enumerate}
\def\labelenumi{\alph{enumi}.}
\item
  Stimulates red blood cell production and thus enhanced oxygen
  delivery.
\item
  Stimulates the activity of other endocrine tissues.
\item
  Accelerates the turnover of minerals in bone.
\item
  Elevates rates of oxygen consumption and energy consumption in cells;
  increase basal metabolic rates.
\item
  Increases heart rate and force of contraction resulting in increased
  blood pressure.
\item
  Increases sensitivity to sympathetic stimulation.
\item
  Maintains the normal sensitivity of respiratory centers to changes in
  oxygen and carbon dioxide concentrations in the blood.
\end{enumerate}

\begin{enumerate}
\def\labelenumi{\Alph{enumi}.}
\setcounter{enumi}{2}
\item
  Between the follicles is a second population of endocrine cells called
  \textbf{parafollicular cells}, or \textbf{C (clear) cells}. Clear
  cells produce \textbf{calcitonin} (CT) which lowers blood calcium
  levels when they are too high. Calcitonin works by increasing the
  amount of calcium excreted in urine and increasing the deposition of
  calcium in bone by stimulating \textbf{osteoblast} activity.
\end{enumerate}

17.5 Parathyroid Gland

\begin{enumerate}
\def\labelenumi{\Alph{enumi}.}
\item
  Two pair of glands embedded in the posterior surfaces of the thyroid
  gland.
\item
  Composed of two cell populations
\end{enumerate}

\begin{enumerate}
\def\labelenumi{\arabic{enumi}.}
\item
  \textbf{Oxyphil cells} -- have no known function.
\item
  \textbf{Chief cells} -- produce \textbf{parathyroid hormone} (PTH)
  which increases blood calcium levels when they are too low.
\end{enumerate}

\begin{enumerate}
\def\labelenumi{\Alph{enumi}.}
\setcounter{enumi}{2}
\item
  PTH and calcitonin work as antagonists to maintain homeostasis of
  blood calcium levels. PTH specifically targets:
\end{enumerate}

\begin{enumerate}
\def\labelenumi{\arabic{enumi}.}
\item
  Bones -- activates \textbf{osteoclasts} causing calcium and phosphate
  ions to be released into the blood.
\item
  Intestine~-- increases calcium absorption from food.
\item
  Kidneys~-- promotes activation of vitamin D and increases calcium
  reabsorption in the kidney tubules.

  \begin{enumerate}
  \def\labelenumii{\arabic{enumii}.}
  \setcounter{enumii}{5}
  \item
    Adrenal Gland
  \end{enumerate}
\end{enumerate}

\begin{enumerate}
\def\labelenumi{\Alph{enumi}.}
\item
  Located retroperitoneal and superior to the kidney. Composed of two
  distinct regions: the adrenal cortex (outer) and adrenal medulla
  (inner).
\item
  \textbf{Adrenal cortex} produces steroid hormones from cholesterol
  (\textbf{corticosteroids}) and is divided into three regions:
\end{enumerate}

\begin{enumerate}
\def\labelenumi{\arabic{enumi}.}
\item
  \textbf{Zona glomerulosa} (outer) -- releases
  \emph{mineralocorticoids}, principally \textbf{aldosterone}, which
  controls electrolyte balance in the kidneys.
\item
  \textbf{Zona fasciculate} (middle) -- produces \emph{glucocorticoids}
  such as \textbf{cortisol} and \textbf{cortisone} which influence
  metabolism of glucose, protein, and fat; controlled by ACTH.
\item
  \textbf{Zona reticularis} (inner) -- produces \emph{androgens} or
  \emph{adrenal} \emph{sex} \emph{hormones} such as
  \textbf{testosterone} which influence masculinization.
\end{enumerate}

\begin{enumerate}
\def\labelenumi{\Alph{enumi}.}
\setcounter{enumi}{2}
\item
  \textbf{Adrenal medulla} releases hormones when the body is under
  stress and consists of hormone-producing cells called
  \textbf{chromaffin} \textbf{cells}.
\end{enumerate}

\begin{enumerate}
\def\labelenumi{\arabic{enumi}.}
\item
  \textbf{Epinephrine (aka adrenaline)} -- (80\%) elevates blood sugar,
  regulates body during stress or anger; raises blood pressure, heart
  beat, glycogen breakdown and increases all other sympathetic effects
  of the nervous system.
\item
  \textbf{Norepinephrine} \textbf{(aka noradrenaline}) -- helps maintain
  blood pressure, and accounts for 20\% of the hormones released by the
  medullary portion of the adrenal gland.
\item
  Effects of Epi/NE:

  \begin{enumerate}
  \def\labelenumii{\alph{enumii}.}
  \item
    Signals the liver and skeletal muscle cells to convert glycogen into
    glucose, resulting in increased blood glucose levels.
  \item
    These hormones increase the heart rate, pulse, and blood pressure to
    prepare the body to fight the perceived threat or flee from it.
  \item
    Dilates the airways, raising blood oxygen levels.
  \item
    Prompts vasodilation, further increasing the oxygenation of
    important organs such as the lungs, brain, heart, and skeletal
    muscle.
  \item
    Triggers vasoconstriction to blood vessels serving less essential
    organs such as the gastrointestinal tract, kidneys, and skin, and
    downregulates some components of the immune system.
  \item
    Other effects include a dry mouth, loss of appetite, pupil dilation,
    and a loss of peripheral vision.
  \end{enumerate}
\end{enumerate}

17.7 Pineal Gland

\begin{enumerate}
\def\labelenumi{\Alph{enumi}.}
\item
  Located in the roof of the 3\textsuperscript{rd} ventricle of the
  brain called the \textbf{epithalamus} region.
\item
  Composed of special secretory cells called \textbf{pinealocytes}.
\item
  The major product is \textbf{melatonin} whose concentrations rise and
  fall in a diurnal cycle. Levels are lowest during daylight hours and
  highest at night.
\end{enumerate}

\begin{enumerate}
\def\labelenumi{\arabic{enumi}.}
\item
  Melatonin appears to maintain the basic \textbf{circadian rhythms} --
  daily changes in physiological processes that follow a regular
  day-night pattern.
\item
  Melatonin protects against tissue damage by acting as an
  \textbf{antioxidant} and that protects the central nervous system from
  \textbf{free radicals} such as hydrogen peroxide.
\item
  Melatonin may inhibit reproductive development and functioning.
\end{enumerate}

17.8 Gonadal and Placental Hormones

\begin{enumerate}
\def\labelenumi{\Alph{enumi}.}
\item
  \textbf{Ovaries}: located in the pelvic cavity. Produce
  \textbf{estrogen} which regulates secondary sex characteristics
  (breast, pubic hair, etc.). Also produce \textbf{progesterone} which
  helps to stimulate the uterus to bring about thickening and
  vascularization of the endometrium in preparation for implantation of
  a fertilized egg.
\item
  \textbf{Testes}: located in the scrotum. Secretes
  \textbf{testosterone}, the male sex hormone, which brings about
  development of secondary sex characteristics, normal sex behaviors,
  and production of sperm. Also produces \textbf{inhibin} which inhibits
  the release of FSH and GnRH when sperm counts are high.
\item
  \textbf{Placenta}: a temporary organ only formed during pregnancy.
  Produces \textbf{hCG} hormone (human chorionic gonadotrophic) with aid
  in maintaining pregnancy and keeping the corpus luteum intact.

  \begin{enumerate}
  \def\labelenumii{\arabic{enumii}.}
  \setcounter{enumii}{8}
  \item
    The Endocrine Pancreas
  \end{enumerate}
\end{enumerate}

\begin{enumerate}
\def\labelenumi{\Alph{enumi}.}
\item
  Located posterior and inferior to the stomach. A unique organ that has
  both and endocrine and exocrine abilities.
\item
  \textbf{Islets of Langerhans}: \emph{endocrine} cells that produce
  hormones.
\end{enumerate}

\begin{enumerate}
\def\labelenumi{\arabic{enumi}.}
\item
  Alpha cells -- produce glucagon
\item
  Beta cells -- produce insulin
\item
  \textbf{Delta cells} -- produce \textbf{somatostatin} which inhibits
  insulin and glucagon secretion and slows the rates of food absorption
  and enzyme secretion along the digestive tract.
\item
  \textbf{PP cells} -- produce the hormone \textbf{pancreatic
  polypeptide} (PP) which inhibits gallbladder contractions and
  regulates the production of some of the digestive enzymes.
\end{enumerate}

\begin{enumerate}
\def\labelenumi{\Alph{enumi}.}
\setcounter{enumi}{2}
\item
  \textbf{Acinar cells}: \emph{exocrine} cells that produce enzymes and
  other digestive chemicals. (must know for GI chapter)
\end{enumerate}

\begin{enumerate}
\def\labelenumi{\arabic{enumi}.}
\item
  \textbf{Sodium bicarbonate} -- serves as a buffer of the HCl produced
  in the stomach.
\item
  \textbf{Proteases} -- secreted as inactive enzymes but become
  activated in the small intestine to form the activated:
  \textbf{carboxypeptidase}, \textbf{chymotrypsin}, and
  \textbf{trypsin}. These enzymes break down large polypeptides into
  oligopeptides, tripeptides, dipeptides and some amino acids.
\item
  \textbf{Pancreatic amylase} (almost identical to salivary amylase) --
  enzymes that further digest the carbohydrates, only briefly started in
  the mouth, into oligosaccharides and disaccharides.
\item
  \textbf{Pancreatic lipase} -- enzyme that breaks down lipids into
  fatty acids and glycerol.
\item
  \textbf{Pancreatic nucleases} -- enzymes that breaks down nucleic
  acids such as deoxyribonuclease (DNA) and ribonuclease (RNA).
\end{enumerate}

\begin{enumerate}
\def\labelenumi{\Alph{enumi}.}
\setcounter{enumi}{3}
\item
  Insulin and glucagon work as antagonists to maintain homeostasis of
  blood sugar.
\end{enumerate}

\begin{enumerate}
\def\labelenumi{\arabic{enumi}.}
\item
  Insulin lowers blood glucose levels by enhancing membrane transport of
  glucose into body cells, converting excess glucose to glycogen for
  short-term storage (glycogenesis) and into fat for long-term storage
  in adipocytes (lipogenesis).
\item
  Glucagon raises blood glucose levels by breaking down glycogen into
  glucose (glycogenolysis), synthesizes glucose from lactic acid and
  other non-carbohydrate molecules (gluconeogenesis), and releases
  glucose to the blood by liver cells.

  \begin{enumerate}
  \def\labelenumii{\arabic{enumii}.}
  \setcounter{enumii}{8}
  \item
    Organs with secondary endocrine function
  \end{enumerate}
\end{enumerate}

\begin{enumerate}
\def\labelenumi{\Alph{enumi}.}
\item
  \textbf{Heart} secretes \textbf{ANP} that literally means ``producing
  salty urine. ANP Inhibits aldosterone release by the adrenal cortex.
  The brain also secretes \textbf{BNP} that performs the same job.
\item
  Gastrointestinal tract possesses cells that produce secretin, gastrin,
  CCK, GIP, VIP, ghrelin, galanin, neuropeptide Y, and many more.
\item
  \textbf{Kidney} secretes \textbf{EPO} for red blood cell production
  and \textbf{rennin} for activation of angiotensin II, a potent
  vasoconstrictor.
\item
  \textbf{Skeleton} produces at least two hormones:

  \begin{enumerate}
  \def\labelenumii{\arabic{enumii}.}
  \item
    \textbf{Fibroblast growth factor 23 (FGF23}) is produced by bone
    cells to triggers the kidneys to inhibit the formation of calcitriol
    from vitamin D3 and to increase phosphorus excretion during high
    blood calcium levels.
  \item
    \textbf{Osteocalcin}, produced by osteoblasts, stimulates the
    pancreatic beta cells to increase insulin production. It also acts
    on peripheral tissues to increase their sensitivity to insulin and
    their utilization of glucose.
  \end{enumerate}
\end{enumerate}

\begin{enumerate}
\def\labelenumi{\Alph{enumi}.}
\setcounter{enumi}{3}
\item
  \textbf{Adipose} \textbf{tissue} releases \textbf{leptin} following
  the uptake of glucose and lipids resulting in satiety.
\item
  \textbf{Skin} produces \textbf{cholecalciferol}, the inactive form of
  vitamin D.
\item
  \textbf{Thymus Gland}: located posterior to the sternum and between
  the lungs. Large in infant, increases in size until puberty and then
  shrinks as the individual continues to age. The major hormonal product
  of the thymus gland is \textbf{thymosin} which appears to be essential
  for the normal development of T lymphocytes and the immune response.
\item
  \textbf{Liver} is responsible for secreting at least four important
  hormones or hormone precursors:

  \begin{enumerate}
  \def\labelenumii{\arabic{enumii}.}
  \item
    \textbf{Insulin-like growth factor (somatomedin}) is the immediate
    stimulus for growth in the body, especially of the bones.
  \item
    \textbf{Angiotensinogen} is the precursor to angiotensin, mentioned
    earlier, which increases blood pressure.
  \item
    \textbf{Thrombopoetin} stimulates the production of the blood's
    platelets.
  \item
    \textbf{Hepcidin} blocks the release of iron from cells in the body,
    helping to regulate iron homeostasis in our body fluids.
  \end{enumerate}

  \begin{enumerate}
  \def\labelenumii{\arabic{enumii}.}
  \setcounter{enumii}{8}
  \item
    Development and Aging of the Endocrine System
  \end{enumerate}
\end{enumerate}

\begin{enumerate}
\def\labelenumi{\Alph{enumi}.}
\item
  Development: The endocrine system arises from all three embryonic germ
  layers.
\end{enumerate}

\begin{enumerate}
\def\labelenumi{\arabic{enumi}.}
\item
  Endoderm: thyroid and parathyroid glands, as well as the pancreas and
  the thymus.
\item
  Mesoderm: The endocrine glands that produce the steroid hormones, such
  as the gonads and adrenal cortex
\item
  Ectoderm: pituitary gland, pineal gland, adrenal medulla
\end{enumerate}

\begin{enumerate}
\def\labelenumi{\Alph{enumi}.}
\setcounter{enumi}{1}
\item
  Aging
\end{enumerate}

As the body ages, changes occur that affect the endocrine system,
sometimes altering the production, secretion, and catabolism of
hormones. For example, the structure of the anterior pituitary gland
changes as vascularization decreases and the connective tissue content
increases with increasing age. This restructuring affects the gland's
hormone production. Certain hormones and gland production decrease with
age: reduced cortisol and aldosterone from adrenal gland, lower estrogen
and progesterone from ovaries, as well as lower testosterone levels.

\textbf{\hfill\break
}

\subsection{Endocrine Glands and Their Major
Hormones}\label{endocrine-glands-and-their-major-hormones}

{\def\LTcaptype{none} % do not increment counter
\begin{longtable}[]{@{}
  >{\raggedright\arraybackslash}p{(\linewidth - 6\tabcolsep) * \real{0.1447}}
  >{\raggedright\arraybackslash}p{(\linewidth - 6\tabcolsep) * \real{0.2280}}
  >{\raggedright\arraybackslash}p{(\linewidth - 6\tabcolsep) * \real{0.1850}}
  >{\raggedright\arraybackslash}p{(\linewidth - 6\tabcolsep) * \real{0.4423}}@{}}
\toprule\noalign{}
\begin{minipage}[b]{\linewidth}\raggedright
\textbf{Endocrine gland}
\end{minipage} & \begin{minipage}[b]{\linewidth}\raggedright
\textbf{Associated hormones}
\end{minipage} & \begin{minipage}[b]{\linewidth}\raggedright
\textbf{Chemical class}
\end{minipage} & \begin{minipage}[b]{\linewidth}\raggedright
\textbf{Effect}
\end{minipage} \\
\midrule\noalign{}
\endhead
\bottomrule\noalign{}
\endlastfoot
Pituitary (anterior) & Growth hormone (GH) & Protein & Promotes growth
of body tissues \\
Pituitary (anterior) & Prolactin (PRL) & Peptide & Promotes milk
production \\
Pituitary (anterior) & Thyroid-stimulating hormone (TSH) & Glycoprotein
& Stimulates thyroid hormone release \\
Pituitary (anterior) & Adrenocorticotropic hormone (ACTH) & Peptide &
Stimulates hormone release by adrenal cortex \\
Pituitary (anterior) & Follicle-stimulating hormone (FSH) & Glycoprotein
& Stimulates gamete production \\
Pituitary (anterior) & Luteinizing hormone (LH) & Glycoprotein &
Stimulates androgen production by gonads \\
Pituitary (posterior) & Antidiuretic hormone (ADH) & Peptide &
Stimulates water reabsorption by kidneys \\
Pituitary (posterior) & Oxytocin (OXY) & Peptide & Stimulates uterine
contractions during childbirth \\
Thyroid & Thyroxine (T\textsubscript{4}), triiodothyronine
(T\textsubscript{3}) & Amine & Stimulate basal metabolic rate \\
Thyroid & Calcitonin & Peptide & Reduces blood
Ca\textsuperscript{2+}~levels \\
Parathyroid & Parathyroid hormone (PTH) & Peptide & Increases blood
Ca\textsuperscript{2+~}levels \\
Adrenal (cortex) & Aldosterone & Steroid & Increases blood
Na\textsuperscript{+}~levels \\
Adrenal (cortex) & Cortisol, corticosterone, cortisone & Steroid &
Increase blood glucose levels \\
Adrenal (medulla) & Epinephrine, norepinephrine & Amine & Stimulate
fight-or-flight response \\
Pineal & Melatonin & Amine & Regulates sleep cycles \\
Pancreas & Insulin & Protein & Reduces blood glucose levels \\
Pancreas & Glucagon & Protein & Increases blood glucose levels \\
Testes & Testosterone & Steroid & Stimulates development of male
secondary sex characteristics and sperm production \\
\end{longtable}
}

\section{}\label{section-14}

\section{}\label{section-15}

\section{}\label{section-16}

\section{}\label{section-17}

\section{\texorpdfstring{Chapter 18 }{Chapter 18 }}\label{chapter-18}

\subsection{18.1. An Overview of Blood}\label{an-overview-of-blood}

\begin{enumerate}
\def\labelenumi{\Roman{enumi}.}
\item
  Background

  \begin{enumerate}
  \def\labelenumii{\arabic{enumii}.}
  \item
    Blood is what type of tissue?
  \item
    What are the three components of the formed elements?
  \end{enumerate}
\item
  Functions of Blood

  \begin{enumerate}
  \def\labelenumii{\arabic{enumii}.}
  \item
    What is the primary function of blood?
  \end{enumerate}

  \begin{enumerate}
  \def\labelenumii{\Alph{enumii}.}
  \item
    Transportation

    \begin{enumerate}
    \def\labelenumiii{\arabic{enumiii}.}
    \item
      What is one substance blood is responsible for transporting?

      \begin{enumerate}
      \def\labelenumiv{\alph{enumiv})}
      \item
        Why is this important?
      \end{enumerate}
    \item
      Blood assists in transportation of blood gases. What gas is
      transported to the lungs for exhalation?
    \item
      Waste products in the blood are transported to \_\_\_\_\_\_\_\_\_
      for excretion.
    \end{enumerate}
  \item
    Defense

    \begin{enumerate}
    \def\labelenumiii{\arabic{enumiii}.}
    \item
      What cells in the blood are responsible for immune defense?
    \item
      What in the blood is responsible for blood clotting?
    \end{enumerate}
  \item
    Maintenance of Homeostasis

    \begin{enumerate}
    \def\labelenumiii{\arabic{enumiii}.}
    \item
      Temperature regulation is regulated by \_\_\_\_\_\_\_\_\_\_
      feedback.
    \item
      Increasing the amount of blood circulating to the periphery would
      have what effect on overall body temperature?
    \item
      Aside from temperature, what other aspects of body homeostasis do
      blood and its components regulate?
    \end{enumerate}
  \end{enumerate}
\end{enumerate}

\begin{enumerate}
\def\labelenumi{\Roman{enumi}.}
\setcounter{enumi}{1}
\item
  Composition of
  Blood\includegraphics[width=3.40347in,height=2.12014in,alt={This figure shows three test tubes with a red and yellow liquid in them. The left panel shows normal blood, the center panel shows anemic blood and the right panel shows polycythemic blood.}]{images/media/image77.jpeg}

  \begin{enumerate}
  \def\labelenumii{\arabic{enumii}.}
  \item
    Erythrocytes are \_\_\_\_\_\_\_ blood cells.
  \item
    What is hematocrit, and what does it measure?
  \item
    The pale, thin layer found above erythrocytes in a centrifuged
    sample is the\_\_\_\_\_\_\_\_\_\_ \_\_\_\_\_\_\_\_, and it contains
    \_\_\_\_\_\_\_\_ blood cells.
  \item
    What is a normal value for Packed Cell Volume (PCV)?
  \item
    The image shows a normal centrifuged sample on the left, and a
    sample with anemia in the middle. What is different between the
    samples, and by extension what is the pathophysiology associated
    with anemia?
  \end{enumerate}
\item
  Characteristics of Blood

  \begin{enumerate}
  \def\labelenumii{\arabic{enumii}.}
  \item
    What pigment in the blood is responsible for the coloration?
  \item
    Do you expect blood to be more or less viscous than water?
  \item
    Blood is normally higher or lower than body temperature?
  \item
    What is a normal value for the blood volume of an adult?
  \end{enumerate}
\item
  Blood Plasma

  \begin{enumerate}
  \def\labelenumii{\arabic{enumii}.}
  \item
    Proteins make up about 7 percent of the volume of plasma, and the
    rest of the volume is from \_\_\_\_\_\_\_\_\_.
  \item
    Blood contains numerous substances that are all suspended within
    what medium?
  \end{enumerate}

  \begin{enumerate}
  \def\labelenumii{\Alph{enumii}.}
  \item
    Plasma Proteins

    \begin{enumerate}
    \def\labelenumiii{\arabic{enumiii}.}
    \item
      Why and how does albumin help to transport lipids?

      \begin{enumerate}
      \def\labelenumiv{\alph{enumiv})}
      \item
        Why can't lipids travel directly in the plasma?
      \end{enumerate}
    \item
      Gamma globulins are involved in immunity, and also known as
      \_\_\_\_\_\_\_\_\_\_ or \_\_\_\_\_\_\_\_\_\_\_\_.
    \item
      What plasma protein is essential for blood clotting?
    \item
      Albumin, most of the globulins, and fibrinogen are all produced by
      what organ?
    \end{enumerate}
  \item
    Other Plasma Solutes

    \begin{enumerate}
    \def\labelenumiii{\arabic{enumiii}.}
    \item
      What is responsible for transporting oxygen and some carbon
      dioxide in the blood?
    \item
      What is the difference between plasma and the formed elements?
    \end{enumerate}
  \end{enumerate}
\end{enumerate}

\subsection{18.2. Production of the Formed
Elements}\label{production-of-the-formed-elements}

\subsubsection{Background}\label{background}

\begin{enumerate}
\def\labelenumi{\arabic{enumi}.}
\item
  What is hemopoiesis?
\end{enumerate}

\subsubsection{Sites of Hemopoiesis}\label{sites-of-hemopoiesis}

\begin{enumerate}
\def\labelenumi{\arabic{enumi}.}
\item
  In adulthood hemopoiesis primarily occurs in bones, but there is some
  extramedullary hemopoiesis in the \_\_\_\_\_\_\_\_\_\_\_ and
  \_\_\_\_\_\_\_\_\_\_\_.
\end{enumerate}

\subsubsection{Differentiation of Formed Elements from Stem
Cells}\label{differentiation-of-formed-elements-from-stem-cells}

\begin{enumerate}
\def\labelenumi{\arabic{enumi}.}
\item
  All of the formed elements of the blood originate from hemocytoblasts
  or \_\_\_\_\_\_\_\_\_\_\_ \_\_\_\_\_\_\_\_\_\_\_ \_\_\_\_\_\_.
\item
  The hemopoietic growth factors stimulate hemopoietic stem cells to
  \_\_\_\_\_\_\_\_\_\_ and \_\_\_\_\_\_\_\_\_\_.
\end{enumerate}

\begin{enumerate}
\def\labelenumi{\Alph{enumi}.}
\item
  Lymphoid Stem Cells

  \begin{enumerate}
  \def\labelenumii{\arabic{enumii}.}
  \item
    Lymphoid stems cells produce what mature cells types?
  \item
    Lymphoid stem cells migrate where in the body?
  \end{enumerate}
\item
  Myeloid Stem Cells

  \begin{enumerate}
  \def\labelenumii{\arabic{enumii}.}
  \item
    Myeloid stem cells give rise to \_\_\_\_\_\_\_\_ blood cells,
    \_\_\_\_\_\_\_\_\_\_ that produce platelets, monocytes, and granular
    leukocytes.
  \end{enumerate}
\end{enumerate}

\begin{quote}
\includegraphics[width=6.5in,height=4.9375in,alt={This flowchart shows the pathways in which a multipotent hemotopoietic stem cell differentiates into the different cell types found in blood.}]{images/media/image78.jpeg}
\end{quote}

\subsubsection{Hemopoietic Growth
Factors}\label{hemopoietic-growth-factors}

\begin{enumerate}
\def\labelenumi{\Alph{enumi}.}
\item
  Erythropoietin (EPO)

  \begin{enumerate}
  \def\labelenumii{\arabic{enumii}.}
  \item
    Low oxygen stimulates the production of EPO in the kidneys. EPO
    signals for increase in what cell type in the blood?
  \end{enumerate}
\item
  Thrombopoietin

  \begin{enumerate}
  \def\labelenumii{\arabic{enumii}.}
  \item
    Thrombopoietin stimulates the production of what element in the
    blood responsible for blood clotting?
  \end{enumerate}
\item
  Cytokines

  \begin{enumerate}
  \def\labelenumii{\arabic{enumii}.}
  \item
    Cytokines are signaling molecules that help with resistance to
    disease. The two major subtypes of cytokines are the
    \_\_\_\_\_\_\_\_\_\_\_\_\_\_\_\_\_\_\_\_\_\_\_\_\_\_\_\_\_ and the
    \_\_\_\_\_\_\_\_\_\_\_\_\_\_\_.
  \item
    IL-1, IL-2, and IL-3 are all what subtype of cytokine?
  \end{enumerate}
\end{enumerate}

\subsubsection{Bone Marrow Sampling and
Transplants}\label{bone-marrow-sampling-and-transplants}

\begin{enumerate}
\def\labelenumi{\arabic{enumi}.}
\item
  Taking a sample of bone marrow is done through a \_\_\_\_\_\_\_\_\_
  \_\_\_\_\_\_\_\_\_ \_\_\_\_\_\_\_\_\_\_\_.
\end{enumerate}

18.3. Erythrocytes

\begin{enumerate}
\def\labelenumi{\Roman{enumi}.}
\item
  Background

  \begin{enumerate}
  \def\labelenumii{\arabic{enumii}.}
  \item
    What is the primary function of erythrocytes?
  \item
    Do erythrocytes leave the vascular network? Do leukocytes?
  \end{enumerate}
\item
  Shape and Structure of Erythrocytes

  \begin{enumerate}
  \def\labelenumii{\arabic{enumii}.}
  \item
    A \_\_\_\_\_\_\_\_\_\_ is an immature erythrocyte.
  \item
    What are erythrocytes missing that most cells in the body have? Why?
  \item
    What fills erythrocytes?
  \item
    Why is the biconcave shape of red blood cells important to their
    function?
  \end{enumerate}
\item
  Hemoglobin

  \begin{enumerate}
  \def\labelenumii{\arabic{enumii}.}
  \item
    Hemoglobin is made up of four folded protein chains known as
    \_\_\_\_\_\_\_\_\_\_\_, and a red pigment molecule that contains
    iron known as \_\_\_\_\_\_\_\_\_\_\_.
  \end{enumerate}
\end{enumerate}

\includegraphics[width=6.61458in,height=3.33333in]{images/media/image79.png}

\begin{enumerate}
\def\labelenumi{\arabic{enumi}.}
\setcounter{enumi}{1}
\item
  When is oxygen is bound to the iron in hemoglobin is forms
  \_\_\_\_\_\_\_\_\_\_\_\_\_\_\_, and when oxygen is released to the
  tissue it forms \_\_\_\_\_\_\_\_\_\_\_\_\_\_\_.
\item
  Insufficient hematopoiesis results in anemia, and excess hematopoiesis
  or an overproduction of red blood cells results in
  \_\_\_\_\_\_\_\_\_\_\_\_\_\_\_\_\_.
\item
  Binding of carbon dioxide to the amino acids in hemoglobin forms
  \_\_\_\_\_\_\_\_\_\_\_\_\_\_\_\_\_\_.
\item
  A pulse oximeter reading of 85 would indicate low blood oxygen or
  \_\_\_\_\_\_\_\_\_\_\_\_.
\item
  Erythropoietin (EPO), is produced in the kidneys and stimulates the
  production of the \_\_\_\_\_\_\_\_\_\_\_\_\_\_\_.
\end{enumerate}

\begin{enumerate}
\def\labelenumi{\Roman{enumi}.}
\setcounter{enumi}{3}
\item
  Lifecycle of Erythrocytes

  \begin{enumerate}
  \def\labelenumii{\arabic{enumii}.}
  \item
    Where are erythrocytes produced?
  \item
    Iron is necessary for red blood cell production. Where is iron
    stored in the body?
  \item
    \_\_\_\_\_\_\_\_\_\_\_\_\_\_\_ and \_\_\_\_\_\_\_\_\_\_\_\_\_\_\_
    are proteins used to help store iron.
  \item
    Old red blood cells are degraded by
    \_\_\_\_\_\_\_\_\_\_\_\_\_\_\_\_.
  \item
    After degradation of a red blood cell the globin is broken down into
    \_\_\_\_\_\_\_\_ \_\_\_\_\_\_\_\_\_\_.
  \item
    \includegraphics[width=5.0625in,height=6.57639in]{images/media/image80.png}After
    degradation of a red blood cell the non-iron portion of heme is
    degraded into a green pigment \_\_\_\_\_\_\_\_\_\_\_\_\_, and then
    into a yellow pigment \_\_\_\_\_\_\_\_\_\_\_\_\_\_.
  \end{enumerate}
\item
  Disorders of Erythrocytes

  \begin{enumerate}
  \def\labelenumii{\arabic{enumii}.}
  \item
    After a diagnosis of anemia what symptoms could an afflicted
    individual expect?
  \item
    \_\_\_\_\_\_\_\_ \_\_\_\_\_\_\_\_\_ \_\_\_\_\_\_\_\_ results in
    blood cells with a characteristic crescent shape due to a mutation
    in a hemoglobin gene.
  \item
    Anemia can occur due to lack of \_\_\_\_\_\_\_, a key mineral found
    inside of heme.
  \item
    \_\_\_\_\_\_\_\_\_\_\_\_\_\_\_ is a disease common in the
    Mediterranean and Middle East where red blood cells do not mature.
  \item
    Polycythemia can occur transiently in a person when they do not have
    enough \_\_\_\_\_\_\_\_\_\_\_\_ intake.
  \end{enumerate}
\end{enumerate}

\subsection[18.4. Leukocytes and Platelets]{\texorpdfstring{18.4.
Leukocytes and
Platelets\protect\includegraphics[width=3.92361in,height=4.92083in]{images/media/image81.png}}{18.4. Leukocytes and Platelets}}\label{leukocytes-and-platelets}

\begin{enumerate}
\def\labelenumi{\Roman{enumi}.}
\item
  Background

  \begin{enumerate}
  \def\labelenumii{\arabic{enumii}.}
  \item
    White blood cells, also known as \_\_\_\_\_\_\_\_\_\_\_\_\_\_,
    protect against invading microorganisms, and the body's own cells
    with mutated DNA.
  \end{enumerate}

  \begin{enumerate}
  \def\labelenumii{\Alph{enumii}.}
  \item
    Characteristics of Leukocytes

    \begin{enumerate}
    \def\labelenumiii{\arabic{enumiii}.}
    \item
      Compared to erythrocytes, are leukocytes less numerous or more
      numerous?
    \item
      In order for leukocytes to leave the vasculature and reach a
      destination in the tissue they are able to utilize emigration or
      \_\_\_\_\_\_\_\_\_\_\_\_\_.
    \item
      Attraction of leukocytes through \_\_\_\_\_\_\_\_\_\_\_\_\_\_
      \_\_\_\_\_\_\_\_\_\_\_\_\_ to sites in need of an immune response
      is due to chemical messages.
    \end{enumerate}
  \end{enumerate}
\item
  Classification of Leukocytes

  \begin{enumerate}
  \def\labelenumii{\arabic{enumii}.}
  \item
    Neutrophils, eosinophils, and basophils all contain numerous
    granules and are \_\_\_\_\_\_\_\_\_\_\_\_\_\_ leukocytes.
  \item
    Monocytes and lymphocytes are \_\_\_\_\_\_\_\_\_\_\_\_\_\_\_\_
    leukocytes.
  \end{enumerate}

  \begin{enumerate}
  \def\labelenumii{\Alph{enumii}.}
  \item
    Granular Leukocytes

    \begin{enumerate}
    \def\labelenumiii{\arabic{enumiii}.}
    \item
      What is the most abundant leukocyte in a healthy individual?
    \item
      Older neutrophils are polymorphonuclear. What does this mean, and
      how does it help to identify them?
    \item
      What strategies do neutrophils employ to deal with invading
      bacterial pathogens?
    \item
      What granular leukocyte is best suited to handle a parasitic worm
      infection?
    \item
      \includegraphics[width=5.71875in,height=1.8625in,alt={The left image shows a neutrophil, the middle image shows an eosinophil, and the right image shows a basophil.}]{images/media/image82.jpeg}Which
      leukocyte is the least abundant, and contains granules with
      histamine and heparin?
    \end{enumerate}
  \end{enumerate}
\item
  Agranular Leukocytes

  \begin{enumerate}
  \def\labelenumii{\arabic{enumii}.}
  \item
    What does the name lymphocyte tell us about the cells?
  \item
    What are the three major categories of lymphocytes?
  \item
    Cells that do no express ``self'' can be recognized by
    \_\_\_\_\_\_\_\_ \_\_\_\_\_\_\_\_\_\_ \_\_\_\_\_\_\_\_, which help
    provide some nonspecific immunity.
  \item
    \_\_\_ Lymphocytes produce antibodies as a part of humoral immunity.
  \item
    \_\_\_\_\_\_\_ Cells live for many years and allow the body to have
    a tailored response to a specific pathogen after a previous
    exposure.
  \item
    Macrophages found in the tissue originate as
    \_\_\_\_\_\_\_\_\_\_\_\_.
  \end{enumerate}
\item
  Lifecycle of Leukocytes

  \begin{enumerate}
  \def\labelenumii{\arabic{enumii}.}
  \item
    What is the general lifespan of a leukocyte?
  \end{enumerate}
\item
  Disorder of Leukocytes

  \begin{enumerate}
  \def\labelenumii{\arabic{enumii}.}
  \item
    \_\_\_\_\_\_\_\_\_\_ is when too few leukocytes are produced, as
    compared to excessive leukocyte production in
    \_\_\_\_\_\_\_\_\_\_\_.\includegraphics[width=2.34931in,height=3.39514in,alt={This flowchart shows a myeloid stem cell differentiating into platelets.}]{images/media/image83.jpeg}
  \item
    Cancers of leukocytes include \_\_\_\_\_\_\_\_\_\_ which involves an
    abundance of leukocytes, and \_\_\_\_\_\_\_\_\_\_\_\_ where
    malignant and T and B cells accumulate in tissues.
  \end{enumerate}
\item
  Platelets

  \begin{enumerate}
  \def\labelenumii{\arabic{enumii}.}
  \item
    Platelets, also known as \_\_\_\_\_\_\_\_\_\_\_\_\_, are produced as
    a cell fragment from a \_\_\_\_\_\_\_\_\_\_\_\_\_\_\_\_\_\_\_\_\_\_.
  \item
    What is the primary function of platelets?
  \end{enumerate}
\item
  Disorders of Platelets

  \begin{enumerate}
  \def\labelenumii{\arabic{enumii}.}
  \item
    Thrombocytosis is when there are too few or too many platelets?
  \item
    Thrombocytopenia is when there is an insufficient number of
    platelets. Why would this be dangerous?
  \end{enumerate}
\end{enumerate}

18.5. Hemostasis

\begin{enumerate}
\def\labelenumi{\Roman{enumi}.}
\item
  Background

  \begin{enumerate}
  \def\labelenumii{\arabic{enumii}.}
  \item
    What is hemostasis?
  \item
    What happens if hemostasis fails?
  \end{enumerate}
\item
  Vascular Spasm

  \begin{enumerate}
  \def\labelenumii{\arabic{enumii}.}
  \item
    Vascular Spasm is the first step of hemostasis. What happens during
    this step to help reduce the loss of blood?
  \end{enumerate}
\item
  Formation of the Platelet Plug

  \begin{enumerate}
  \def\labelenumii{\arabic{enumii}.}
  \item
    What is a platelet plug?
  \item
    What signals for platelets to form a platelet plug?
  \item
    What substances do platelets release to contribute to hemostasis?
  \end{enumerate}
\item
  Coagulation

  \begin{enumerate}
  \def\labelenumii{\arabic{enumii}.}
  \item
    \_\_\_\_\_\_\_\_\_\_\_\_\_\_ is formation of a blood clot.
  \item
    \includegraphics[width=3.9625in,height=4.77083in]{images/media/image84.png}\_\_\_\_\_\_\_\_,
    the end product of the coagulation cascade is a protein used to bind
    the clot together.
  \end{enumerate}

  \begin{enumerate}
  \def\labelenumii{\Alph{enumii}.}
  \item
    Clotting Factors Involved in Coagulation

    \begin{enumerate}
    \def\labelenumiii{\arabic{enumiii}.}
    \item
      Why are clotting factors crucial to blood clotting?
    \item
      What organ is responsible for the production of most of the
      clotting factors?
    \item
      What purpose does Vitamin K serve in clotting?
    \end{enumerate}
  \item
    Extrinsic Pathway

    \begin{enumerate}
    \def\labelenumiii{\arabic{enumiii}.}
    \item
      What triggers the extrinsic pathway to coagulation?
    \end{enumerate}
  \item
    Intrinsic Pathway

    \begin{enumerate}
    \def\labelenumiii{\arabic{enumiii}.}
    \item
      What triggers the intrinsic pathway to coagulation?
    \item
      Is this pathway faster or slower than the extrinsic pathway?
    \end{enumerate}
  \item
    Common Pathway

    \begin{enumerate}
    \def\labelenumiii{\arabic{enumiii}.}
    \item
      What factor, once activated, starts the common pathway?
    \end{enumerate}
  \end{enumerate}
\item
  Fibrinolysis

  \begin{enumerate}
  \def\labelenumii{\arabic{enumii}.}
  \item
    \_\_\_\_\_\_\_\_ is blood plasma without clotting factors.
  \item
    Fibrinolysis is degradation of what?
  \item
    Plasmin plays what role in fibrinolysis?
  \end{enumerate}
\item
  Plasma Anticoagulants

  \begin{enumerate}
  \def\labelenumii{\arabic{enumii}.}
  \item
    Antithrombin and heparin are what kinds of substances?
  \item
    Why would an anticoagulant be administered to a patient in surgery?
  \end{enumerate}
\item
  Disorders of Clotting

  \begin{enumerate}
  \def\labelenumii{\arabic{enumii}.}
  \item
    What is hemophilia?
  \item
    What is a thrombus? What is an embolus? What is the difference?
  \item
    Why would a physician recommend for a patient to take a low dose of
    aspirin?
  \end{enumerate}
\end{enumerate}

\subsubsection{18.6. Blood Typing}\label{blood-typing}

\paragraph{Background}\label{background-1}

\begin{enumerate}
\def\labelenumi{\arabic{enumi}.}
\item
  Why is blood typing important? When is it necessary to know about
  blood groups?
\end{enumerate}

\paragraph{Antigens, Antibodies, and Transfusion
Reactions}\label{antigens-antibodies-and-transfusion-reactions}

\begin{enumerate}
\def\labelenumi{\arabic{enumi}.}
\item
  What are antigens?
\item
  What are antibodies?
\item
  \_\_\_\_\_\_\_\_\_\_\_\_\_\_\_ is when antibodies bind to non-self
  erythrocytes and result in clumping.
\item
  Clumps of erythrocytes are degraded through a process known as
  \_\_\_\_\_\_\_\_\_\_\_\_\_\_\_\_\_
\item
  Why is degradation of a large quantity of red blood cells all at once
  dangerous?\includegraphics[width=4.10764in,height=3.75556in,alt={This figure shows an umbilical artery and vein passing through the placenta on the top left. The top right panel shows the first exposure to Rh+ antibodies in the mother. The bottom right panel shows the response when the second exposure in the form of another fetus takes place. Textboxes detail the steps in each process.}]{images/media/image85.jpeg}
\end{enumerate}

\paragraph{The ABO Blood Group}\label{the-abo-blood-group}

\begin{enumerate}
\def\labelenumi{\arabic{enumi}.}
\item
  A person with blood type A would express what antibodies in their
  blood plasma?
\item
  A person with blood type O would express what antigens on their red
  blood cells?
\end{enumerate}

\paragraph{\texorpdfstring{Rh Blood Groups
}{Rh Blood Groups }}\label{rh-blood-groups}

\begin{enumerate}
\def\labelenumi{\arabic{enumi}.}
\item
  If someone has antigen D present on their red blood cells are they
  considered Rh positive or Rh negative?
\item
  When are Rh antibodies formed, and how is this timing different from
  standard ABO antibodies?
\item
  Hemolytic Disease of the newborn (HDN) can occur during the second
  pregnancy of a Rh+ or Rh- mother?
\end{enumerate}

\paragraph{Determining ABO Blood
Types}\label{determining-abo-blood-types}

\begin{enumerate}
\def\labelenumi{\Alph{enumi}.}
\item
  How would you determine someone's blood type?
\item
  What blood type is the sample below?
\end{enumerate}

\includegraphics[width=6.34375in,height=2.57292in,alt={This figure shows three different red blood cells with different blood types.}]{images/media/image86.jpeg}

\paragraph{ABO Transfusion Protocols}\label{abo-transfusion-protocols}

\begin{enumerate}
\def\labelenumi{\arabic{enumi}.}
\item
  What blood type is a universal donor? Why?
\item
  What blood type is a universal recipient? Why?
\end{enumerate}

\includegraphics[width=5.19792in,height=3.97917in,alt={This table shows the different blood types, the antibodies in plasma, the antigens in the red blood cell, and the blood compatible blood types in an emergency.}]{images/media/image87.jpeg}

\section{}\label{section-18}

\section{\texorpdfstring{Chapter 19 }{Chapter 19 }}\label{chapter-19}

\subsection{19.1. Heart Anatomy}\label{heart-anatomy}

\subsubsection{Background}\label{background-2}

\begin{enumerate}
\def\labelenumi{\arabic{enumi}.}
\item
  Why is the heart an essential organ?
\end{enumerate}

\subsubsection{Location of the Heart}\label{location-of-the-heart}

\begin{enumerate}
\def\labelenumi{\arabic{enumi}.}
\item
  Where specifically is the heart located within the mediastinum?
\item
  Where is the base of the heart?
\item
  Laterally the heart is bordered by the lungs. How does the left lung
  accommodate the apex of the heart?
\end{enumerate}

\includegraphics[width=7in,height=6.34722in]{images/media/image88.png}

\subsubsection{Shape and size of the
Heart}\label{shape-and-size-of-the-heart}

\begin{enumerate}
\def\labelenumi{\arabic{enumi}.}
\item
  What is hypertrophy?
\item
  What effect does aerobic exercise have on the heart?
\item
  Pathophysiological enlargement of the heart is known as
  \_\_\_\_\_\_\_\_\_\_\_\_\_ \_\_\_\_\_\_\_\_\_\_\_\_\_\_\_\_.
\end{enumerate}

\subsubsection{Chambers and Circulation through the
Heart}\label{chambers-and-circulation-through-the-heart}

\begin{enumerate}
\def\labelenumi{\arabic{enumi}.}
\item
  The upper chambers of the heart are known as the \_\_\_\_\_\_\_\_\_\_.
\item
  The \_\_\_\_\_\_\_\_\_\_\_\_ are the lower chambers of the heart, and
  the primary pumping chambers.
\item
  The pulmonary circuit transports oxygenated blood to, and deoxygenated
  blood from the \_\_\_\_\_\_\_\_\_\_\_\_.
\item
  The \_\_\_\_\_\_\_\_\_\_\_ \_\_\_\_\_\_\_\_\_\_\_\_\_ transports
  oxygenated blood to the body and returns deoxygenated blood back to
  the heart.
\item
  Upon contraction the right ventricle pumps blood into the
  \_\_\_\_\_\_\_\_\_ \_\_\_\_\_\_\_\_\_ which bifurcates into the left
  and right pulmonary arteries.
\item
  The pulmonary arteries deliver blood to the lungs, and through gas
  exchange this blood gets oxygenated at the \_\_\_\_\_\_\_\_\_\_\_
  \_\_\_\_\_\_\_\_\_\_\_\_\_\_.
\item
  After oxygenation, the blood is returned to the heart through the
  \_\_\_\_\_\_\_\_\_\_\_\_\_\_\_ \_\_\_\_\_\_\_\_\_\_\_\_\_.
\item
  Blood returns to the heart, and specifically the right atrium, via the
  \_\_\_\_\_\_\_\_ \_\_\_\_\_\_\_\_\_\_ \_\_\_\_\_\_\_\_\_\_\_ and
  \_\_\_\_\_\_\_\_\_\_ \_\_\_\_\_\_\_\_\_\_
  \_\_\_\_\_\_\_\_\_\_\_\_\_\_\_.
\end{enumerate}

\includegraphics[width=7in,height=7.72847in]{images/media/image89.png}

\subsubsection{Membranes, Surface Features, and
Layers}\label{membranes-surface-features-and-layers}

\begin{enumerate}
\def\labelenumi{\Alph{enumi}.}
\item
  Membranes

  \begin{enumerate}
  \def\labelenumii{\arabic{enumii}.}
  \item
    The sac around the heart is known as the
    \_\_\_\_\_\_\_\_\_\_\_\_\_\_, and it both protects the heart and
    helps to maintain its position within the thorax.
  \item
    The \_\_\_\_\_\_\_\_\_\_\_, also known as the visceral pericardium,
    is the outer layer of the heart wall.
  \item
    What is responsible for secreting the serous fluid found in the
    pericardial cavity?

    \begin{enumerate}
    \def\labelenumiii{\alph{enumiii})}
    \item
      What purpose does this fluid serve?
    \end{enumerate}
  \end{enumerate}
\end{enumerate}

\includegraphics[width=7in,height=4.86597in]{images/media/image90.png}

\begin{enumerate}
\def\labelenumi{\Alph{enumi}.}
\item
  Surface Features of the Heart

  \begin{enumerate}
  \def\labelenumii{\arabic{enumii}.}
  \item
    The ``ear like'' superficial extension of the atria are known as
    \_\_\_\_\_\_\_\_\_\_\_\_\_\_\_\_.
  \item
    The grooves located on the heart are known as sulci. The groove
    between the atria and the ventricles is the
    \_\_\_\_\_\_\_\_\_\_\_\_\_\_\_.
  \item
    The groove on the anterior side of the heart between the left and
    right ventricles is the \_\_\_\_\_\_\_\_\_\_\_\_\_\_\_
    \_\_\_\_\_\_\_\_\_\_\_\_\_\_\_\_\_\_\_\_ \_\_\_\_\_\_\_\_\_\_\_ and
    the posterior groove between the ventricles is the
    \_\_\_\_\_\_\_\_\_\_\_\_\_\_\_\_\_\_\_\_
    \_\_\_\_\_\_\_\_\_\_\_\_\_\_\_\_\_\_ \_\_\_\_\_\_\_\_\_\_\_\_\_\_\_.
  \end{enumerate}
\end{enumerate}

\includegraphics[width=5.52083in,height=5.1875in,alt={The top panel shows the anterior view of the heart and the bottom panel shows the posterior view of the human heart. In both panels, the main parts of the heart are labeled.}]{images/media/image91.jpeg}

\begin{enumerate}
\def\labelenumi{\Alph{enumi}.}
\setcounter{enumi}{1}
\item
  Layers

  \begin{enumerate}
  \def\labelenumii{\arabic{enumii}.}
  \item
    The layers of the heart from deep to superficial are the
    \_\_\_\_\_\_\_\_\_\_\_\_\_\_\_\_\_\_\_,
    \_\_\_\_\_\_\_\_\_\_\_\_\_\_\_\_\_\_, and
    \_\_\_\_\_\_\_\_\_\_\_\_\_\_\_\_\_\_\_\_\_\_.
  \item
    The thickest layer of the heart, and the layer that is responsible
    for pumping is the \_\_\_\_\_\_\_\_\_\_\_\_\_\_\_\_\_\_\_\_\_\_.
  \item
    Why is the swirling pattern of the heart muscle useful?
  \end{enumerate}
\end{enumerate}

\begin{quote}
\includegraphics[width=2.91667in,height=2.875in,alt={This diagram shows the muscles in the heart.}]{images/media/image92.jpeg}
\end{quote}

\begin{enumerate}
\def\labelenumi{\arabic{enumi}.}
\setcounter{enumi}{3}
\item
  Which chamber is more muscular, the left or the right ventricle? Why?
  \includegraphics[width=5in,height=2.90625in,alt={In this figure the left panel shows the muscles of the heart in the relaxed position, and the right panel shows the muscles of the heart in contracted position.}]{images/media/image93.jpeg}
\item
  The endocardium, the innermost layer of the heart is coated with
  squamous epithelium known as \_\_\_\_\_\_\_\_\_\_\_\_\_\_\_\_. This
  epithelium is continuous with the lining of the vessel that attaches
  to the heart.
\end{enumerate}

\subsubsection{Internal Structure of the
Heart}\label{internal-structure-of-the-heart}

\begin{enumerate}
\def\labelenumi{\Alph{enumi}.}
\item
  Septa of the Heart

  \begin{enumerate}
  \def\labelenumii{\arabic{enumii}.}
  \item
    What structure divides the left and right atria?
  \item
    What is the fossa ovalis, and why was the foramen ovale necessary in
    the fetal heart?
  \item
    What structure divides the left and right ventricle?

    \begin{enumerate}
    \def\labelenumiii{\alph{enumiii})}
    \item
      How is this structure different than the interatrial septum?
    \end{enumerate}
  \item
    The atrioventricular septum contains openings to allow blood to
    flow, but only in one direction due to the presence of
    \_\_\_\_\_\_\_\_\_\_\_.
  \item
    Valves between the atria and ventricles are known as
    \_\_\_\_\_\_\_\_\_\_\_\_\_ \_\_\_\_\_\_\_\_\_\_\_.
  \item
    Valves between the ventricles and the pulmonary trunk or aorta are
    the \_\_\_\_\_\_\_\_\_\_\_\_\_\_\_\_\_\_ \_\_\_\_\_\_\_\_\_\_\_\_\_.
  \item
    Due to the openings present, the atrioventricular septum is
    reinforced by dense connective tissue known as the
    \_\_\_\_\_\_\_\_\_\_\_\_\_\_ \_\_\_\_\_\_\_\_\_\_\_\_\_\_\_\_.
  \end{enumerate}
\end{enumerate}

\begin{enumerate}
\def\labelenumi{\Alph{enumi}.}
\item
  Right Atrium

  \begin{enumerate}
  \def\labelenumii{\arabic{enumii}.}
  \item
    In addition to the deoxygenated blood supplied by the superior and
    inferior vena cavae, the right atrium also receives blood from the
    coronary circulation via the \_\_\_\_\_\_\_\_\_\_\_\_\_
    \_\_\_\_\_\_\_\_\_\_\_\_\_\_\_.
  \item
    The inferior vena cava drains blood from what part of the body?
  \item
    The \_\_\_\_\_\_\_\_\_\_\_\_\_ \_\_\_\_\_\_\_\_\_\_ is muscle that
    lines the auricle and anterior surface of the right atrium.
  \end{enumerate}
\item
  Right Ventricle

  \begin{enumerate}
  \def\labelenumii{\arabic{enumii}.}
  \item
    The right ventricle receives blood from the \_\_\_\_\_\_\_\_\_\_\_
    \_\_\_\_\_\_\_\_\_\_.
  \item
    What valve separates the right atrium from the right ventricle?
  \item
    What do the chordae tendineae, in conjunction with the papillary
    muscles, prevent from happening?
  \end{enumerate}
\end{enumerate}

\includegraphics[width=4.16667in,height=2.875in,alt={This photo shows the inside of the heart with the main muscles labeled.}]{images/media/image94.jpeg}

\begin{enumerate}
\def\labelenumi{\arabic{enumi}.}
\setcounter{enumi}{3}
\item
  \_\_\_\_\_\_\_\_\_\_\_\_\_\_ \_\_\_\_\_\_\_\_\_\_\_\_\_ are the ridges
  of cardiac muscle found in the left and right ventricle.
\item
  Upon contraction the right ventricle ejects blood into what vessel?
\end{enumerate}

\begin{enumerate}
\def\labelenumi{\Alph{enumi}.}
\setcounter{enumi}{2}
\item
  Left Atrium

  \begin{enumerate}
  \def\labelenumii{\arabic{enumii}.}
  \item
    What vessels return oxygenated blood from the lungs back to the left
    atrium of the heart?
  \item
    What valve is found between the left atrium and left ventricle?
  \end{enumerate}
\item
  Left Ventricle

  \begin{enumerate}
  \def\labelenumii{\arabic{enumii}.}
  \item
    The left and right ventricle pump the same amount of blood, so why
    is the left ventricle more muscular?
  \end{enumerate}
\item
  Heart Valve Structure and Function
\end{enumerate}

\includegraphics[width=4.08333in,height=3.25857in]{images/media/image95.png}

\begin{enumerate}
\def\labelenumi{\arabic{enumi}.}
\item
  The right atrioventricular valve is also known as the
  \_\_\_\_\_\_\_\_\_\_ \_\_\_\_\_\_\_\_ because it has three flaps.
\item
  The \-\-\-\-\_\_\_\_\_\_\_\_\_\_\_\_\_\_ semilunar valve is found
  between the right ventricle and pulmonary trunk.
\item
  Where is the mitral, or bicuspid, valve is located where in the heart?
\end{enumerate}

\includegraphics[width=5in,height=4.8125in,alt={The left panel of this figure shows the anterior view of the heart with the different valves, and the right panel of this figure shows the location of the mitral valve in the open position in the heart.}]{images/media/image96.jpeg}

\begin{enumerate}
\def\labelenumi{\arabic{enumi}.}
\setcounter{enumi}{3}
\item
  The bicuspid and tricuspid valves are open in the illustration above.
  That means blood can flow from the \_\_\_\_\_\_\_\_\_\_ into the
  \_\_\_\_\_\_\_\_\_\_\_\_\_\_\_.
\end{enumerate}

\includegraphics[width=5in,height=4.8125in,alt={The left panel of this figure shows the anterior view of the heart with the different valves, and the right panel of this figure shows the location of the mitral valve in the closed position in the heart.}]{images/media/image97.jpeg}

\begin{enumerate}
\def\labelenumi{\arabic{enumi}.}
\setcounter{enumi}{4}
\item
  The semilunar valves are open in the picture above which means that
  blood could flow from the \_\_\_\_\_\_\_\_\_\_\_\_\_ to the pulmonary
  trunk and aorta.
\end{enumerate}

\subsubsection{Coronary Circulation}\label{coronary-circulation}

\begin{enumerate}
\def\labelenumi{\arabic{enumi}.}
\item
  What are cardiomyocytes?
\item
  Why do cardiomyocytes require a reliable supply of oxygen and
  nutrients?
\item
  What circulation supplies cardiomyocytes with blood?
\end{enumerate}

\begin{enumerate}
\def\labelenumi{\Alph{enumi}.}
\item
  Coronary Arteries

  \begin{enumerate}
  \def\labelenumii{\arabic{enumii}.}
  \item
    What is the purpose of coronary arteries?
  \item
    Where are epicardial coronary arteries found?
  \item
    Where is the circumflex artery located?
  \item
    What does the name anterior interventricular artery tell you about
    its location?
  \item
    What is an anastomosis? When would an anastomosis be beneficial?
  \item
    What part of the heart is supplied with blood by the marginal
    arteries?
  \end{enumerate}
\end{enumerate}

\includegraphics[width=5.72917in,height=5.25in,alt={The top panel of this figure shows the anterior view of the heart while the bottom panel shows the posterior view of the heart. The different blood vessels are labeled.}]{images/media/image98.jpeg}

\begin{enumerate}
\def\labelenumi{\Alph{enumi}.}
\item
  Coronary Veins

  \begin{enumerate}
  \def\labelenumii{\arabic{enumii}.}
  \item
    What purpose do coronary veins serve?
  \item
    Blood from the great, middle, small, and anterior cardiac veins all
    return their deoxygenated blood to what chamber of the heart?
  \end{enumerate}
\end{enumerate}

\subsection{19.2. Cardiac Muscle and Electrical
Activity}\label{cardiac-muscle-and-electrical-activity}

\subsubsection{Background}\label{background-3}

\begin{enumerate}
\def\labelenumi{\arabic{enumi}.}
\item
  The ability of the heart to initiate an action potential at a fixed
  rate is known as \_\_\_\_\_\_\_\_\_\_\_\_\_\_\_\_\_.
\item
  The two types of cardiac muscle cells are the
  \_\_\_\_\_\_\_\_\_\_\_\_\_\_\_\_\_ \_\_\_\_\_\_\_\_\_\_\_\_
  \_\_\_\_\_\_\_\_\_\_ responsible for contraction, and the
  \_\_\_\_\_\_\_\_\_\_\_\_\_\_\_\_ \_\_\_\_\_\_\_\_\_\_\_\_\_\_\_\_\_
  \_\_\_\_\_\_\_\_ that form the conduction system of the heart.
\end{enumerate}

\subsubsection{Structure of Cardiac
Muscle}\label{structure-of-cardiac-muscle}

\begin{enumerate}
\def\labelenumi{\arabic{enumi}.}
\item
  Cardiac muscle cells are striated like skeletal muscle, but are
  shorter and branched. These branched cells are joined at junctions
  known as \_\_\_\_\_\_\_\_\_\_\_\_\_ \_\_\_\_\_\_\_\_\_\_ which help to
  synchronize the contraction of the muscle cells.
\end{enumerate}

\includegraphics[width=7in,height=6.97153in]{images/media/image99.png}

\subsubsection{Conduction System of the
Heart}\label{conduction-system-of-the-heart}

\begin{enumerate}
\def\labelenumi{\arabic{enumi}.}
\item
  The components of the conduction system of the heart includes the
  sinoatrial node, \_\_\_\_\_\_\_\_\_\_\_\_\_
  \_\_\_\_\_\_\_\_\_\_\_\_\_\_\_, the atrioventricular bundle, and the
  \_\_\_\_\_\_\_\_\_\_\_\_\_\_\_\_\_\_ \_\_\_\_\_\_\_\_\_\_\_.
\end{enumerate}

\includegraphics[width=5.20833in,height=3.47917in,alt={This image shows the anterior view of the frontal section of the heart with the major parts labeled.}]{images/media/image100.jpeg}

\begin{enumerate}
\def\labelenumi{\Alph{enumi}.}
\item
  Sinoatrial (SA) Node

  \begin{enumerate}
  \def\labelenumii{\arabic{enumii}.}
  \item
    Where is the sinoatrial node found within the heart?
  \item
    The sinoatrial node is the pacemaker of the heart, and as such it
    initiates the \_\_\_\_\_\_\_\_\_\_ \_\_\_\_\_\_\_\_\_\_\_ or the
    electrical pattern that leads to contraction of the heart.
  \end{enumerate}
\end{enumerate}

\includegraphics[width=4.04167in,height=4.0557in]{images/media/image101.png}

\begin{enumerate}
\def\labelenumi{\Alph{enumi}.}
\item
  Atrioventricular (AV) Node

  \begin{enumerate}
  \def\labelenumii{\arabic{enumii}.}
  \item
    What role does the atrioventricular septum play in the conduction
    pathway? In order words, what does it prevent?
  \item
    Where is the AV node located in the heart?
  \end{enumerate}
\item
  Atrioventricular Bundle (Bundle of His), Bundle Branches, and Purkinje
  Fibers

  \begin{enumerate}
  \def\labelenumii{\arabic{enumii}.}
  \item
    The atrioventricular bundle pass through which septum of the heart?
  \item
    What part of the heart do the different atrioventricular bundle
    branches supply? Which is larger, and why?
  \end{enumerate}
\item
  Membrane Potentials and Ion Movement in Cardiac Conductive Cells

  \begin{enumerate}
  \def\labelenumii{\arabic{enumii}.}
  \item
    Skeletal muscle cells and neurons have a stable resting membrane
    potential. How does this compare the membrane potential of cardiac
    conductive cells?
  \item
    What is spontaneous depolarization (prepotential depolarization).
    Why is it important in cardiac conductive cells?
  \end{enumerate}
\end{enumerate}

\includegraphics[width=4.21875in,height=2.02108in,alt={This graph shows the change in membrane potential as a function of time.}]{images/media/image102.jpeg}

\begin{enumerate}
\def\labelenumi{\Alph{enumi}.}
\setcounter{enumi}{3}
\item
  Membrane Potentials and Ion Movement in Cardiac Contractile Cells

  \begin{enumerate}
  \def\labelenumii{\arabic{enumii}.}
  \item
    What does the long refractory period in cardiac muscle cells allow
    time for?
  \item
    What ion movement allows for the plateau phase seen in cardiac
    contractile cells?
  \item
    How long is a cardiac contractile cell's action potential compared
    to skeletal muscle?
  \end{enumerate}
\end{enumerate}

\includegraphics[width=4.40228in,height=4.83333in,alt={The top panel of this figure shows millivolts as a function of time with the various stages labeled. The bottom left panel shows action potential and tension as a function of time for skeletal muscle, and the bottom right panel shows the action potential and tension as a function of time for cardiac muscle.}]{images/media/image103.jpeg}

\begin{enumerate}
\def\labelenumi{\Alph{enumi}.}
\setcounter{enumi}{4}
\item
  Calcium Ions

  \begin{enumerate}
  \def\labelenumii{\arabic{enumii}.}
  \item
    What are the two critical roles of calcium ions in cardiac muscle
    cells?
  \end{enumerate}
\item
  Comparative Rates of Conductive System Firing

  \begin{enumerate}
  \def\labelenumii{\arabic{enumii}.}
  \item
    Without external control does the SA node reach threshold at a
    faster or slow rate than the other conductive cells (AV node, AV
    bundle, and Purkinje Fibers)?
  \end{enumerate}
\end{enumerate}

\subsubsection{Electrocardiogram}\label{electrocardiogram}

\begin{enumerate}
\def\labelenumi{\arabic{enumi}.}
\item
  What is an electrocardiogram (ECG), and what is it used for?
\item
  What do the following ECG deflections represent in terms of electrical
  activity in the heart?

  \begin{enumerate}
  \def\labelenumii{\alph{enumii})}
  \item
    P Wave
  \item
    QRS Complex
  \item
    T Wave
  \end{enumerate}
\item
  What are segments and intervals in the ECG?

  \begin{enumerate}
  \def\labelenumii{\alph{enumii})}
  \item
    What is the difference between them?
  \end{enumerate}
\end{enumerate}

\includegraphics[width=4.47917in,height=2.92708in,alt={This figure shows a graph of millivolts over time and the heart cycles during an ECG.}]{images/media/image104.jpeg}

\begin{enumerate}
\def\labelenumi{\arabic{enumi}.}
\setcounter{enumi}{3}
\item
  What is a heart block? Why might these be dangerous?
\item
  What is an example of a condition that would require an artificial
  pacemaker?
\end{enumerate}

\subsubsection{Cardiac Muscle
Metabolism}\label{cardiac-muscle-metabolism}

\begin{enumerate}
\def\labelenumi{\arabic{enumi}.}
\item
  Under normal conditions is metabolism in cardiac myocytes aerobic or
  anaerobic?
\end{enumerate}

19.3. Cardiac Cycle

\paragraph{Background}\label{background-4}

\begin{enumerate}
\def\labelenumi{\arabic{enumi}.}
\item
  The cardiac cycle encompasses what period of time?
\item
  What is systole?
\item
  The period of time when the chambers fill with blood is known as
  \_\_\_\_\_\_\_\_\_\_\_\_.
\end{enumerate}

\includegraphics[width=3.9375in,height=4.28438in,alt={This pie chart shows the different phases of the cardiac cycle and details the atrial and ventricular stages.}]{images/media/image105.jpeg}

\paragraph{Pressure and Flow}\label{pressure-and-flow}

\begin{enumerate}
\def\labelenumi{\arabic{enumi}.}
\item
  Fluid move along their pressure gradient from \_\_\_\_\_\_\_\_\_\_
  pressure to \_\_\_\_\_\_\_\_\_\_\_ pressure.
\item
  During ventricular systole blood moves from the \_\_\_\_\_\_\_\_\_\_
  pressure in the ventricles to the \_\_\_\_\_\_\_\_ pressure in the
  pulmonary trunk and aorta.
\end{enumerate}

\paragraph{Phases of the Cardiac
Cycle}\label{phases-of-the-cardiac-cycle}

\begin{enumerate}
\def\labelenumi{\arabic{enumi}.}
\item
  At the beginning of the cardiac cycle when both the atria and the
  ventricles are in diastole what valves are open, and what valves are
  closed?

  \begin{enumerate}
  \def\labelenumii{\alph{enumii})}
  \item
    What does this tell you about the pressure in the chambers?
  \end{enumerate}
\end{enumerate}

\begin{enumerate}
\def\labelenumi{\Alph{enumi}.}
\item
  Atrial Systole and Diastole

  \begin{enumerate}
  \def\labelenumii{\arabic{enumii}.}
  \item
    During atrial systole blood flow from what chamber to what chamber
    in the heart?
  \item
    What valves are open and closed during atrial systole?
  \end{enumerate}
\item
  Ventricular Systole

  \begin{enumerate}
  \def\labelenumii{\arabic{enumii}.}
  \item
    What ECG event precedes ventricular systole?
  \item
    What is end diastolic volume (EDV) or preload?

    \begin{enumerate}
    \def\labelenumiii{\alph{enumiii})}
    \item
      What is a normal end diastolic volume for a standing adult?
    \end{enumerate}
  \item
    What is isovolumic contraction?

    \begin{enumerate}
    \def\labelenumiii{\alph{enumiii})}
    \item
      What is the difference between isovolumic contraction and
      ventricular ejection?
    \end{enumerate}
  \item
    After ventricular ejection there is still blood left in the heart
    (normally 50-60mL). This blood remains in the ventricle is the
    \_\_\_\_\_\_\_\_\_\_\_\_ \_\_\_\_\_\_\_\_\_\_ \_\_\_\_\_\_\_\_\_\_
    (\_\_\_\_).
  \end{enumerate}
\item
  Ventricular Diastole

  \begin{enumerate}
  \def\labelenumii{\arabic{enumii}.}
  \item
    What is the isovolumic ventricular relaxation phase, and why is no
    there no blood volume change in the ventricles?
  \item
    What set of valves open during late ventricular diastole? Why?
  \end{enumerate}
\end{enumerate}

\includegraphics[width=4.45833in,height=2.11771in,alt={This image shows the correlation between the cardiac cycle and the different stages in a electrocardiogram.}]{images/media/image106.jpeg}

\paragraph{Heart Sounds}\label{heart-sounds}

\begin{enumerate}
\def\labelenumi{\arabic{enumi}.}
\item
  What causes the following heart sounds?

  \begin{enumerate}
  \def\labelenumii{\alph{enumii})}
  \item
    S\textsubscript{1} (lub) -
  \item
    S\textsubscript{2} (dub) --
  \end{enumerate}
\end{enumerate}

\includegraphics[width=3.80208in,height=2.87533in]{images/media/image107.png}

\begin{enumerate}
\def\labelenumi{\arabic{enumi}.}
\setcounter{enumi}{1}
\item
  Upon auscultation with a stethoscope a physician hears a murmur, what
  does that mean?
\item
  In order to hear the mitral valve clearly would it be best to place
  the bell of the stethoscope near the base or the apex of the heart?
\end{enumerate}

\includegraphics[width=3.22917in,height=2.57693in]{images/media/image108.png}

\subsection{19.4. Cardiac Physiology}\label{cardiac-physiology}

\subsubsection{Background}\label{background-5}

\begin{enumerate}
\def\labelenumi{\arabic{enumi}.}
\item
  What body systems help to regulate cardiac function?
\end{enumerate}

\subsubsection{Resting Cardiac Output}\label{resting-cardiac-output}

\begin{enumerate}
\def\labelenumi{\arabic{enumi}.}
\item
  What is cardiac output, and how is it calculated?
\item
  The amount of blood pumped by each ventricle per beat is the
  \_\_\_\_\_\_\_\_\_\_ \_\_\_\_\_\_\_\_\_\_.
\item
  What is a normal heart rate for an individual at rest?
\end{enumerate}

\includegraphics[width=4.28125in,height=2.50674in]{images/media/image109.png}

\begin{enumerate}
\def\labelenumi{\arabic{enumi}.}
\setcounter{enumi}{3}
\item
  During each heart beat only a portion of the blood in the ventricle is
  ejected. This is known as the \_\_\_\_\_\_\_\_\_\_\_\_
  \_\_\_\_\_\_\_\_\_\_\_\_ and can be calculated by dividing the stroke
  volume by the end diastolic volume.
\end{enumerate}

\subsubsection{Exercise and Maximum Cardiac
Output}\label{exercise-and-maximum-cardiac-output}

\begin{enumerate}
\def\labelenumi{\arabic{enumi}.}
\item
  What happens to heart rate and stroke volume during exercise?
\item
  The difference between resting and maximal cardiac output (CO) is
  known as \_\_\_\_\_\_\_\_\_\_\_\_ \_\_\_\_\_\_\_\_\_\_\_\_\_\_
\end{enumerate}

\subsubsection{Heart Rates}\label{heart-rates}

\begin{enumerate}
\def\labelenumi{\arabic{enumi}.}
\item
  What is the trend of maximal heart rate as we age?
\end{enumerate}

\subsubsection{Correlation between Heart Rates and Cardiac
Output}\label{correlation-between-heart-rates-and-cardiac-output}

\begin{enumerate}
\def\labelenumi{\arabic{enumi}.}
\item
  At very high heart rates why does cardiac output decrease?
\item
  The range in which both the heart and lungs receive the maximal
  benefit from aerobic exercise is the \_\_\_\_\_\_\_\_\_\_\_\_
  \_\_\_\_\_\_\_\_\_\_\_ \_\_\_\_\_\_\_\_\_\_\_\_.
\end{enumerate}

\subsubsection[Cardiovascular
Centers]{\texorpdfstring{\protect\includegraphics[width=2.30208in,height=4.29167in,alt={This figure shows the brain and the nerves connecting the brain to the heart.}]{images/media/image110.jpeg}Cardiovascular
Centers}{This figure shows the brain and the nerves connecting the brain to the heart.Cardiovascular Centers}}\label{this-figure-shows-the-brain-and-the-nerves-connecting-the-brain-to-the-heart.cardiovascular-centers}

\begin{enumerate}
\def\labelenumi{\arabic{enumi}.}
\item
  What is autonomic tone?
\item
  The network of nerve fibers found at the base of the heart is the
  \_\_\_\_\_\_\_\_\_\_\_\_ \_\_\_\_\_\_\_\_\_\_\_\_ and it contains both
  sympathetic and parasympathetic nerve fibers.
\item
  The sympathetic nervous system has what effect on heart rate?
\item
  Adult resting heart rate is normally \textless100 bpm, but without any
  nervous system stimulation the SA node will fire at 100 bpm. Is the
  sympathetic or parasympathetic nervous system acting on the SA node at
  rest to drive heart rate down?
\end{enumerate}

\includegraphics[width=4.28714in,height=6.45833in]{images/media/image111.png}

\subsubsection{Input to the Cardiovascular
Center}\label{input-to-the-cardiovascular-center}

\begin{enumerate}
\def\labelenumi{\arabic{enumi}.}
\item
  What is a cardiac reflex?
\item
  In the baroreceptor reflex, if increased stretch and pressure are
  detected how will the cardiac center respond?
\item
  According to the atrial or Bainbridge reflex if increased venous
  return stretches the atria what will reflexively happen to heart rate?
\item
  The limbic system and therefore emotional state can also influence
  heart rate. What effect would a stressful situation have on heart
  rate?
\end{enumerate}

\subsubsection{Other Factors Influencing Heart
Rate}\label{other-factors-influencing-heart-rate}

\begin{enumerate}
\def\labelenumi{\arabic{enumi}.}
\item
  Beyond autorhythmicity and innervation what are some other factors
  impact heart rate?
\end{enumerate}

\begin{enumerate}
\def\labelenumi{\Alph{enumi}.}
\item
  Epinephrine and Norepinephrine

  \begin{enumerate}
  \def\labelenumii{\arabic{enumii}.}
  \item
    What hormones are released from the adrenal medulla?

    \begin{enumerate}
    \def\labelenumiii{\alph{enumiii})}
    \item
      What effect do they have on heart rate?
    \end{enumerate}
  \end{enumerate}
\end{enumerate}

\begin{enumerate}
\def\labelenumi{\Alph{enumi}.}
\item
  Thyroid Hormones

  \begin{enumerate}
  \def\labelenumii{\arabic{enumii}.}
  \item
    An increase in thyroid hormones would have what effect on heart
    rate?
  \end{enumerate}
\item
  Calcium

  \begin{enumerate}
  \def\labelenumii{\arabic{enumii}.}
  \item
    Calcium impacts both heart rate and contractility. What change would
    you expect from someone taking a calcium channel blocker?
  \end{enumerate}
\item
  Caffeine and Nicotine

  \begin{enumerate}
  \def\labelenumii{\arabic{enumii}.}
  \item
    Both caffeine and nicotine have a stimulatory effect on cardiac
    centers. What effect do you expect these substances to have on heart
    rate?
  \end{enumerate}
\item
  Factors Decreasing Heart Rate

  \begin{enumerate}
  \def\labelenumii{\arabic{enumii}.}
  \item
    Electrolyte balance is critical to heart function. Which of the
    following ions is of greater clinical significance to heart function
    sodium or potassium?
  \item
    What effect does hypothermia have on heart rate?
  \item
    How might hypothermia be used in a surgical setting?
  \end{enumerate}
\end{enumerate}

\subsubsection{Stroke Volume}\label{stroke-volume}

\begin{enumerate}
\def\labelenumi{\arabic{enumi}.}
\item
  How is stroke volume calculated?
\end{enumerate}

\begin{enumerate}
\def\labelenumi{\Alph{enumi}.}
\item
  Preload

  \begin{enumerate}
  \def\labelenumii{\arabic{enumii}.}
  \item
    Increased filling time would have what effect on stroke volume?
  \item
    What is the Frank-Starling mechanism? How is it related to preload?
  \end{enumerate}
\end{enumerate}

\begin{enumerate}
\def\labelenumi{\Alph{enumi}.}
\item
  Contractility

  \begin{enumerate}
  \def\labelenumii{\arabic{enumii}.}
  \item
    What is contractility? Why is it important to cardiac function?
  \item
    As contractions becomes more forceful what happens to stroke volume
    and therefore end systolic volume (ESV)?
  \item
    Is sympathetic stimulation a positive or negative inotrope?
  \end{enumerate}
\item
  Afterload

  \begin{enumerate}
  \def\labelenumii{\arabic{enumii}.}
  \item
    What is afterload?
  \item
    As afterload increases what would you expect to happen to stroke
    volume?
  \end{enumerate}
\end{enumerate}

\includegraphics[width=5.38542in,height=3.06937in]{images/media/image112.png}

19.5. Development of the Heart

\paragraph{Background}\label{background-6}

\begin{enumerate}
\def\labelenumi{\arabic{enumi}.}
\item
  When does the heart begin to beat in development?
\item
  The anterior surface of the embryo features a prominent protrusion
  where the heart is developing. This is known as the
  \_\_\_\_\_\_\_\_\_\_ \_\_\_\_\_\_\_\_\_\_\_\_\_.
\item
  The heart forms from what germ layer?
\item
  As the cardiogenic cords develop a lumen forms within them and they
  are known as \_\_\_\_\_\_\_\_\_\_\_\_\_\_\_ \_\_\_\_\_\_\_\_\_\_.
  These then fuse to become the primitive heart tube.
\end{enumerate}

\begin{quote}
\includegraphics[width=5.46875in,height=4.71354in]{images/media/image113.png}
\end{quote}

\begin{enumerate}
\def\labelenumi{\arabic{enumi}.}
\setcounter{enumi}{4}
\item
  The \_\_\_\_\_\_\_\_\_\_\_\_ \_\_\_\_\_\_\_\_\_ portion of the heart
  tube develops into the aorta and pulmonary trunk.
\item
  The primitive ventricle develops to become the \_\_\_\_\_\_\_\_
  ventricle.
\item
  The primitive atrium becomes what part of the right and left atria?
\item
  The SA node, a critical part of the conductions system, develops from
  the \_\_\_\_\_\_\_\_\_\_\_\_ \_\_\_\_\_\_\_\_\_\_\_\_\_.
\item
  Which valves develop first, the atrioventricular valves or the
  semilunar valves?
\end{enumerate}

\section{}\label{section-19}

\section{\texorpdfstring{Chapter 20 }{Chapter 20 }}\label{chapter-20}

\subsection{20.1. Structure and Function of Blood
Vessels}\label{structure-and-function-of-blood-vessels}

\subsubsection{Background}\label{background-7}

\begin{enumerate}
\def\labelenumi{\arabic{enumi}.}
\item
  Arteries carry blood \_\_\_\_\_\_\_\_\_\_\_ \_\_\_\_\_\_\_\_\_\_ the
  heart
\item
  Capillaries are where nutrients and wastes are
  \_\_\_\_\_\_\_\_\_\_\_\_\_\_\_.
\item
  Veins \_\_\_\_\_\_\_\_\_\_\_\_ blood to the heart.
\item
  Blood is transported in two circuits, the \_\_\_\_\_\_\_\_\_\_ circuit
  and the \_\_\_\_\_\_\_\_\_\_\_\_\_\_\_ circuit.
\end{enumerate}

\includegraphics[width=5.41667in,height=3.66667in,alt={This diagram shows how oxygenated and deoxygenated blood flow through the major organs in the body.}]{images/media/image114.jpeg}

\subsubsection{Shared Structures}\label{shared-structures}

\begin{enumerate}
\def\labelenumi{\arabic{enumi}.}
\item
  Due to the higher pressure in arteries do they have thicker or thinner
  walls than veins?
\item
  The hollow passageway that blood flows through in arteries,
  capillaries, and veins is the \_\_\_\_\_\_\_\_\_\_.
\end{enumerate}

\includegraphics[width=4.53215in,height=6.14583in]{images/media/image115.png}

\begin{enumerate}
\def\labelenumi{\arabic{enumi}.}
\setcounter{enumi}{2}
\item
  Valves contain \_\_\_\_\_\_\_\_\_\_\_\_\_ that provide unidirectional
  flow of blood back to the heart.
\item
  Large arteries and veins need a blood supply separate from what passes
  through their lumen, and this blood supply the ``vessels of the
  vessel'' is the \_\_\_\_\_\_\_\_\_\_\_ \_\_\_\_\_\_\_\_\_\_\_\_\_\_.
\item
  What are the three layers, or tunics, of arteries and veins from the
  lumen out?

  \begin{enumerate}
  \def\labelenumii{\alph{enumii})}
  \item
    Tunica \_\_\_\_\_\_\_\_\_\_\_\_\_
  \item
    Tunica \_\_\_\_\_\_\_\_\_\_\_\_\_
  \item
    Tunica \_\_\_\_\_\_\_\_\_\_\_\_\_
  \end{enumerate}
\end{enumerate}

\begin{enumerate}
\def\labelenumi{\Alph{enumi}.}
\item
  Tunica Intima

  \begin{enumerate}
  \def\labelenumii{\arabic{enumii}.}
  \item
    What composes the tunica intima?
  \item
    What type of arteries have a thick internal elastic membrane?

    \begin{enumerate}
    \def\labelenumiii{\alph{enumiii})}
    \item
      What purpose do the elastic fibers serve?
    \end{enumerate}
  \end{enumerate}
\item
  Tunica Media

  \begin{enumerate}
  \def\labelenumii{\arabic{enumii}.}
  \item
    What kind of muscle is found in the tunica media?
  \item
    Vasoconstriction has what effect on blood flow?
  \item
    The \_\_\_\_\_\_\_\_\_\_\_\_ \_\_\_\_\_\_\_\_\_\_\_\_\_, or ``nerves
    of the vessel'' help to regulate vasoconstriction and vasodilation
    in the vessels.
  \item
    The external elastic membrane separates the tunica media from what?
  \end{enumerate}
\item
  Tunica Externa

  \begin{enumerate}
  \def\labelenumii{\arabic{enumii}.}
  \item
    Why is it important for the tunica externa to hold the vessel in
    place?
  \end{enumerate}
\end{enumerate}

\subsubsection{Arteries}\label{arteries}

\begin{enumerate}
\def\labelenumi{\arabic{enumi}.}
\item
  Compared to veins, are arteries exposed to higher or lower pressures?
\item
  The aorta, the largest artery in the body is an
  \_\_\_\_\_\_\_\_\_\_\_\_\_ \_\_\_\_\_\_\_\_\_\_\_\_\_.
\end{enumerate}

\includegraphics[width=5.72917in,height=1.45833in,alt={The left panel shows the cross-section of an elastic artery, the middle panel shows the cross section of a muscular artery, and the right panel shows the cross-section of an arteriole.}]{images/media/image116.jpeg}

\begin{enumerate}
\def\labelenumi{\arabic{enumi}.}
\setcounter{enumi}{2}
\item
  Beyond elastic arteries are \_\_\_\_\_\_\_\_\_\_\_\_
  \_\_\_\_\_\_\_\_\_\_\_\_\_ which have a thick tunica media. These are
  also known as distributing arteries.
\end{enumerate}

\subsubsection{Arterioles}\label{arterioles}

\begin{enumerate}
\def\labelenumi{\arabic{enumi}.}
\item
  Arterioles lead to \_\_\_\_\_\_\_\_\_\_\_\_\_\_\_\_\_.
\item
  Arterioles are both the primary site for \_\_\_\_\_\_\_\_\_\_\_\_\_\_
  and regulation of \_\_\_\_\_\_\_\_\_\_\_\_\_
  \_\_\_\_\_\_\_\_\_\_\_\_\_\_\_\_.
\end{enumerate}

\subsubsection{Capillaries}\label{capillaries}

\begin{enumerate}
\def\labelenumi{\arabic{enumi}.}
\item
  What is perfusion?
\item
  What do capillaries exchange with the tissue?
\item
  \_\_\_\_\_\_\_\_\_\_\_\_\_\_\_ is used to describe flow through the
  capillaries.
\end{enumerate}

\begin{enumerate}
\def\labelenumi{\Alph{enumi}.}
\item
  Continuous Capillaries

  \begin{enumerate}
  \def\labelenumii{\arabic{enumii}.}
  \item
    How common are continuous capillaries in the body?
  \item
    What type of junctions are found between the endothelial cells of
    continuous capillaries?

    \begin{enumerate}
    \def\labelenumiii{\alph{enumiii})}
    \item
      Do the junctions here mean these capillaries do not participate in
      exchange?
    \end{enumerate}
  \end{enumerate}
\end{enumerate}

\includegraphics[width=5.20833in,height=1.95833in,alt={The left panel shows the structure of a continuous capillary, the middle panel shows a fenestrated capillary, and the right panel shows a sinusoid capillary.}]{images/media/image117.jpeg}

\begin{enumerate}
\def\labelenumi{\Alph{enumi}.}
\setcounter{enumi}{1}
\item
  Fenestrated Capillaries

  \begin{enumerate}
  \def\labelenumii{\arabic{enumii}.}
  \item
    What is a fenestration?

    \begin{enumerate}
    \def\labelenumiii{\alph{enumiii})}
    \item
      Why are these beneficial to have in capillaries in the small
      intestine?
    \end{enumerate}
  \end{enumerate}
\item
  Sinusoid Capillaries

  \begin{enumerate}
  \def\labelenumii{\arabic{enumii}.}
  \item
    Why are sinusoid capillaries necessary in bone marrow?
  \end{enumerate}
\end{enumerate}

\subsubsection{Metarterioles and Capillary
Beds}\label{metarterioles-and-capillary-beds}

\begin{enumerate}
\def\labelenumi{\arabic{enumi}.}
\item
  A metarteriole arises from a \_\_\_\_\_\_\_\_\_\_\_\_\_
  \_\_\_\_\_\_\_\_\_\_\_\_\_ and supplies a
  \_\_\_\_\_\_\_\_\_\_\_\_\_\_\_\_\_\_\_.
\item
  What purpose do the precapillary sphincters serve in the metarteriole?
\item
  When precapillary sphincters are closed blood will pass through a
  \_\_\_\_\_\_\_\_\_\_\_\_\_\_ \_\_\_\_\_\_\_\_\_\_\_\_\_ to bypass the
  capillary bed. This is known as a vascular shunt.
\item
  The irregular pulsating flow of blood through a capillary bed is
  \_\_\_\_\_\_\_\_\_\_\_\_\_.
\end{enumerate}

\includegraphics[width=5in,height=2.94792in,alt={This diagram shows a capillary bed connecting an arteriole and a venule.}]{images/media/image118.jpeg}

\subsubsection{Venules}\label{venules}

\begin{enumerate}
\def\labelenumi{\arabic{enumi}.}
\item
  \includegraphics[width=2.78125in,height=3.76538in,alt={The top panel shows the cross-section of a large vein, the middle panel shows the cross-section of a medium sized vein, and the bottom panel shows the cross-section of a venule.}]{images/media/image119.jpeg}After
  a capillary, small veins known as \_\_\_\_\_\_\_\_\_\_\_\_ carry blood
  towards larger veins.
\end{enumerate}

\subsubsection{Veins}\label{veins}

\begin{enumerate}
\def\labelenumi{\arabic{enumi}.}
\item
  Veins conduct blood \_\_\_\_\_\_\_\_\_\_\_ \_\_\_\_\_\_\_\_\_\_ the
  heart.
\item
  The pressure of blood in veins is \_\_\_\_\_\_\_\_\_\_ compared to
  arteries.
\end{enumerate}

\subsubsection{Veins as Blood
Reservoirs}\label{veins-as-blood-reservoirs}

\begin{enumerate}
\def\labelenumi{\arabic{enumi}.}
\item
  Veins hold a significant portion of the body's blood volume at any
  given time. This makes veins high \_\_\_\_\_\_\_\_\_\_\_\_\_\_ vessels
  due to their ability to distend.
\item
  Venous reserve is blood located within the venous networks and within
  what organs?

  \begin{enumerate}
  \def\labelenumii{\alph{enumii})}
  \item
    How can this reserve volume be utilized by the body?
  \end{enumerate}
\end{enumerate}

20.2. Blood Flow, Blood Pressure, and Resistance

\paragraph{Background}\label{background-8}

\begin{enumerate}
\def\labelenumi{\arabic{enumi}.}
\item
  What is blood flow, and what are the units?
\item
  Blood flows from \_\_\_\_\_\_\_\_ pressure to \_\_\_\_\_\_\_\_
  pressure.
\item
  \_\_\_\_\_\_\_\_\_\_\_\_\_ is the factor that impedes blood flow.
\item
  The term blood pressure without a specific designator indicates
  pressure measured in the \_\_\_\_\_\_\_\_\_\_\_\_\_\_.
\end{enumerate}

\paragraph{Components or Arterial Blood
Pressure}\label{components-or-arterial-blood-pressure}

\begin{enumerate}
\def\labelenumi{\Alph{enumi}.}
\item
  Systolic and Diastolic Pressures

  \begin{enumerate}
  \def\labelenumii{\arabic{enumii}.}
  \item
    What is systolic pressure? What is a normal value for systolic
    pressure?
  \item
    What is diastolic pressure? What is occurring in the heart during
    diastole?
  \end{enumerate}
\end{enumerate}

\includegraphics[width=4.625in,height=3.219in,alt={This graph shows the value of pulse pressure in different types of blood vessels.}]{images/media/image120.jpeg}

\begin{enumerate}
\def\labelenumi{\Alph{enumi}.}
\setcounter{enumi}{1}
\item
  Pulse Pressure

  \begin{enumerate}
  \def\labelenumii{\arabic{enumii}.}
  \item
    The difference between systolic pressure and diastolic pressure is
    \_\_\_\_\_\_\_\_\_\_\_\_ \_\_\_\_\_\_\_\_\_\_\_\_.
  \item
    Why would a high (100 mm Hg) pulse pressure not be conducive to good
    health?
  \end{enumerate}
\item
  Mean Arterial Pressure

  \begin{enumerate}
  \def\labelenumii{\arabic{enumii}.}
  \item
    What is Mean arterial pressure (MAP)?
  \item
    How is MAP calculated?
  \item
    What happens when MAP is too low? How could this damage tissue?
  \end{enumerate}
\end{enumerate}

\paragraph[Pulse]{\texorpdfstring{\protect\includegraphics[width=2.34375in,height=3.59583in,alt={This image shows the pulse points in a woman's body.}]{images/media/image121.jpeg}Pulse}{This image shows the pulse points in a woman's body.Pulse}}\label{this-image-shows-the-pulse-points-in-a-womans-body.pulse}

\begin{enumerate}
\def\labelenumi{\arabic{enumi}.}
\item
  What is pulse, and what is responsible for causing a pulse?
\item
  Measurement of pulse is measurement of heart \_\_\_\_\_\_\_.
\end{enumerate}

\paragraph{Measurement of Blood
Pressure}\label{measurement-of-blood-pressure}

\begin{enumerate}
\def\labelenumi{\arabic{enumi}.}
\item
  What causes Korotkoff sounds?
\item
  How can we use Korotkoff sounds and a sphygmomanometer to measure
  blood pressure?
\item
  What is the first Korotkoff sound heard during a blood pressure
  measurement?
\item
  What is the point at which the last sound is heart? What does that
  indicate?
\end{enumerate}

\includegraphics[width=5.52083in,height=3.44066in]{images/media/image122.png}

\paragraph{Variables Affecting Blood Flow and Blood
Pressure}\label{variables-affecting-blood-flow-and-blood-pressure}

\begin{enumerate}
\def\labelenumi{\arabic{enumi}.}
\item
  What are the five variables that affect blood flow and blood pressure?
\end{enumerate}

\begin{enumerate}
\def\labelenumi{\Alph{enumi}.}
\item
  Cardiac Output

  \begin{enumerate}
  \def\labelenumii{\arabic{enumii}.}
  \item
    An increase in cardiac output will have what impact on blood
    pressure and blood flow?
  \end{enumerate}
\end{enumerate}

\begin{enumerate}
\def\labelenumi{\Alph{enumi}.}
\item
  Compliance

  \begin{enumerate}
  \def\labelenumii{\arabic{enumii}.}
  \item
    What is compliance?
  \item
    Are veins or arteries more compliant?
  \end{enumerate}
\item
  A Mathematical Approach to Factors Affecting Blood Flow

  \begin{enumerate}
  \def\labelenumii{\arabic{enumii}.}
  \item
    Which factor in Poiseuille's equation will have the greatest effect
    on blood pressure and blood flow with the smallest change? Why?
  \end{enumerate}
\item
  Blood Volume

  \begin{enumerate}
  \def\labelenumii{\arabic{enumii}.}
  \item
    An increase in blood volume will have what effect on blood pressure
    and blood flow?
  \item
    What is hypovolemia? Why might it occur?
  \item
    What could cause hypervolemia?
  \end{enumerate}
\item
  Blood Viscosity

  \begin{enumerate}
  \def\labelenumii{\arabic{enumii}.}
  \item
    What is viscosity and why is it important to blood pressure and
    blood flow?
  \end{enumerate}
\item
  Vessel Length and Diameter

  \begin{enumerate}
  \def\labelenumii{\arabic{enumii}.}
  \item
    Would an increase in vessel length increase or decrease resistance?
  \item
    How does weight gain change vessel length in the body?
  \item
    How can blood vessels change in diameter?
  \item
    What would happen to blood flow if vascular tone is reduced and the
    vessel diameter increases?
  \end{enumerate}
\item
  The Roles of Vessel Diameter and Total Area in Blood Flow and Blood
  Pressure

  \begin{enumerate}
  \def\labelenumii{\arabic{enumii}.}
  \item
    Where in the body is the velocity of blood flow the fastest, and
    where is it the slowest?

    \begin{enumerate}
    \def\labelenumiii{\alph{enumiii})}
    \item
      How does this relate to cross-sectional area?
    \end{enumerate}
  \end{enumerate}
\end{enumerate}

\includegraphics[width=4.8125in,height=3.30397in,alt={This figure shows four graphs. The top left graph shows the vessel diameter for different types of blood vessels. The top right panel shows cross-sectional area for different blood vessels. The bottom left panel shows the average blood pressure for different blood vessels, and the bottom right panel shows the velocity of blood flow in different blood vessels.}]{images/media/image123.jpeg}

\paragraph{Venous System}\label{venous-system}

\begin{enumerate}
\def\labelenumi{\arabic{enumi}.}
\item
  What factors help to maintain the pressure gradient between the veins
  and the heart?
\end{enumerate}

\begin{enumerate}
\def\labelenumi{\Alph{enumi}.}
\item
  Skeletal Muscle Pump

  \begin{enumerate}
  \def\labelenumii{\arabic{enumii}.}
  \item
    When contracting skeletal muscles how does the skeletal muscle pump
    help to return blood to the heart?
  \end{enumerate}
\end{enumerate}

\includegraphics[width=3.05208in,height=2.66936in]{images/media/image124.png}

\begin{enumerate}
\def\labelenumi{\Alph{enumi}.}
\item
  Respiratory Pump

  \begin{enumerate}
  \def\labelenumii{\arabic{enumii}.}
  \item
    What pressure changes occur during inhalation to help move blood
    into the thorax?
  \end{enumerate}
\item
  Pressure Relationship in the Venous System

  \begin{enumerate}
  \def\labelenumii{\arabic{enumii}.}
  \item
    Why does the cross-sectional area decrease as blood moves from
    venules to veins?

    \begin{enumerate}
    \def\labelenumiii{\alph{enumiii})}
    \item
      What does this mean in terms of blood velocity through the veins
      as compared to the venules?
    \end{enumerate}
  \end{enumerate}
\item
  The Role of Venoconstriction in Resistance, Blood Pressure, and Flow

  \begin{enumerate}
  \def\labelenumii{\arabic{enumii}.}
  \item
    How is the outcome of venoconstriction different than the outcome of
    vasoconstriction?
  \end{enumerate}
\end{enumerate}

\subsection{20.3. Capillary Exchange}\label{capillary-exchange}

\subsubsection{Background}\label{background-9}

\begin{enumerate}
\def\labelenumi{\arabic{enumi}.}
\item
  What is the primary purpose of the cardiovascular system?

  \begin{enumerate}
  \def\labelenumii{\alph{enumii})}
  \item
    What is the role of the capillaries in this purpose?
  \end{enumerate}
\end{enumerate}

\subsubsection{Bulk Flow}\label{bulk-flow}

\begin{enumerate}
\def\labelenumi{\arabic{enumi}.}
\item
  What are the two pressure driven mechanisms involved in bulk flow?
\item
  \_\_\_\_\_\_\_\_\_\_\_\_\_ is movement of fluid from the capillaries
  to the tissue bed, and \_\_\_\_\_\_\_\_\_\_\_\_\_ is movement from the
  tissue bed to the capillaries.
\end{enumerate}

\begin{enumerate}
\def\labelenumi{\Alph{enumi}.}
\item
  Hydrostatic Pressure

  \begin{enumerate}
  \def\labelenumii{\arabic{enumii}.}
  \item
    The force exerted by the blood confined within blood vessels or
    heart chambers is \_\_\_\_\_\_\_\_\_\_\_\_
    \_\_\_\_\_\_\_\_\_\_\_\_\_\_\_\_\_\_\_\_
    \_\_\_\_\_\_\_\_\_\_\_\_\_\_.
  \item
    Pressure exerted against the walls of capillaries is
    \_\_\_\_\_\_\_\_\_\_\_\_\_\_\_ \_\_\_\_\_\_\_\_\_\_\_\_\_\_\_\_
    \_\_\_\_\_\_\_\_\_\_\_\_\_\_\_\_\_\_.
  \item
    Interstitial fluid hydrostatic pressure (IFHP) is the pressure in
    the tissue and it opposes capillary hydrostatic pressure. In order
    for fluid to filter out of the capillaries which pressure is
    normally higher CHP, or IFHP?
  \end{enumerate}
\item
  Osmotic Pressure

  \begin{enumerate}
  \def\labelenumii{\arabic{enumii}.}
  \item
    Osmotic pressure is the pressure that drives
    \_\_\_\_\_\_\_\_\_\_\_\_\_\_\_\_.
  \item
    The proteins in the blood are exert pressure known as
    \_\_\_\_\_\_\_\_\_\_\_ \_\_\_\_\_\_\_\_\_\_\_\_\_\_\_
    \_\_\_\_\_\_\_\_\_\_\_\_\_\_\_ \_\_\_\_\_\_\_\_\_\_\_\_ (BCOP).
  \item
    BCOP draws water into the capillary because it is higher than
    \_\_\_\_\_\_\_\_\_\_\_\_\_\_\_ \_\_\_\_\_\_\_\_\_\_\_\_\_\_
    \_\_\_\_\_\_\_\_\_\_\_\_ \_\_\_\_\_\_\_\_\_\_\_\_\_\_\_\_
    \_\_\_\_\_\_\_\_\_\_\_ (ICOP).
  \end{enumerate}
\item
  Interaction of Hydrostatic and Osmotic Pressures

  \begin{enumerate}
  \def\labelenumii{\arabic{enumii}.}
  \item
    What happens to capillary hydrostatic pressure (CHP) at the
    beginning of the capillary compared to the end of the capillary?
  \item
    What is net filtration pressure (NFP)?
  \item
    Compare the CHP at the beginning and end of the capillary to the
    NFP.
  \end{enumerate}
\end{enumerate}

\includegraphics[width=5.41667in,height=2.52083in,alt={This diagram shows the process of fluid exchange in a capillary from the arterial end to the venous end.}]{images/media/image125.jpeg}

\subsubsection{The Role of Lymphatic
Capillaries}\label{the-role-of-lymphatic-capillaries}

\begin{enumerate}
\def\labelenumi{\arabic{enumi}.}
\item
  Why is the amount of fluid filtered at the capillaries not equal to
  the amount reabsorbed? Where does the difference in fluid go?
\item
  Does the fluid filtered into the lymph return to the blood? If so,
  where?
\end{enumerate}

20.4. Homeostatic Regulation of the Vascular System

\paragraph{Background}\label{background-10}

\begin{enumerate}
\def\labelenumi{\arabic{enumi}.}
\item
  Is there enough blood flow to distribute blood equally to all tissues?
\item
  What happens to blood flow during exercise?
\end{enumerate}

\paragraph{Neural Regulation}\label{neural-regulation}

\begin{enumerate}
\def\labelenumi{\arabic{enumi}.}
\item
  What is the primary site in the brain that regulate vascular
  homeostasis?
\end{enumerate}

\begin{enumerate}
\def\labelenumi{\Alph{enumi}.}
\item
  The Cardiovascular Centers in the Brain

  \begin{enumerate}
  \def\labelenumii{\arabic{enumii}.}
  \item
    What are the three distinct parts of the cardiovascular control
    center?

    \begin{enumerate}
    \def\labelenumiii{\alph{enumiii})}
    \item
      The \_\_\_\_\_\_\_\_\_\_\_\_\_\_\_ \_\_\_\_\_\_\_\_\_\_\_\_\_ that
      regulate heart rate and stroke volume
    \item
      The \_\_\_\_\_\_\_\_\_\_\_\_\_\_\_ \_\_\_\_\_\_\_\_\_\_\_\_\_ that
      decrease heart rate and stroke volume
    \item
      The \_\_\_\_\_\_\_\_\_\_\_\_\_\_\_ \_\_\_\_\_\_\_\_\_\_\_\_\_ that
      control diameter of the vessels.
    \end{enumerate}
  \end{enumerate}
\end{enumerate}

\begin{enumerate}
\def\labelenumi{\Alph{enumi}.}
\item
  Baroreceptor Reflexes

  \begin{enumerate}
  \def\labelenumii{\arabic{enumii}.}
  \item
    What is a baroreceptor?

    \begin{enumerate}
    \def\labelenumiii{\alph{enumiii})}
    \item
      How does it measure pressure?
    \end{enumerate}
  \item
    In response to low blood pressure what happens to the firing of the
    baroreceptors?

    \begin{enumerate}
    \def\labelenumiii{\alph{enumiii})}
    \item
      What is the body's response to increase pressure back to normal?
    \end{enumerate}
  \end{enumerate}
\end{enumerate}

\includegraphics[width=5.52083in,height=4.10417in,alt={This flow chart shows what happens when blood pressure is increased or decreased. The top panel shows the events that take place when blood pressure is increased, and the bottom panel shows the events that take place when blood pressure is decreased.}]{images/media/image126.jpeg}

\begin{enumerate}
\def\labelenumi{\arabic{enumi}.}
\setcounter{enumi}{2}
\item
  What is the atrial reflex and how does it help to maintain appropriate
  cardiac output?
\end{enumerate}

\begin{enumerate}
\def\labelenumi{\Alph{enumi}.}
\setcounter{enumi}{1}
\item
  Chemoreceptor Reflexes

  \begin{enumerate}
  \def\labelenumii{\arabic{enumii}.}
  \item
    What do chemoreceptors measure?
  \item
    Where are the chemoreceptors located in the body?
  \item
    When carbon dioxide and hydrogen ion levels increase in the blood
    how does the body respond?
  \end{enumerate}
\end{enumerate}

\paragraph{Endocrine}\label{endocrine}

\begin{enumerate}
\def\labelenumi{\arabic{enumi}.}
\item
  What hormones help to regulate the vascular system?
\end{enumerate}

\begin{enumerate}
\def\labelenumi{\Alph{enumi}.}
\item
  Epinephrine and Norepinephrine

  \begin{enumerate}
  \def\labelenumii{\arabic{enumii}.}
  \item
    What releases epinephrine and norepinephrine?
  \item
    Are these hormones part of the sympathetic or parasympathetic
    response?
  \item
    What effect do these hormones have on heart rate and force of
    contraction?
  \end{enumerate}
\end{enumerate}

\begin{enumerate}
\def\labelenumi{\Alph{enumi}.}
\item
  Antidiuretic Hormone

  \begin{enumerate}
  \def\labelenumii{\arabic{enumii}.}
  \item
    What triggers the release of Antidiuretic Hormone/Vasopressin?
  \item
    How does this hormone modify blood volume and blood pressure?
  \end{enumerate}
\item
  Renin-Angiotensin-Aldosterone Mechanism

  \begin{enumerate}
  \def\labelenumii{\arabic{enumii}.}
  \item
    Angiotensin II is a potent \_\_\_\_\_\_\_\_\_\_\_\_\_\_\_\_\_\_\_\_.
  \item
    How does aldosterone raise blood pressure?
  \end{enumerate}
\end{enumerate}

\includegraphics[width=5.83333in,height=2.52083in,alt={This flow chart shows the action of decreased blood pressure in the short and long term.}]{images/media/image127.jpeg}

\begin{enumerate}
\def\labelenumi{\Alph{enumi}.}
\setcounter{enumi}{2}
\item
  Erythropoietin

  \begin{enumerate}
  \def\labelenumii{\arabic{enumii}.}
  \item
    EPO stimulates the production of what in bone marrow?
  \item
    As the number of red blood cells increases what happens to blood
    viscosity?
  \end{enumerate}
\item
  Atrial Natriuretic Hormone

  \begin{enumerate}
  \def\labelenumii{\arabic{enumii}.}
  \item
    How does Atrial Natriuretic Hormone decrease blood pressure?
  \end{enumerate}
\end{enumerate}

\paragraph{Autoregulation of
Perfusion}\label{autoregulation-of-perfusion}

\begin{enumerate}
\def\labelenumi{\arabic{enumi}.}
\item
  What is autoregulation?
\item
  Why do tissues need to regulate their own blood flow?
\end{enumerate}

\begin{enumerate}
\def\labelenumi{\Alph{enumi}.}
\item
  Chemical Signals Involved in Autoregulation

  \begin{enumerate}
  \def\labelenumii{\arabic{enumii}.}
  \item
    Decreased oxygen concentration, or increased carbon dioxide
    concentration leads to precapillary sphincter
    \_\_\_\_\_\_\_\_\_\_\_\_\_\_\_.
  \item
    Release of endothelin would result in what action at the
    precapillary sphincter? How would this impact blood flow to the
    local capillary bed?
  \end{enumerate}
\end{enumerate}

\begin{enumerate}
\def\labelenumi{\Alph{enumi}.}
\item
  The Myogenic Response

  \begin{enumerate}
  \def\labelenumii{\arabic{enumii}.}
  \item
    The \_\_\_\_\_\_\_\_\_\_\_\_\_\_ \_\_\_\_\_\_\_\_\_\_\_\_\_\_\_\_
    stabilizes blood flow and protects against dramatic fluctuations in
    blood pressure and blood flow.
  \end{enumerate}
\end{enumerate}

\paragraph{Effect of Exercise on Vascular
Homeostasis}\label{effect-of-exercise-on-vascular-homeostasis}

\begin{enumerate}
\def\labelenumi{\arabic{enumi}.}
\item
  What happens to cardiac output during exercise? Why?
\item
  What happens to blood pressure during exercise? Why?
\item
  Which areas of the body experience vasoconstriction and vasodilation
  during intense exercise?
\item
  What are some of the long-term benefits of aerobic exercise?
\end{enumerate}

\paragraph{Clinical Considerations in Vascular
Homeostasis}\label{clinical-considerations-in-vascular-homeostasis}

\begin{enumerate}
\def\labelenumi{\Alph{enumi}.}
\item
  Hypertension and Hypotension

  \begin{enumerate}
  \def\labelenumii{\arabic{enumii}.}
  \item
    What is hypertension?

    \begin{enumerate}
    \def\labelenumiii{\alph{enumiii})}
    \item
      What is a normal blood pressure and how does that compare to a
      hypertensive blood pressure?
    \end{enumerate}
  \end{enumerate}
\end{enumerate}

\begin{enumerate}
\def\labelenumi{\Alph{enumi}.}
\item
  Hemorrhage

  \begin{enumerate}
  \def\labelenumii{\arabic{enumii}.}
  \item
    How is hemorrhage different from minor blood loss?
  \item
    How does the body try to compensate during hemorrhage?
  \end{enumerate}
\end{enumerate}

\includegraphics[width=5.10417in,height=2.6875in,alt={This flowchart shows the action of decreased blood pressure and volume in the neural and endocrine mechanisms.}]{images/media/image128.jpeg}

\begin{enumerate}
\def\labelenumi{\Alph{enumi}.}
\setcounter{enumi}{1}
\item
  Circulatory Shock

  \begin{enumerate}
  \def\labelenumii{\arabic{enumii}.}
  \item
    What leads to circulatory shock?
  \item
    What is hypovolemic shock?

    \begin{enumerate}
    \def\labelenumiii{\alph{enumiii})}
    \item
      What can cause this type of shock?
    \end{enumerate}
  \item
    What is cardiogenic shock and how is it different than hypovolemic
    shock?
  \item
    What are three types of vascular shock?

    \begin{enumerate}
    \def\labelenumiii{\alph{enumiii})}
    \item
      What do they all have in common?
    \end{enumerate}
  \end{enumerate}
\end{enumerate}

\subparagraph{20.5. Circulatory Pathways}\label{circulatory-pathways}

Background

\begin{enumerate}
\def\labelenumi{\arabic{enumi}.}
\item
  Connect one other system covered to the circulatory system? How do
  they interact?
\end{enumerate}

\includegraphics[width=5.41667in,height=6.53125in,alt={This table outlines the role of the circulatory system in the other organ systems in the body.}]{images/media/image129.jpeg}

\begin{enumerate}
\def\labelenumi{\arabic{enumi}.}
\setcounter{enumi}{1}
\item
  What is a trunk in the vascular system?
\end{enumerate}

Pulmonary Circulation

\begin{enumerate}
\def\labelenumi{\arabic{enumi}.}
\item
  What is the purpose of the pulmonary circuit?
\item
  The \_\_\_\_\_\_\_\_\_\_\_\_ \_\_\_\_\_\_\_\_ is the vessel that exits
  the right ventricle.
\item
  The pulmonary arteries branch from the pulmonary trunk and after many
  divisions delivers \_\_\_\_\_\_\_\_\_\_\_\_\_ blood to the capillaries
  of the lungs for oxygenation.
\item
  What vessels return oxygenated blood back to the left atrium?
\end{enumerate}

\includegraphics[width=4.375in,height=2.51042in,alt={This diagram shows the network of blood vessels in the lungs.}]{images/media/image130.jpeg}

Overview of Systemic Arteries

\begin{enumerate}
\def\labelenumi{\arabic{enumi}.}
\item
  Blood pumped from the left ventricle passes through the
  \_\_\_\_\_\_\_\_\_\_\_ which delivers blood to the rest of the body
  through the systemic arteries.
\end{enumerate}

\includegraphics[width=3.768in,height=6.10417in,alt={This diagrams shows the major arteries in the human body.}]{images/media/image131.jpeg}

The Aorta

\begin{enumerate}
\def\labelenumi{\arabic{enumi}.}
\item
  The ascending aorta leads to the \_\_\_\_\_\_\_\_\_\_\_\_
  \_\_\_\_\_\_\_\_\_\_ which then continues on to the descending aorta.
\item
  \includegraphics[width=3.95833in,height=4.20833in,alt={This diagram shows the aorta and the major parts are labeled.}]{images/media/image132.jpeg}Superior
  to the diaphragm the aorta is known as the \_\_\_\_\_\_\_\_\_\_\_\_
  \_\_\_\_\_\_\_\_\_\_\_, and inferior to the diaphragm the aorta is the
  \_\_\_\_\_\_\_\_\_\_\_\_\_\_\_ \_\_\_\_\_\_\_\_\_\_\_\_\_.
\end{enumerate}

\begin{enumerate}
\def\labelenumi{\Alph{enumi}.}
\item
  Coronary Circulation

  \begin{enumerate}
  \def\labelenumii{\arabic{enumii}.}
  \item
    What are the first vessels that branch from the ascending aorta?
  \end{enumerate}
\end{enumerate}

\begin{enumerate}
\def\labelenumi{\Alph{enumi}.}
\item
  Aortic Arch Branches

  \begin{enumerate}
  \def\labelenumii{\arabic{enumii}.}
  \item
    What are the three major branches of the aortic arch?
  \item
    Which of the three major branches is only found on the right side of
    the body and has no left counterpart?
  \item
    What do the subclavian arteries provide blood to?
  \item
    The common carotid arteries supply what part of the body with blood?

    \begin{enumerate}
    \def\labelenumiii{\alph{enumiii})}
    \item
      More specifically what do the external and internal carotid
      arteries provide?
    \end{enumerate}
  \item
    What is a transient ischemic attack (TIA)?

    \begin{enumerate}
    \def\labelenumiii{\alph{enumiii})}
    \item
      Compare a TIA to a cerebrovascular accident (CVA)
    \end{enumerate}
  \item
    How does the arterial circle (circle of Willis) help to prevent a
    disruption in blood flow to the brain?
  \item
    The internal carotid artery has an anterior cerebral artery, middle
    cerebral artery, and ophthalmic artery branch. What do these supply
    with blood?
  \end{enumerate}
\end{enumerate}

\includegraphics[width=4.3125in,height=4.71765in]{images/media/image133.png}

\includegraphics[width=4.6875in,height=2.76042in,alt={This diagram shows the arteries of the brain.}]{images/media/image134.jpeg}

\begin{enumerate}
\def\labelenumi{\Alph{enumi}.}
\setcounter{enumi}{1}
\item
  Thoracic Aorta and Major Branches

  \begin{enumerate}
  \def\labelenumii{\arabic{enumii}.}
  \item
    What are the visceral branches of the thoracic aorta?
  \item
    The \_\_\_\_\_\_\_\_\_\_\_\_\_\_ \_\_\_\_\_\_\_\_\_\_\_\_ supply
    blood to the lungs and visceral pleura.
  \item
    What does the pericardial artery supply?
  \item
    The superior phrenic artery supplies what muscle with blood?
  \end{enumerate}
\end{enumerate}

\includegraphics[width=3.83333in,height=3.61771in,alt={This diagram shows the arteries in the thoracic and abdominal cavity.}]{images/media/image135.jpeg}

\begin{enumerate}
\def\labelenumi{\Alph{enumi}.}
\setcounter{enumi}{2}
\item
  Abdominal Aorta and Major Branches

  \begin{enumerate}
  \def\labelenumii{\arabic{enumii}.}
  \item
    What are the branches of the celiac trunk?
  \item
    What do the superior and inferior mesenteric arteries supply?
  \item
    The adrenal arteries supply blood to the \_\_\_\_\_\_\_\_\_\_\_\_\_
    \_\_\_\_\_\_\_\_\_\_\_\_\_.
  \item
    The kidneys are supplied with blood from the
    \_\_\_\_\_\_\_\_\_\_\_\_\_\_\_ artery
  \item
    The gonads are supplied with blood from the
    \_\_\_\_\_\_\_\_\_\_\_\_\_\_\_ artery
  \item
    What supplies blood to the abdominal wall?
  \item
    Inferiorly what does the abdominal aorta divide to become?

    \begin{enumerate}
    \def\labelenumiii{\alph{enumiii})}
    \item
      What small vessel continues inferiorly after the division?
    \end{enumerate}
  \item
    Does the external or internal iliac artery supply the lower limbs
    with blood?
  \end{enumerate}
\end{enumerate}

\includegraphics[width=4.08333in,height=5.7922in]{images/media/image136.png}\includegraphics[width=3.88889in,height=3.19213in,alt={This flowchart shows the different branches into which that the abdominal aorta is divided.}]{images/media/image137.jpeg}

\paragraph[Arteries Serving the Upper
Limbs]{\texorpdfstring{\protect\includegraphics[width=2.30556in,height=4.08333in,alt={A picture containing text, map Description automatically generated}]{images/media/image138.jpeg}Arteries
Serving the Upper
Limbs}{A picture containing text, map Description automatically generatedArteries Serving the Upper Limbs}}\label{a-picture-containing-text-map-description-automatically-generatedarteries-serving-the-upper-limbs}

\begin{enumerate}
\def\labelenumi{\arabic{enumi}.}
\item
  The subclavian artery supplies what artery in the armpit?
\item
  The brachial artery supplies what part of the body?
\item
  The arteries of the forearm share their names with what other bodily
  structures?
\item
  Palmar arches supply the \_\_\_\_\_\_\_\_\_\_\_\_
  \_\_\_\_\_\_\_\_\_\_\_\_\_ of the fingers with blood.
\end{enumerate}

\paragraph{Arteries Serving the Lower
Limbs}\label{arteries-serving-the-lower-limbs}

\begin{enumerate}
\def\labelenumi{\arabic{enumi}.}
\item
  What are the branches of the femoral artery?
\item
  What is the name of the artery posterior to the knee?
\item
  What does the anterior tibial artery supply?
\item
  Based upon the name where is the posterior tibial artery located?
\item
  What two arches supply blood to the foot and toes?
\end{enumerate}

\paragraph{Overview of Systemic Veins}\label{overview-of-systemic-veins}

\begin{enumerate}
\def\labelenumi{\arabic{enumi}.}
\item
  Systemic veins return blood to the \_\_\_\_\_\_\_\_\_
  \_\_\_\_\_\_\_\_.

  \begin{enumerate}
  \def\labelenumii{\alph{enumii})}
  \item
    Do they return oxygenated or deoxygenated blood?
  \end{enumerate}
\item
  Are deep or superficial veins normally named similar to their arterial
  counterparts?
\end{enumerate}

\begin{enumerate}
\def\labelenumi{\Alph{enumi}.}
\item
  The Superior Vena Cava

  \begin{enumerate}
  \def\labelenumii{\arabic{enumii}.}
  \item
    What part of the body does the superior vena cava receive blood
    from?
  \item
    There is one brachiocephalic artery, but \_\_\_\_\_\_\_
    brachiocephalic veins.

    \begin{enumerate}
    \def\labelenumiii{\alph{enumiii})}
    \item
      What part of the body do the brachiocephalic veins receive blood
      from?
    \end{enumerate}
  \item
    The subclavian vein, external and internal jugular veins fuse to
    form the \_\_\_\_\_\_\_\_\_\_\_\_\_\_\_\_\_\_\_
    \_\_\_\_\_\_\_\_\_\_\_\_\_\_.
  \item
    What veins drain into the azygos vein?

    \begin{enumerate}
    \def\labelenumiii{\alph{enumiii})}
    \item
      What vessel does the azygos vein drain into?
    \end{enumerate}
  \end{enumerate}
\end{enumerate}

\begin{enumerate}
\def\labelenumi{\Alph{enumi}.}
\item
  Veins of the Head and Neck

  \begin{enumerate}
  \def\labelenumii{\arabic{enumii}.}
  \item
    Blood from the brain is drained primarily via what vein?
  \item
    What does the external jugular vein receive blood from?
  \end{enumerate}
\item
  Venous Drainage of the Brain

  \begin{enumerate}
  \def\labelenumii{\arabic{enumii}.}
  \item
    What is the largest sinus in the brain?
  \item
    What sinus drains the eye socket?
  \end{enumerate}
\item
  Veins Draining the Upper Limbs

  \begin{enumerate}
  \def\labelenumii{\arabic{enumii}.}
  \item
    Digital veins drain into the palmar venous arches which in turn
    drain into the \_\_\_\_\_\_\_\_\_\_\_\_ and \_\_\_\_\_\_\_\_\_\_\_
    veins.
  \item
    What are three superficial veins of the arm?
  \item
    What branch of the basilic vein joins the cephalic vein? What is
    this branch commonly used for?
  \end{enumerate}
\item
  The Inferior Vena Cava

  \begin{enumerate}
  \def\labelenumii{\arabic{enumii}.}
  \item
    The inferior vena cava drains blood from what part of the body?
  \item
    What drains blood from the kidneys to the inferior vena cava?
  \item
    Which of the gonadal veins drains into the inferior vena cava?
  \item
    The \_\_\_\_\_\_\_\_\_\_\_\_\_\_ \_\_\_\_\_\_\_\_\_\_ drains blood
    from the liver to the inferior vena cava.
  \end{enumerate}
\item
  Veins Draining the Lower Limbs

  \begin{enumerate}
  \def\labelenumii{\arabic{enumii}.}
  \item
    The inferior surface of the foot is drained by what veins?
  \item
    The anterior tibial vein, posterior tibial vein, and fibular vein
    combine to form what vein behind the knee?
  \item
    Which saphenous vein is located on the medial side of the leg?
  \item
    The femoral vein drains into which branch of the common iliac vein?
  \item
    The common iliac veins become what vein at L5?
  \end{enumerate}
\end{enumerate}

\paragraph{Hepatic Portal System}\label{hepatic-portal-system}

\begin{enumerate}
\def\labelenumi{\arabic{enumi}.}
\item
  The \_\_\_\_\_\_\_\_\_\_\_\_\_ \_\_\_\_\_\_\_\_\_\_\_\_\_\_\_
  \_\_\_\_\_\_\_\_\_\_\_\_\_\_ allows the liver to process absorbed
  digestive materials and certain wastes.
\item
  What veins drain into the hepatic portal vein?
\item
  Is the hepatic portal vein the only blood vessel that supplies the
  liver?
\end{enumerate}

\includegraphics[width=5.69579in,height=3.66667in,alt={This diagram shows the veins in the digestive system.}]{images/media/image139.jpeg}

\subparagraph{20.6. Development of Blood Vessels and Fetal
Circulation}\label{development-of-blood-vessels-and-fetal-circulation}

Background

\begin{enumerate}
\def\labelenumi{\arabic{enumi}.}
\item
  The hemangioblasts precursor cells become \_\_\_\_\_\_\_\_\_\_\_\_\_\_
  which give rise to blood vessels and pluripotent stem cells.
\item
  What is angiogenesis? Do all blood vessels form from one single
  vascular tube?
\item
  What kind of blood (oxygenated/deoxygenated) do the umbilical vein and
  arteries carry?
\item
  What are the three shunts present in the fetal circulatory system that
  divert blood away from the pulmonary circuit and towards the systemic
  circuit?
\item
  The foramen ovale connects what two chambers of the heart?
\item
  What remains after the ductus arteriosus closes?
\item
  Where is the ductus venosus located?
\end{enumerate}

\includegraphics[width=5.65844in,height=5.8642in,alt={This figure shows the blood vessels in a fetus.}]{images/media/image140.jpeg}

\section{}\label{section-20}

\section{\texorpdfstring{Chapter 21 }{Chapter 21 }}\label{chapter-21}

\subsection{21.1. Anatomy of the Lymphatic and Immune
Systems}\label{anatomy-of-the-lymphatic-and-immune-systems}

\subsubsection{}\label{section-21}

\begin{enumerate}
\def\labelenumi{\Roman{enumi}.}
\item
  Background

  \begin{enumerate}
  \def\labelenumii{\arabic{enumii}.}
  \item
    What are the general components of the immune system?
  \item
    What is the purpose of the lymphatic system?
  \end{enumerate}
\item
  Functions of the Lymphatic System

  \begin{enumerate}
  \def\labelenumii{\arabic{enumii}.}
  \item
    What is lymph?
  \item
    What is transported within lymph?
  \item
    What purpose do lymph nodes serve?
  \end{enumerate}
\item
  Structure of the Lymphatic System

  \begin{enumerate}
  \def\labelenumii{\arabic{enumii}.}
  \item
    Lymphatic vessels are blind ended. What does this mean, and how is
    it different from blood capillaries?
  \item
    The heart pumps blood throughout the body, does it also pump lymph?
  \item
    Where does lymph flow to?
  \end{enumerate}

  \begin{enumerate}
  \def\labelenumii{\Alph{enumii}.}
  \item
    Lymphatic Capillaries

    \begin{enumerate}
    \def\labelenumiii{\arabic{enumiii}.}
    \item
      Where does the lymph come from that enters lymphatic capillaries?
    \item
      What purpose do the flaps of overlapping cells serve in lymphatic
      capillaries?
    \item
      Aside from transporting lymph, what do lymphatic capillaries in
      the intestines (lacteals) transport as well?

      \begin{enumerate}
      \def\labelenumiv{\alph{enumiv})}
      \item
        What is chyle?
      \end{enumerate}
    \end{enumerate}
  \end{enumerate}

  \begin{enumerate}
  \def\labelenumii{\Alph{enumii}.}
  \item
    Larger Lymphatic Vessels, Trunks, and Ducts

    \begin{enumerate}
    \def\labelenumiii{\arabic{enumiii}.}
    \item
      What purpose do valves serve in lymphatic vessels?
    \item
      How are lymphatic trunks formed?
    \item
      What parts of the body drain into the right lymphatic duct and the
      thoracic duct?
    \item
      What blood vessel receives the lymph from the lymphatic ducts?
    \item
      What is the cisterna chyli?
    \end{enumerate}
  \end{enumerate}
\end{enumerate}

\begin{enumerate}
\def\labelenumi{\Roman{enumi}.}
\setcounter{enumi}{3}
\item
  The Organization of Immune Function

  \begin{enumerate}
  \def\labelenumii{\arabic{enumii}.}
  \item
    What is an example of a barrier defense?
  \item
    After the barrier defenses and the innate immune response what is
    the last temporal phase of immune function?
  \item
    What is the difference between a phagocytic cell and a lymphocyte?
  \end{enumerate}
\item
  Lymphocytes: B Cells, T Cells, Plasma Cells, and Natural Killer Cells

  \begin{enumerate}
  \def\labelenumii{\arabic{enumii}.}
  \item
    Both B and T cells both develop in the bone marrow, but where they
    mature is different. Where do B and T cells mature?
  \end{enumerate}

  \begin{enumerate}
  \def\labelenumii{\Alph{enumii}.}
  \item
    B Cells

    \begin{enumerate}
    \def\labelenumiii{\arabic{enumiii}.}
    \item
      B Cells produce \_\_\_\_\_\_\_\_\_\_\_\_\_\_\_\_\_.
    \item
      What is an antigen?
    \end{enumerate}
  \end{enumerate}

  \begin{enumerate}
  \def\labelenumii{\Alph{enumii}.}
  \item
    T Cells

    \begin{enumerate}
    \def\labelenumiii{\arabic{enumiii}.}
    \item
      Do T cells secrete antibodies like B Cells?
    \item
      What do T cells secrete?
    \end{enumerate}
  \item
    Plasma Cells

    \begin{enumerate}
    \def\labelenumiii{\arabic{enumiii}.}
    \item
      A plasma cell is a differentiated \_\_\_\_\_ cell.
    \end{enumerate}
  \item
    Natural Killer Cells

    \begin{enumerate}
    \def\labelenumiii{\arabic{enumiii}.}
    \item
      Natural Killer (NK) cells participate in what part of the immune
      response?
    \end{enumerate}
  \end{enumerate}
\end{enumerate}

\begin{enumerate}
\def\labelenumi{\Roman{enumi}.}
\setcounter{enumi}{5}
\item
  Primary Lymphoid Organs and Lymphocyte Development

  \begin{enumerate}
  \def\labelenumii{\arabic{enumii}.}
  \item
    The primary lymphoid organs are the \_\_\_\_\_\_\_\_\_\_\_
    \_\_\_\_\_\_\_\_\_\_\_\_ and the thymus gland.
  \item
    The lymphoid organs are where lymphocytes
    \_\_\_\_\_\_\_\_\_\_\_\_\_\_\_,
    \_\_\_\_\_\_\_\_\_\_\_\_\_\_\_\_\_\_\_, and are selected.
  \end{enumerate}

  \begin{enumerate}
  \def\labelenumii{\Alph{enumii}.}
  \item
    \includegraphics[width=2.68056in,height=2.48264in,alt={Figure 21.6 Bone Marrow Red bone marrow fills the head of the femur, and a spot of yellow bone marrow is visible in the center. The white reference bar is 1 cm.}]{images/media/image141.jpeg}Bone
    Marrow

    \begin{enumerate}
    \def\labelenumiii{\arabic{enumiii}.}
    \item
      Is red or yellow bone marrow responsible for hemopoiesis?
    \item
      Do B or T cells complete the majority of their development in bone
      marrow?
    \item
      Thymocytes leave the bone marrow and mature in the
      \_\_\_\_\_\_\_\_\_\_\_\_\_\_\_\_\_.
    \end{enumerate}
  \end{enumerate}

  \begin{enumerate}
  \def\labelenumii{\Alph{enumii}.}
  \item
    Thymus

    \begin{enumerate}
    \def\labelenumiii{\arabic{enumiii}.}
    \item
      Where is the thymus found in the body?
    \item
      Which part of the thymus is where thymocytes migrate before
      leaving, the cortex or the medulla?
    \end{enumerate}
  \end{enumerate}
\end{enumerate}

\begin{enumerate}
\def\labelenumi{\Roman{enumi}.}
\setcounter{enumi}{6}
\item
  Secondary Lymphoid Organs and their Roles in Active Immune Response

  \begin{enumerate}
  \def\labelenumii{\arabic{enumii}.}
  \item
    After development and maturation in the primary lymphoid organs,
    what is the purpose of secondary lymphoid organs?
  \item
    What is a naïve lymphocyte?
  \item
    What are some features secondary lymphoid organs have in common?
  \item
    What is a germinal center?

    \begin{enumerate}
    \def\labelenumiii{\alph{enumiii})}
    \item
      What type of immune cell is found within?
    \end{enumerate}
  \end{enumerate}

  \begin{enumerate}
  \def\labelenumii{\Alph{enumii}.}
  \item
    Lymph Nodes

    \begin{enumerate}
    \def\labelenumiii{\arabic{enumiii}.}
    \item
      What is the function of lymph nodes?
    \item
      What role doe dendritic and macrophages play?
    \item
      What role do T and B cells play?
    \item
      What are the pathways into and out of the lymph nodes?
    \end{enumerate}
  \end{enumerate}

  \begin{enumerate}
  \def\labelenumii{\Alph{enumii}.}
  \item
    Spleen

    \begin{enumerate}
    \def\labelenumiii{\arabic{enumiii}.}
    \item
      Why is the spleen known as the ``filter of the blood?''
    \item
      What is primarily found in the red pulp of the spleen?
    \item
      Where in the spleen are adaptive immune responses mounted?
    \end{enumerate}
  \item
    Lymphoid Nodules

    \begin{enumerate}
    \def\labelenumiii{\arabic{enumiii}.}
    \item
      Where are lymphoid nodules found?
    \item
      Why are tonsils important?
    \item
      Swelling of a tonsil indicates what?
    \item
      Why might palatine tonsils be removed?
    \end{enumerate}
  \item
    Mucosa-associated lymphoid tissue (MALT)

    \begin{enumerate}
    \def\labelenumiii{\arabic{enumiii}.}
    \item
      What is MALT, and where is it found?
    \item
      What is BALT?

      \begin{enumerate}
      \def\labelenumiv{\alph{enumiv})}
      \item
        What type of pathogens does it help protect against?
      \end{enumerate}
    \end{enumerate}
  \end{enumerate}
\end{enumerate}

\subsection{21.2. Barrier Defense and the Innate Immune
Response}\label{barrier-defense-and-the-innate-immune-response}

\begin{enumerate}
\def\labelenumi{\Roman{enumi}.}
\item
  Background

  \begin{enumerate}
  \def\labelenumii{\arabic{enumii}.}
  \item
    What are the two overlapping mechanisms that the body uses to
    destroy a pathogen?
  \item
    Which of the two generates a specific response to a
    pathogen?\includegraphics[width=3.98611in,height=3.50833in,alt={Figure 21.12 Cooperation between Innate and Adaptive Immune Responses The innate immune system enhances adaptive immune responses so they can be more effective.}]{images/media/image142.jpeg}
  \item
    How do barrier defenses handle infections?
  \item
    What is an example of a barrier defense, and how does it protect the
    body?
  \end{enumerate}
\item
  Cells of the Innate Immune Response

  \begin{enumerate}
  \def\labelenumii{\arabic{enumii}.}
  \item
    What is phagocytosis, and how is it used as part of the immune
    response?
  \end{enumerate}

  \begin{enumerate}
  \def\labelenumii{\Alph{enumii}.}
  \item
    Phagocytes: Macrophages and Neutrophils

    \begin{enumerate}
    \def\labelenumiii{\arabic{enumiii}.}
    \item
      What are the three major phagocytic cells of the immune system?
    \item
      Macrophages have different names depending on their location. What
      are they known as in the liver?
    \item
      How are neutrophils different from macrophages?

      \begin{enumerate}
      \def\labelenumiv{\alph{enumiv})}
      \item
        Which is an agranulocyte?
      \end{enumerate}
    \item
      What types of cells does a monocyte differentiate in to?
    \end{enumerate}
  \item
    Natural Killer Cells

    \begin{enumerate}
    \def\labelenumiii{\arabic{enumiii}.}
    \item
      Natural killer (NK) cells can induce apoptosis. What is apoptosis,
      and why is it useful in the immune process?
    \item
      How does the fas ligand help the NK cells to trigger apoptosis in
      infected cells?
    \item
      How does perforin help to destroy and infected cell?
    \end{enumerate}
  \end{enumerate}
\item
  Recognition of Pathogens

  \begin{enumerate}
  \def\labelenumii{\arabic{enumii}.}
  \item
    What is the purpose of a pattern recognition receptor (PRR)?

    \begin{enumerate}
    \def\labelenumiii{\alph{enumiii})}
    \item
      How does it allow for phagocytic cells to identify pathogens?
    \end{enumerate}
  \end{enumerate}
\item
  Soluble Mediators of the Innate Immune Response

  \begin{enumerate}
  \def\labelenumii{\Alph{enumii}.}
  \item
    Cytokines and Chemokines

    \begin{enumerate}
    \def\labelenumiii{\arabic{enumiii}.}
    \item
      What is a cytokine?
    \item
      How is a chemokine different from a cytokine?
    \end{enumerate}
  \item
    Early Induced Proteins

    \begin{enumerate}
    \def\labelenumiii{\arabic{enumiii}.}
    \item
      What are early induced proteins? When are they present within the
      body?
    \item
      When do cells secrete interferons?
    \item
      The liver makes C-reactive protein. How does this protein help
      with a bacterial infection?
    \item
      What is opsonization, and how does it help the body to deal with
      an infection?
    \end{enumerate}
  \item
    \includegraphics[width=4.04167in,height=4.22708in,alt={Figure 21.13 Complement Cascade and Function The classical pathway, used during adaptive immune responses, occurs when C1 reacts with antibodies that have bound an antigen.}]{images/media/image143.jpeg}Complement
    System

    \begin{enumerate}
    \def\labelenumiii{\arabic{enumiii}.}
    \item
      Why is complement not considered a part of the early immune
      response?
    \item
      What are the two pathways to activate the complement cascade?
    \item
      What is the membrane attack complex, and what happens at the end
      of the complement cascade?
    \end{enumerate}
  \end{enumerate}
\item
  Inflammatory Response

  \begin{enumerate}
  \def\labelenumii{\arabic{enumii}.}
  \item
    What is the hallmark of the innate immune response?
  \item
    Why is the inflammatory response important?
  \item
    Acute inflammation can lead to \_\_\_\_\_\_\_\_
    \_\_\_\_\_\_\_\_\_\_\_\_\_\_\_\_\_\_ that persists.
  \item
    What are the four important parts of the inflammatory response?
  \end{enumerate}
\end{enumerate}

\subsection{21.3. The Adaptive Immune Response: T Lymphocytes and Their
Functional
Types}\label{the-adaptive-immune-response-t-lymphocytes-and-their-functional-types}

\begin{enumerate}
\def\labelenumi{\Roman{enumi}.}
\item
  Background

  \begin{enumerate}
  \def\labelenumii{\arabic{enumii}.}
  \item
    Why is the adaptive immune response necessary? Do the innate and
    early induced responses normally completely eliminate pathogens?
  \end{enumerate}
\item
  The Benefits of the Adaptive Immune Response

  \begin{enumerate}
  \def\labelenumii{\arabic{enumii}.}
  \item
    How is the adaptive immune response different from the innate immune
    response?

    \begin{enumerate}
    \def\labelenumiii{\alph{enumiii})}
    \item
      The adaptive immune response is to specific
      \_\_\_\_\_\_\_\_\_\_\_\_\_\_, and is versatile enough to respond
      to nearly any pathogen.
    \end{enumerate}
  \end{enumerate}

  \begin{enumerate}
  \def\labelenumii{\Alph{enumii}.}
  \item
    Primary Disease and Immunological Memory

    \begin{enumerate}
    \def\labelenumiii{\arabic{enumiii}.}
    \item
      What is different between a primary and secondary adaptive
      response?

      \begin{enumerate}
      \def\labelenumiv{\alph{enumiv})}
      \item
        What role does immunological memory play?
      \end{enumerate}
    \end{enumerate}
  \item
    Self Recognition

    \begin{enumerate}
    \def\labelenumiii{\arabic{enumiii}.}
    \item
      Why is self recognition important?
    \item
      What happens in autoimmune disease?
    \end{enumerate}
  \end{enumerate}
\item
  \includegraphics[width=3.73611in,height=3.41806in,alt={Figure 21.15 Alpha-beta T Cell Receptor Notice the constant and variable regions of each chain, anchored by the transmembrane region.}]{images/media/image144.jpeg}T
  Cell-Mediated Immune Responses

  \begin{enumerate}
  \def\labelenumii{\arabic{enumii}.}
  \item
    Why are T cells a critical part of the adaptive immune response?
  \item
    What is an alpha-beta T cell receptor, and what is it used for?
  \item
    What is the difference between the variable and constant region
    domains?
  \end{enumerate}
\item
  Antigens

  \begin{enumerate}
  \def\labelenumii{\arabic{enumii}.}
  \item
    How many antigenic determinants (epitopes) might a pathogen contain?
  \item
    Why are protein antigens more complex than carbohydrate antigens?
  \end{enumerate}

  \begin{enumerate}
  \def\labelenumii{\Alph{enumii}.}
  \item
    Antigen Processing and Presentation

    \begin{enumerate}
    \def\labelenumiii{\arabic{enumiii}.}
    \item
      What is an antigen presenting cell, and why is it important for T
      cells to recognize antigens?
    \item
      What is MHC, and how does it help in antigen presentation?
    \item
      MHC class II molecules process what type of antigen?
    \end{enumerate}
  \item
    Professional Antigen-presenting Cells

    \begin{enumerate}
    \def\labelenumiii{\arabic{enumiii}.}
    \item
      ``Professional'' antigen presenting cells are the only cells that
      express MHC class \_\_\_ molecules.
    \item
      What tare the three types of professional antigen presenting
      cells?
    \end{enumerate}
  \end{enumerate}
\item
  T Cell Development and Differentiation

  \begin{enumerate}
  \def\labelenumii{\arabic{enumii}.}
  \item
    T cell tolerance eliminates T cells that might do what?
  \item
    What is positive selection?

    \begin{enumerate}
    \def\labelenumiii{\alph{enumiii})}
    \item
      Why is it a crucial step in T cell maturation?
    \end{enumerate}
  \item
    What is negative selection?

    \begin{enumerate}
    \def\labelenumiii{\alph{enumiii})}
    \item
      Why is it important that the cells do not bind to self-antigens?
    \end{enumerate}
  \end{enumerate}
\item
  Mechanisms of T Cell-mediated Immune Responses

  \begin{enumerate}
  \def\labelenumii{\arabic{enumii}.}
  \item
    When is a mature T cell activated?
  \item
    What is clonal expansion, and why is it necessary for a robust
    immune response?
  \end{enumerate}
\item
  Clonal Selection and Expansion

  \begin{enumerate}
  \def\labelenumii{\arabic{enumii}.}
  \item
    In clonal selection what T cells proliferate?
  \item
    If multiple antigens on the pathogen generate a response it is a
    \_\_\_\_\_\_\_\_\_\_\_\_\_\_\_\_ \_\_\_\_\_\_\_\_\_\_\_\_\_\_.
    Multiple T cell clones will respond.
  \end{enumerate}
\item
  The Cellular Basis of Immunological Memory

  \begin{enumerate}
  \def\labelenumii{\arabic{enumii}.}
  \item
    Why are both memory T cells and effector T cells needed for a
    primary adaptive immune response?
  \end{enumerate}
\item
  T Cell Types and their Functions

  \begin{enumerate}
  \def\labelenumii{\arabic{enumii}.}
  \item
    Mature T cells express one of two markers. What are those markers?
  \item
    CD4 T Cells are associated with \_\_\_\_\_\_\_\_\_\_\_ functions and
    CD8 T cells are associated with \_\_\_\_\_\_\_\_\_\_\_\_\_.
  \end{enumerate}

  \begin{enumerate}
  \def\labelenumii{\Alph{enumii}.}
  \item
    Helper T Cells and Their Cytokines

    \begin{enumerate}
    \def\labelenumiii{\arabic{enumiii}.}
    \item
      How do helper T cells (Th) function?
    \item
      What is the difference in cytokine secretion between Th1 and Th2
      cells?
    \end{enumerate}
  \item
    Cytotoxic T Cells

    \begin{enumerate}
    \def\labelenumiii{\arabic{enumiii}.}
    \item
      Why is it important for a cytotoxic T cell (Tc) to induce
      apoptosis in a virally infected cell prior to virus replication?
    \end{enumerate}
  \item
    Regulatory T Cells

    \begin{enumerate}
    \def\labelenumiii{\arabic{enumiii}.}
    \item
      Why do regulatory T cells (Treg) modulate the immune response?
      What happens with an unrestrained T cell response?
    \end{enumerate}
  \end{enumerate}
\end{enumerate}

\subsection{21.4. The Adaptive Immune Response: B-Lymphocytes and
Antibodies}\label{the-adaptive-immune-response-b-lymphocytes-and-antibodies}

\subsubsection{Background}\label{background-13}

\begin{enumerate}
\def\labelenumi{\arabic{enumi}.}
\item
  What is immunoglobin?
\item
  What secretes antibody?

  \begin{enumerate}
  \def\labelenumii{\alph{enumii})}
  \item
    What are the five classes of antibodies found in humans?
  \end{enumerate}
\end{enumerate}

\subsubsection{B Cell Differentiation and
Activation}\label{b-cell-differentiation-and-activation}

\begin{enumerate}
\def\labelenumi{\arabic{enumi}.}
\item
  What is central tolerance, and why is it necessary in B cell
  development?
\item
  What happens to B cells that bind to self-antigens in clonal deletion?
\item
  What happens to B cells during peripheral tolerance that bind to
  self-antigen, but do not receive a signal from a Th2 cell?
\end{enumerate}

\subsubsection{Antibody Structure}\label{antibody-structure}

\begin{enumerate}
\def\labelenumi{\arabic{enumi}.}
\item
  What are the two polypeptide chains that compose an antibody?
\end{enumerate}

\begin{enumerate}
\def\labelenumi{\Alph{enumi}.}
\item
  Four-chain Models of Antibody Structures

  \begin{enumerate}
  \def\labelenumii{\arabic{enumii}.}
  \item
    Why is the Fc region of the antibody important?
  \item
    Where on an antibody does an antigen bind?
  \end{enumerate}
\end{enumerate}

\includegraphics[width=4.79167in,height=1.90625in,alt={This diagram shows the four chain structure of a generic antibody.}]{images/media/image145.jpeg}

\begin{enumerate}
\def\labelenumi{\Alph{enumi}.}
\setcounter{enumi}{1}
\item
  Five Classes of Antibodies and their Functions

  \begin{enumerate}
  \def\labelenumii{\arabic{enumii}.}
  \item
    What are the two basic function of antibodies?
  \item
    What two types of antibodies can act as the antigen receptor for
    naïve B cells?
  \end{enumerate}
\end{enumerate}

\includegraphics[width=5in,height=4.67708in,alt={This table shows the five classes of the immunoglobulins. The table shows the molecular weight, number of antigen binding sites, and their function.}]{images/media/image146.jpeg}

\begin{enumerate}
\def\labelenumi{\arabic{enumi}.}
\setcounter{enumi}{2}
\item
  IgM is the largest of the antibody molecules. What is the structure of
  this antibody?
\item
  What is class switching? When does it occur during the response to an
  infection?
\item
  IgG can cross the placenta. How is this potentially beneficial?
\item
  IgA is unique in that it is the only antibody that leaves the body.
  Why is this useful?
\item
  IgE is associated with \_\_\_\_\_\_\_\_\_\_\_\_, and the severe
  reaction of anaphylaxis.
\end{enumerate}

\begin{enumerate}
\def\labelenumi{\Alph{enumi}.}
\setcounter{enumi}{2}
\item
  Clonal Selection of B Cells

  \begin{enumerate}
  \def\labelenumii{\arabic{enumii}.}
  \item
    What B Cells are selected and expanded during clonal selection?
  \end{enumerate}
\end{enumerate}

\includegraphics[width=5in,height=4.57292in,alt={This flow chart shows how the clonal selection of B cells takes place. The left panel shows the primary response and the right panel shows the secondary response.}]{images/media/image147.jpeg}

\begin{enumerate}
\def\labelenumi{\Alph{enumi}.}
\setcounter{enumi}{3}
\item
  Primary versus Secondary B Cell Responses

  \begin{enumerate}
  \def\labelenumii{\arabic{enumii}.}
  \item
    What is different between a primary and secondary B cell response?

    \begin{enumerate}
    \def\labelenumiii{\alph{enumiii})}
    \item
      How does this impact the ability to deal with the second exposure
      to a pathogen?
    \end{enumerate}
  \end{enumerate}
\end{enumerate}

\includegraphics[width=3.95833in,height=2.75in,alt={This graph shows the antibody concentration as a function of time in primary and secondary response.}]{images/media/image148.jpeg}

\subsubsection{Active versus Passive
Immunity}\label{active-versus-passive-immunity}

\begin{enumerate}
\def\labelenumi{\arabic{enumi}.}
\item
  What are two ways a person can acquire active immunity?
\item
  What is passive immunity, and how can a person receive passive
  immunity?
\end{enumerate}

\subsubsection{T cell-dependent versus T cell-independent
Antigens}\label{t-cell-dependent-versus-t-cell-independent-antigens}

\begin{enumerate}
\def\labelenumi{\arabic{enumi}.}
\item
  T cell-independent antigens are usually \_\_\_\_\_\_\_\_\_\_\_\_\_\_
  moieties.
\item
  What are the two signals necessary for T cell-dependent antigens to
  elicit a response?
\end{enumerate}

\includegraphics[width=5in,height=1.84375in,alt={This diagram shows the binding of a B cell and a T cell.}]{images/media/image149.jpeg}

\subsection{21.5. The Immune Response Against
Pathogens}\label{the-immune-response-against-pathogens}

\begin{enumerate}
\def\labelenumi{\Roman{enumi}.}
\item
  Background

  \begin{enumerate}
  \def\labelenumii{\arabic{enumii}.}
  \item
    During the first 4-5 days of infection what part of the immune
    system partially controls pathogen growth?
  \item
    What is seroconversion?
  \item
    What happens in HIV disease that results in fluctuations in the
    amount of anti-HIV antibodies?
  \end{enumerate}
\item
  The mucosal Immune Response

  \begin{enumerate}
  \def\labelenumii{\arabic{enumii}.}
  \item
    \_\_\_\_\_\_\_\_\_\_\_\_\_\_\_\_\_\_\_\_\_ is a process that coats a
    pathogen with antibodies and prevents the pathogen from binding to
    receptors.
  \item
    What role does neutralization play in the influenza vaccine?
  \item
    How do the microfold, or M cells, of the intestines help the body to
    mount an immune response if necessary?
  \end{enumerate}
\item
  Defenses against Bacteria and Fungi

  \begin{enumerate}
  \def\labelenumii{\arabic{enumii}.}
  \item
    \emph{Mycobacterium leprae} can survive inside of macrophages. The
    \_\_\_\_\_\_\_\_\_\_\_\_\_\_\_\_\_\_\_
    \_\_\_\_\_\_\_\_\_\_\_\_\_\_\_\_\_\_
    \_\_\_\_\_\_\_\_\_\_\_\_\_\_\_\_\_\_\_ pathways helps macrophages to
    clear the bacteria through the use of nitric oxide.
  \item
    What are opportunistic infections?
  \end{enumerate}
\item
  Defenses against Parasites

  \begin{enumerate}
  \def\labelenumii{\arabic{enumii}.}
  \item
    IgE and eosinophils are both used to deal with what type of
    infection?
  \item
    How does IgE labeling of a parasite helps to clear the infection?
  \end{enumerate}
\item
  Defenses against Viruses

  \begin{enumerate}
  \def\labelenumii{\arabic{enumii}.}
  \item
    What are the three primary mechanisms the body uses against viruses?
  \item
    Why are antibodies ineffective against viruses inside of cells?
  \item
    Do interferons usually clear viral infections?
  \item
    How do cytotoxic T cells clear viral infections?
  \end{enumerate}
\item
  Evasion of the Immune System by Pathogens

  \begin{enumerate}
  \def\labelenumii{\arabic{enumii}.}
  \item
    Why is tuberculosis a chronic infection and not cleared by the
    immune system?
  \item
    Why is a \emph{Staphylococcus aureus} infection easier to treat than
    a methicillin resistant \emph{Staphylococcus aureus} strain?
  \item
    How does mutation and genetic recombination potentially lead to
    better evasion of the immune system in some pathogens?
  \end{enumerate}
\end{enumerate}

\subsection{21.6. Diseases Associated with Depressed or Overactive
Immune
Responses}\label{diseases-associated-with-depressed-or-overactive-immune-responses}

\subsubsection{}\label{section-22}

\begin{enumerate}
\def\labelenumi{\Roman{enumi}.}
\item
  Background
\item
  Immunodeficiencies

  \begin{enumerate}
  \def\labelenumii{\arabic{enumii}.}
  \item
    What is the difference between inherited and acquired
    immunodeficiencies?
  \end{enumerate}

  \begin{enumerate}
  \def\labelenumii{\Alph{enumii}.}
  \item
    Inherited Immunodeficiencies

    \begin{enumerate}
    \def\labelenumiii{\arabic{enumiii}.}
    \item
      Why is SCID a sever inherited immunodeficiency? What portion of
      the immune system is compromised?

      \begin{enumerate}
      \def\labelenumiv{\alph{enumiv})}
      \item
        Why is a bone marrow transplant a treatment method for SCID?
      \end{enumerate}
    \end{enumerate}
  \item
    Human Immunodeficiency Virus/AIDS

    \begin{enumerate}
    \def\labelenumiii{\arabic{enumiii}.}
    \item
      Why is HIV unique among viruses that depress the immune system?
    \item
      How does HIV screening work?
    \item
      Why are the CD4\textsuperscript{+} receptor positive cells
      depleted in HIV?
    \end{enumerate}
  \end{enumerate}
\end{enumerate}

\begin{enumerate}
\def\labelenumi{\Roman{enumi}.}
\setcounter{enumi}{2}
\item
  Hypersensitivities

  \begin{enumerate}
  \def\labelenumii{\arabic{enumii}.}
  \item
    What is a ``hypersensitivity?''
  \end{enumerate}

  \begin{enumerate}
  \def\labelenumii{\Alph{enumii}.}
  \item
    Immediate (Type I) Hypersensitivity

    \begin{enumerate}
    \def\labelenumiii{\arabic{enumiii}.}
    \item
      Antigens that cause allergic responses are called
      \_\_\_\_\_\_\_\_\_\_\_\_\_\_\_.

      \begin{enumerate}
      \def\labelenumiv{\alph{enumiv})}
      \item
        The immediate hypersensitivity is due to which immunoglobulin?
      \end{enumerate}
    \item
      How quickly does a type I hypersensitivity onset?
    \item
      What are some potential triggers to a type I hypersensitivity
      reaction?
    \end{enumerate}
  \item
    Type II and Type III Hypersensitivities

    \begin{enumerate}
    \def\labelenumiii{\arabic{enumiii}.}
    \item
      Type \_\_\_\_\_\_ hypersensitivity occurs due to IgG mediated
      lysis of a cell by complement, and can occur due to a mismatched
      blood transfusion.
    \item
      Type \_\_\_\_\_\_ hypersensitivity occurs when a disease process
      results in an accumulation of DNA and other cellular materials
      along with antibodies accumulate and precipitate causing
      inflammation.
    \end{enumerate}
  \item
    Delayed (Type IV) Hypersensitivity

    \begin{enumerate}
    \def\labelenumiii{\arabic{enumiii}.}
    \item
      In delayed hypersensitivity the first exposure results in
      \_\_\_\_\_\_\_\_\_\_\_\_\_\_\_\_\_\_\_\_\_\_\_\_, and re-exposure
      results in a response of higher magnitude.
    \item
      How is a delayed hypersensitivity used to test for tuberculosis?
    \end{enumerate}
  \end{enumerate}
\item
  Autoimmune Responses

  \begin{enumerate}
  \def\labelenumii{\arabic{enumii}.}
  \item
    When tolerance breaks down an
    \_\_\_\_\_\_\_\_\_\_\_\_\_\_\_\_\_\_\_\_ response can occur.
  \item
    How could an environmental trigger leader to an autoimmune disease?
  \item
    Are there genetic factors in autoimmune disease? What genes play a
    role?
  \end{enumerate}
\end{enumerate}

\subsection{21.7. Transplantation and Cancer
Immunology}\label{transplantation-and-cancer-immunology}

\begin{enumerate}
\def\labelenumi{\Roman{enumi}.}
\item
  Background

  \begin{enumerate}
  \def\labelenumii{\arabic{enumii}.}
  \item
    What is tissue typing? What does it have to do with MHC molecules?
  \end{enumerate}
\item
  The Rh Factor

  \begin{enumerate}
  \def\labelenumii{\arabic{enumii}.}
  \item
    Would a person who is B- or B+ express the Rh antigen?
  \item
    Why would a mother who is Rh- mount an immune response against a Rh-
    fetus' red blood cells in erythroblastosis fetalis?
  \end{enumerate}
\item
  Tissue Transplantation

  \begin{enumerate}
  \def\labelenumii{\arabic{enumii}.}
  \item
    What are the two characteristics of MHC molecules that are important
    to consider in organ transplantation?
  \item
    What happens if a donor and recipient do not match?
  \item
    Why is graft-versus-host disease unique to bone marrow transplants
    and not to other organ transplants?
  \end{enumerate}
\item
  Immune Responses Against Cancer

  \begin{enumerate}
  \def\labelenumii{\arabic{enumii}.}
  \item
    What are some cancers that can be caused by a virus?
  \item
    What are the three stages of the immune response to cancer?
  \item
    What happens in escape that allows the cancer to continue growing?
  \end{enumerate}
\end{enumerate}

\section{}\label{section-23}

\section{\texorpdfstring{Chapter 22 }{Chapter 22 }}\label{chapter-22}

22.1. Organs and Structures of the Respiratory System

\begin{enumerate}
\def\labelenumi{\Roman{enumi}.}
\item
  Background

  \begin{enumerate}
  \def\labelenumii{\arabic{enumii}.}
  \item
    What are the primary functions of the respiratory system?
  \item
    What the two functional sections of the respiratory system, and what
    is the difference between them?
  \end{enumerate}
\item
  Conducting Zone

  \begin{enumerate}
  \def\labelenumii{\arabic{enumii}.}
  \item
    The conducting zone provides a route for air to pass through,
    removes debris, and what else?
  \end{enumerate}

  \begin{enumerate}
  \def\labelenumii{\Alph{enumii}.}
  \item
    The Nose and its Adjacent Structures

    \begin{enumerate}
    \def\labelenumiii{\arabic{enumiii}.}
    \item
      What feature of the external nose connects the root to the rest of
      the nose?
    \item
      What bones form the bridge of the nose?
    \item
      The nose is divided into the left and right sections by the
      \_\_\_\_\_\_\_\_\_\_\_\_\_ \_\_\_\_\_\_\_\_\_\_\_\_.
    \item
      What purposes do the conchae and nasal meatuses serve?
    \item
      Paranasal sinuses \_\_\_\_\_\_\_\_\_\_\_\_\_ and
      \_\_\_\_\_\_\_\_\_\_\_\_\_\_ incoming air.
    \item
      What purposes do the conchae and nasal meatuses serve?
    \item
      \_\_\_\_\_\_\_\_\_\_\_\_\_\_\_
      \_\_\_\_\_\_\_\_\_\_\_\_\_\_\_\_\_\_ lines the conchae, meatuses,
      and paranasal sinuses and is composed of pseudostratified ciliated
      columnar epithelium.
    \end{enumerate}
  \item
    Pharynx

    \begin{enumerate}
    \def\labelenumiii{\arabic{enumiii}.}
    \item
      What are the three major regions of the pharynx?
    \item
      What purpose does the nasopharynx serve? What passes through it?
    \item
      The \_\_\_\_\_\_\_\_\_\_\_\_\_\_\_\_\_\_\_\_\_\_\_ is a passageway
      for both air and food.
    \item
      What is the oropharynx connected to anteriorly?
    \item
      The laryngopharynx is connected to the
      \_\_\_\_\_\_\_\_\_\_\_\_\_\_ anteriorly and the
      \_\_\_\_\_\_\_\_\_\_\_\_\_\_\_ posteriorly.
    \end{enumerate}
  \item
    Larynx

    \begin{enumerate}
    \def\labelenumiii{\arabic{enumiii}.}
    \item
      The larynx connects the pharynx to the
      \_\_\_\_\_\_\_\_\_\_\_\_\_\_\_\_.
    \item
      What is the largest piece of cartilage in the thyroid?
    \item
      Where does the epiglottis rest when in the ``closed'' position?
    \item
      What purpose doe the true vocal cords serve?
    \item
      When swallowing how does the epiglottis prevent food from entering
      the trachea?
    \end{enumerate}
  \item
    Trachea

    \begin{enumerate}
    \def\labelenumiii{\arabic{enumiii}.}
    \item
      The trachea connects the larynx to the
      \_\_\_\_\_\_\_\_\_\_\_\_\_\_.
    \item
      What happens with the fibroelastic membrane of the trachea when
      inhaling and exhaling?
    \item
      What purpose do the C-shaped rings of cartilage in the trachea
      serve?
    \end{enumerate}
  \item
    Bronchial Tree

    \begin{enumerate}
    \def\labelenumiii{\arabic{enumiii}.}
    \item
      The trachea branches into the left and right primary bronchi at
      the carina. What functional purpose does the carina serve?
    \item
      What is the bronchial tree?
    \item
      Why are the walls of the bronchioles muscular?
    \end{enumerate}
  \end{enumerate}
\end{enumerate}

\begin{enumerate}
\def\labelenumi{\Roman{enumi}.}
\setcounter{enumi}{1}
\item
  \includegraphics[width=4.75903in,height=3.63472in,alt={Figure 22.10 Respiratory Zone Bronchioles lead to alveolar sacs in the respiratory zone, where gas exchange occurs.}]{images/media/image150.jpeg}Respiratory
  Zone

  \begin{enumerate}
  \def\labelenumii{\arabic{enumii}.}
  \item
    What is different functionally between the conducting and the
    respiratory zone?
  \item
    Where does the respiratory zone begin?
  \end{enumerate}

  \begin{enumerate}
  \def\labelenumii{\Alph{enumii}.}
  \item
    Alveoli

    \begin{enumerate}
    \def\labelenumiii{\arabic{enumiii}.}
    \item
      An \_\_\_\_\_\_\_\_\_\_\_\_\_\_\_\_ \_\_\_\_\_\_\_\_\_\_ leads to
      a cluster of alveoli, an alveolar sac.
    \item
      What cell type makes up the majority of the surface area of an
      alveolus?
    \item
      Type \_\_\_\_\_\_ alveolar cells are responsible for producing
      surfactant.
    \item
      What immune cell is present within the lungs?
    \item
      What makes up the respiratory membrane? What purpose does it
      serve?
    \end{enumerate}
  \end{enumerate}
\end{enumerate}

\subsection{22.2. The Lungs}\label{the-lungs}

\begin{enumerate}
\def\labelenumi{\Roman{enumi}.}
\item
  Background

  \begin{enumerate}
  \def\labelenumii{\arabic{enumii}.}
  \item
    What is the main function of the lungs?
  \item
    What is the epithelial surface of the lungs permeable to?
  \end{enumerate}
\item
  Gross Anatomy of the Lungs

  \begin{enumerate}
  \def\labelenumii{\arabic{enumii}.}
  \item
    What borders the lungs inferiorly?
  \item
    What encloses the lungs?
  \item
    Where is the cardiac notch in the lungs and why is it necessary?
  \item
    Lungs are divided into lobes. \_\_\_\_\_\_\_\_\_\_\_\_\_ separate
    the lobes.
  \item
    The left lung has \_\_\_\_\_\_\_ lobes, and the right lung has
    \_\_\_\_\_ lobes.
  \end{enumerate}
\item
  Blood Supply and Nervous Innervation of the Lungs

  \begin{enumerate}
  \def\labelenumii{\Alph{enumii}.}
  \item
    Blood Supply

    \begin{enumerate}
    \def\labelenumiii{\arabic{enumiii}.}
    \item
      Why is blood supply necessary for the lungs?
    \item
      The pulmonary artery carries
      \_\_\_\_\_\_\_\_\_\_\_\_\_\_\_\_\_\_\_\_ blood to the alveoli of
      the lungs.
    \item
      The pulmonary veins exit the lungs through the
      \_\_\_\_\_\_\_\_\_\_\_\_\_\_ and drain oxygenated blood to the
      left side of the heart.
    \end{enumerate}
  \item
    Nervous Innervation

    \begin{enumerate}
    \def\labelenumiii{\arabic{enumiii}.}
    \item
      Bronchoconstriction is controlled by the
      \_\_\_\_\_\_\_\_\_\_\_\_\_\_\_\_\_\_\_\_ system, and
      bronchodilation is controlled by the
      \_\_\_\_\_\_\_\_\_\_\_\_\_\_\_\_\_ system.
    \item
      What is the purpose of the pulmonary plexus?
    \end{enumerate}
  \end{enumerate}
\item
  Pleura of the Lungs

  \begin{enumerate}
  \def\labelenumii{\arabic{enumii}.}
  \item
    The \_\_\_\_\_\_\_\_\_\_\_\_\_\_ pleura is a serous membrane that
    surrounds the lung.
  \item
    What are the two layers of the pleura? What is the space between
    them?
  \item
    The pleura produce pleural fluid to lubricate and reduce
    \_\_\_\_\_\_\_\_\_\_\_\_\_\_\_\_\_, as well as help the lungs to
    enlarge due to surface tension.
  \end{enumerate}
\end{enumerate}

22.3. The Process of Breathing

\begin{enumerate}
\def\labelenumi{\Roman{enumi}.}
\item
  Background

  \begin{enumerate}
  \def\labelenumii{\arabic{enumii}.}
  \item
    Breathing, or \_\_\_\_\_\_\_\_\_\_\_\_\_\_\_
    \_\_\_\_\_\_\_\_\_\_\_\_\_\_\_\_, is movement of air into and out of
    the lungs.
  \item
    What are the three major pressures that drive pulmonary ventilation?
  \end{enumerate}
\item
  Mechanisms of Breathing

  \begin{enumerate}
  \def\labelenumii{\arabic{enumii}.}
  \item
    What pressures depend on physical features of the lungs?
  \end{enumerate}

  \begin{enumerate}
  \def\labelenumii{\Alph{enumii}.}
  \item
    Pressure Relationships

    \begin{enumerate}
    \def\labelenumiii{\arabic{enumiii}.}
    \item
      Inspiration and expiration depend on the difference between
      pressures where?
    \item
      What does Boyle's law tell us about the relationship between
      pressure and volume of a gas?
    \item
      What is a normal atmospheric pressure?
    \item
      Intra-alveolar pressure always equalizes with
      \_\_\_\_\_\_\_\_\_\_\_\_\_\_\_\_\_\_\_
      \_\_\_\_\_\_\_\_\_\_\_\_\_\_\_\_.
    \item
      \_\_\_\_\_\_\_\_\_\_\_\_\_\_\_\_\_\_ pressure remains about -4mm
      Hg lower than intra-alveolar pressure.
    \item
      The difference between intrapleural and intra-alveolar pressure is
      \_\_\_\_\_\_\_\_\_\_\_\_\_\_\_\_\_\_ pressure.
    \end{enumerate}
  \item
    Physical Factors Affecting Ventilation

    \begin{enumerate}
    \def\labelenumiii{\arabic{enumiii}.}
    \item
      Contraction and relaxation of what muscle fibers controls
      breathing?
    \item
      What is the relationship between flow and resistance?

      \begin{enumerate}
      \def\labelenumiv{\alph{enumiv})}
      \item
        What happens if the tube the air passes through narrows?
      \end{enumerate}
    \item
      What is thoracic wall compliance?
    \end{enumerate}
  \end{enumerate}
\end{enumerate}

\begin{enumerate}
\def\labelenumi{\Roman{enumi}.}
\setcounter{enumi}{1}
\item
  Pulmonary Ventilation

  \begin{enumerate}
  \def\labelenumii{\arabic{enumii}.}
  \item
    Air flows \_\_\_\_\_\_\_\_\_\_\_\_\_\_\_\_\_ a pressure gradient,
    from high pressure to low pressure.
  \item
    A respiratory cycle is on sequence of
    \_\_\_\_\_\_\_\_\_\_\_\_\_\_\_\_\_\_ and
    \_\_\_\_\_\_\_\_\_\_\_\_\_\_\_\_\_\_\_\_\_\_\_.
  \item
    What happens to the volume and pressure of the thoracic cavity when
    the diaphragm contracts?
  \item
    During quiet breathing is expiration active or passive? Why?
  \item
    During forced breathing is expiration active or passive? Why?
  \end{enumerate}
\item
  Respiratory Volumes and Capacities

  \begin{enumerate}
  \def\labelenumii{\arabic{enumii}.}
  \item
    What are the four major types of respiratory volumes?
  \item
    What is the respiratory volume that corresponds to movement of air
    during quiet breathing?
  \item
    \_\_\_\_\_\_\_\_\_\_\_\_\_\_\_\_ \_\_\_\_\_\_\_\_\_\_\_\_\_ (RV) is
    air that remains in the lungs to prevent the alveoli from
    collapsing.
  \item
    What are respiratory capacities?
  \item
    What capacity is the amount of air a person can move in and out of
    their lungs?
  \item
    \_\_\_\_\_\_\_\_\_\_\_\_\_\_\_\_\_\_\_\_\_\_
    \_\_\_\_\_\_\_\_\_\_\_\_\_\_\_\_\_\_\_
    \_\_\_\_\_\_\_\_\_\_\_\_\_\_\_ (FRC) is the amount of air that
    remains in the lung after a normal tidal expiration. It is the sum
    of the reserve and residual volumes.
  \item
    What is the difference between anatomical and alveolar dead space?
  \end{enumerate}
\item
  Respiratory Rate and Control of Ventilation

  \begin{enumerate}
  \def\labelenumii{\arabic{enumii}.}
  \item
    What is the normal respiratory rate of an adult?
  \end{enumerate}

  \begin{enumerate}
  \def\labelenumii{\Alph{enumii}.}
  \item
    Ventilation Control Centers

    \begin{enumerate}
    \def\labelenumiii{\arabic{enumiii}.}
    \item
      Where are the major brain regions responsible for pulmonary
      ventilation located in the brain?
    \item
      The \_\_\_\_\_\_\_\_\_\_\_\_\_\_\_\_\_\_\_
      \_\_\_\_\_\_\_\_\_\_\_\_\_\_\_\_\_\_
      \_\_\_\_\_\_\_\_\_\_\_\_\_\_\_\_\_\_\_ (DRG) is involved in
      constant breathing and the \_\_\_\_\_\_\_\_\_\_\_\_\_\_\_\_\_\_
      \_\_\_\_\_\_\_\_\_\_\_\_\_\_\_\_ \_\_\_\_\_\_\_\_\_\_\_\_\_\_\_
      (VRG) is involved in forced breathing.
    \item
      The apneustic center, located in the pons controls the
      \_\_\_\_\_\_\_\_\_\_\_\_\_\_\_ of inspiration.
    \item
      The pneumotaxic center controls the \_\_\_\_\_\_\_\_\_\_ of
      breathing.
    \end{enumerate}
  \item
    Factors That Affect the Rate and Depth of Respiration

    \begin{enumerate}
    \def\labelenumiii{\arabic{enumiii}.}
    \item
      The concentration of what gas is the major factor in stimulating
      respiration?
    \item
      Where are central and peripheral chemoreceptors located?
    \item
      Increased carbon dioxide in the blood leads to what effect on
      respiration?
    \item
      If peripheral chemoreceptors sense inadequate oxygenation what
      happens?
    \end{enumerate}
  \end{enumerate}
\end{enumerate}

\subsection{22.4. The Process of
Breathing}\label{the-process-of-breathing}

\begin{enumerate}
\def\labelenumi{\Roman{enumi}.}
\item
  Background

  \begin{enumerate}
  \def\labelenumii{\arabic{enumii}.}
  \item
    What is the purpose of the respiratory system?
  \item
    What gases are exchanged at the respiratory membrane?
  \end{enumerate}
\item
  Gas Exchange

  \begin{enumerate}
  \def\labelenumii{\Alph{enumii}.}
  \item
    Gas Laws and Air Composition

    \begin{enumerate}
    \def\labelenumiii{\arabic{enumiii}.}
    \item
      What two gases make up the vast majority of the air we breathe?

      \begin{enumerate}
      \def\labelenumiv{\alph{enumiv})}
      \item
        Which gas is the most abundant?
      \end{enumerate}
    \item
      What is partial pressure?

      \begin{enumerate}
      \def\labelenumiv{\alph{enumiv})}
      \item
        How can we use partial pressure to figure our total pressure?
      \end{enumerate}
    \item
      What is Dalton's Law?
    \item
      How can we use partial pressure to predict where a gas will move?
    \end{enumerate}
  \item
    Solubility of Gases in Liquids

    \begin{enumerate}
    \def\labelenumiii{\arabic{enumiii}.}
    \item
      What does Henry's law tell us about the solubility of a gas in a
      liquid?
    \item
      Why is the amount of water vapor greater in the lungs than in the
      atmosphere?
    \item
      Why is the concentration of carbon dioxide greater in the alveoli
      than in the atmosphere?
    \end{enumerate}
  \item
    Ventilation and Perfusion

    \begin{enumerate}
    \def\labelenumiii{\arabic{enumiii}.}
    \item
      \_\_\_\_\_\_\_\_\_\_\_\_\_\_\_\_\_ is the movement of air into and
      out of the lungs, and \_\_\_\_\_\_\_\_\_\_\_\_\_\_\_\_\_\_ is the
      flow of blood in the pulmonary capillaries.
    \item
      How does the body handle a situation in which the partial pressure
      difference across the respiratory membrane is narrowing?
    \item
      Ventilation is regulated by the diameter of the
      \_\_\_\_\_\_\_\_\_\_\_\_\_\_\_\_\_\_\_\_\_, and perfusion is
      regulated by the diameter of the \_\_\_\_\_\_\_\_\_\_\_\_\_\_\_
      \_\_\_\_\_\_\_\_\_\_\_\_\_\_\_\_\_.
    \end{enumerate}
  \end{enumerate}
\end{enumerate}

\begin{enumerate}
\def\labelenumi{\Roman{enumi}.}
\setcounter{enumi}{1}
\item
  Gas Exchange

  \begin{enumerate}
  \def\labelenumii{\arabic{enumii}.}
  \item
    Where are the two locations in the body where gas exchange occurs?
  \item
    What is the difference between internal and external respiration?
  \end{enumerate}

  \begin{enumerate}
  \def\labelenumii{\Alph{enumii}.}
  \item
    External Respiration

    \begin{enumerate}
    \def\labelenumiii{\arabic{enumiii}.}
    \item
      When gas exchange occurs across the respiratory membrane where
      does the majority of oxygen end up in the blood?

      \begin{enumerate}
      \def\labelenumiv{\alph{enumiv})}
      \item
        Does it primarily dissolve into the plasma?
      \end{enumerate}
    \item
      Why does oxygen diffuse from the alveoli to the blood? What is the
      partial pressure difference that drives this movement?
    \item
      Why does carbon dioxide diffuse from the blood to the alveoli?
      What is the partial pressure difference?
    \end{enumerate}
  \item
    Internal Respiration

    \begin{enumerate}
    \def\labelenumiii{\arabic{enumiii}.}
    \item
      Why does oxygen move from the blood to the tissues in internal
      respiration?
    \item
      Why is there more carbon dioxide in the tissues than in the blood?
    \end{enumerate}
  \end{enumerate}
\end{enumerate}

22.5. Transport of Gases

\begin{enumerate}
\def\labelenumi{\Roman{enumi}.}
\item
  Background

  \begin{enumerate}
  \def\labelenumii{\arabic{enumii}.}
  \item
    What is respiration? How is it different from ventilation?
  \end{enumerate}
\item
  Oxygen Transport in the Blood

  \begin{enumerate}
  \def\labelenumii{\arabic{enumii}.}
  \item
    Does oxygen dissolve into the blood easily? Is it soluble?
  \item
    What is the primary way oxygen is transported in the blood?
  \item
    What part of hemoglobin carries oxygen?
  \item
    \_\_\_\_\_\_\_\_\_\_\_\_\_\_\_\_\_\_ (Hb-O\textsubscript{2}) is
    formed when oxygen binds to hemoglobin.
  \end{enumerate}
\end{enumerate}

\includegraphics[width=3.125in,height=3.20833in,alt={This diagram shows a red blood cell and the structure of a hemoglobin molecule.}]{images/media/image151.jpeg}

\begin{enumerate}
\def\labelenumi{\Alph{enumi}.}
\item
  Function of Hemoglobin

  \begin{enumerate}
  \def\labelenumii{\arabic{enumii}.}
  \item
    How many protein subunits compose hemoglobin?
  \item
    As oxygen binds does that make is easier or more difficult for the
    next oxygen molecule to bind?
  \item
    What is a normal hemoglobin saturation in a healthy individual?
  \end{enumerate}
\item
  Oxygen Dissociation from Hemoglobin

  \begin{enumerate}
  \def\labelenumii{\arabic{enumii}.}
  \item
    What does an oxygen-hemoglobin dissociated curve describe?
  \item
    As the partial pressure of oxygen increases what happens to oxygen
    saturation of hemoglobin?
  \end{enumerate}
\end{enumerate}

\begin{quote}
\includegraphics[width=4.16667in,height=3.28125in,alt={The top panel of this figure shows a graph with oxygen saturation of the y-axis and partial pressure of oxygen on the x-axis.}]{images/media/image152.jpeg}
\end{quote}

\includegraphics[width=4.16667in,height=2.88542in,alt={The middle panel shows oxygen saturation versus partial pressure of oxygen as a function of pH.}]{images/media/image153.jpeg}

\begin{quote}
\includegraphics[width=4.16667in,height=2.88542in,alt={The bottom panel shows the same relationship as a function of temperature.}]{images/media/image154.jpeg}
\end{quote}

\begin{enumerate}
\def\labelenumi{\arabic{enumi}.}
\setcounter{enumi}{2}
\item
  Do highly metabolizing tissues need more or less oxygen than their
  normal metabolism counterparts?
\item
  Higher temperature promotes hemoglobin and oxygen to dissociate
  \_\_\_\_\_\_\_\_\_\_\_\_\_, and lower temperature
  \_\_\_\_\_\_\_\_\_\_\_\_\_\_\_\_ dissociation.
\item
  What is the Bohr effect?
\end{enumerate}

\begin{enumerate}
\def\labelenumi{\Alph{enumi}.}
\setcounter{enumi}{2}
\item
  Hemoglobin of the Fetus

  \begin{enumerate}
  \def\labelenumii{\arabic{enumii}.}
  \item
    Where do the mother and fetus exchange oxygen?
  \item
    How does the fetus receive enough oxygen when the difference between
    the maternal and the fetal blood is not large?
  \end{enumerate}
\end{enumerate}

\begin{quote}
\includegraphics[width=4.47917in,height=3.40625in,alt={This graph shows the oxygen saturation versus the partial pressure of oxygen in fetal hemoglobin and adult hemoglobin.}]{images/media/image155.jpeg}
\end{quote}

\begin{enumerate}
\def\labelenumi{\Roman{enumi}.}
\setcounter{enumi}{1}
\item
  Carbon Dioxide Transport in the Blood

  \begin{enumerate}
  \def\labelenumii{\arabic{enumii}.}
  \item
    What are the three major mechanisms used to transport carbon dioxide
    in the blood?
  \end{enumerate}
\end{enumerate}

\includegraphics[width=4.53125in,height=2.22917in,alt={This figure shows how carbon dioxide is transported from the tissue to the red blood cell.}]{images/media/image156.jpeg}

\begin{enumerate}
\def\labelenumi{\Alph{enumi}.}
\item
  Dissolved Carbon Dioxide

  \begin{enumerate}
  \def\labelenumii{\arabic{enumii}.}
  \item
    How much carbon dioxide is usually dissolved into the blood plasma?
  \end{enumerate}
\item
  Bicarbonate Buffer

  \begin{enumerate}
  \def\labelenumii{\arabic{enumii}.}
  \item
    How much of the carbon dioxide is bound up with bicarbonate?
  \item
    What enzyme allows for the rapid formation of carbonic acid from
    carbon dioxide and water?
  \item
    Carbonic acid dissociates into what two molecules?
  \item
    Why is the chloride shift important? What does it help to maintain?
  \end{enumerate}
\item
  Carbaminohemoglobin

  \begin{enumerate}
  \def\labelenumii{\arabic{enumii}.}
  \item
    How does hemoglobin transport carbon dioxide?
  \item
    What happens to the color of hemoglobin when it is not transporting
    oxygen?
  \item
    Why does carbon dioxide move from the tissue to the blood at
    systemic capillaries, and from the capillaries to the alveoli in the
    lungs?
  \item
    What is the Haldane effect, and what happens to the ability of
    carbon dioxide to bind to hemoglobin when its saturated with oxygen?
  \end{enumerate}
\end{enumerate}

\subsubsection{22.6. Modifications in Respiratory
Functions}\label{modifications-in-respiratory-functions}

\begin{enumerate}
\def\labelenumi{\Roman{enumi}.}
\item
  Background
\item
  Hyperpnea

  \begin{enumerate}
  \def\labelenumii{\arabic{enumii}.}
  \item
    \_\_\_\_\_\_\_\_\_\_\_\_\_\_\_\_\_\_ is an increase in rate and
    depth of ventilation.
  \item
    What might cause hyperpnea?
  \item
    How is hyperventilation different from hyperpnea?
  \item
    What might be responsible for hyperpnea in exercise developing
    before a drop in oxygen?
  \end{enumerate}
\item
  High Altitude Effects

  \begin{enumerate}
  \def\labelenumii{\arabic{enumii}.}
  \item
    What happens at higher altitudes that leads to lower hemoglobin
    saturation?
  \item
    What mechanisms help maintain oxygenation to the tissue at rest
    while at high altitudes?
  \item
    Why is it important to drink more fluid at altitude?
  \item
    What is AMS, and what can cause it?
  \end{enumerate}

  \begin{enumerate}
  \def\labelenumii{\Alph{enumii}.}
  \item
    Acclimatization

    \begin{enumerate}
    \def\labelenumiii{\arabic{enumiii}.}
    \item
      What is acclimatization and how can it help to prevent AMS?
    \item
      How does erythropoietin (EPO) help with acclimatization?
    \end{enumerate}
  \end{enumerate}
\end{enumerate}

22.7. Embryonic Development of the Respiratory System

\begin{enumerate}
\def\labelenumi{\Roman{enumi}.}
\item
  Background

  \begin{enumerate}
  \def\labelenumii{\arabic{enumii}.}
  \item
    When are breathing movements observed in development?

    \begin{enumerate}
    \def\labelenumiii{\alph{enumiii})}
    \item
      How is the fetus provided with oxygen until birth?
    \end{enumerate}
  \end{enumerate}
\item
  Time Line

  \begin{enumerate}
  \def\labelenumii{\arabic{enumii}.}
  \item
    When does the respiratory system being to develop?
  \end{enumerate}

  \begin{enumerate}
  \def\labelenumii{\Alph{enumii}.}
  \item
    Weeks 4-7

    \begin{enumerate}
    \def\labelenumiii{\arabic{enumiii}.}
    \item
      The olfactory pit enlarges to become what part of the respiratory
      tract?
    \item
      The lung bud becomes the \_\_\_\_\_\_\_\_\_\_\_\_\_\_\_\_.
    \item
      The bronchial buds become what segment of the respiratory tract?
    \end{enumerate}
  \end{enumerate}
\end{enumerate}

\begin{quote}
\includegraphics[width=4.58333in,height=3.47917in,alt={This flowchart shows the embryonic development of the respiratory system and correlates the gestational age to the appearance of new features.}]{images/media/image157.jpeg}
\end{quote}

\begin{enumerate}
\def\labelenumi{\Alph{enumi}.}
\setcounter{enumi}{1}
\item
  Weeks 7-16

  \begin{enumerate}
  \def\labelenumii{\arabic{enumii}.}
  \item
    When do respiratory bronchioles form?
  \end{enumerate}
\item
  Weeks 16-24

  \begin{enumerate}
  \def\labelenumii{\arabic{enumii}.}
  \item
    When do the cells of the respiratory tract differentiate?
  \item
    When are fetal breathing movements observed?
  \end{enumerate}
\item
  Weeks 24-Term

  \begin{enumerate}
  \def\labelenumii{\arabic{enumii}.}
  \item
    Pulmonary capillaries form for what purpose?
  \item
    At what point would a premature baby be expected to be able to
    breathe on its own?
  \item
    Is the respiratory system completely developed at birth?
  \end{enumerate}
\end{enumerate}

\begin{enumerate}
\def\labelenumi{\Roman{enumi}.}
\setcounter{enumi}{1}
\item
  Fetal ``Breathing''

  \begin{enumerate}
  \def\labelenumii{\arabic{enumii}.}
  \item
    What is inhaled and exhaled during fetal breathing?
  \item
    What purpose might fetal breathing serve?
  \end{enumerate}
\item
  Birth

  \begin{enumerate}
  \def\labelenumii{\arabic{enumii}.}
  \item
    What fills the lungs prior to birth?
  \item
    Why is pulmonary surfactant critical immediately after birth?
  \end{enumerate}
\item
\end{enumerate}

\section{}\label{section-24}

\section{\texorpdfstring{Chapter 23 }{Chapter 23 }}\label{chapter-23}

\subsection{23.1. Overview of the Digestive
System}\label{overview-of-the-digestive-system}

\begin{enumerate}
\def\labelenumi{\Roman{enumi}.}
\item
  Background

  \begin{enumerate}
  \def\labelenumii{\arabic{enumii}.}
  \item
    What is the function of the digestive system?
  \item
    What is an example of cooperation between the digestive system and
    another body system?
  \end{enumerate}
\item
  Digestive System Organs

  \begin{enumerate}
  \def\labelenumii{\arabic{enumii}.}
  \item
    What is the difference between the organs of the alimentary canal
    and the accessory digestive organs?
  \end{enumerate}

  \begin{enumerate}
  \def\labelenumii{\Alph{enumii}.}
  \item
    Alimentary Canal Organs

    \begin{enumerate}
    \def\labelenumiii{\arabic{enumiii}.}
    \item
      Where does the alimentary canal begin and end?
    \item
      As food passes from mouth to anus how can it pass from outside the
      body to the body's ``inner space.''
    \end{enumerate}
  \item
    Accessory Structures

    \begin{enumerate}
    \def\labelenumiii{\arabic{enumiii}.}
    \item
      What role do accessory digestive organs play in digestion?
    \item
      What are some of the accessory digestive organs?
    \end{enumerate}
  \end{enumerate}
\end{enumerate}

\begin{enumerate}
\def\labelenumi{\Roman{enumi}.}
\setcounter{enumi}{1}
\item
  Histology of the Alimentary Canal

  \begin{enumerate}
  \def\labelenumii{\arabic{enumii}.}
  \item
    What are the layers of the alimentary canal from lumen outwards?
  \item
    The \_\_\_\_\_\_\_\_\_\_\_\_\_\_\_\_\_\_ consists of epithelium and
    directly contacts the digested food.
  \item
    Why is it advantageous for the epithelium in the alimentary canal to
    have a short lifespan?
  \item
    Why is it beneficial for the MALT to be found in the \emph{lamina
    propria}?
  \item
    What is the purpose of the smooth muscle found in the
    \emph{muscularis mucosa}?
  \item
    What types of tissue are found in the submucosa?
  \item
    What is different about the muscularis externa of the stomach as
    compared to the small intestines?
  \item
    Where is the serosa found?

    \begin{enumerate}
    \def\labelenumiii{\alph{enumiii})}
    \item
      Where is the serosa not found?
    \end{enumerate}
  \end{enumerate}
\item
  Nerve Supply

  \begin{enumerate}
  \def\labelenumii{\arabic{enumii}.}
  \item
    Why are receptors and nerves necessary in the mouth?
  \item
    They myenteric and submucosal plexuses regulate what parts of
    digestion?
  \item
    What effect does sympathetic and parasympathetic activation have on
    digestion?
  \end{enumerate}
\item
  Blood Supply

  \begin{enumerate}
  \def\labelenumii{\arabic{enumii}.}
  \item
    What are the two function of blood vessels serving the digestive
    system?
  \item
    Lacteals absorb what digested components?
  \item
    What purpose does the hepatic portal system, and the liver, serve in
    blood supply to the digestive system?
  \end{enumerate}
\item
  The Peritoneum

  \begin{enumerate}
  \def\labelenumii{\arabic{enumii}.}
  \item
    What is the purpose of the fluid found between the visceral and
    parietal peritoneum?
  \item
    What does retroperitoneal mean, and what is an example of a
    retroperitoneal organ?
  \item
    What are the five major peritoneal folds?
  \end{enumerate}
\end{enumerate}

23.2 Digestive System Processes and Regulation

\begin{enumerate}
\def\labelenumi{\Roman{enumi}.}
\item
  Background

  \begin{enumerate}
  \def\labelenumii{\arabic{enumii}.}
  \item
    Does the mouth perform mechanical digestion, chemical digestion, or
    both?
  \item
    List 3 accessory organs and include their function.
  \item
    What is the major function of the large intestine?
  \end{enumerate}
\item
  Digestive Processes

  \begin{enumerate}
  \def\labelenumii{\arabic{enumii}.}
  \item
    What are the six processes of digestion?
  \item
    Where does ingestion occur?
  \item
    How does peristalsis move food through the digestive tract?
  \item
    Mastication and segmentation are examples of
    \_\_\_\_\_\_\_\_\_\_\_\_\_\_\_\_ digestion.
  \item
    Enzymes assist with \_\_\_\_\_\_\_\_\_\_\_\_\_\_\_\_\_\_ digestion.
  \item
    Where does the majority of absorption in the digestive tract occur?
  \item
    Where does defecation occur at in the digestive tract?
  \end{enumerate}
\item
  Regulatory Mechanisms

  \begin{enumerate}
  \def\labelenumii{\arabic{enumii}.}
  \item
    The digestive system is controlled both by
    \_\_\_\_\_\_\_\_\_\_\_\_\_\_\_\_\_\_ and
    \_\_\_\_\_\_\_\_\_\_\_\_\_\_\_\_\_\_\_ regulatory mechanisms.
  \end{enumerate}

  \begin{enumerate}
  \def\labelenumii{\Alph{enumii}.}
  \item
    Neural Controls

    \begin{enumerate}
    \def\labelenumiii{\arabic{enumiii}.}
    \item
      What do mechanoreceptors, chemoreceptors, and osmoreceptors
      detect?
    \item
      Extrinsic nerve plexuses stimulate \_\_\_\_\_\_\_\_\_\_\_\_
      reflexes, and intrinsic nerve plexuses stimulate
      \_\_\_\_\_\_\_\_\_\_\_\_\_\_\_\_\_ reflexes.
    \item
      What type of reflex is initiated when the sight or smell of food
      increase secretion of digestive juices?
    \end{enumerate}
  \item
    Hormonal Controls

    \begin{enumerate}
    \def\labelenumiii{\arabic{enumiii}.}
    \item
      What is the main digestive hormone of the stomach?
    \item
      CCK and secretin are secreted by what organ?
    \end{enumerate}
  \end{enumerate}
\end{enumerate}

\subsection{23.3 The Mouth, Pharynx, and
Esophagus}\label{the-mouth-pharynx-and-esophagus}

\begin{enumerate}
\def\labelenumi{\Roman{enumi}.}
\item
  Background
\item
  The Mouth

  \begin{enumerate}
  \def\labelenumii{\arabic{enumii}.}
  \item
    The cheeks, tongue, and palate make up the
    \_\_\_\_\_\_\_\_\_\_\_\_\_\_\_\_
    \_\_\_\_\_\_\_\_\_\_\_\_\_\_\_\_\_\_.
  \item
    What makes labia, or lips red?
  \item
    The \_\_\_\_\_\_\_\_\_\_\_\_\_\_ \_\_\_\_\_\_\_\_\_\_\_\_\_\_\_\_\_
    attaches the inner surface of the lip to the gum.
  \item
    The opening between the oral cavity and oropharynx is the
    \_\_\_\_\_\_\_\_\_\_\_\_\_\_\_\_.
  \item
    The maxillary and palatine bones make up the
    \_\_\_\_\_\_\_\_\_\_\_\_ palate, and skeletal muscle primarily
    composes the \_\_\_\_\_\_\_\_\_\_\_\_\_\_\_\_ palate.
  \item
    Between the palatoglossal and palatopharyngeal arches are the
    \_\_\_\_\_\_\_\_\_\_\_\_\_\_\_\_\_ tonsils.
  \end{enumerate}
\item
  \includegraphics[width=3.42986in,height=3.05in,alt={Figure 23.8 Tongue This superior view of the tongue shows the locations and types of lingual papillae.}]{images/media/image158.jpeg}The
  Tongue

  \begin{enumerate}
  \def\labelenumii{\arabic{enumii}.}
  \item
    What are at least three things the tongue doe to help in the
    digestive process?
  \item
    Why is it important to have a flexible tongue?
  \item
    How does the tongue aid in swallowing of a bolus of food?
  \item
    Fungiform papillae contain \_\_\_\_\_\_\_\_\_\_\_
    \_\_\_\_\_\_\_\_\_\_\_, and filiform papillae have
    \_\_\_\_\_\_\_\_\_\_\_\_\_\_\_\_\_
    \_\_\_\_\_\_\_\_\_\_\_\_\_\_\_\_\_.
  \item
    Lingual lipase helps to break down
    \_\_\_\_\_\_\_\_\_\_\_\_\_\_\_\_\_\_\_\_\_ in the stomach.
  \item
    The \_\_\_\_\_\_\_\_\_\_\_\_\_\_\_\_\_\_\_
    \_\_\_\_\_\_\_\_\_\_\_\_\_\_\_\_\_\_\_ attaches the tongue to the
    floor of the mouth.
  \end{enumerate}
\item
  The Salivary Glands

  \begin{enumerate}
  \def\labelenumii{\arabic{enumii}.}
  \item
    Does saliva secretion increase or decrease when eating?
  \end{enumerate}

  \begin{enumerate}
  \def\labelenumii{\Alph{enumii}.}
  \item
    The Major Salivary Glands

    \begin{enumerate}
    \def\labelenumiii{\arabic{enumiii}.}
    \item
      Where are the parotid glands located?
    \end{enumerate}
  \item
    Saliva

    \begin{enumerate}
    \def\labelenumiii{\arabic{enumiii}.}
    \item
      What is the major component of saliva?
    \item
      Salivary amylase initiates the breakdown of
      \_\_\_\_\_\_\_\_\_\_\_\_\_\_\_\_\_\_\_\_\_\_\_\_.
    \item
      What is the role of immunoglobin A and lysozyme in saliva?
    \end{enumerate}
  \item
    Regulation of Salivation

    \begin{enumerate}
    \def\labelenumiii{\arabic{enumiii}.}
    \item
      What effect do the parasympathetic and sympathetic branches of the
      autonomic nervous system have on salivation?
    \item
      How does the site or smell of food stimulate salivation?
    \end{enumerate}
  \end{enumerate}
\end{enumerate}

\begin{enumerate}
\def\labelenumi{\Roman{enumi}.}
\setcounter{enumi}{3}
\item
  \includegraphics[width=2.29375in,height=3.86042in,alt={Figure 23.10 Permanent and Deciduous Teeth This figure of two human dentitions shows the arrangement of teeth in the maxilla and mandible, and the relationship between the deciduous and permanent teeth.}]{images/media/image159.jpeg}The
  Teeth

  \begin{enumerate}
  \def\labelenumii{\arabic{enumii}.}
  \item
    Teeth are used for what purpose in the digestive process?
  \end{enumerate}

  \begin{enumerate}
  \def\labelenumii{\Alph{enumii}.}
  \item
    Types of Teeth

    \begin{enumerate}
    \def\labelenumiii{\arabic{enumiii}.}
    \item
      How many dentitions throughout a life?
    \item
      Baby teeth are \_\_\_\_\_\_\_\_\_\_\_\_\_\_\_\_\_\_\_\_\_\_\_\_\_
      teeth, and they are replaced by
      \_\_\_\_\_\_\_\_\_\_\_\_\_\_\_\_\_\_\_ teeth.
    \item
      What teeth are the most anterior and used for biting food?
    \item
      Canines, also known as \_\_\_\_\_\_\_\_\_\_\_\_\_\_\_\_ are best
      suited for piercing tough food.
    \item
      Premolars are just anterior to \_\_\_\_\_\_\_\_\_\_\_\_\_ which
      are used to crush food.
    \end{enumerate}
  \item
    Anatomy of a Tooth

    \begin{enumerate}
    \def\labelenumiii{\arabic{enumiii}.}
    \item
      \_\_\_\_\_\_\_\_\_\_\_\_\_\_\_\_, or the gums, surround the necks
      of the teeth.
    \item
      The top portion of a tooth above the gum is the
      \_\_\_\_\_\_\_\_\_\_\_\_\_\_\_\_\_, and the portion embedded is
      the \_\_\_\_\_\_\_\_\_\_\_\_\_\_\_\_\_\_.
    \item
      The inside of the tooth contains the
      \_\_\_\_\_\_\_\_\_\_\_\_\_\_\_
      \_\_\_\_\_\_\_\_\_\_\_\_\_\_\_\_\_\_\_\_\_.
    \item
      Around the pulp cavity is the bone like tissue known as
      \_\_\_\_\_\_\_\_\_\_\_\_\_\_\_\_\_\_\_\_.
    \item
      What is cementum, and where is it located in relationship to
      enamel?
    \end{enumerate}
  \end{enumerate}
\end{enumerate}

\begin{enumerate}
\def\labelenumi{\Roman{enumi}.}
\setcounter{enumi}{4}
\item
  The Pharynx

  \begin{enumerate}
  \def\labelenumii{\arabic{enumii}.}
  \item
    The throat, also known as the
    \_\_\_\_\_\_\_\_\_\_\_\_\_\_\_\_\_\_\_\_, receives food from the
    mouth.
  \item
    Which segments of the pharynx are for air only, and which sections
    are for food an air?
  \item
    How is food directed towards the esophagus during swallowing?
  \end{enumerate}
\item
  \includegraphics[width=3.11389in,height=4.14653in,alt={Figure 23.13 Esophagus The upper esophageal sphincter controls the movement of food from the pharynx to the esophagus. The lower esophageal sphincter controls the movement of food from the esophagus to the stomach.}]{images/media/image160.jpeg}The
  Esophagus

  \begin{enumerate}
  \def\labelenumii{\arabic{enumii}.}
  \item
    The esophagus connects the \_\_\_\_\_\_\_\_\_\_\_\_\_\_\_\_\_\_ to
    the \_\_\_\_\_\_\_\_\_\_\_\_\_\_\_\_\_\_\_\_.
  \end{enumerate}

  \begin{enumerate}
  \def\labelenumii{\Alph{enumii}.}
  \item
    Passage of Food through the Esophagus

    \begin{enumerate}
    \def\labelenumiii{\arabic{enumiii}.}
    \item
      What happens when the upper esophageal sphincter relaxes?
    \item
      What is the purpose of the lower esophageal sphincter?
    \item
      What happens when the lower esophageal sphincter malfunctions?
    \end{enumerate}
  \item
    Histology of the Esophagus

    \begin{enumerate}
    \def\labelenumiii{\arabic{enumiii}.}
    \item
      Why does the esophagus have an adventitia and not a serosa?
    \end{enumerate}
  \end{enumerate}
\end{enumerate}

\begin{enumerate}
\def\labelenumi{\Roman{enumi}.}
\setcounter{enumi}{6}
\item
  Deglutition

  \begin{enumerate}
  \def\labelenumii{\arabic{enumii}.}
  \item
    Deglutition moves food from the \_\_\_\_\_\_\_\_\_\_\_\_\_ to the
    \_\_\_\_\_\_\_\_\_\_\_\_\_\_\_\_\_\_\_\_\_.
  \item
    What are the three stages of deglutition?
  \end{enumerate}

  \begin{enumerate}
  \def\labelenumii{\Alph{enumii}.}
  \item
    The Voluntary Phase

    \begin{enumerate}
    \def\labelenumiii{\arabic{enumiii}.}
    \item
      How does the tongue facilitate the voluntary phase?
    \end{enumerate}
  \item
    The Pharyngeal Phase

    \begin{enumerate}
    \def\labelenumiii{\arabic{enumiii}.}
    \item
      During the pharyngeal phase deglutition apnea occurs. What is
      deglutition apnea?
    \end{enumerate}
  \item
    The Esophageal Phase

    \begin{enumerate}
    \def\labelenumiii{\arabic{enumiii}.}
    \item
      What is the role of peristalsis in the esophageal phase?
    \item
      How do the circular and longitudinal muscles work together to move
      the bolus of food to the stomach?
    \item
      What is the purpose of the mucous lining the esophagus?
    \end{enumerate}
  \end{enumerate}
\end{enumerate}

23.4 The Stomach

\begin{enumerate}
\def\labelenumi{\Roman{enumi}.}
\item
  Background

  \begin{enumerate}
  \def\labelenumii{\arabic{enumii}.}
  \item
    The stomach is attached to the inferior end of the esophagus and
    leads to the first part of the small intestine, the
    \_\_\_\_\_\_\_\_\_\_\_\_\_\_\_\_\_\_\_\_\_\_\_.
  \item
    The stomach can expand more than 75 times its empty volume, but when
    empty it is about the size of a \_\_\_\_\_\_\_\_\_\_\_\_\_.
  \item
    Is anything absorbed in the stomach?
  \end{enumerate}
\item
  Structure

  \begin{enumerate}
  \def\labelenumii{\arabic{enumii}.}
  \item
    What are the four main regions of the stomach?
  \item
    The \_\_\_\_\_\_\_\_\_\_\_\_\_\_\_\_\_ connects to the esophagus.
  \item
    The dome shaped region of the stomach that extends above the cardia
    is the \_\_\_\_\_\_\_\_\_\_\_\_\_\_\_\_\_\_\_.
  \item
    Below the fundus is the main part of the stomach which is the
    \_\_\_\_\_\_\_\_\_\_\_\_\_\_\_\_\_.
  \item
    The pyloric canal connects the stomach to the
    \_\_\_\_\_\_\_\_\_\_\_\_\_\_\_\_\_\_\_\_\_\_\_\_.
  \item
    The pyloric sphincter controls what?
  \item
    What are rugae and what purpose do they serve?
  \end{enumerate}
\item
  Histology

  \begin{enumerate}
  \def\labelenumii{\arabic{enumii}.}
  \item
    What additional layer of the muscularis is present in the stomach
    and why?
  \item
    What are gastric pits and gastric glands?
  \item
    What hormone is secreted by the gastric glands?
  \item
    \_\_\_\_\_\_\_\_\_\_\_\_\_ cells secrete hydrochloric (HCl) acid and
    intrinsic factor?
  \item
    Chief cells secrete pepsinogen, the inactive form of
    \_\_\_\_\_\_\_\_\_\_\_\_\_\_.
  \item
    Mucous neck cells secrete \_\_\_\_\_\_\_\_\_\_\_\_\_.
  \item
    Enteroendocrine cells, or G cells, secrete the hormone
    \_\_\_\_\_\_\_\_\_\_\_\_\_\_\_\_\_\_\_.
  \end{enumerate}
\item
  Gastric Secretion

  \begin{enumerate}
  \def\labelenumii{\arabic{enumii}.}
  \item
    The \_\_\_\_\_\_\_\_\_\_\_\_\_\_\_ phase happens before food enters
    the stomach.
  \item
    The gastric phase is initiated by the entry of \_\_\_\_\_\_\_\_\_\_
    in the stomach.
  \item
    The intestinal phase initially stimulates stomach secretory
    activity, but later inhibits it after chyme distends the
    \_\_\_\_\_\_\_\_\_\_\_\_\_\_\_\_\_\_\_\_.
  \end{enumerate}
\item
  The Mucosal Barrier

  \begin{enumerate}
  \def\labelenumii{\arabic{enumii}.}
  \item
    Why does the stomach need a mucosal barrier?
  \item
    How does bicarbonate help to prevent damage to the stomach?
  \item
    How frequently does the stomach lining get replaced? Why?
  \end{enumerate}
\item
  Digestive Functions of the Stomach

  \begin{enumerate}
  \def\labelenumii{\arabic{enumii}.}
  \item
    What digestive activities does the stomach not participate in?
  \end{enumerate}

  \begin{enumerate}
  \def\labelenumii{\Alph{enumii}.}
  \item
    Mechanical Digestion

    \begin{enumerate}
    \def\labelenumiii{\arabic{enumiii}.}
    \item
      What are mixing waves, and why are they essential to stomach
      digestion?
    \item
      How does gastric emptying occur? How much chyme is passed out of
      the stomach at a time?
    \end{enumerate}
  \item
    Chemical Digestion

    \begin{enumerate}
    \def\labelenumiii{\arabic{enumiii}.}
    \item
      What role does the fundus play in digestion?
    \item
      When is salivary amylase active, and lingual lipase active?
    \item
      What is the purpose of intrinsic factor?
    \end{enumerate}
  \end{enumerate}
\end{enumerate}

\subsubsection{23.5 The Small and Large
Intestines}\label{the-small-and-large-intestines}

\begin{enumerate}
\def\labelenumi{\Roman{enumi}.}
\item
  Background

  \begin{enumerate}
  \def\labelenumii{\arabic{enumii}.}
  \item
    What digestive function do the intestines not perform?
  \end{enumerate}
\item
  The Small Intestine

  \begin{enumerate}
  \def\labelenumii{\arabic{enumii}.}
  \item
    Why is the small intestine the primary digestive organ?
  \item
    Chyme that enters the small intestine comes from what organ?
  \end{enumerate}

  \begin{enumerate}
  \def\labelenumii{\Alph{enumii}.}
  \item
    Structure

    \begin{enumerate}
    \def\labelenumiii{\arabic{enumiii}.}
    \item
      What is the sequence that chyme pass through the small intestine?
      Firs the \_\_\_\_\_\_\_\_\_\_\_\_\_\_\_\_\_\_, then
      \_\_\_\_\_\_\_\_\_\_\_\_\_\_\_\_\_\_ and finally the
      \_\_\_\_\_\_\_\_\_\_\_\_\_\_\_\_\_.
    \item
      The bile duct and main pancreatic duct pass their contents into
      the duodenum and the
      \_\_\_\_\_\_\_\_\_\_\_\_\_\_\_\_\_\_\_\_\_\_\_\_\_\_\_\_\_\_\_\_
      \_\_\_\_\_\_\_\_\_\_\_\_\_ also known as the ampulla of Vater.
    \item
      What sphincter regulates the flow of bile and pancreatic juice
      into the duodenum?
    \item
      What is the middle section of the small intestine?
    \item
      The ileocecal sphincter is the junction between what two parts of
      the digestive tract?
    \item
      What artery supplies blood to the small intestines?
    \end{enumerate}
  \item
    Histology

    \begin{enumerate}
    \def\labelenumiii{\arabic{enumiii}.}
    \item
      What are some ways the mucosa and submucosa are modified to
      increase the absorptive surface area?
    \item
      What is a circular fold, and how does it modify how food passes
      through the intestines?
    \item
      What is found within the villi?

      \begin{enumerate}
      \def\labelenumiv{\alph{enumiv})}
      \item
        What purpose do lacteals serve?
      \end{enumerate}
    \item
      What composes the brush border?
    \item
      Intestinal juice is produced in \_\_\_\_\_\_\_\_\_\_\_\_\_\_\_
      \_\_\_\_\_\_\_\_\_\_\_\_\_\_\_\_.
    \item
      What do duodenal glands produce? Why is this product important in
      the duodenum?
    \item
      MALT is found in what layer of the intestines?
    \end{enumerate}
  \item
    \includegraphics[width=2.32847in,height=2.54931in,alt={Figure 23.20 Segmentation Segmentation separates chyme and then pushes it back together, mixing it and providing time for digestion and absorption.}]{images/media/image161.jpeg}Mechanical
    Digestion in the Small Intestine

    \begin{enumerate}
    \def\labelenumiii{\arabic{enumiii}.}
    \item
      What is the purpose of segmentation in the small intestines?
    \item
      What does motilin signal for?
    \item
      The migrating motility complex serves what purpose in intestinal
      digestion?
    \item
      What is the gastroileal reflex?
    \end{enumerate}
  \item
    Chemical Digestion in the Small Intestine

    \begin{enumerate}
    \def\labelenumiii{\arabic{enumiii}.}
    \item
      Do lipids undergo any substantial digestion before entering the
      small intestines?
    \item
      Why is chyme delivered the intestines in small amounts?
    \end{enumerate}
  \end{enumerate}
\end{enumerate}

\begin{enumerate}
\def\labelenumi{\Roman{enumi}.}
\setcounter{enumi}{2}
\item
  The Large Intestine

  \begin{enumerate}
  \def\labelenumii{\arabic{enumii}.}
  \item
    What is the purpose of large intestine?
  \end{enumerate}

  \begin{enumerate}
  \def\labelenumii{\Alph{enumii}.}
  \item
    Structure

    \begin{enumerate}
    \def\labelenumiii{\arabic{enumiii}.}
    \item
      The large intestine starts and end where in the digestive tract?
    \end{enumerate}
  \item
    Subdivisions

    \begin{enumerate}
    \def\labelenumiii{\arabic{enumiii}.}
    \item
      The first part of the large intestines, the
      \_\_\_\_\_\_\_\_\_\_\_\_\_\_\_, receives digested contents from
      the small intestines.
    \item
      What has the appendix been postulated to do?
    \item
      What is the pathway of food through the segments of the colon?
    \item
      What are the flexures and where are they located?
    \end{enumerate}
  \end{enumerate}
\end{enumerate}

\begin{quote}
\includegraphics[width=3.64583in,height=2.45833in,alt={This image shows the large intestine; the major parts of the large intestine are labeled.}]{images/media/image162.jpeg}
\end{quote}

\begin{enumerate}
\def\labelenumi{\arabic{enumi}.}
\setcounter{enumi}{4}
\item
  What is the purpose of the rectal valves?
\item
  What are the two sphincters of the anal canal?

  \begin{enumerate}
  \def\labelenumii{\alph{enumii})}
  \item
    Which is voluntary?
  \end{enumerate}
\end{enumerate}

\begin{enumerate}
\def\labelenumi{\Alph{enumi}.}
\setcounter{enumi}{2}
\item
  Histology

  \begin{enumerate}
  \def\labelenumii{\arabic{enumii}.}
  \item
    Compared to the small intestines, what do the large intestines not
    have?
  \item
    What is absorbed by the enterocytes of the large intestine?
  \end{enumerate}
\end{enumerate}

\begin{quote}
\includegraphics[width=4.63542in,height=3.9401in,alt={This image shows the histological cross section of the large intestine. The left panel shows a small region of the large intestine. The center panel shows a magnified view of this region, highlighting the openings of the intestinal glands. The right panel shows a further magnified view, with the microvilli and goblet cells.}]{images/media/image163.jpeg}
\end{quote}

\begin{enumerate}
\def\labelenumi{\Alph{enumi}.}
\setcounter{enumi}{3}
\item
  Anatomy

  \begin{enumerate}
  \def\labelenumii{\arabic{enumii}.}
  \item
    The teniae coli have tonic contractions that bunch the colon up into
    \_\_\_\_\_\_\_\_\_\_\_\_\_\_\_\_\_.
  \end{enumerate}
\end{enumerate}

\begin{quote}
\includegraphics[width=3in,height=3.29063in,alt={This image shows the Taenia Coli, haustra and epiploic appendages, which are parts of the large intestine.}]{images/media/image164.jpeg}
\end{quote}

\begin{enumerate}
\def\labelenumi{\arabic{enumi}.}
\setcounter{enumi}{1}
\item
  What is an anal column?
\item
  Why is there a difference in sensitivity above and below the pectinate
  line?
\end{enumerate}

\begin{enumerate}
\def\labelenumi{\Alph{enumi}.}
\setcounter{enumi}{4}
\item
  Bacterial Flora

  \begin{enumerate}
  \def\labelenumii{\arabic{enumii}.}
  \item
    What do the normal bacterial flora provide that is beneficial?
  \end{enumerate}
\item
  Digestive Functions of the Large Intestine

  \begin{enumerate}
  \def\labelenumii{\arabic{enumii}.}
  \item
    What is the difference between haustral contractions and mass
    movement?
  \item
    How does distension in the stomach lead to increased motility in the
    large intestines?
  \item
    What is saccharolytic fermentation that bacteria take part in? How
    does it lead to flatus?
  \end{enumerate}
\item
  Absorption, Feces Formation, and Defecation

  \begin{enumerate}
  \def\labelenumii{\arabic{enumii}.}
  \item
    What do the large intestines remove to convert liquid chyme into
    feces?
  \item
    Why does the Valsalva maneuver aid in defecation?
  \end{enumerate}
\end{enumerate}

23.6 Accessory Organs in Digestion: The Liver, Pancreas, and Gallbladder

\begin{enumerate}
\def\labelenumi{\Roman{enumi}.}
\item
  \includegraphics[width=2.69792in,height=3.35486in,alt={This diagram shows the accessory organs of the digestive system. The liver, spleen, pancreas, gallbladder and their major parts are shown.}]{images/media/image165.jpeg}Background

  \begin{enumerate}
  \def\labelenumii{\arabic{enumii}.}
  \item
    What are three accessory organs that contribute to digestion in the
    small intestine?
  \end{enumerate}
\item
  The Liver

  \begin{enumerate}
  \def\labelenumii{\arabic{enumii}.}
  \item
    Where is the liver located in the abdomen?
  \item
    The hepatic artery and hepatic portal vein enter the liver at the
    \_\_\_\_\_\_\_\_\_\_\_\_\_\_ \_\_\_\_\_\_\_\_\_\_\_\_\_\_\_\_\_.
  \item
    What does the hepatic portal vein deliver to the liver?
  \end{enumerate}

  \begin{enumerate}
  \def\labelenumii{\Alph{enumii}.}
  \item
    Histology

    \begin{enumerate}
    \def\labelenumiii{\arabic{enumiii}.}
    \item
      Plates of hepatocytes are found in each
      \_\_\_\_\_\_\_\_\_\_\_\_\_\_\_\_\_\_
      \_\_\_\_\_\_\_\_\_\_\_\_\_\_\_\_\_.
    \item
      Bile created by hepatocytes accumulates in
      \_\_\_\_\_\_\_\_\_\_\_\_\_\_\_
      \_\_\_\_\_\_\_\_\_\_\_\_\_\_\_\_\_\_\_\_\_\_\_\_\_\_ to be taken
      to the bile ductules and bile ducts.
    \item
      The common hepatic duct joins with the cystic duct to form the
      \_\_\_\_\_\_\_\_\_\_\_\_\_\_\_\_\_\_\_
      \_\_\_\_\_\_\_\_\_\_\_\_\_\_\_\_\_\_\_\_\_
      \_\_\_\_\_\_\_\_\_\_\_\_\_\_\_\_\_\_\_\_\_\_\_\_ which takes bile
      to the small intestines.
    \item
      What is a hepatic sinusoid?
    \item
      \includegraphics[width=3.34375in,height=3.82143in,alt={This image shows the microscopic anatomy of the liver. The top panel shows the liver; the center panel shows a magnified view of the connective tissue and the lobules. The bottom panel shows a further magnified view of a lobule, identifying the veins, bile duct and the sinusoids.}]{images/media/image166.jpeg}The
      hepatic sinusoids combine and send blood to a central veins which
      then flows through the \_\_\_\_\_\_\_\_\_\_\_\_\_\_\_\_\_
      \_\_\_\_\_\_\_\_\_\_\_\_\_\_\_\_ to get to the inferior vena cava.
      Bile flows the opposite way towards the ducts.
    \item
      What role do reticuloendothelial cells play in the liver?
    \item
      What are the three components of a portal triad?
    \end{enumerate}
  \item
    Bile

    \begin{enumerate}
    \def\labelenumiii{\arabic{enumiii}.}
    \item
      Bile emulsifies lipids. What is emulsification?
    \item
      The bile salts and phospholipids in bile increases the surface
      area available for what to interact with the separated lipids?
    \item
      How are bile salts reclaimed from the digestive tract?
    \item
      How is bilirubin produced in the body?
    \item
      How is bilirubin excreted from the body?
    \item
      When not actively digesting where is bile stored?
    \end{enumerate}
  \end{enumerate}
\end{enumerate}

\begin{enumerate}
\def\labelenumi{\Roman{enumi}.}
\setcounter{enumi}{1}
\item
  The Pancreas

  \begin{enumerate}
  \def\labelenumii{\arabic{enumii}.}
  \item
    What are the exocrine and endocrine functions of the pancreas?
  \end{enumerate}
\end{enumerate}

\begin{quote}
\includegraphics[width=3.64583in,height=4.72917in,alt={This figure shows the pancreas and its major parts. A magnified view of a small region of the pancreas shows the pancreatic islet cells, the acinar cells and the pancreatic duct.}]{images/media/image167.jpeg}
\end{quote}

\begin{enumerate}
\def\labelenumi{\arabic{enumi}.}
\setcounter{enumi}{1}
\item
  What do the acini in the pancreas produce?
\item
  The smaller pancreatic duct is the
  \_\_\_\_\_\_\_\_\_\_\_\_\_\_\_\_\_\_ duct.
\end{enumerate}

\begin{enumerate}
\def\labelenumi{\Alph{enumi}.}
\item
  Pancreatic Juice

  \begin{enumerate}
  \def\labelenumii{\arabic{enumii}.}
  \item
    What do the enzymes in pancreatic juice help the body to digest?
  \item
    Enteropeptidase from the brush border of the intestines is critical
    to the function of the protein degrading enzymes from the pancreas
    in what way?
  \end{enumerate}
\item
  Pancreatic Secretion

  \begin{enumerate}
  \def\labelenumii{\arabic{enumii}.}
  \item
    What regulates pancreatic secretion?
  \item
    What does the pancreas release to neutralize acidic chyme?
  \end{enumerate}
\end{enumerate}

\begin{enumerate}
\def\labelenumi{\Roman{enumi}.}
\setcounter{enumi}{2}
\item
  The Gallbladder

  \begin{enumerate}
  \def\labelenumii{\arabic{enumii}.}
  \item
    What is the purpose of the gallbladder?
  \item
    Bile from the gallbladder is ejected through the
    \_\_\_\_\_\_\_\_\_\_\_\_\_\_ duct.
  \end{enumerate}
\end{enumerate}

\begin{quote}
\includegraphics[width=3.45833in,height=2.81217in,alt={This figure shows the gallbladder and its major parts are labeled.}]{images/media/image168.jpeg}
\end{quote}

\paragraph[23.7 Chemical Digestion and Absorption: A Closer
Look]{\texorpdfstring{\protect\includegraphics[width=2.92708in,height=3.62958in,alt={This diagram identifies the functions of mechanical and chemical digestion and absorption at each organ. Next to each organ, a callout identifies which steps of digestion take place in that particular organ.}]{images/media/image169.jpeg}23.7
Chemical Digestion and Absorption: A Closer
Look}{This diagram identifies the functions of mechanical and chemical digestion and absorption at each organ. Next to each organ, a callout identifies which steps of digestion take place in that particular organ.23.7 Chemical Digestion and Absorption: A Closer Look}}\label{this-diagram-identifies-the-functions-of-mechanical-and-chemical-digestion-and-absorption-at-each-organ.-next-to-each-organ-a-callout-identifies-which-steps-of-digestion-take-place-in-that-particular-organ.23.7-chemical-digestion-and-absorption-a-closer-look}

\begin{enumerate}
\def\labelenumi{\Roman{enumi}.}
\item
  Background

  \begin{enumerate}
  \def\labelenumii{\arabic{enumii}.}
  \item
    What is the difference between mechanical and chemical digestion?
  \end{enumerate}
\item
  Chemical Digestion

  \begin{enumerate}
  \def\labelenumii{\Alph{enumii}.}
  \item
    Carbohydrate Digestion

    \begin{enumerate}
    \def\labelenumiii{\arabic{enumiii}.}
    \item
      Where does chemical digestion of carbohydrates begin?
    \item
      Pancreatic \_\_\_\_\_\_\_\_\_\_\_\_\_\_ breaks down carbohydrates
      in the small intestines.
    \item
      \_\_\_\_\_\_\_\_\_\_\_\_\_\_\_\_\_\_\_ breaks down α-dextrin one
      glucose at a time.
    \item
      Sucrose, lactose, and maltose are broken down by what three
      enzymes?
    \end{enumerate}
  \end{enumerate}
\end{enumerate}

\begin{quote}
\includegraphics[width=3.62135in,height=2.37975in,alt={This flow chart shows the steps in digestion of carbohydrates. The different levels shown are starch and glycogen, disaccharides and monosaccharides. Under each type of sugar, examples and the enzymes responsible for digestion are listed.}]{images/media/image170.jpeg}
\end{quote}

\begin{enumerate}
\def\labelenumi{\Alph{enumi}.}
\setcounter{enumi}{1}
\item
  Protein Digestion

  \begin{enumerate}
  \def\labelenumii{\arabic{enumii}.}
  \item
    Where does protein digestion start in the digestive system?
  \item
    What enzymes are secreted by the brush border to help breakdown
    proteins?
  \end{enumerate}
\end{enumerate}

\begin{quote}
\includegraphics[width=1.5in,height=3.09167in,alt={This flow chart shows the different steps in the digestion of protein. The four steps shown are protein, large polypeptides, short peptides and amino acids and amino acids.}]{images/media/image171.jpeg}\includegraphics[width=3.275in,height=4.09375in,alt={This diagrams shows the human digestive system and identifies the role of each organ in protein digestion. A text call-out next to each organ details the specific function.}]{images/media/image172.jpeg}
\end{quote}

\begin{enumerate}
\def\labelenumi{\Alph{enumi}.}
\setcounter{enumi}{2}
\item
  Lipid Digestion

  \begin{enumerate}
  \def\labelenumii{\arabic{enumii}.}
  \item
    Which lipase is the most important to digestion?

    \begin{enumerate}
    \def\labelenumiii{\alph{enumiii})}
    \item
      What organ produces it?
    \end{enumerate}
  \end{enumerate}
\item
  Nucleic Acid Digestion

  \begin{enumerate}
  \def\labelenumii{\arabic{enumii}.}
  \item
    What are the two types of pancreatic nucleases?

    \begin{enumerate}
    \def\labelenumiii{\alph{enumiii})}
    \item
      What do they digest?
    \end{enumerate}
  \item
    Nucleosidase and phosphatase are produced by what in the digestive
    tract?
  \end{enumerate}
\end{enumerate}

\begin{enumerate}
\def\labelenumi{\Roman{enumi}.}
\setcounter{enumi}{2}
\item
  \includegraphics[width=3.29097in,height=4.22639in,alt={This image shows the human digestive system. Next to each organ, a text callout identifies how water and digestive secretions such as saliva and bile are processed.}]{images/media/image173.jpeg}Absorption

  \begin{enumerate}
  \def\labelenumii{\arabic{enumii}.}
  \item
    Is 10 liters of food a reasonable amount to absorb per day?
  \item
    What are the five mechanisms used for absorption?
  \item
    What is different about absorption of water-soluble vs lipid-soluble
    nutrients?
  \end{enumerate}

  \begin{enumerate}
  \def\labelenumii{\Alph{enumii}.}
  \item
    Carbohydrate Absorption

    \begin{enumerate}
    \def\labelenumiii{\arabic{enumiii}.}
    \item
      Carbohydrates are broken down and absorbed in the form of
      \_\_\_\_\_\_\_\_\_\_\_\_\_\_\_\_\_\_\_\_\_\_\_\_.
    \item
      What happens to indigestible fiber?
    \end{enumerate}
  \item
    Protein Absorption

    \begin{enumerate}
    \def\labelenumiii{\arabic{enumiii}.}
    \item
      Where are most proteins absorbed in the digestive tract?
    \item
      How long of amino acid chains are usually absorbed?
    \end{enumerate}
  \item
    Lipid Absorption

    \begin{enumerate}
    \def\labelenumiii{\arabic{enumiii}.}
    \item
      Where are the majority of lipids absorbed?
    \item
      What is a micelle?
    \item
      Chylomicrons are \_\_\_\_\_\_\_\_\_\_\_\_-soluble and enter into
      the lacteals after passing out of the cell.
    \item
      \_\_\_\_\_\_\_\_\_\_\_\_\_\_\_\_\_\_\_\_\_\_\_\_\_
      \_\_\_\_\_\_\_\_\_\_\_\_\_\_ breaks down the triglycerides of the
      chylomicrons.
    \end{enumerate}
  \item
    Nucleic Acid Absorption

    \begin{enumerate}
    \def\labelenumiii{\arabic{enumiii}.}
    \item
      How are nucleic acids absorbed?
    \end{enumerate}
  \item
    Mineral Absorption

    \begin{enumerate}
    \def\labelenumiii{\arabic{enumiii}.}
    \item
      All ions except \_\_\_\_\_\_\_\_\_\_\_\_\_\_\_ and
      \_\_\_\_\_\_\_\_\_\_\_\_\_\_\_\_ are always absorbed in the
      intestines even when not needed.
    \item
      What type of transport is used to absorb iron?
    \item
      Why do women have more iron transporters in their intestinal
      epithelium than men?
    \item
      What effect does PTH have on calcium absorption of dietary
      calcium?
    \end{enumerate}
  \item
    Vitamin Absorption

    \begin{enumerate}
    \def\labelenumiii{\arabic{enumiii}.}
    \item
      Where are fat soluble vitamins (A, D, E, and K) primarily get
      absorbed?
    \item
      How is absorption of B\textsubscript{12} different than other
      vitamins?

      \begin{enumerate}
      \def\labelenumiv{\alph{enumiv})}
      \item
        Why is intrinsic factor necessary to assist in the absorption?
      \end{enumerate}
    \end{enumerate}
  \item
    Water Absorption

    \begin{enumerate}
    \def\labelenumiii{\arabic{enumiii}.}
    \item
      Why does water move from the chyme into the epithelial cells?
    \end{enumerate}
  \end{enumerate}
\end{enumerate}

\section{}\label{section-25}

\section{\texorpdfstring{Chapter 24 }{Chapter 24 }}\label{chapter-24}

24.1. Overview of Metabolic Reactions

\begin{enumerate}
\def\labelenumi{\Roman{enumi}.}
\item
  Background

  \begin{enumerate}
  \def\labelenumii{\arabic{enumii}.}
  \item
    \_\_\_\_\_\_\_\_\_\_\_\_\_\_\_\_\_\_\_\_\_ is the sum of a chemical
    reactions in the body.
  \item
    What is the difference between catabolism and anabolism?
  \end{enumerate}
\item
  Catabolic Reactions

  \begin{enumerate}
  \def\labelenumii{\arabic{enumii}.}
  \item
    Why is ATP an important product of catabolic reactions?
  \item
    How can the energy stored in ATP be used? What portion of the ATP is
    removed?
  \end{enumerate}
\end{enumerate}

\begin{quote}
\includegraphics[width=4.42708in,height=3.25189in,alt={This diagram shows the chemical structure of adenosine triphosphate, and how different reactions add or remove phosphate groups.}]{images/media/image174.jpeg}

\includegraphics[width=3.97917in,height=3.65012in,alt={This flowchart shows how food is modified into lipids, carbohydrates, and protein, and the various catabolic reactions which convert food into energy.}]{images/media/image175.jpeg}
\end{quote}

\begin{enumerate}
\def\labelenumi{\arabic{enumi}.}
\setcounter{enumi}{2}
\item
  What is the most common source of energy to fuel the body?
\item
  β-Oxidation is used to breakdown what type of macromolecule?
\item
  Amino acids are monomers of \_\_\_\_\_\_\_\_\_\_\_\_\_\_\_\_\_\_\_\_.
\end{enumerate}

\begin{enumerate}
\def\labelenumi{\Roman{enumi}.}
\setcounter{enumi}{2}
\item
  Anabolic Reactions

  \begin{enumerate}
  \def\labelenumii{\arabic{enumii}.}
  \item
    Anabolic reactions are also known and
    \_\_\_\_\_\_\_\_\_\_\_\_\_\_\_\_\_\_\_\_\_\_\_\_\_\_ reactions.
  \end{enumerate}
\item
  Hormonal Regulation of Metabolism

  \begin{enumerate}
  \def\labelenumii{\arabic{enumii}.}
  \item
    What is an example of a catabolic hormone?
  \item
    Insulin, testosterone, and estrogen are all examples of
    \_\_\_\_\_\_\_\_\_\_\_\_\_\_\_ hormones.
  \end{enumerate}
\item
  Oxidation-Reduction Reactions

  \begin{enumerate}
  \def\labelenumii{\arabic{enumii}.}
  \item
    The loss of an electron is \_\_\_\_\_\_\_\_\_\_\_\_\_\_\_\_\_\_, and
    the electron is passed on to another molecule which is
    \_\_\_\_\_\_\_\_\_\_\_\_\_\_\_\_\_\_.

    \begin{enumerate}
    \def\labelenumiii{\alph{enumiii})}
    \item
      These two reactions happen together in an
      \_\_\_\_\_\_\_\_\_\_\_\_\_\_\_\_\_\_\_\_-\_\_\_\_\_\_\_\_\_\_\_\_\_\_
      reaction.
    \end{enumerate}
  \item
    NAD is reduced to \_\_\_\_\_\_\_\_\_\_\_\_\_\_\_\_\_\_\_, and FAD is
    reduced to \_\_\_\_\_\_\_\_\_\_\_\_\_\_\_\_.
  \end{enumerate}
\end{enumerate}

\subsection{24.2. Carbohydrate
Metabolism}\label{carbohydrate-metabolism}

\begin{enumerate}
\def\labelenumi{\Roman{enumi}.}
\item
  Background

  \begin{enumerate}
  \def\labelenumii{\arabic{enumii}.}
  \item
    What are polysaccharides and monosaccharides?

    \begin{enumerate}
    \def\labelenumiii{\alph{enumiii})}
    \item
      What is the difference between them?
    \end{enumerate}
  \item
    What purpose does salivary amylase serve in the mouth?
  \item
    What is produced form cellular respiration that can be used for
    energy?
  \end{enumerate}
\end{enumerate}

\begin{quote}
\includegraphics[width=3.88542in,height=4.33373in,alt={This figure shows the different pathways of cellular respiration. The pathways shown are glycolysis, the pyruvic acid cycle, the Krebs cycle, and oxidative phosphorylation.}]{images/media/image176.jpeg}
\end{quote}

\begin{enumerate}
\def\labelenumi{\Roman{enumi}.}
\setcounter{enumi}{1}
\item
  Glycolysis

  \begin{enumerate}
  \def\labelenumii{\arabic{enumii}.}
  \item
    Glycolysis transfers some energy from glucose to ADP to produce
    \_\_\_\_\_\_\_\_\_\_.
  \item
    What is the product at the last step of glycolysis?
  \item
    Glucose, a 6 carbon sugar, is broken down into what three carbon
    molecules?
  \item
    What is the citric acid cycle or the TCA?

    \begin{enumerate}
    \def\labelenumiii{\alph{enumiii})}
    \item
      When does pyruvate enter the cycle?
    \end{enumerate}
  \item
    What is the energy-consuming phase of glycolysis?

    \begin{enumerate}
    \def\labelenumiii{\alph{enumiii})}
    \item
      \includegraphics[width=4.29236in,height=6.16667in,alt={This flowchart shows the different steps in glycolysis in detail. The top panel shows the energy-consuming phase, the middle panel shows the coupling of phosphorylation with oxidation, and the bottom panel shows the energy-releasing phase.}]{images/media/image177.jpeg}Beyond
      the energy-consuming phase what is the net gain or ATP from
      glycolysis?
    \end{enumerate}
  \item
    Why is the addition of a phosphate to glucose to create
    glucose-6-phosphate important?
  \item
    Hexokinase is found nearly everywhere in the body, but where is
    glucokinase found?
  \item
    What happens overall in the energy-yielding phase of glycolysis?

    \begin{enumerate}
    \def\labelenumiii{\alph{enumiii})}
    \item
      Why is this important?
    \end{enumerate}
  \item
    Does the pyruvate produced from glycolysis always enter the citric
    acid cycle?
  \end{enumerate}

  \begin{enumerate}
  \def\labelenumii{\Alph{enumii}.}
  \item
    Anaerobic Respiration

    \begin{enumerate}
    \def\labelenumiii{\arabic{enumiii}.}
    \item
      When oxygen is not available pyruvate is converted to
      \_\_\_\_\_\_\_\_\_\_\_\_\_ \_\_\_\_\_\_\_\_\_\_\_\_\_\_ through
      anaerobic means.
    \item
      Why does exercise lead to the production of lactic acid?
    \end{enumerate}
  \item
    Aerobic Respiration

    \begin{enumerate}
    \def\labelenumiii{\arabic{enumiii}.}
    \item
      In aerobic respiration oxygen serves as the
      \_\_\_\_\_\_\_\_\_\_\_\_\_\_ \_\_\_\_\_\_\_\_\_\_\_\_\_\_\_
      acceptor.
    \end{enumerate}
  \end{enumerate}
\end{enumerate}

\begin{enumerate}
\def\labelenumi{\Roman{enumi}.}
\setcounter{enumi}{1}
\item
  Krebs Cycle/Citric Acid Cycle/Tricarboxylic Acid Cycle

  \begin{enumerate}
  \def\labelenumii{\arabic{enumii}.}
  \item
    The Krebs cycle coupled with the electron transport chain are used
    to produce \_\_\_\_\_\_\_\_\_ for cellular energy.
  \item
    Where in the cell does pyruvate get converted into acetyl CoA?
  \item
    How much ATP is directly produced from acetyl CoA?
  \item
    Is oxaloacetate used up through the Krebs cycle?
  \item
    How is carbon dioxide produced through the Krebs cycle?
  \end{enumerate}
\item
  Oxidative Phosphorylation and the Electron Transport Chain

  \begin{enumerate}
  \def\labelenumii{\arabic{enumii}.}
  \item
    What does the electron transport chain (ETC) use to produce ATP?
  \item
    \includegraphics[width=3.57222in,height=4.56875in,alt={The top panel of this figure shows the transformation of pyruvate to acetyl-CoA, and the bottom panel shows the steps in Krebs cycle.}]{images/media/image178.jpeg}What
    is oxidative phosphorylation?
  \end{enumerate}
\end{enumerate}

\begin{quote}
\includegraphics[width=4.90638in,height=3.37202in,alt={This image shows the mitochondrial membrane with proton pumps and ATP synthase embedded in the membrane. Arrows show the direction of flow of proteins and electrons across the membrane.}]{images/media/image179.jpeg}
\end{quote}

\begin{enumerate}
\def\labelenumi{\arabic{enumi}.}
\setcounter{enumi}{2}
\item
  \includegraphics[width=3.59931in,height=5.60694in,alt={This figure shows the different steps in which carbohydrates are metabolized and lists the number of ATP molecules produced in each step. The different steps shown are glycolysis, transformation of pyruvate to acetyl-CoA, the Krebs cycle, and the electron transport chain.}]{images/media/image180.jpeg}The
  accumulation of H\textsuperscript{+} where creates a gradient compared
  to the mitochondrial matrix?
\item
  How does ATP synthase use the proton gradient to create ATP?
\item
  How much ATP is produced in glycolysis, TCA, and ETC?

  \begin{enumerate}
  \def\labelenumii{\alph{enumii})}
  \item
    How much ATP is produced total?
  \end{enumerate}
\end{enumerate}

\begin{enumerate}
\def\labelenumi{\Roman{enumi}.}
\setcounter{enumi}{3}
\item
  Gluconeogenesis

  \begin{enumerate}
  \def\labelenumii{\arabic{enumii}.}
  \item
    Where does gluconeogenesis primarily take place at?
  \item
    What is an example of an organ that can only use glucose for energy?
  \item
    Why is it important that hexokinase and phosphofructokinase-1 are
    not used in gluconeogenesis?
  \end{enumerate}
\end{enumerate}

\begin{quote}
\includegraphics[width=3.17195in,height=4.93671in,alt={This figure shows the different steps in gluconeogenesis, where pyruvate is converted to glucose.}]{images/media/image181.jpeg}
\end{quote}

\includegraphics[width=2.90625in,height=3.22271in,alt={The top image shows the chemical formula for a triglyceride, and the bottom panel shows the formula for a monoglyceride.}]{images/media/image182.jpeg}24.3.
Lipid Metabolism

\begin{enumerate}
\def\labelenumi{\Roman{enumi}.}
\item
  Background

  \begin{enumerate}
  \def\labelenumii{\arabic{enumii}.}
  \item
    Oxidation of lipids can be used both to generate
    \_\_\_\_\_\_\_\_\_\_\_\_\_\_\_ and to synthesize new
    \_\_\_\_\_\_\_\_\_\_\_\_ from smaller constituent molecules?
  \item
    Lipid metabolism breaks triglycerides into monoglycerides through
    what enzymes?
  \item
    What is responsible for emulsifying ingested fats in the intestines?
  \item
    What role does CCK play in digestion?
  \item
    Why are chylomicrons essential for transport of fats and
    cholesterol?
  \end{enumerate}
\end{enumerate}

\begin{quote}
\includegraphics[width=2.55696in,height=2.03884in,alt={This figure shows a chylomicron containing triglycerides and cholesterol molecules as well as other lipids.}]{images/media/image183.jpeg}
\end{quote}

\begin{enumerate}
\def\labelenumi{\Roman{enumi}.}
\setcounter{enumi}{1}
\item
  Lipolysis

  \begin{enumerate}
  \def\labelenumii{\arabic{enumii}.}
  \item
    Where does lipolysis occur at within the cell?
  \item
    How much energy can be produced from fat compared to carbohydrates?
  \item
    Fatty acid oxidation converts fatty acids into
    \_\_\_\_\_\_\_\_\_\_\_\_\_\_\_\_ \_\_\_\_\_\_\_\_\_\_\_\_\_
    molecules.
  \end{enumerate}
\item
  Ketogenesis

  \begin{enumerate}
  \def\labelenumii{\arabic{enumii}.}
  \item
    When are ketone bodies created?
  \item
    Excess acetyl CoA is converted into
    \_\_\_\_\_\_\_\_\_\_\_\_\_\_\_\_\_\_\_\_\_\_\_\_\_\_\_\_\_\_
    \_\_\_\_\_\_\_\_\_\_\_\_\_\_\_\_ (HMG CoA)
  \end{enumerate}
\item
  Ketone Body Oxidation

  \begin{enumerate}
  \def\labelenumii{\arabic{enumii}.}
  \item
    When does the brain use ketone bodies as a source for energy?
  \item
    What is a symptom of ketogenesis?
  \item
    When β-hydroxybutyrate is oxidized acetoacetate and
    \_\_\_\_\_\_\_\_\_\_\_\_ is released.
  \end{enumerate}
\end{enumerate}

\includegraphics[width=3.99955in,height=3.43038in,alt={This figure shows the reactions in which ketone is oxidized to acetyl-CoA.}]{images/media/image184.jpeg}

\begin{enumerate}
\def\labelenumi{\Roman{enumi}.}
\setcounter{enumi}{4}
\item
  Lipogenesis

  \begin{enumerate}
  \def\labelenumii{\arabic{enumii}.}
  \item
    When does lipogenesis occur?
  \item
    What types of cells have lipogenesis occur within them?
  \item
    Why is the creation of citrate necessary for lipogenesis?
  \end{enumerate}
\end{enumerate}

\begin{quote}
\includegraphics[width=4.39315in,height=4.48101in,alt={This figure shows the different reactions that take place for lipid metabolism.}]{images/media/image185.jpeg}
\end{quote}

\subsection[24.4. Protein
Metabolism]{\texorpdfstring{\protect\includegraphics[width=4.08056in,height=3.23958in,alt={The left panel shows the main organs of the digestive system, and the right panel shows a magnified view of the intestine. Text callouts indicate the different protein digesting enzymes produced in different organs.}]{images/media/image186.jpeg}24.4.
Protein
Metabolism}{The left panel shows the main organs of the digestive system, and the right panel shows a magnified view of the intestine. Text callouts indicate the different protein digesting enzymes produced in different organs.24.4. Protein Metabolism}}\label{the-left-panel-shows-the-main-organs-of-the-digestive-system-and-the-right-panel-shows-a-magnified-view-of-the-intestine.-text-callouts-indicate-the-different-protein-digesting-enzymes-produced-in-different-organs.24.4.-protein-metabolism}

\begin{enumerate}
\def\labelenumi{\Roman{enumi}.}
\item
  Background

  \begin{enumerate}
  \def\labelenumii{\arabic{enumii}.}
  \item
    Are proteins stored in the body?
  \item
    What effect does pepsin have on polypeptides?
  \item
    HCl denatures proteins but can be harmful to the intestines. What is
    used to neutralize the acidity?
  \item
    How does the pancreas contribute to protein digestion?
  \item
    Why are inactive proenzymes important for the health of digestive
    organs?
  \item
    What are the inactive forms of trypsin and chymotrypsin?

    \begin{enumerate}
    \def\labelenumiii{\alph{enumiii})}
    \item
      What is the role of these proteolytic enzymes?
    \end{enumerate}
  \item
    What is formed when amino acids are decomposed?
  \end{enumerate}
\item
  \includegraphics[width=2.94792in,height=4.025in,alt={This image shows the reactions of the urea cycle and the organelles in which they take place.}]{images/media/image187.jpeg}Urea
  Cycle

  \begin{enumerate}
  \def\labelenumii{\arabic{enumii}.}
  \item
    Why is the urea cycle necessary?
  \item
    What are the two products of transamination?
  \item
    How is urea eliminated?
  \item
    \includegraphics[width=3.35417in,height=3.45139in,alt={This figure shows the different reactions in which products of carbohydrate breakdown are converted into different amino acids.}]{images/media/image188.jpeg}How
    can amino acids be used as a source of energy?
  \end{enumerate}
\end{enumerate}

\begin{quote}
\includegraphics[width=2.46624in,height=3.09161in,alt={This diagram shows the different metabolic pathways, and how they are connected.}]{images/media/image189.jpeg}
\end{quote}

24.5. Metabolic States of the Body

\begin{enumerate}
\def\labelenumi{\Roman{enumi}.}
\item
  Background
\item
  The Absorptive State

  \begin{enumerate}
  \def\labelenumii{\arabic{enumii}.}
  \item
    Is the absorptive state when a person is fed or fasted?
  \item
    How does insulin help with absorption?
  \item
    If ingested fats and carbohydrates are not used immediately what
    happens to them?
  \end{enumerate}
\item
  The Postabsorptive State

  \begin{enumerate}
  \def\labelenumii{\arabic{enumii}.}
  \item
    During the postabsorptive state what is the storage form of sugar
    the body initially relies upon?
  \item
    Glucagon results in the liver producing sugar through
    \_\_\_\_\_\_\_\_\_\_\_\_\_\_\_\_\_\_\_\_\_\_.
  \item
    After a period of fasting when food is ingested why does the liver
    continue to undergo gluconeogenesis?
  \end{enumerate}
\item
  Starvation

  \begin{enumerate}
  \def\labelenumii{\arabic{enumii}.}
  \item
    What organ is prioritized when in a starvation state?
  \item
    Why are ketones used to supply the needs of glucose dependent
    organs?

    \begin{enumerate}
    \def\labelenumiii{\alph{enumiii})}
    \item
      What does this spare?
    \end{enumerate}
  \item
    \includegraphics[width=3.57083in,height=4.66181in,alt={This figure shows the postabsorptive stage where no nutrients enter the blood stream from the digestive system and its effects of liver cells, muscle cells, and adipose cells.}]{images/media/image190.jpeg}\includegraphics[width=3.68403in,height=4.59097in,alt={This figure shows how nutrients are absorbed by the body. The diagram shows digested nutrients entering the blood stream and being absorbed by liver cells, muscle cells, and adipose cells. Underneath each panel, text details the process taking place in each cell type.}]{images/media/image191.jpeg}When
    are proteins finally used for energy in starvation?
  \end{enumerate}
\end{enumerate}

\subsubsection{24.6. Energy and Heat
Balance}\label{energy-and-heat-balance}

\begin{enumerate}
\def\labelenumi{\Roman{enumi}.}
\item
  Background

  \begin{enumerate}
  \def\labelenumii{\arabic{enumii}.}
  \item
    \_\_\_\_\_\_\_\_\_\_\_\_\_\_\_\_\_\_\_\_\_\_\_\_ is the process in
    which the body regulates its temperature.
  \item
    When producing ATP how much energy is lost as heat?
  \item
    Is thermoregulation governed by positive or negative feedback?
  \item
    What can the hypothalamus do to help decrease body temperature in a
    hot environment?
  \item
    What can the body do in an environment that is colder than body
    temperature?
  \item
    \includegraphics[width=3.69653in,height=6.225in,alt={This figure shows the pathways in which body temperature is controlled by the hypothalamus.}]{images/media/image192.jpeg}What
    is a thermoneutral environment?
  \end{enumerate}
\item
  Mechanisms of Heat Exchange

  \begin{enumerate}
  \def\labelenumii{\arabic{enumii}.}
  \item
    Heat flows from \_\_\_\_\_\_\_\_\_\_\_\_\_ concentration to
    \_\_\_\_\_\_\_\_\_\_\_ concentration.
  \item
    \_\_\_\_\_\_\_\_\_\_\_\_\_\_\_\_\_ is the transfer of heat between
    two objects in direct contact.
  \item
    What is convection and how is it different from conduction?
  \item
    How do the rays from the sun warm a room?
  \item
    How does evaporative cooling work?
  \end{enumerate}
\item
  Metabolic Rate

  \begin{enumerate}
  \def\labelenumii{\arabic{enumii}.}
  \item
    What is metabolic rate?
  \item
    When at rest the amount of energy expended is known as
    \_\_\_\_\_\_\_\_\_\_\_\_\_\_ \_\_\_\_\_\_\_\_\_\_\_\_\_\_\_\_
    \_\_\_\_\_\_\_\_\_\_\_ or BMR.
  \item
    How much of the BMR is typically dedicated to thermoregulation?
  \end{enumerate}
\end{enumerate}

24.7. Nutrition and Diet

\begin{enumerate}
\def\labelenumi{\Roman{enumi}.}
\item
  Background

  \begin{enumerate}
  \def\labelenumii{\arabic{enumii}.}
  \item
    How is energy primarily stored inside the body?
  \end{enumerate}
\item
  Food and Metabolism

  \begin{enumerate}
  \def\labelenumii{\arabic{enumii}.}
  \item
    What is a Calorie ©?
  \item
    How many calories does the average person need to carry out usual
    daily activities?
  \item
    \includegraphics[width=2.14085in,height=1.70656in,alt={The figure shows a plate with different food groups assigned different portion sizes.}]{images/media/image193.jpeg}How
    many excess calories would result in an additional pound of body
    weight?
  \item
    Which takes more energy to digest, carbohydrates or proteins?
  \item
    How much of a plate full of food should be made up of fruits and
    vegetables?
  \end{enumerate}
\item
  Vitamins

  \begin{enumerate}
  \def\labelenumii{\arabic{enumii}.}
  \item
    What are vitamins?
  \item
    The \_\_\_\_\_ vitamins play the largest role of any vitamins in
    metabolism.
  \item
    Are vitamins only available from the diet, or can some by formed?
  \item
    What vitamins are fat-soluble?
  \item
    How are excess water-soluble vitamins eliminated?
  \end{enumerate}
\item
  Minerals

  \begin{enumerate}
  \def\labelenumii{\arabic{enumii}.}
  \item
    What are minerals?
  \item
    Can minerals be made in the body?
  \item
    What are the most common minerals?

    \begin{enumerate}
    \def\labelenumiii{\alph{enumiii})}
    \item
      Where are they stored?
    \end{enumerate}
  \end{enumerate}
\end{enumerate}

\section{}\label{section-26}

\section{\texorpdfstring{Chapter 25 }{Chapter 25 }}\label{chapter-25}

\subsection{25.1. Physical Characteristics of
Urine}\label{physical-characteristics-of-urine}

\begin{enumerate}
\def\labelenumi{\Roman{enumi}.}
\item
  \includegraphics[width=2.59792in,height=4.40486in,alt={Figure 25.2 Urine Color}]{images/media/image194.jpeg}Background

  \begin{enumerate}
  \def\labelenumii{\arabic{enumii}.}
  \item
    What allows the urinary system to filter blood?
  \item
    The glomeruli filter the blood mostly based on size. What is too big
    to be filtered, and therefore not found in the urine?
  \item
    What is a normal volume of urine to produce a day?
  \item
    Should red blood cells be found in the urine?
  \item
    What is a urinalysis? Why is it useful in assessing renal health?

    \begin{enumerate}
    \def\labelenumiii{\alph{enumiii})}
    \item
      What diseases could result in a large quantity of urine?
    \end{enumerate}
  \item
    What pigment gives urine its characteristic color?
  \item
    What is an example of a color change of urine that may indicate a
    disease state?
  \item
    How much urine is excreted in oliguria, anuria, and polyuria?
  \item
    Would diabetes insipidus lead to oliguria, anuria, or polyuria?
  \item
    Why is it important to regulate urine pH? What happens when urine pH
    is chronically high or low?
  \item
    What is specific gravity? What does it tell us about urine?
  \item
    If leukocyte esterase is found in the urine what may be occurring?
  \item
    What does the presence of ketones in the urine potentially tell us
    about what is being used for energy?
  \item
    Nitrites, if found in the urine, may indicate what kind of
    infection?
  \end{enumerate}
\end{enumerate}

25.2. Gross Anatomy of Urine Transport

\begin{enumerate}
\def\labelenumi{\Roman{enumi}.}
\item
  Background

  \begin{enumerate}
  \def\labelenumii{\arabic{enumii}.}
  \item
    What is filtered to form urine?
  \item
    What does the urinary system protect the body against?
  \end{enumerate}
\item
  Urethra

  \begin{enumerate}
  \def\labelenumii{\arabic{enumii}.}
  \item
    The urethra transports urine from the bladder to where?
  \end{enumerate}
\end{enumerate}

\includegraphics[width=5.83333in,height=2.13542in,alt={The top panel of this figure shows the organs in the female urinary system.}]{images/media/image195.jpeg}

\begin{enumerate}
\def\labelenumi{\arabic{enumi}.}
\setcounter{enumi}{1}
\item
  What is the trigone and where is it located?
\item
  Is the internal urinary sphincter or the external urinary sphincter
  under voluntary control?
\end{enumerate}

\begin{enumerate}
\def\labelenumi{\Alph{enumi}.}
\item
  Female Urethra

  \begin{enumerate}
  \def\labelenumii{\arabic{enumii}.}
  \item
    What anatomically makes a UTI more likely in women?
  \end{enumerate}
\item
  Male Urethra

  \begin{enumerate}
  \def\labelenumii{\arabic{enumii}.}
  \item
    What part of the male urethra passes through the prostate?
  \item
    Why are mucous glands present in the male urethra?
  \end{enumerate}
\end{enumerate}

\begin{enumerate}
\def\labelenumi{\Roman{enumi}.}
\setcounter{enumi}{2}
\item
  Bladder

  \begin{enumerate}
  \def\labelenumii{\arabic{enumii}.}
  \item
    What role does the bladder serve in the urinary tract?
  \item
    The bladder is a retroperitoneal organ. What does retroperitoneal
    mean?
  \end{enumerate}
\end{enumerate}

\includegraphics[width=5.20833in,height=3.1875in,alt={The left panel of this figure shows the cross section of the bladder and the major parts are labeled. The right panel shows a micrograph of the bladder.}]{images/media/image196.jpeg}

\begin{enumerate}
\def\labelenumi{\arabic{enumi}.}
\setcounter{enumi}{2}
\item
  Is the detrusor muscle the only way to create the force necessary to
  void urine?
\end{enumerate}

\begin{enumerate}
\def\labelenumi{\Alph{enumi}.}
\item
  Micturition Reflex

  \begin{enumerate}
  \def\labelenumii{\arabic{enumii}.}
  \item
    What is micturition?
  \item
    When is incontinence likely to occur?
  \item
    The sacral micturition center is necessary for voluntary or
    involuntary micturition?
  \end{enumerate}
\end{enumerate}

\includegraphics[width=4.42708in,height=3.21875in,alt={This image shows the female urinary system and identifies the nerves that are important in this system.}]{images/media/image197.jpeg}

\begin{enumerate}
\def\labelenumi{\Roman{enumi}.}
\setcounter{enumi}{2}
\item
  Ureters

  \begin{enumerate}
  \def\labelenumii{\arabic{enumii}.}
  \item
    \includegraphics[width=2.98548in,height=2.50078in,alt={A micrograph shows the lumen of the ureter.}]{images/media/image198.jpeg}The
    ureters are retroperitoneal, which makes sense as they drain urine
    from the \_\_\_\_\_\_\_\_\_\_\_\_\_\_\_\_\_\_\_ to the
    \_\_\_\_\_\_\_\_\_\_\_\_\_\_\_\_\_ which are also retroperitoneal.
  \item
    The oblique angle that the ureters join the bladder creates a
    \_\_\_\_\_\_\_\_\_\_\_\_\_\_\_\_\_\_ sphincter, or one-way valve.
  \item
    What type of epithelium lines the ureters?
  \item
    Are the ureters passive, or do they actively move urine through
    muscular contractions?
  \end{enumerate}
\end{enumerate}

\subsection{25.3. Gross Anatomy of the
Kidney}\label{gross-anatomy-of-the-kidney}

\begin{enumerate}
\def\labelenumi{\Roman{enumi}.}
\item
  Background

  \begin{enumerate}
  \def\labelenumii{\arabic{enumii}.}
  \item
    What protects the kidneys?
  \item
    Estimate how much of the heart's cardiac output do the kidneys
    receive at rest?
  \end{enumerate}
\item
  External Anatomy

  \begin{enumerate}
  \def\labelenumii{\arabic{enumii}.}
  \item
    Why is the left kidney located higher than the right?
  \item
    What purpose does the renal fat pad serve?
  \end{enumerate}
\end{enumerate}

\includegraphics[width=4.42708in,height=3.26042in,alt={This image shows a human torso and shows the location of the kidneys within the torso.}]{images/media/image199.jpeg}

\begin{enumerate}
\def\labelenumi{\arabic{enumi}.}
\setcounter{enumi}{2}
\item
  What glands are found on top of the kidneys?
\end{enumerate}

\begin{enumerate}
\def\labelenumi{\Roman{enumi}.}
\setcounter{enumi}{2}
\item
  Internal Anatomy

  \begin{enumerate}
  \def\labelenumii{\arabic{enumii}.}
  \item
    Which is found deeper inside the kidney the renal cortex or medulla?
  \item
    What separates the renal pyramids?
  \item
    The renal \_\_\_\_\_\_\_\_\_\_\_\_\_\_\_\_\_\_\_\_ are bundles of
    collecting ducts that transport urine to the calyces.
  \end{enumerate}
\end{enumerate}

\includegraphics[width=4.94792in,height=2.82292in,alt={The left panel of this figure shows the location of the kidneys in the abdomen. The right panel shows the cross section of the kidney.}]{images/media/image200.jpeg}

\begin{enumerate}
\def\labelenumi{\Roman{enumi}.}
\setcounter{enumi}{3}
\item
  Renal Hilum

  \begin{enumerate}
  \def\labelenumii{\arabic{enumii}.}
  \item
    What enters and exits the kidney at the renal hilum?
  \end{enumerate}

  \begin{enumerate}
  \def\labelenumii{\Alph{enumii}.}
  \item
    Nephrons and Vessels

    \begin{enumerate}
    \def\labelenumiii{\arabic{enumiii}.}
    \item
      What arterioles supply the nephrons of the kidney?
    \end{enumerate}
  \end{enumerate}
\end{enumerate}

\includegraphics[width=4.94792in,height=3.96875in,alt={This figure shows the network of blood vessels and the blood flow in the kidneys.}]{images/media/image201.jpeg}

\begin{enumerate}
\def\labelenumi{\arabic{enumi}.}
\setcounter{enumi}{1}
\item
  What purpose do nephrons serve in the kidney?
\item
  The glomerulus is formed from a tuft of what arteriole?
\item
  The Bowman's capsule and glomerulus form the
  \_\_\_\_\_\_\_\_\_\_\_\_\_\_\_ \_\_\_\_\_\_\_\_\_\_\_\_\_\_\_\_.
\item
  Blood that is not filtered passes from the glomerulus on to the
  \_\_\_\_\_\_\_\_\_\_\_\_\_\_\_\_\_ arteriole which then becomes the
  peritubular capillaries and vasa recta.
\end{enumerate}

\begin{enumerate}
\def\labelenumi{\Alph{enumi}.}
\setcounter{enumi}{1}
\item
  Cortex

  \begin{enumerate}
  \def\labelenumii{\arabic{enumii}.}
  \item
    What components of the nephron are found within the renal cortex?
  \item
    The loop of Henle in juxtamedullary nephrons descends into what part
    of the kidney?
  \end{enumerate}
\end{enumerate}

25.4. Microscopic Anatomy of the Kidney

\begin{enumerate}
\def\labelenumi{\Roman{enumi}.}
\item
  \includegraphics[width=2.36458in,height=3.33472in,alt={This image shows the blood vessels and the direction of blood flow in the nephron.}]{images/media/image202.jpeg}Background
\item
  Nephrons: The Functional Unit

  \begin{enumerate}
  \def\labelenumii{\arabic{enumii}.}
  \item
    What is the term forming urine used to describe?
  \item
    What are the three principle functions that the nephrons carry out?
  \item
    The kidneys can help to regulate blood pressure via the production
    of \_\_\_\_\_\_\_\_\_\_\_\_\_\_\_\_.
  \end{enumerate}

  \begin{enumerate}
  \def\labelenumii{\Alph{enumii}.}
  \item
    Renal Corpuscle

    \begin{enumerate}
    \def\labelenumiii{\arabic{enumiii}.}
    \item
      How do the podocytes, and their pedicels, help with filtration
      from the glomerulus to the proximal convoluted tubule (PCT)?
    \item
      What part of the podocytes forms the filtration slits?
    \end{enumerate}
  \end{enumerate}
\end{enumerate}

\includegraphics[width=5in,height=2.33333in,alt={The left panel of this figure shows an image of a podocyte. The right panel shows a tube-like structure that illustrates the filtration slits and the cell bodies.}]{images/media/image203.jpeg}

\includegraphics[width=3.59375in,height=2.71875in,alt={The top panel of this figure shows a tube-like structure with the basement membrane and other parts labeled.}]{images/media/image204.jpeg}

\begin{enumerate}
\def\labelenumi{\arabic{enumi}.}
\setcounter{enumi}{2}
\item
  Why are the capillaries in the glomerulus fenestrated?
\item
  How do mesangial cells regulate the rate of filtration?
\item
  The juxtaglomerular apparatus (JGA) is the where the
  \_\_\_\_\_\_\_\_\_\_\_\_\_\_\_\_\_
  \_\_\_\_\_\_\_\_\_\_\_\_\_\_\_\_\_\_\_\_\_\_\_\_
  \_\_\_\_\_\_\_\_\_\_\_\_\_\_\_\_\_\_\_ (DCT) comes into contact with
  the arterioles.
\item
  The \_\_\_\_\_\_\_\_\_\_\_\_\_\_\_\_\_\_\_\_\_\_
  \_\_\_\_\_\_\_\_\_\_\_\_\_\_\_\_\_\_\_\_\_ is the wall of the DCT that
  forms a part of the JGA.
\item
  How does the macula densa help to regulate flow through the nephrons?
\end{enumerate}

\includegraphics[width=5.72917in,height=2.11458in,alt={The top panel of this image shows the cross section of the juxtaglomerular apparatus. The major parts are labeled.}]{images/media/image205.jpeg}

\begin{enumerate}
\def\labelenumi{\arabic{enumi}.}
\setcounter{enumi}{7}
\item
  How does the juxtaglomerular cells help to regulate blood flow to the
  glomerulus?
\item
  What happens when to glomerular filtration rate (GFR) when the
  osmolarity of the filtrate is too high?
\end{enumerate}

\includegraphics[width=5.23958in,height=3.30094in,alt={This diagram shows the pathway of action of the renin-aldosterone-angiotensin system. An arrow in the center of the image shows the sequence of events that take place, and branching off from this arrow are indications of where in the body these events take place.}]{images/media/image206.jpeg}

\begin{enumerate}
\def\labelenumi{\Alph{enumi}.}
\setcounter{enumi}{1}
\item
  Proximal Convoluted Tubule (PCT)

  \begin{enumerate}
  \def\labelenumii{\arabic{enumii}.}
  \item
    What purpose does the brush border in the PCT serve?
  \end{enumerate}
\item
  Loop of Henle

  \begin{enumerate}
  \def\labelenumii{\arabic{enumii}.}
  \item
    What is different between the descending and ascending limb of the
    loop of Henle?
  \end{enumerate}
\item
  Distal Convoluted Tubule (DCT)

  \begin{enumerate}
  \def\labelenumii{\arabic{enumii}.}
  \item
    Why does the DCT need a lot of mitochondria?

    \begin{enumerate}
    \def\labelenumiii{\alph{enumiii})}
    \item
      How does this compare to the PCT?
    \end{enumerate}
  \end{enumerate}
\item
  Collecting Ducts

  \begin{enumerate}
  \def\labelenumii{\arabic{enumii}.}
  \item
    Are the collecting ducts a part of the nephron?
  \item
    What effect does the insertion of an aquaporin have on the recovery
    of water from the collecting ducts?
  \item
    Are more collecting ducts present when antidiuretic hormone is
    present or absent?
  \end{enumerate}
\end{enumerate}

\includegraphics[width=4.01042in,height=2.01042in,alt={This figure shows an aquaporin water channel in the bilayer membrane with water molecules passing through.}]{images/media/image207.jpeg}

\subsubsection{25.5. Physiology of Urine
Formation}\label{physiology-of-urine-formation}

\begin{enumerate}
\def\labelenumi{\Roman{enumi}.}
\item
  Background

  \begin{enumerate}
  \def\labelenumii{\arabic{enumii}.}
  \item
    What are the three processes that produce urine?
  \item
    The production of urine modifies the composition of the
    \_\_\_\_\_\_\_\_\_\_\_\_\_\_\_\_.
  \end{enumerate}
\item
  Glomerular Filtration Rate (GFR)

  \begin{enumerate}
  \def\labelenumii{\arabic{enumii}.}
  \item
    What is Glomerular Filtration Rate (GFR)?
  \item
    What is a normal GFR for both kidneys in one day?
  \item
    \_\_\_\_\_\_\_\_\_\_\_\_\_\_\_\_ forces fluid and solutes through a
    semipermeable barrier.
  \item
    What is hydrostatic pressure?

    \begin{enumerate}
    \def\labelenumiii{\alph{enumiii})}
    \item
      What exerts hydrostatic pressure on the filtration membrane in
      renal capsule?
    \end{enumerate}
  \item
    What is colloid osmotic pressure?
  \item
    Can cells or large proteins pass through the fenestrations or
    between the podocyte processes?
  \item
    What does a positive net filtration pressure (NFP) tell us about
    filtration?
  \end{enumerate}
\end{enumerate}

\includegraphics[width=3.64583in,height=3.77083in,alt={This figure shows the different pressures acting across the glomerulus.}]{images/media/image208.jpeg}

\begin{enumerate}
\def\labelenumi{\arabic{enumi}.}
\setcounter{enumi}{7}
\item
  What happens to the body when there is an insufficient amount of
  plasma proteins in the blood?
\item
  What is systemic Edema?
\end{enumerate}

\begin{enumerate}
\def\labelenumi{\Roman{enumi}.}
\setcounter{enumi}{2}
\item
  Net Filtration Pressure (NFP)

  \begin{enumerate}
  \def\labelenumii{\arabic{enumii}.}
  \item
    How does the kidney deal with a wide range of blood pressure?
  \item
    At what point is blood pressure insufficient for filtration to
    occur?

    \begin{enumerate}
    \def\labelenumiii{\alph{enumiii})}
    \item
      What is shock?
    \end{enumerate}
  \item
    How can inulin be used to figure out GFR?
  \end{enumerate}
\end{enumerate}

\includegraphics[width=2.05208in,height=3.91362in,alt={This diagram shows the different ions and chemicals that are secreted and reabsorbed along the nephron. Arrows show the direction of the movement of the substance.}]{images/media/image209.jpeg}25.6.
Tubular Reabsorption

\begin{enumerate}
\def\labelenumi{\Roman{enumi}.}
\item
  Background

  \begin{enumerate}
  \def\labelenumii{\arabic{enumii}.}
  \item
    Where is most water recovered in the nephron?
  \end{enumerate}
\item
  Mechanisms of Recovery

  \begin{enumerate}
  \def\labelenumii{\arabic{enumii}.}
  \item
    How does active transport move substances from low concentration to
    high concentration?
  \item
    Diffusion moves substances from \_\_\_\_\_\_\_\_\_\_\_\_\_\_
    concentration to \_\_\_\_\_\_\_\_\_\_\_\_ concentration.
  \item
    Symport move two or more substances in the \_\_\_\_\_\_\_\_\_\_\_\_
    direction, and antiporters move substances in the
    \_\_\_\_\_\_\_\_\_\_\_\_\_\_\_\_\_\_\_ direction.
  \item
    Acid-base balance is maintained through the
    \_\_\_\_\_\_\_\_\_\_\_\_\_ and the \_\_\_\_\_\_\_\_\_\_\_\_\_\_.
  \end{enumerate}
\item
  Reabsorption and Secretion in the PCT

  \begin{enumerate}
  \def\labelenumii{\arabic{enumii}.}
  \item
    What does reabsorbed mean in the context of the nephrons?
  \item
    \includegraphics[width=3.90347in,height=2.79444in,alt={This diagram shows the different substances that are secreted and reabsorbed by the proximal collecting tubule. Arrows show the direction of the movement of the substance.}]{images/media/image210.jpeg}Where
    is water ``obliged to follow sodium?
  \item
    The apical surface of the cells in the PCT face what?
  \item
    How much of sodium and potassium are absorbed in the PCT?
  \item
    How much of filtered glucose is reabsorbed in the PCT?

    \begin{enumerate}
    \def\labelenumiii{\alph{enumiii})}
    \item
      What is glycosuria?
    \end{enumerate}
  \item
    Why does bicarbonate need to be recovered?
  \end{enumerate}
\end{enumerate}

\includegraphics[width=4.16667in,height=2.41667in,alt={This diagram shows the process of reabsorption of bicarbonate by the proximal collecting tubule.}]{images/media/image211.jpeg}

\begin{enumerate}
\def\labelenumi{\Roman{enumi}.}
\setcounter{enumi}{3}
\item
  Reabsorption and Secretion in the Loop of Henle

  \begin{enumerate}
  \def\labelenumii{\arabic{enumii}.}
  \item
    What is the difference between a cortical and juxtamedullary
    nephron?
  \item
    The descending and ascending portions of the loop of Henle are
    specialized to recover what ion?
  \end{enumerate}

  \begin{enumerate}
  \def\labelenumii{\Alph{enumii}.}
  \item
    Descending Loop

    \begin{enumerate}
    \def\labelenumiii{\arabic{enumiii}.}
    \item
      The osmolarity of the interstitium increases the further the
      descending loop is into the renal medulla. How does this help to
      move water into the nephron?
    \end{enumerate}
  \item
    Ascending Loop

    \begin{enumerate}
    \def\labelenumiii{\arabic{enumiii}.}
    \item
      The thick portion of the ascending limb is impermeable to
      \_\_\_\_\_\_\_\_\_\_ due to the absence of aquaporins.
    \item
      The reabsorption of NaCl, and the inability for water to move
      creates what kind of filtrate by the time it reaches the DCT?
    \item
      Leaky tight junctions allow certain solutes to move according to
      their \_\_\_\_\_\_\_\_\_\_\_\_\_\_\_ gradient.
    \end{enumerate}
  \item
    \includegraphics[width=3.45in,height=3.09722in,alt={The left panel of this image shows the location of the loop of Henle. The right panel shows the interstitial osmolality and the exchange of sodium and chloride ions, as well as water and urea.}]{images/media/image212.jpeg}Countercurrent
    Multiplier System

    \begin{enumerate}
    \def\labelenumiii{\arabic{enumiii}.}
    \item
      The reabsorption of NaCl and the presence of urea in the
      interstitial space creates a \_\_\_\_\_\_\_\_\_\_\_ osmolar
      environment in the depths of the medulla.
    \item
      What is the net result of the countercurrent multiplier system?
    \item
      Why is it important that blood flows slowly in the vasa recta?
    \end{enumerate}
  \end{enumerate}
\end{enumerate}

\begin{enumerate}
\def\labelenumi{\Roman{enumi}.}
\setcounter{enumi}{3}
\item
  Reabsorption and Secretion in the Distal Convoluted Tubule

  \begin{enumerate}
  \def\labelenumii{\arabic{enumii}.}
  \item
    How much water has been recovered by the time the filtrate reaches
    the DCT?
  \item
    How does the movement of sodium out of the lumen of the collecting
    duct help to move chloride ions?
  \item
    Parathyroid hormone (PTH) signals for the reabsorption of what ion?
  \end{enumerate}
\item
  Collecting Ducts and Recovery of Water

  \begin{enumerate}
  \def\labelenumii{\arabic{enumii}.}
  \item
    Principal cells allow for recovery or loss of
    \_\_\_\_\_\_\_\_\_\_\_\_\_\_\_.
  \item
    Intercalated cells help to regulate blood pH by secreting or
    absorbing \_\_\_\_\_\_\_\_\_\_\_ or
    \_\_\_\_\_\_\_\_\_\_\_\_\_\_\_\_\_\_\_\_\_\_\_\_\_.
  \item
    The presence of vasopressin would allow for more or less water
    reabsorption\textgreater{}

    \begin{enumerate}
    \def\labelenumiii{\alph{enumiii})}
    \item
      Would urine be dilute or concentrated?
    \end{enumerate}
  \item
    Aldosterone assists with recovery of what ion?
  \end{enumerate}
\end{enumerate}

\paragraph{25.7. Regulation of Renal Blood
Flow}\label{regulation-of-renal-blood-flow}

\begin{enumerate}
\def\labelenumi{\Roman{enumi}.}
\item
  Background

  \begin{enumerate}
  \def\labelenumii{\arabic{enumii}.}
  \item
    Why is it important that the flow of blood through the kidney is
    maintained?
  \end{enumerate}
\item
  Sympathetic Nerves

  \begin{enumerate}
  \def\labelenumii{\arabic{enumii}.}
  \item
    During rest sympathetic stimulation to the kidneys results in
    \_\_\_\_\_\_\_\_\_\_\_\_\_\_\_\_\_\_\_ and increased blood flow
    through the kidneys.
  \item
    Why during times of stress do the kidneys receive less blood?
  \end{enumerate}
\item
  Autoregulation

  \begin{enumerate}
  \def\labelenumii{\arabic{enumii}.}
  \item
    What is autoregulation?

    \begin{enumerate}
    \def\labelenumiii{\alph{enumiii})}
    \item
      What mechanisms do the kidneys have to maintain their rate of
      blood flow?
    \end{enumerate}
  \end{enumerate}

  \begin{enumerate}
  \def\labelenumii{\Alph{enumii}.}
  \item
    Arteriole Myogenic Mechanism

    \begin{enumerate}
    \def\labelenumiii{\arabic{enumiii}.}
    \item
      According to the myogenic mechanism what is the smooth muscle
      response to stretch?

      \begin{enumerate}
      \def\labelenumiv{\alph{enumiv})}
      \item
        How does this response help to maintain flow?
      \end{enumerate}
    \end{enumerate}
  \item
    Tubuloglomerular Feedback

    \begin{enumerate}
    \def\labelenumiii{\arabic{enumiii}.}
    \item
      Increased fluid flow is detected how in the macula densa?
    \item
      ATP and adenosine acting as paracrine factors have what effect on
      the afferent arteriole and GFR?
    \item
      NO has what effect on the afferent arteriole and GFR?
    \end{enumerate}
  \end{enumerate}
\end{enumerate}

25.8. Endocrine Regulation of Kidney Function

\begin{enumerate}
\def\labelenumi{\Roman{enumi}.}
\item
  Background

  \begin{enumerate}
  \def\labelenumii{\arabic{enumii}.}
  \item
    What is the difference between endocrine and paracrine acting
    messengers?
  \end{enumerate}
\item
  Renin-Angiotensin-Aldosterone

  \begin{enumerate}
  \def\labelenumii{\arabic{enumii}.}
  \item
    Renin converts angiotensinogen into what?
  \item
    ACE converts Angiotensin I into what?
  \item
    Angiotensin II is a vaso\_\_\_\_\_\_\_\_\_\_\_\_\_\_\_. Therefore,
    it raises blood pressure.
  \item
    What effect does Angiotensin II have on the efferent and afferent
    arterioles?
  \item
    What can stimulate the release of aldosterone?

    \begin{enumerate}
    \def\labelenumiii{\alph{enumiii})}
    \item
      Aldosterone promotes the reabsorption of what ion?
    \end{enumerate}
  \end{enumerate}
\item
  Antidiuretic Hormone (ADH)

  \begin{enumerate}
  \def\labelenumii{\arabic{enumii}.}
  \item
    Diuretics such as coffee, tea, and alcohol all promote water
    \_\_\_\_\_\_\_\_\_\_\_\_\_\_\_\_.
  \item
    ADH promotes water recovery through what mechanism?
  \end{enumerate}
\item
  Endothelin

  \begin{enumerate}
  \def\labelenumii{\arabic{enumii}.}
  \item
    Endothelins are potent
    vaso\_\_\_\_\_\_\_\_\_\_\_\_\_\_\_\_\_\_\_\_\_\_.
  \item
    Chronically elevated endothelins, such as in diabetic kidney
    disease, has what effect on GFR?
  \end{enumerate}
\item
  Natriuretic Hormones

  \begin{enumerate}
  \def\labelenumii{\arabic{enumii}.}
  \item
    What is natriuresis?
  \item
    Natriuretic hormones have what effect on ADH release?
  \item
    Under what conditions is atrial natriuretic hormone released?
  \end{enumerate}
\item
  Parathyroid Hormone

  \begin{enumerate}
  \def\labelenumii{\arabic{enumii}.}
  \item
    Parathyroid hormone (PTH) is released in response to a decreased
    amount of what ion?
  \item
    Why does parathyroid hormone block the reabsorption of phosphate?
  \end{enumerate}
\end{enumerate}

\subparagraph{25.9. Regulation of Fluid Volume and
Composition}\label{regulation-of-fluid-volume-and-composition}

\begin{enumerate}
\def\labelenumi{\Roman{enumi}.}
\item
  Background

  \begin{enumerate}
  \def\labelenumii{\arabic{enumii}.}
  \item
    What aspect of blood volume maintenance is the kidney able to effect
    a change over?
  \end{enumerate}
\item
  Volume-sensing Mechanisms

  \begin{enumerate}
  \def\labelenumii{\arabic{enumii}.}
  \item
    Why is blood pressure used as an indicator for blood volume in the
    body?

    \begin{enumerate}
    \def\labelenumiii{\alph{enumiii})}
    \item
      What do the baroreceptors measure?
    \end{enumerate}
  \item
    How do the kidneys respond to low blood pressure?
  \item
    How does the heart, specifically the atria, respond to an increase
    in stretch such as when blood pressure rises?
  \item
    How can ADH or vasopressin help in a situation in which a
    significant amount of blood is lost?
  \end{enumerate}
\item
  Diuretics and Fluid Volume

  \begin{enumerate}
  \def\labelenumii{\arabic{enumii}.}
  \item
    How do caffeine and alcohol increase urine volume?
  \item
    How do osmotic diuretics work?
  \end{enumerate}
\item
  Regulation of Extracellular Na\textsuperscript{+}

  \begin{enumerate}
  \def\labelenumii{\arabic{enumii}.}
  \item
    An increase in sodium has what effect on water retention?

    \begin{enumerate}
    \def\labelenumiii{\alph{enumiii})}
    \item
      What happens to blood pressure when blood volume increases?
    \end{enumerate}
  \item
    What hormone(s) may signal to retain more sodium?
  \end{enumerate}
\item
  Regulation of Extracellular K\textsuperscript{+}

  \begin{enumerate}
  \def\labelenumii{\arabic{enumii}.}
  \item
    When more sodium is reabsorbed what tends to happen to potassium?
  \end{enumerate}
\item
  Regulation of Cl\textsuperscript{-}

  \begin{enumerate}
  \def\labelenumii{\arabic{enumii}.}
  \item
    Chloride is important for
    \_\_\_\_\_\_\_\_\_\_\_\_-\_\_\_\_\_\_\_\_\_\_\_\_\_ balance.
  \end{enumerate}
\item
  Regulation of Ca\textsuperscript{++} and Phosphate

  \begin{enumerate}
  \def\labelenumii{\arabic{enumii}.}
  \item
    Parathyroid Hormone (PTH) has what effect on the kidneys?
  \end{enumerate}
\item
  Regulation of H\textsuperscript{+}, Bicarbonate, and pH

  \begin{enumerate}
  \def\labelenumii{\arabic{enumii}.}
  \item
    Why is bicarbonate an important buffer in the body?
  \item
    The kidneys can rid the body of both \_\_\_\_\_\_\_\_\_\_\_\_\_\_
    and \_\_\_\_\_\_\_\_\_\_\_\_\_\_.
  \end{enumerate}
\item
  Regulation of Nitrogen Wastes

  \begin{enumerate}
  \def\labelenumii{\arabic{enumii}.}
  \item
    How do the kidneys help to eliminate nitrogenous wastes from the
    body?
  \end{enumerate}
\item
  Elimination of Drugs and Hormones

  \begin{enumerate}
  \def\labelenumii{\arabic{enumii}.}
  \item
    How are water soluble drugs eliminated?
  \item
    Are there some drugs that can't be filtered by the kidneys?
  \end{enumerate}
\end{enumerate}

25.10. The Urinary System and Homeostasis

\begin{enumerate}
\def\labelenumi{\Roman{enumi}.}
\item
  Background
\item
  Vitamin D Synthesis

  \begin{enumerate}
  \def\labelenumii{\arabic{enumii}.}
  \item
    Why is the kidney necessary for vitamin D to be active?
  \item
    Without active vitamin D what can happen pathophysiologically?
  \end{enumerate}
\item
  Erythropoiesis

  \begin{enumerate}
  \def\labelenumii{\arabic{enumii}.}
  \item
    \includegraphics[width=3.64583in,height=3.17708in,alt={Alt text to come.}]{images/media/image213.jpeg}How
    does EPO help the body to maintain sufficient oxygenation at high
    altitudes?
  \item
    What produces EPO in the body?
  \item
    How can exercise influence EPO production?
  \end{enumerate}
\item
  Blood Pressure Regulation

  \begin{enumerate}
  \def\labelenumii{\arabic{enumii}.}
  \item
    What happens to blood pressure if the kidneys are unable to recover
    water in the collecting ducts?
  \item
    How do the kidneys, lungs, and liver all work together to allow the
    renin-angiotensin-aldosterone system to function?
  \item
    What effect does Angiotensin II have on blood pressure?
  \end{enumerate}
\item
  Regulation of Osmolarity

  \begin{enumerate}
  \def\labelenumii{\arabic{enumii}.}
  \item
    How does hypo-osmolarity lead to edema?
  \item
    Severe dehydration can lead to what osmotic state within the body?
  \end{enumerate}
\item
  Recovery of Electrolytes

  \begin{enumerate}
  \def\labelenumii{\arabic{enumii}.}
  \item
    Why is regulation of blood potassium important?
  \end{enumerate}
\item
  pH Regulation

  \begin{enumerate}
  \def\labelenumii{\arabic{enumii}.}
  \item
    Kidneys work with the lungs to regulate the pH of the blood. What
    happens to enzyme when they are outside their optimum pH range?
  \end{enumerate}
\end{enumerate}

\section{}\label{section-27}

\section{\texorpdfstring{Chapter 26 }{Chapter 26 }}\label{chapter-26}

\subsection{26.1. Body Fluids and Fluid
Compartments}\label{body-fluids-and-fluid-compartments}

\includegraphics[width=2.9375in,height=4.16295in]{images/media/image214.png}

\begin{enumerate}
\def\labelenumi{\Roman{enumi}.}
\item
  Background

  \begin{enumerate}
  \def\labelenumii{\arabic{enumii}.}
  \item
    Substances dissolved into the water of the body are known as
    \_\_\_\_\_\_\_\_\_\_\_\_\_\_\_.
  \item
    What are some examples of electrolytes?
  \item
    Water moves from regions of \_\_\_\_\_\_\_\_\_\_\_\_\_\_\_
    concentration to \_\_\_\_\_\_\_\_\_\_\_\_\_\_\_ concentration.
  \end{enumerate}
\item
  Body Water Content

  \begin{enumerate}
  \def\labelenumii{\arabic{enumii}.}
  \item
    What happens to the percent of the body made of water as we age?
  \item
    What part of the body is composed of a significant amount of water,
    and what has a small proportion of water?
  \end{enumerate}
\item
  Fluid Compartments

  \begin{enumerate}
  \def\labelenumii{\arabic{enumii}.}
  \item
    What is a fluid compartment?
  \item
    Fluid found inside the cells of the body is known as
    \_\_\_\_\_\_\_\_\_\_\_\_\_\_\_\_ \_\_\_\_\_\_\_\_\_\_\_\_\_\_\_\_
    (ICF)
  \item
    What are the two components of extracellular fluid (ECF)?
  \end{enumerate}
\end{enumerate}

\includegraphics[width=3.64583in,height=2.45551in]{images/media/image215.png}

\begin{enumerate}
\def\labelenumi{\Alph{enumi}.}
\item
  Intracellular Fluid

  \begin{enumerate}
  \def\labelenumii{\arabic{enumii}.}
  \item
    Which fluid compartment has the highest volume?
  \end{enumerate}
\end{enumerate}

\begin{quote}
\includegraphics[width=2.78125in,height=2.31283in,alt={This pie chart shows that about 55\% of water in the human body is intracellular fluid. About 30\% of the water in the human body is interstitial fluid. Most of the remaining 15\% of water is plasma, along with a small percentage labeled ``other fluid''.}]{images/media/image216.jpeg}
\end{quote}

\begin{enumerate}
\def\labelenumi{\Alph{enumi}.}
\setcounter{enumi}{1}
\item
  Extracellular Fluid

  \begin{enumerate}
  \def\labelenumii{\arabic{enumii}.}
  \item
    What travels in the plasma?
  \item
    What are some examples of other water-based portions of ECF?
  \end{enumerate}
\end{enumerate}

\begin{enumerate}
\def\labelenumi{\Roman{enumi}.}
\setcounter{enumi}{2}
\item
  Composition of Body Fluids

  \begin{enumerate}
  \def\labelenumii{\arabic{enumii}.}
  \item
    What is different between the ECF and interstitial fluid (IF)?
  \item
    What is found in high concentration in ICF that is in smaller
    amounts in ECF?
  \end{enumerate}
\end{enumerate}

\includegraphics[width=2.57292in,height=2.51875in,alt={This bar graph shows the concentration of several ions and proteins in intracellular fluid, interstitial fluid and plasma. The ions and proteins are categories on the X axis . The Y axis shows concentration, in milliequivalents per liter, ranging from zero to 160. Three different colored bars are shown above each compound on the X axis. One bar represents intracellular fluid (ICF), a second bar represents interstitial fluid (IF, which is part of ECF) and the third bar represents plasma (ECF). Intracellular fluid contains high concentrations of K plus and HPO four two minus. It has lower concentrations of MG two plus and protein, and negligible amounts of the other compounds. Interstitial fluid contains high concentrations of NA plus and CL minus, along with a smaller amount of HCO 3 minus, and negligible amounts of the other compounds. Plasma contains large concentrations of NA plus and CL minus, with smaller concentrations of HCO 3 minus and protein, and negligible amounts of the other compounds.}]{images/media/image217.jpeg}

\begin{enumerate}
\def\labelenumi{\arabic{enumi}.}
\setcounter{enumi}{2}
\item
  Do most body fluids have a charge associated with them?
\item
  How are the high levels of potassium and low levels of sodium
  maintained in the ICF?
\end{enumerate}

\includegraphics[width=4.16667in,height=1.92708in]{images/media/image218.png}

\begin{enumerate}
\def\labelenumi{\Roman{enumi}.}
\setcounter{enumi}{3}
\item
  Fluid Movement between Compartments

  \begin{enumerate}
  \def\labelenumii{\arabic{enumii}.}
  \item
    What is hydrostatic pressure?

    \begin{enumerate}
    \def\labelenumiii{\alph{enumiii})}
    \item
      How does this pressure help capillaries with filtration?
    \end{enumerate}
  \item
    What allows for reabsorption of fluid at the venous end of the
    capillary?
  \end{enumerate}
\end{enumerate}

\includegraphics[width=4.25903in,height=1.99642in,alt={Alt text to come.}]{images/media/image219.jpeg}

\begin{enumerate}
\def\labelenumi{\arabic{enumi}.}
\setcounter{enumi}{2}
\item
  Why is hydrostatic pressure vital for kidney filtration?
\item
  How does the osmotic gradient between fluid compartments allow for
  redistribution of water?
\item
  When sweating and losing water from tissue, how is this fluid
  replenished?
\end{enumerate}

\begin{enumerate}
\def\labelenumi{\Roman{enumi}.}
\setcounter{enumi}{4}
\item
  Solute Movement between Compartments

  \begin{enumerate}
  \def\labelenumii{\arabic{enumii}.}
  \item
    Active transport of solutes allows for movement from
    \_\_\_\_\_\_\_\_\_\_\_\_\_\_\_ concentration to
    \_\_\_\_\_\_\_\_\_\_\_\_ concentration.
  \item
    Passive movement of solutes moves them from \_\_\_\_\_\_\_\_\_\_\_
    concentration to \_\_\_\_\_\_\_\_\_\_\_\_ concentration.
  \item
    What can pass through the cell membrane without facilitation?
  \end{enumerate}
\end{enumerate}

\includegraphics[width=3.98958in,height=1.9283in,alt={This diagram shows a carrier protein embedded in the plasma membrane between the cytoplasm and the extracellular fluid. There are several glucose molecules in the extracellular fluid. In the first step, the carrier protein is open to the extracellular fluid and closed to the cytosol. One of the glucose molecules travels from the extracellular fluid into the carrier protein. The protein then changes shape, closing at both ends. This pushes the glucose down into the carrier protein, closer to the cytosol end. The protein then opens on the cytosol side and closes on the extracellular fluid side, allowing the glucose to enter the cytosol.}]{images/media/image220.jpeg}

26.2. Water balance

\begin{enumerate}
\def\labelenumi{\Roman{enumi}.}
\item
  Background

  \begin{enumerate}
  \def\labelenumii{\arabic{enumii}.}
  \item
    Is all water in the body ingested, or can some be generated?
  \item
    How does most water leave the body?
  \end{enumerate}
\item
  \includegraphics[width=2.61458in,height=4.27966in,alt={This figure is a top-to bottom flowchart describing the thirst response. The topmost box of the chart states that there is insufficient water in the body, which has two effects. The left branch of the chart leads to decreased blood volume, which leads to decreased blood pressure. This triggers an increase in angiotensin two. Angiotensin two stimulates the thirst center in the hypothalamus. On the right branch, insufficient water in the body leads to increased blood osmolality, which causes dry mouth. Increased blood osmolality and dry mouth is sensed by osmoreceptors in the hypothalamus. This stimulates the thirst center in the hypothalamus to increase thirst, giving a person the urge to drink. Drinking decreases blood osmolality back to homeostatic levels.}]{images/media/image221.jpeg}Regulation
  of Water Intake

  \begin{enumerate}
  \def\labelenumii{\arabic{enumii}.}
  \item
    What is plasma osmolality?
  \item
    In a state of dehydration how does the body act to conserve water?
  \item
    What triggers thirst?
  \item
    How are baroreceptors able to detect changes that occur due to
    dehydration?
  \item
    How does the renin-angiotensin-aldosterone system help to conserve
    water during dehydration?
  \item
    What can happen after a prolonged period of dehydration?
  \end{enumerate}
\item
  Regulation of Water Output

  \begin{enumerate}
  \def\labelenumii{\arabic{enumii}.}
  \item
    If the body cannot produce the minimum necessary amount of urine
    what happens to metabolic wastes?
  \item
    What is diuresis?
  \end{enumerate}
\item
  Role of ADH

  \begin{enumerate}
  \def\labelenumii{\arabic{enumii}.}
  \item
    Where is ADH or vasopressin produced?

    \begin{enumerate}
    \def\labelenumiii{\alph{enumiii})}
    \item
      How do osmoreceptors help to determine how much vasopressin is
      produced?
    \end{enumerate}
  \item
    What are the two major effects of ADH vasopressin?
  \item
    \includegraphics[width=3in,height=2.42639in,alt={This set of diagrams shows the effects of ADH on various structures within the body. In the brain, ADH affects the cerebrum by influencing social behavior in some mammals. ADH is also produced in the brain by the hypothalamus and released in the posterior pituitary. ADH also constricts arterioles in the body, which are the small arteries that enter into capillary beds. Finally, a kidney is shown because ADH increases the reabsorption of water in the kidneys.}]{images/media/image222.jpeg}\includegraphics[width=3.29167in,height=2.39306in,alt={This diagram depicts a cross section of the right wall of a kidney collecting tubule. The wall is composed of three block-shaped cells arranged vertically one on top of each other. The lumen of the collecting tubule is to the left of the three cells. Yellow-colored urine is flowing through the lumen. There is a small strip of blue interstitial fluid to the right of the three cells. To the right of the interstitial fluid is a cross section of a blood vessel. Arrows show that water in the urine is entering the left side of the wall cells through aquaporins. The water travels through the cells and then leaves the kidney tubule through additional aquaporins in the right side of the wall cells. The water travels through the interstitial space and enters into the blood in the blood vessel. The aquaporins in the wall cells are being released from aquaporin storage vesicles within their cytoplasm.}]{images/media/image223.jpeg}When
    ADH vasopressin are present would you expect more or less aquaporins
    in the collecting tubules?
  \item
    What are some examples of diuretics that are regularly consumed in
    everyday life?
  \end{enumerate}
\end{enumerate}

\subsection{26.3. Electrolyte Balance}\label{electrolyte-balance}

\begin{enumerate}
\def\labelenumi{\Roman{enumi}.}
\item
  Background

  \begin{enumerate}
  \def\labelenumii{\arabic{enumii}.}
  \item
    What are some of the roles ions, or electrolytes, fulfill in the
    body?
  \item
    What are the six most important electrolytes?
  \end{enumerate}
\item
  Role of Electrolytes

  \begin{enumerate}
  \def\labelenumii{\arabic{enumii}.}
  \item
    Where are most calcium and phosphate found in the body?
  \item
    How are most ions excreted?
  \end{enumerate}

  \begin{enumerate}
  \def\labelenumii{\Alph{enumii}.}
  \item
    Sodium

    \begin{enumerate}
    \def\labelenumiii{\arabic{enumiii}.}
    \item
      Sodium is the major cation of
      \_\_\_\_\_\_\_\_\_\_\_\_\_\_\_\_\_\_\_\_\_\_\_\_\_\_\_\_ fluid.
    \item
      Excess sodium can contribute to what pathophysiological condition?
    \item
      What can cause hyponatremia?
    \item
      How can a relative decrease in sodium result in swelling of cells?
    \item
      What can cause hypernatremia?
    \end{enumerate}
  \item
    Potassium

    \begin{enumerate}
    \def\labelenumiii{\arabic{enumiii}.}
    \item
      Potassium is the major
      \_\_\_\_\_\_\_\_\_\_\_\_\_\_\_\_\_\_\_\_\_\_\_\_ cation.
    \item
      What is hypokalemia?
    \item
      What can lead to hypokalemia?
    \item
      What is hyperkalemia?
    \item
      Why is hyperkalemia dangerous for the function of the heart?
    \end{enumerate}
  \item
    Chloride

    \begin{enumerate}
    \def\labelenumiii{\arabic{enumiii}.}
    \item
      Is the majority of chloride normally found inside and outside of
      cells?
    \item
      Hyperchloremia can occur due to what?
    \end{enumerate}
  \item
    Bicarbonate

    \begin{enumerate}
    \def\labelenumiii{\arabic{enumiii}.}
    \item
      Bicarbonate has what charge?
    \item
      Water and what else can combine to create carbonic acid and
      therefore bicarbonate?
    \item
      Tissues that are highly metabolizing produce more or less carbon
      dioxide?
    \end{enumerate}
  \item
    Calcium

    \begin{enumerate}
    \def\labelenumiii{\arabic{enumiii}.}
    \item
      Aside from bones, what else is calcium used for in the body?
    \item
      How can a deficiency in vitamin D lead to hypocalcemia?
    \end{enumerate}
  \item
    Phosphate

    \begin{enumerate}
    \def\labelenumiii{\arabic{enumiii}.}
    \item
      \includegraphics[width=2.3625in,height=3.56319in,alt={This flow chart shows how potassium and sodium ion concentrations in the blood are regulated by aldosterone. Rising K plus and falling NA plus levels in the blood trigger aldosterone release from the adrenal cortex. Aldosterone targets the kidneys, causing a decrease in K plus release from the kidneys, which reduces the amount of K plus in the blood back to homeostatic levels. Aldosterone also increases sodium reabsorption by the kidneys, which increases the amount of NA plus in the blood back to homeostatic levels.}]{images/media/image224.jpeg}Where
      is most of the phosphate in the body found?
    \item
      Aside from bones what is phosphate used for in the body?
    \item
      What can lead to hypophosphatemia?
    \end{enumerate}
  \end{enumerate}
\end{enumerate}

\begin{enumerate}
\def\labelenumi{\Roman{enumi}.}
\setcounter{enumi}{1}
\item
  Regulation of Sodium and Potassium

  \begin{enumerate}
  \def\labelenumii{\arabic{enumii}.}
  \item
    Sodium is \_\_\_\_\_\_\_\_\_\_\_\_\_\_\_\_\_\_\_ from the renal
    filtrate, and potassium is \_\_\_\_\_\_\_\_\_\_ into the filtrate.
  \end{enumerate}

  \begin{enumerate}
  \def\labelenumii{\Alph{enumii}.}
  \item
    Aldosterone

    \begin{enumerate}
    \def\labelenumiii{\arabic{enumiii}.}
    \item
      Aldosterone increases the excretion of
      \_\_\_\_\_\_\_\_\_\_\_\_\_\_\_ and the reabsorption of
      \_\_\_\_\_\_\_\_\_\_\_\_\_\_\_.
    \item
      Overall aldosterone increases \_\_\_\_\_\_\_\_\_\_\_\_ levels in
      the plasma.
    \end{enumerate}
  \item
    Angiotensin II

    \begin{enumerate}
    \def\labelenumiii{\arabic{enumiii}.}
    \item
      The vasoconstriction and increase in blood pressure has what
      effect on GFR?
    \item
      Angiotensin II signal for an increase in what hormone from the
      adrenal cortex?
    \end{enumerate}
  \end{enumerate}
\end{enumerate}

\begin{quote}
\includegraphics[width=4.84811in,height=2.2125in,alt={This figure shows the hormone cascade that that increases kidney reabsorption of NA plus and water. In the first step, the kidneys release renin into the blood stream. The blood stream is depicted with a red arrow pointing from left to right. At the same time, the liver releases angiotensinogen into the blood, which combines with the renin, yielding angiotensin one. The blood flow then leads to the lungs. Within the pulmonary blood, angiotensin-converting enzyme (ACE) converts angiotensin one to angiotensin two. The blood then flows to the adrenal cortex, where angiotensin two stimulates the adrenal cortex to secrete aldosterone. Aldosterone causes the kidney tubules to increase reabsorption of NA plus and water into the blood.}]{images/media/image225.jpeg}
\end{quote}

\begin{enumerate}
\def\labelenumi{\Roman{enumi}.}
\setcounter{enumi}{2}
\item
  Regulation of Calcium and Phosphate

  \begin{enumerate}
  \def\labelenumii{\arabic{enumii}.}
  \item
    What is the overall effect of parathyroid hormone (PTH) on blood
    calcium?
  \item
    What effect does dihydroxyvitamin D (calcitriol) have on the
    intestinal epithelial cells?
  \item
    What effect does calcitonin have on blood calcium?
  \end{enumerate}
\end{enumerate}

26.4. Acid-Base Balance

\begin{enumerate}
\def\labelenumi{\Roman{enumi}.}
\item
  Background

  \begin{enumerate}
  \def\labelenumii{\arabic{enumii}.}
  \item
    Why are buffers important to regulating blood pH?
  \item
    What are the most common buffers? Weak
    \_\_\_\_\_\_\_\_\_\_\_\_\_\_\_\_\_ or weak
    \_\_\_\_\_\_\_\_\_\_\_\_\_\_\_\_.
  \end{enumerate}
\item
  Buffer Systems in the Body

  \begin{enumerate}
  \def\labelenumii{\arabic{enumii}.}
  \item
    How fast are buffers systems able to act relative to the respiratory
    or renal mechanisms that regulate blood pH?
  \item
    Plasma proteins, phosphate, and bicarbonate are all
    \_\_\_\_\_\_\_\_\_\_\_ systems in the blood.
  \end{enumerate}

  \begin{enumerate}
  \def\labelenumii{\Alph{enumii}.}
  \item
    Proteins Buffers in Blood Plasma and Cells

    \begin{enumerate}
    \def\labelenumiii{\arabic{enumiii}.}
    \item
      Protein buffering accounts for how much of the total buffering
      power of the blood and within cells?
    \end{enumerate}
  \item
    Hemoglobin as a Buffer

    \begin{enumerate}
    \def\labelenumiii{\arabic{enumiii}.}
    \item
      Hemoglobin is found inside of what type of cell?
    \item
      The conversion of CO\textsubscript{2} into
      \_\_\_\_\_\_\_\_\_\_\_\_ allows for buffering.
    \end{enumerate}
  \item
    Phosphate Buffer

    \begin{enumerate}
    \def\labelenumiii{\arabic{enumiii}.}
    \item
      Phosphates can buffer both \_\_\_\_\_\_\_\_\_\_\_\_\_\_\_ and
      \_\_\_\_\_\_\_\_\_\_\_\_\_\_\_\_\_.
    \end{enumerate}
  \item
    Bicarbonate-Carbonic Acid Buffer

    \begin{enumerate}
    \def\labelenumiii{\arabic{enumiii}.}
    \item
      The weak acid or weak bases of carbonic acid and bicarbonate
      prevent a significant change in \_\_\_\_\_\_ of the blood.
    \item
      Is there normally more bicarbonate or carbonic acid in the blood?
    \item
      Carbonic acid levels are controlled by exhalation of
      CO\textsubscript{2}, but what controls the levels of bicarbonate
      in the blood?
    \end{enumerate}
  \end{enumerate}
\end{enumerate}

\begin{enumerate}
\def\labelenumi{\Roman{enumi}.}
\setcounter{enumi}{1}
\item
  \includegraphics[width=2.91667in,height=5.3125in,alt={This top to bottom flowchart describes the regulation of PH in the blood. The left branch shows acidosis, which is when the PH of the blood drops. Acidosis stimulates brain and arterial receptors, triggering an increase in respiratory rate. This causes a drop in blood CO two and H two CO three. A drop in these two acidic compounds causes the blood PH to rise back to homeostatic levels. The right branch shows alkalosis which is when the PH of the blood rises. Alkalosis also stimulates brain and arterial receptors, but these now trigger a decrease in respiratory rate. This causes an increase in blood CO two and H two CO three, which lowers the PH of the blood back to homeostatic levels.}]{images/media/image226.jpeg}Respiratory
  Regulation of Acid-Base Balance

  \begin{enumerate}
  \def\labelenumii{\arabic{enumii}.}
  \item
    When holding your breath CO2 in the blood rises and therefore
    \_\_\_\_\_\_\_\_\_\_\_\_\_\_ \_\_\_\_\_\_\_\_\_\_\_\_\_\_ does as
    well.
  \item
    Hyperventilation has what effect on carbonic acid in the blood?
  \item
    Why does respiratory rate and depth increase with exercise?

    \begin{enumerate}
    \def\labelenumiii{\alph{enumiii})}
    \item
      How do chemoreceptors play a role?
    \end{enumerate}
  \item
    What is hyper and hypocapnia?
  \item
    What substances or drugs can cause hypercapnia?
  \end{enumerate}
\item
  Renal Regulation of Acid-Base Balance

  \begin{enumerate}
  \def\labelenumii{\arabic{enumii}.}
  \item
    Renal regulation addresses the \_\_\_\_\_\_\_\_\_\_\_\_\_\_\_\_
    component of the buffer system.
  \item
    Inhibition of carbonic anhydrase can have what effect on blood
    bicarbonate?
  \item
    Cells of the renal tubules are not permeable to bicarbonate so how
    does the body maintain enough bicarbonate?
  \end{enumerate}
\end{enumerate}

\includegraphics[width=3.71875in,height=2.18392in,alt={This diagram depicts a cross section of the left wall of a kidney proximal tubule. The wall is composed of two block-shaped cells arranged vertically one on top of each other. The lumen of the proximal tubule is to the left of the two cells. Yellow-colored urine is flowing through the lumen. There is a small strip of blue interstitial fluid to the right of the two cells. To the right of the interstitial fluid is a cross section of a blood vessel. A loop of chemical reactions is occurring in the diagram. Within the lumen of the proximal tubule, HCO three minus is combining with an H plus ion that enters the lumen from a proximal tubule cell. This reaction forms H two CO three. H two CO three then breaks into H two O and CO two, a reaction catalyzed by the enzyme carbonic anhydrase. The CO two then moves from the lumen of the proximal tubule into one of the proximal tubule cells. There, the reaction runs in reverse, with CO two combining with H two O to form H two CO three. The H two CO three then splits into H plus and HCO three minus. The H plus moves into the lumen, reinitiating the first step of the loop. The HCO three minus leaves the proximal tubule cell and enters the blood stream.}]{images/media/image227.jpeg}

\begin{enumerate}
\def\labelenumi{\arabic{enumi}.}
\setcounter{enumi}{3}
\item
  What happens to bicarbonate if sulfates or phosphates capture hydrogen
  ions and prevent them from being used to create carbonic acid?
\item
  Hydrogen ions also compete with what ion in the renal tubules?
\item
  Why can bicarbonate be used in place of chloride in neutralizing
  positive ion charges?
\end{enumerate}

\subsubsection{26.5. Disorders of Acid-Base
Balance}\label{disorders-of-acid-base-balance}

\begin{enumerate}
\def\labelenumi{\Roman{enumi}.}
\item
  Background

  \begin{enumerate}
  \def\labelenumii{\arabic{enumii}.}
  \item
    What is physiological acidosis?

    \begin{enumerate}
    \def\labelenumiii{\alph{enumiii})}
    \item
      What are some of the symptoms?
    \end{enumerate}
  \item
    Why is the respiratory contribution to acid-base balance discussed
    in terms of CO\textsubscript{2}?
  \end{enumerate}
\end{enumerate}

\includegraphics[width=4.83923in,height=3.13542in,alt={This figure points out the symptoms of acidosis and alkalosis on a silhouette of a human torso. The effects of acidosis on the central nervous system include headache, sleepiness, confusion, loss of consciousness and coma. The effects of acidosis are given on the left side of the diagram. The effects of acidosis on the respiratory system include shortness of breath and coughing. The effects of acidosis on the heart include arrhythmia and increased heart rate. The effects of acidosis on the muscular system include seizures and weakness. The effects of acidosis on the digestive system include nausea, vomiting and diarrhea. The right side of the diagram describes the symptoms of alkalosis. The effects of alkalosis on the central nervous system include confusion, light-headedness, stupor, and coma. The effects of alkalosis on the peripheral nervous system include hand tremor and numbness or tingling in the face, hands, and feet. The effects of alkalosis on the muscular system include twitching and prolonged spasms. The effects of alkalosis on the digestive system include nausea and vomiting.}]{images/media/image228.jpeg}

\begin{enumerate}
\def\labelenumi{\Roman{enumi}.}
\setcounter{enumi}{1}
\item
  Metabolic Acidosis: Primary Bicarbonate Deficiency

  \begin{enumerate}
  \def\labelenumii{\arabic{enumii}.}
  \item
    When there is a bicarbonate deficiency it leads to a
    \_\_\_\_\_\_\_\_\_\_\_\_\_\_\_\_\_\_ acidosis.
  \item
    Are all causes of metabolic acidosis caused medical or
    pathophysiological conditions?
  \end{enumerate}
\item
  Metabolic Alkalosis: Primary Bicarbonate Excess

  \begin{enumerate}
  \def\labelenumii{\arabic{enumii}.}
  \item
    What common over the counter medication can cause a transient
    metabolic alkalosis?
  \end{enumerate}
\item
  Respiratory Acidosis: Primary Carbonic Acid/CO\textsubscript{2} Excess

  \begin{enumerate}
  \def\labelenumii{\arabic{enumii}.}
  \item
    Respiratory acidosis occurs due to an excess of
    \_\_\_\_\_\_\_\_\_\_\_\_\_\_\_\_ acid.
  \end{enumerate}
\item
  Respiratory Alkalosis: Primary Carbonic Acid/CO\textsubscript{2}
  Deficiency

  \begin{enumerate}
  \def\labelenumii{\arabic{enumii}.}
  \item
    How can hyperventilation lead to a respiratory alkalosis?
  \end{enumerate}
\item
  Compensation Mechanisms

  \begin{enumerate}
  \def\labelenumii{\arabic{enumii}.}
  \item
    What are three compensatory mechanisms that keep blood pH in the
    appropriate range?
  \end{enumerate}

  \begin{enumerate}
  \def\labelenumii{\Alph{enumii}.}
  \item
    Respiratory Compensation

    \begin{enumerate}
    \def\labelenumiii{\arabic{enumiii}.}
    \item
      How rapidly can respiratory compensation occur?
    \item
      Why is the respiratory system better suited to handle a metabolic
      acidosis rather than a metabolic alkalosis?
    \end{enumerate}
  \item
    Metabolic Compensation

    \begin{enumerate}
    \def\labelenumiii{\arabic{enumiii}.}
    \item
      In a respiratory acidosis how do the kidneys help to compensate?

      \begin{enumerate}
      \def\labelenumiv{\alph{enumiv})}
      \item
        Can this compensation continue to be increase indefinitely?
      \end{enumerate}
    \end{enumerate}
  \item
    Diagnosing Acidosis and Alkalosis

    \begin{enumerate}
    \def\labelenumiii{\arabic{enumiii}.}
    \item
      In metabolic acidosis pCO\textsubscript{2} is normal and then what
      happens?
    \item
      Why would bicarbonate levels in the blood increase as a
      compensation to respiratory acidosis?
    \item
      Respiratory alkalosis would have what renal compensation?
    \end{enumerate}
  \end{enumerate}
\end{enumerate}

\section{}\label{section-28}

\section[Chapter 27
]{\texorpdfstring{\protect\includegraphics[width=3.67639in,height=3.99722in,alt={This figure shows the different organs in the male reproductive system. The top panel shows the side view of a man and an uncircumcised and a circumcised penis. The bottom panel shows the lateral view of the male reproductive system and the major parts are labeled.}]{images/media/image229.jpeg}Chapter
27
}{This figure shows the different organs in the male reproductive system. The top panel shows the side view of a man and an uncircumcised and a circumcised penis. The bottom panel shows the lateral view of the male reproductive system and the major parts are labeled.Chapter 27 }}\label{this-figure-shows-the-different-organs-in-the-male-reproductive-system.-the-top-panel-shows-the-side-view-of-a-man-and-an-uncircumcised-and-a-circumcised-penis.-the-bottom-panel-shows-the-lateral-view-of-the-male-reproductive-system-and-the-major-parts-are-labeled.chapter-27}

27.1. Anatomy and Physiology of the Male Reproductive System

\begin{enumerate}
\def\labelenumi{\Roman{enumi}.}
\item
  Background

  \begin{enumerate}
  \def\labelenumii{\arabic{enumii}.}
  \item
    What is a gamete?
  \item
    A sperm has how many chromosomes compared to a somatic cell (cell of
    the body)?
  \item
    What is the purpose of sperm?
  \end{enumerate}
\item
  Scrotum

  \begin{enumerate}
  \def\labelenumii{\arabic{enumii}.}
  \item
    Why is the location of the sperm important?
  \item
    How do the dartos and cremaster muscles help to regulate the
    temperature of the testes?
  \end{enumerate}
\end{enumerate}

\includegraphics[width=5.39773in,height=2.375in,alt={This figure shows the scrotum and testes. The left panel shows the external view of the scrotum, the middle panel shows the muscle layer and the right panel shows the deep tissues of the scrotum.}]{images/media/image230.jpeg}

\begin{enumerate}
\def\labelenumi{\Roman{enumi}.}
\setcounter{enumi}{2}
\item
  Testes

  \begin{enumerate}
  \def\labelenumii{\arabic{enumii}.}
  \item
    The testes are the male gonads. What are gonads?
  \item
    The tunica albuginea both covers the testis and helps to divide the
    testis into \_\_\_\_\_\_\_\_\_\_\_\_\_\_\_\_\_\_\_\_.
  \item
    When do the testis descend into the scrotal cavity?
  \item
    \includegraphics[width=2.97153in,height=2.74028in,alt={This diagram shows the cross section of the testis.}]{images/media/image231.jpeg}What
    develops in the seminiferous tubules?
  \item
    Formed sperm are released into where within the testes?
  \end{enumerate}

  \begin{enumerate}
  \def\labelenumii{\Alph{enumii}.}
  \item
    Sertoli Cells

    \begin{enumerate}
    \def\labelenumiii{\arabic{enumiii}.}
    \item
      \_\_\_\_\_\_\_\_\_\_\_\_ cells, or sustentacular cells, secrete
      what types of signals?
    \item
      Why is a blood-testis barrier necessary?
    \end{enumerate}
  \item
    Germ Cells

    \begin{enumerate}
    \def\labelenumiii{\arabic{enumiii}.}
    \item
      What are the stem cells of the testis?
    \item
      The division of germ cells produces what type of spermatocyte?
    \item
      What is spermatogenesis?
    \end{enumerate}
  \item
    Spermatogenesis

    \begin{enumerate}
    \def\labelenumiii{\arabic{enumiii}.}
    \item
      When in the lifecycle does spermatogenesis start?
    \item
      Mitosis begins with a diploid spermatogonia, but the mature
      gametes are \_\_\_\_\_\_\_\_\_\_\_\_\_\_ and contain 23
      chromosomes.
    \item
      \_\_\_\_\_\_\_\_\_\_\_\_\_\_\_\_\_\_\_\_ is the process of
      cellular division to produce haploid gametes.
    \item
      \includegraphics[width=4.63472in,height=3.34514in,alt={This figure shows the steps in spermatogenesis. The left panel shows a flow chart that outlines the different steps in the formation of sperm. The right panel shows a micrograph with the cross section of a seminiferous tubule.}]{images/media/image232.jpeg}After
      mitosis of the spermatogonia the
      \_\_\_\_\_\_\_\_\_\_\_\_\_\_\_\_\_\_\_ spermatocyte begins meiosis
      I.
    \item
      \_\_\_\_\_\_\_\_\_\_\_\_\_\_\_\_\_\_\_ spermatocytes each with
      half the number of chromosomes undergo meiosis II.
    \item
      After the second meiotic division the new cells produced are
      \_\_\_\_\_\_\_\_\_\_\_\_\_\_\_\_.
    \item
      What is the process that transforms the early spermatids into true
      sperm?
    \end{enumerate}
  \end{enumerate}
\end{enumerate}

\begin{enumerate}
\def\labelenumi{\Roman{enumi}.}
\setcounter{enumi}{2}
\item
  Structure of Formed Sperm

  \begin{enumerate}
  \def\labelenumii{\arabic{enumii}.}
  \item
    Where is the haploid nucleus of a sperm found?
  \item
    What fills the acrosomal cap? Why?
  \item
    Why is the mid-piece of the sperm filled with tightly packed
    mitochondria?
  \item
    The flagellum helps the sperm to perform what function?
  \end{enumerate}
\end{enumerate}

\includegraphics[width=4.125in,height=1.1175in,alt={This diagram shows the structure of sperm; the major parts are labeled.}]{images/media/image233.jpeg}

\begin{enumerate}
\def\labelenumi{\Roman{enumi}.}
\setcounter{enumi}{3}
\item
  Sperm Transport

  \begin{enumerate}
  \def\labelenumii{\arabic{enumii}.}
  \item
    Where are sperm transported from?

    \begin{enumerate}
    \def\labelenumiii{\alph{enumiii})}
    \item
      Where are they transported to?
    \end{enumerate}
  \end{enumerate}

  \begin{enumerate}
  \def\labelenumii{\Alph{enumii}.}
  \item
    Role of the Epididymis

    \begin{enumerate}
    \def\labelenumiii{\arabic{enumiii}.}
    \item
      After passing through the seminiferous tubules, sperm continue to
      mature in the
      \_\_\_\_\_\_\_\_\_\_\_\_\_\_\_\_\_\_\_\_\_\_\_\_\_\_.
    \item
      As sperm move through the epididymis, they acquire the ability to
      move on their own. Why is this important?
    \end{enumerate}
  \item
    Duct System

    \begin{enumerate}
    \def\labelenumiii{\arabic{enumiii}.}
    \item
      During ejaculation where do sperm pass through to pass towards the
      urethra?
    \item
      The spermatic cord includes the ductus deferens and what else?
    \item
      What is a vasectomy?
    \item
      What is the inguinal canal, why does the ductus deferens pass
      through it?
    \item
      What glands contribute to the contents of semen?
    \end{enumerate}
  \item
    Seminal Vesicles

    \begin{enumerate}
    \def\labelenumiii{\arabic{enumiii}.}
    \item
      The seminal vesicles contribute a large amount of fructose to
      semen. Why is this necessary?
    \item
      The ejaculatory ducts transport the semen to what gland next?
    \end{enumerate}
  \item
    Prostate Gland

    \begin{enumerate}
    \def\labelenumiii{\arabic{enumiii}.}
    \item
      The prostate gland helps to thicken the sperm why is this
      beneficial in the female reproductive tract?
    \item
      What are some of the symptoms of an enlarged prostate?
    \end{enumerate}
  \item
    Bulbourethral Gland

    \begin{enumerate}
    \def\labelenumiii{\arabic{enumiii}.}
    \item
      The fluid from the bulbourethral glands helps to do what?
    \end{enumerate}
  \end{enumerate}
\end{enumerate}

\begin{enumerate}
\def\labelenumi{\Roman{enumi}.}
\setcounter{enumi}{4}
\item
  The Penis

  \begin{enumerate}
  \def\labelenumii{\arabic{enumii}.}
  \item
    The penis is \_\_\_\_\_\_\_\_\_\_\_\_\_ for non-sexual actions, and
    \_\_\_\_\_\_\_\_\_\_\_\_\_ with sexual arousal.
  \item
    What are the chambers found inside the shaft of the penis?
  \item
    The two larger chambers are the \_\_\_\_\_\_\_\_\_\_\_\_\_\_\_\_\_
    \_\_\_\_\_\_\_\_\_\_\_\_\_\_\_\_\_\_\_\_\_\_\_\_\_.
  \item
    The corpus spongiosum surrounds the \_\_\_\_\_\_\_\_\_\_\_\_\_\_\_.
  \item
    The end of the penis is the \_\_\_\_\_\_\_\_\_\_\_\_\_\_
    \_\_\_\_\_\_\_\_\_\_\_\_\_\_\_\_.
  \item
    The foreskin or \_\_\_\_\_\_\_\_\_\_\_\_\_\_\_ is a collar around
    the penis.
  \item
    What is the role of Nitric Oxide in the erection process?
  \end{enumerate}
\end{enumerate}

\includegraphics[width=4.125in,height=3.7875in,alt={ This multipart diagram shows the cross section of the penis. The top left panel shows the lateral view of the flaccid penis and the top right panel shows the transverse view. The bottom left panel shows the lateral view of the erect penis and the bottom right panel shows the transverse view.}]{images/media/image234.jpeg}

\begin{enumerate}
\def\labelenumi{\Roman{enumi}.}
\setcounter{enumi}{5}
\item
  Testosterone

  \begin{enumerate}
  \def\labelenumii{\arabic{enumii}.}
  \item
    What cells produce testosterone?
  \end{enumerate}

  \begin{enumerate}
  \def\labelenumii{\Alph{enumii}.}
  \item
    Functions of Testosterone

    \begin{enumerate}
    \def\labelenumiii{\arabic{enumiii}.}
    \item
      Why is testosterone important for the male reproductive system?

      \begin{enumerate}
      \def\labelenumiv{\alph{enumiv})}
      \item
        What happens when there are low levels?
      \end{enumerate}
    \item
      Is all testosterone produced in the testes?
    \end{enumerate}
  \item
    Control of Testosterone
  \end{enumerate}
\end{enumerate}

\includegraphics[width=4.10417in,height=2.75352in,alt={This figure shows the steps in the regulation of testosterone production. The top panel shows the hypothalamus and the bottom panel shows two micrographs. The left micrograph is that of sertoli cells and the right micrograph is that of Leydig cells.}]{images/media/image235.jpeg}

\begin{enumerate}
\def\labelenumi{\arabic{enumi}.}
\item
  How does gonadotropin-releasing hormone (GnRH) lead to production of
  testosterone?
\item
  What effect does follicle-stimulating hormone have on the male
  reproductive system?
\item
  What happens to GnRH secretion when testosterone is high? Low?

  \begin{enumerate}
  \def\labelenumii{\alph{enumii})}
  \item
    What type of feedback governs this system?
  \end{enumerate}
\end{enumerate}

\subsection{27.2. Anatomy and Physiology of the Female Reproductive
System}\label{anatomy-and-physiology-of-the-female-reproductive-system}

\begin{enumerate}
\def\labelenumi{\Roman{enumi}.}
\item
  \includegraphics[width=2.83542in,height=4.03125in]{images/media/image236.png}Background

  \begin{enumerate}
  \def\labelenumii{\arabic{enumii}.}
  \item
    What is similar and different between the male and female
    reproductive systems?
  \item
    The ovaries are the female gonads. What gamete do they produce?
  \end{enumerate}
\item
  External Female Genitalia

  \begin{enumerate}
  \def\labelenumii{\arabic{enumii}.}
  \item
    What is the vulva?
  \item
    What changes about the mons pubis after puberty?
  \item
    What role do the labia minora serve?
  \item
    Why does the clitoris have an abundant nerve supply?
  \item
    Why is the hymen only a partial membrane?
  \item
    The greater vestibular glands that are found on either side of the
    vaginal opening are known as \_\_\_\_\_\_\_\_\_\_\_\_\_\_\_\_\_\_\_
    \_\_\_\_\_\_\_\_\_\_\_\_.
  \end{enumerate}
\end{enumerate}

\includegraphics[width=4.21681in,height=2.46875in,alt={This figure shows the parts of the vulva. The right panel shows the external anterior view and the left panel shows the internal anteriolateral view. The major parts are labeled. }]{images/media/image237.jpeg}

\begin{enumerate}
\def\labelenumi{\Roman{enumi}.}
\setcounter{enumi}{2}
\item
  Vagina

  \begin{enumerate}
  \def\labelenumii{\arabic{enumii}.}
  \item
    The vagina allows for the exit of what materials?
  \item
    The middle and inner layers of the vagina, including the rugae,
    allow for what?
  \item
    Microorganisms that live inside the vagina protect against what?
  \end{enumerate}
\item
  Ovaries

  \begin{enumerate}
  \def\labelenumii{\arabic{enumii}.}
  \item
    The mesovarium supports the ovaries and connects them to the
    \_\_\_\_\_\_\_\_\_\_\_ ligament.
  \item
    The oocyte and the supporting cells are collectively known as a
    \_\_\_\_\_\_\_\_\_\_\_\_.
  \end{enumerate}
\item
  The Ovarian Cycle

  \begin{enumerate}
  \def\labelenumii{\arabic{enumii}.}
  \item
    How long is the ovarian cycle?

    \begin{enumerate}
    \def\labelenumiii{\alph{enumiii})}
    \item
      Is this the same as the menstrual cycle?
    \end{enumerate}
  \item
    What is oogenesis and folliculogenesis? Are they the same?
  \end{enumerate}

  \begin{enumerate}
  \def\labelenumii{\Alph{enumii}.}
  \item
    Oogenesis

    \begin{enumerate}
    \def\labelenumiii{\arabic{enumiii}.}
    \item
      Gametogenesis in females in oogenesis. What is the ovarian stem
      cell used for this process?
    \item
      When do oogonia form primary oocytes?

      \begin{enumerate}
      \def\labelenumiv{\alph{enumiv})}
      \item
        \includegraphics[width=3.78125in,height=3.1625in,alt={This flowchart shows the formation of oocytes in the female. The top half of the flowchart is before birth and the bottom half is after puberty. A callout to the left also shows the eggs before and after sperm penetration. }]{images/media/image238.jpeg}What
        happens to these primary oocytes until puberty?
      \item
        What happens to the number of primary oocytes throughout the
        lifetime?
      \end{enumerate}
    \item
      What is ovulation?
    \item
      What triggers the resumption of meiosis in a primary oocyte?
    \item
      What are polar bodies?

      \begin{enumerate}
      \def\labelenumiv{\alph{enumiv})}
      \item
        Are these viable gametes?
      \end{enumerate}
    \item
      When is meiosis of a secondary oocyte completed?

      \begin{enumerate}
      \def\labelenumiv{\alph{enumiv})}
      \item
        After meiosis II a haploid \_\_\_\_\_\_\_\_\_\_\_ is produced.
      \end{enumerate}
    \item
      \includegraphics[width=3.32431in,height=4.37014in,alt={This multipart figure shows how follicles are generated. The top panel shows the six stages of folliculogenesis. In each stage, the major cell types are labeled. The bottom part shows a micrograph of a secondary follicle and the major parts are labeled. }]{images/media/image239.jpeg}Does
      the sperm contribute cytoplasm or organelles during fertilization?
    \end{enumerate}
  \item
    Folliculogenesis

    \begin{enumerate}
    \def\labelenumiii{\arabic{enumiii}.}
    \item
      \_\_\_\_\_\_\_\_\_\_\_\_\_\_\_\_\_\_ is the growth and development
      of follicles.
    \item
      Primordial follicles are at a resting state and have a single
      layer of cells known as the \_\_\_\_\_\_\_\_\_\_\_\_\_\_\_
      \_\_\_\_\_\_\_\_\_.
    \item
      Primordial follicles that respond to recruitment signals develop
      into \_\_\_\_\_\_\_\_\_\_\_\_\_ \_\_\_\_\_\_\_\_\_\_\_\_\_\_\_.
    \item
      Secondary follicles develop connective tissue, blood vessels, and
      theca cells. Theca cells produce \_\_\_\_\_\_\_\_\_\_\_\_\_\_\_\_.
    \item
      Follicles in the antrum that are fully formed are
      \_\_\_\_\_\_\_\_\_\_\_\_\_\_ follicles.
    \item
      What is atresia?
    \end{enumerate}
  \item
    Hormonal Control over the Ovarian Cycle

    \begin{enumerate}
    \def\labelenumiii{\arabic{enumiii}.}
    \item
      GnRH signal the anterior pituitary to produce what two hormones?
    \item
      \includegraphics[width=3.5in,height=4.09818in,alt={This figure shows three flowcharts. The flowchart on the top left shows the hormonal regulation of the follicular phase. The flowchart on the top right shows the hormonal regulation of the ovulation phase. The bottom flowchart shows the hormonal regulation of luteal phase. }]{images/media/image240.jpeg}FSH
      stimulates follicles to do what?
    \item
      LH results in the production of what hormone?
    \item
      The dominant follicle releases so much estrogen the negative
      feedback doesn't occur. What happens instead?
    \item
      A surge of what hormone leads to ovulation?
    \item
      After ovulation the remainder of the follicle forms the
      \_\_\_\_\_\_\_\_\_\_\_ \_\_\_\_\_\_\_\_\_\_\_\_ or ``yellowish
      body.''
    \item
      The corpus luteum does not release estrogen, but instead it
      releases the hormone \_\_\_\_\_\_\_\_\_\_\_\_\_\_\_\_\_\_\_\_\_
      which triggers the negative feedback in the hypothalamus and
      pituitary.
    \item
      If pregnancy does not occur the corpus luteum degrades into the
      \_\_\_\_\_\_\_\_\_\_\_\_\_ \_\_\_\_\_\_\_\_\_\_\_\_\_\_\_.
    \end{enumerate}
  \end{enumerate}
\end{enumerate}

\begin{enumerate}
\def\labelenumi{\Roman{enumi}.}
\setcounter{enumi}{4}
\item
  The Uterine Tubes

  \begin{enumerate}
  \def\labelenumii{\arabic{enumii}.}
  \item
    The uterine or fallopian tubes allow the oocyte to travel from the
    \_\_\_\_\_\_\_\_\_\_\_\_\_ to the \_\_\_\_\_\_\_\_\_\_\_\_\_\_\_\_.
  \item
    The narrow medial end of the uterine tube is the
    \_\_\_\_\_\_\_\_\_\_\_\_\_\_, and the wide distal portion is the
    \_\_\_\_\_\_\_\_\_\_\_\_\_\_\_\_\_\_ and have projections known as
    fimbriae.
  \item
    Where in the uterine tubes does fertilization occur?
  \item
    How are oocytes transported from the ovaries through the uterine
    tubes?
  \end{enumerate}
\end{enumerate}

\includegraphics[width=4.67917in,height=2.246in,alt={This diagram shows the uterus and ovaries in the center. To the left is a micrograph showing the ultrastructure of the ovaries and to the right is a micrograph showing the ultrastructure of the uterus.}]{images/media/image241.jpeg}

\begin{enumerate}
\def\labelenumi{\arabic{enumi}.}
\setcounter{enumi}{4}
\item
  The uterine tubes are open at the end near the ovaries. Why can this
  potentially be problematic in the event of an infection?
\end{enumerate}

\begin{enumerate}
\def\labelenumi{\Roman{enumi}.}
\setcounter{enumi}{5}
\item
  The Uterus and the Cervix

  \begin{enumerate}
  \def\labelenumii{\arabic{enumii}.}
  \item
    What organ nourishes and supports the growing embryo?
  \item
    The fundus is superior to the opening of the uterine tube. The body
    of the uterus leads to the narrow inferior portion of the uterus,
    the \_\_\_\_\_\_\_\_\_\_\_\_.
  \item
    What helps hold the uterus in position?
  \item
    The perimetrium, myometrium, and endometrium are all layers of the
    uterus. Which one contains the most muscle?
  \item
    What happens to the stratum functionalis of the endometrium during
    menstruation?
  \item
    What hormone maintains the thick stratum functionalis?

    \begin{enumerate}
    \def\labelenumiii{\alph{enumiii})}
    \item
      Once this signal stops and the endometrial tissue in the stratum
      functionalis dies it is shed during \_\_\_\_\_\_\_\_\_\_\_\_\_\_.
    \item
      The first menses after puberty is \_\_\_\_\_\_\_\_\_\_\_\_\_\_\_.
    \end{enumerate}
  \end{enumerate}
\item
  The Menstrual Cycle

  \begin{enumerate}
  \def\labelenumii{\arabic{enumii}.}
  \item
    How variable is the length of a menstrual cycle?
  \item
    What are the three phases of the menstrual cycle?
  \end{enumerate}

  \begin{enumerate}
  \def\labelenumii{\Alph{enumii}.}
  \item
    \includegraphics[width=3.32778in,height=5.69097in,alt={The top panel of this image shows the stages in the follicular phase and how one follicle is selected at the end of this phase. The middle part of this image shows the ovarian cycle phases and the uterine cycle phases. The bottom panel shows the levels of different hormones as a function of time. }]{images/media/image242.jpeg}Menses
    Phase

    \begin{enumerate}
    \def\labelenumiii{\arabic{enumiii}.}
    \item
      What happens during the menses phase?
    \item
      FSH and LH are \_\_\_\_\_\_\_\_\_ during the menses phase.
    \item
      Decline in progesterone triggers the shedding of the
      \_\_\_\_\_\_\_\_\_\_\_\_\_\_\_\_\_ of the endometrium.
    \end{enumerate}
  \item
    Proliferative Phase

    \begin{enumerate}
    \def\labelenumiii{\arabic{enumiii}.}
    \item
      The proliferative phase is when the
      \_\_\_\_\_\_\_\_\_\_\_\_\_\_\_\_ proliferates.
    \item
      What hormone leads to the endometrium rebuilding?
    \item
      Normally high estrogen decreases FSH and LH, and maturation of a
      tertiary follicle. The high estrogen from the dominant follicle
      switches the normally negative feedback to \_\_\_\_\_\_\_\_\_\_
      feedback driving the LH surge for ovulation.
    \end{enumerate}
  \item
    Secretory Phase

    \begin{enumerate}
    \def\labelenumiii{\arabic{enumiii}.}
    \item
      What hormone begins the secretory phase of the menstrual cycle?
    \item
      If pregnancy is to occur how does the uterus prepare during the
      secretory phase?
    \item
      If pregnancy does not occur what happens to the endometrial
      lining?
    \end{enumerate}
  \end{enumerate}
\end{enumerate}

\begin{enumerate}
\def\labelenumi{\Roman{enumi}.}
\setcounter{enumi}{7}
\item
  The Breasts

  \begin{enumerate}
  \def\labelenumii{\arabic{enumii}.}
  \item
    \includegraphics[width=3.17708in,height=1.58125in,alt={This figure shows the anatomy of the breast. The left panel shows the front view and the right panel shows the side view. The main parts are labeled. }]{images/media/image243.jpeg}Breasts
    are an accessory reproductive organ, what is their primary function?
  \item
    Why does the areola have raised areolar glands?
  \item
    \_\_\_\_\_\_\_\_\_\_\_\_\_\_\_\_\_ glands produce milk and are
    modified sweat glands.
  \item
    Milk is produced in alveoli and then passes through
    \_\_\_\_\_\_\_\_\_\_\_\_\_\_\_\_\_\_\_ sinus and ducts to exit the
    breast.
  \item
    What is the role of the suspensory ligaments?
  \item
    How do the normal hormonal fluctuations of the menstrual cycle
    affect the breast?
  \end{enumerate}
\item
  Hormonal Birth Control

  \begin{enumerate}
  \def\labelenumii{\arabic{enumii}.}
  \item
    How do birth controls supplying a constant level of estrogen and
    progesterone help to prevent pregnancy?
  \item
    What is the difference in the birth control that have 21 active days
    and 7 placebos versus the pills that have low dose active pills
    every day?
  \item
    What happens when a birth control pill is missed or delayed?
  \end{enumerate}
\end{enumerate}

27.3. Development of the Male and Female Reproductive Systems

\begin{enumerate}
\def\labelenumi{\Roman{enumi}.}
\item
  Background

  \begin{enumerate}
  \def\labelenumii{\arabic{enumii}.}
  \item
    How much does the reproductive system change between infancy and
    puberty?
  \end{enumerate}
\item
  Development of the Sexual Organs in the Embryo and Fetus

  \begin{enumerate}
  \def\labelenumii{\arabic{enumii}.}
  \item
    Without chemical prompting fertilized eggs develop into what sex?
  \item
    What is the role of the SRY gene in sexual development?
  \item
    What happens when bipotential is expressed to testosterone or when
    it is not exposed to testosterone?
  \item
    What tissues in the reproductive tract are not bipotential?
  \item
    The female duct is the \_\_\_\_\_\_\_\_\_\_\_\_\_\_\_\_ duct, and
    the male duct is the \_\_\_\_\_\_\_\_\_\_\_\_\_\_ duct.
  \end{enumerate}
\item
  Further Sexual Development Occurs at Puberty

  \begin{enumerate}
  \def\labelenumii{\arabic{enumii}.}
  \item
    What is puberty?
  \item
    Aside from maturation of the reproductive system during puberty
    there is also a development of secondary sex characteristics. What
    are secondary sexual characteristics?
  \item
    When is LH production detectable? Does this occur at the start of
    physically visible puberty or before?
  \item
    During puberty what happens to the sensitivity of the hypothalamus
    and pituitary gland?

    \begin{enumerate}
    \def\labelenumiii{\alph{enumiii})}
    \item
      What happens to the sensitivity to the of the gonads to FSH or LH?
    \end{enumerate}
  \item
    FSH and LH increase during puberty, which leads to more sex hormones
    and the initiation of \_\_\_\_\_\_\_\_\_\_\_\_\_\_\_\_\_\_\_\_\_\_\_
    and \_\_\_\_\_\_\_\_\_\_\_\_\_\_\_\_\_\_\_\_\_\_\_\_\_.
  \end{enumerate}
\end{enumerate}

\includegraphics[width=6.16314in,height=5.7037in,alt={This flow chart shows the different hormones and the organs they act on at the onset of puberty. The hypothalamus is shown on top. The right half of the flowchart shows the hormones in females and the left half shows the hormones in males.}]{images/media/image244.jpeg}

\begin{enumerate}
\def\labelenumi{\Alph{enumi}.}
\item
  Signs of Puberty

  \begin{enumerate}
  \def\labelenumii{\arabic{enumii}.}
  \item
    As girls develop through puberty development of breast tissue
    typically occurs first, followed by growth of axillary and pubic
    hair, a growth spurt, and then \_\_\_\_\_\_\_\_\_\_\_\_\_\_\_\_, the
    start of puberty.
  \item
    What is usually the first physical sign of puberty in boys?

    \begin{enumerate}
    \def\labelenumiii{\alph{enumiii})}
    \item
      Testosterone stimulates growth of the
      \_\_\_\_\_\_\_\_\_\_\_\_\_\_\_\_ and thickening of the vocal folds
      which cause the voice to drop in pitch.
    \item
      When does the male growth spurt occur relative to the female
      growth spurt?
    \end{enumerate}
  \end{enumerate}
\end{enumerate}

\section{}\label{section-29}

\section{\texorpdfstring{Chapter 28 }{Chapter 28 }}\label{chapter-28}

\subsection{28.1. Fertilization}\label{fertilization}

\begin{enumerate}
\def\labelenumi{\Roman{enumi}.}
\item
  Background

  \begin{enumerate}
  \def\labelenumii{\arabic{enumii}.}
  \item
    What is fertilization?
  \item
    After fertilization a zygote contains what?
  \end{enumerate}
\item
  Transit of Sperm

  \begin{enumerate}
  \def\labelenumii{\arabic{enumii}.}
  \item
    Why are millions of sperm necessary for one to fertilize the egg?
  \item
    What is capacitation? How does this assist in fertilization and give
    sperm the ``capacity'' to fertilize an oocyte?
  \end{enumerate}
\item
  Contact Between Sperm and Oocyte

  \begin{enumerate}
  \def\labelenumii{\arabic{enumii}.}
  \item
    The two protective layers around the oocyte are the
    \_\_\_\_\_\_\_\_\_\_\_\_\_\_\_ \_\_\_\_\_\_\_\_\_\_\_\_\_\_\_ which
    is more external, and the \_\_\_\_\_\_\_\_\_\_\_\_
    \_\_\_\_\_\_\_\_\_\_\_\_\_\_\_\_ that surrounds the plasma membrane.
  \item
    \includegraphics[width=4.32778in,height=2.97917in,alt={This figure shows the process of sperm fertilizing an egg. There are many sperm trying to attach to the egg.}]{images/media/image245.jpeg}What
    purpose do the chemical attractants serve that are released by the
    corona radiata?
  \item
    What is the acrosomal reaction, and what are the enzymes in the
    acrosome used for?
  \item
    Is one sperm sufficient to degrade the corona radiata and zona
    pellucida?
  \item
    How does the oocyte prevent polyspermy?
  \item
    Is the cortical reaction the fast or slow block?
  \item
    What is the fertilization membrane?
  \end{enumerate}
\item
  The Zygote

  \begin{enumerate}
  \def\labelenumii{\arabic{enumii}.}
  \item
    Upon fertilization the oocyte must complete
    \_\_\_\_\_\_\_\_\_\_\_\_\_ in order to become an ovum.
  \item
    After the female and male genetic material intermingle the diploid
    \_\_\_\_\_\_\_\_\_\_\_\_\_\_\_\_\_\_ results.
  \item
    What are fraternal twins, and how do they result?
  \item
    How do identical twins develop?

    \begin{enumerate}
    \def\labelenumiii{\alph{enumiii})}
    \item
      How frequently does this occur?
    \end{enumerate}
  \end{enumerate}
\end{enumerate}

28.2. Embryonic Development

\begin{enumerate}
\def\labelenumi{\Roman{enumi}.}
\item
  Background

  \begin{enumerate}
  \def\labelenumii{\arabic{enumii}.}
  \item
    The time for full development of a fetus in utero is
    \_\_\_\_\_\_\_\_\_\_\_\_\_\_\_\_\_\_\_\_\_.
  \item
    During weeks 3-8 of development the developing human is an
    \_\_\_\_\_\_\_\_\_\_\_\_\_\_\_.
  \item
    When is a developing human a fetus?
  \end{enumerate}
\item
  \includegraphics[width=2.39514in,height=2.09861in,alt={This figure shows the different stages of cell divisions taking place before the embryo is formed. The top panel shows the cell divisions occurring in the uterine tube and the bottom panel shows the cell divisions occurring in the uterus.}]{images/media/image246.jpeg}Pre-implantation
  of Embryonic Development

  \begin{enumerate}
  \def\labelenumii{\arabic{enumii}.}
  \item
    The conceptus is the zygote and what else?
  \item
    The first few divisions, or cleavages of the cells creates daughter
    cells known as \_\_\_\_\_\_\_\_\_\_\_\_\_\_\_\_\_\_.
  \item
    What is the 16 cell conceptus known as?
  \item
    The blastocoel is a fluid-filled cavity inside of the
    \_\_\_\_\_\_\_\_\_\_\_\_\_\_\_\_.
  \item
    The inner cell mass will become the \_\_\_\_\_\_\_\_\_\_ and the
    trophoblasts will develop into the
    \_\_\_\_\_\_\_\_\_\_\_\_\_\_\_\_\_\_.
  \end{enumerate}
\item
  \includegraphics[width=2.97292in,height=2.71875in,alt={This figure shows the different stages in pre-embryonic development. A diagram of the uterus is shown and from this image, eight callouts show the different stages of development.}]{images/media/image247.jpeg}Implantation

  \begin{enumerate}
  \def\labelenumii{\arabic{enumii}.}
  \item
    How long after fertilization until implantation occurs?
  \item
    Where in the uterus does implantation usually occur?
  \item
    Is implantation normally successful?
  \item
    What is the synctiotrophoblast, and how does it help to keep the
    blastocyst adhered?
  \item
    What is hCG, and why is it commonly used in pregnancy tests?
  \item
    \includegraphics[width=2.91667in,height=3.64583in,alt={This figure shows the different steps during implantation. The top panel shows how the blastocyst burrows into the endometrium. The middle panel shows the blastocyst completely surrounded by the endometrium. The bottom panel shows the implanted embryo growing in the uterus.}]{images/media/image248.jpeg}What
    is an ectopic pregnancy and why is this not conducive to a healthy
    pregnancy?
  \item
    Where does implantation occur in placenta previa?
  \end{enumerate}
\item
  Embryonic Membranes

  \begin{enumerate}
  \def\labelenumii{\arabic{enumii}.}
  \item
    The amniotic cavity appears between what two layers?
  \item
    What fills the amnion during the second week?
  \item
    How does the amniotic fluid protect the developing fetus?
  \end{enumerate}
\end{enumerate}

\includegraphics[width=3.15625in,height=2.17615in,alt={This image shows the development of the amniotic cavity and the location of the embryonic disc.}]{images/media/image249.jpeg}

\begin{enumerate}
\def\labelenumi{\arabic{enumi}.}
\setcounter{enumi}{3}
\item
  What does the yolk sac provide?
\item
  That allantois becomes part of what organ?
\end{enumerate}

\begin{enumerate}
\def\labelenumi{\Roman{enumi}.}
\setcounter{enumi}{4}
\item
  \includegraphics[width=3.27083in,height=2.91667in,alt={This image shows the different germ layers. The top panel shows the epiblast and trophoblast cells in the early stages of development. The bottom panel shows the three germ layers: the endoderm, ectoderm, and mesoderm. All the other major parts are also labeled.}]{images/media/image250.jpeg}Embryogenesis

  \begin{enumerate}
  \def\labelenumii{\arabic{enumii}.}
  \item
    The two-layer disc of cells becomes a three layered disc during
    \_\_\_\_\_\_\_\_\_\_\_\_\_\_\_\_\_\_.
  \item
    The primitive streak appears on the \_\_\_\_\_\_\_\_\_\_\_\_\_\_
    surface.
  \item
    What are the three layers that develop during the third week of
    development?
  \item
    What does the endoderm become in the embryo?
  \item
    Which layer becomes the hair, nails, and skin?
  \end{enumerate}
\end{enumerate}

\includegraphics[width=3.33333in,height=2.875in,alt={This image shows the structure of the embryo in the third week of development. Under the image, three callouts list the different organ systems into which each germ layer develops.}]{images/media/image251.jpeg}

\begin{enumerate}
\def\labelenumi{\Roman{enumi}.}
\setcounter{enumi}{5}
\item
  Development of the Placenta

  \begin{enumerate}
  \def\labelenumii{\arabic{enumii}.}
  \item
    What takes over feeding the embryo during weeks 4-12?
  \item
    How does the placenta connect to the conceptus?
  \item
    The chorionic membrane's chorionic villi form what part of the
    placenta?
  \end{enumerate}
\end{enumerate}

\includegraphics[width=5.47707in,height=2.83333in,alt={This figure shows the location and structure of the placenta. The left panel shows a fetus in the womb. The right panel shows a magnified view of a small region including the placenta and the blood vessels.}]{images/media/image252.jpeg}

\begin{enumerate}
\def\labelenumi{\arabic{enumi}.}
\setcounter{enumi}{3}
\item
  What is placentation, and when does it occur?
\item
  What does the placenta provide for the embryo and fetus?
\item
  What are substances can pass through the placenta through diffusion?
\item
  What is moved via active transport?
\item
  Why is it important that maternal and fetal blood do not interact?
\end{enumerate}

\begin{enumerate}
\def\labelenumi{\Roman{enumi}.}
\setcounter{enumi}{6}
\item
  Organogenesis

  \begin{enumerate}
  \def\labelenumii{\arabic{enumii}.}
  \item
    \includegraphics[width=3.15625in,height=3.64167in,alt={This multi-part image shows the formation of the neural tube and the notochord. The top panel shows the ectoderm and mesoderm. The second panel shows the neural plate starting to fold over and the third panel shows the closed neural plate forming the neural tube. The fourth panel shows the mesoderm-derived notochord under the neural tube.}]{images/media/image253.jpeg}Neurulation
    developing rudimentary portion of what body system?
  \item
    The neural tube is composed of the folds of the \_\_\_\_\_\_\_\_\_\_
    \_\_\_\_\_\_\_ converging.
  \item
    The notochord arises from what germ layer?
  \item
    The somites develop into what part of the skeleton?
  \item
    Through embryonic folding the embryo assumes a C-Shape. The yolk sac
    is where in the fold, and becomes what structure?
  \end{enumerate}
\end{enumerate}

\includegraphics[width=3.67708in,height=3.36595in,alt={This multipart image shows the folding of the embryo. Each of the six panels shows a progression of steps in which the embryo folds on itself.}]{images/media/image254.jpeg}

\begin{enumerate}
\def\labelenumi{\arabic{enumi}.}
\setcounter{enumi}{5}
\item
  Development of the rudimentary structures of all organs and tissues is
  a process known as \_\_\_\_\_\_\_\_\_\_\_\_\_\_\_\_\_\_\_\_.
\item
  When does the heart start beating in development, and when does it
  start pumping blood?
\item
  What allows for the paddle shaped hands to develop into discrete
  fingers?
\item
  Ossification allows for what to develop?
\end{enumerate}

\subsection{28.3. Fetal Development}\label{fetal-development}

\includegraphics[width=2.34375in,height=4.41319in,alt={This flow chart shows how the sexual organs develop in embryos. The left side of the flow chart shows the development of male organs and the right side of the flow chart shows the development of female organs.}]{images/media/image255.jpeg}

\begin{enumerate}
\def\labelenumi{\Roman{enumi}.}
\item
  Background

  \begin{enumerate}
  \def\labelenumii{\arabic{enumii}.}
  \item
    When is a developing human a fetus?
  \item
    At the end of fetal development what is the newborn capable of?
  \end{enumerate}
\item
  Sexual Differentiation

  \begin{enumerate}
  \def\labelenumii{\arabic{enumii}.}
  \item
    Does sexual differentiation occur during the embryonic or fetal
    period of development?
  \item
    Development of testes and a vas deferens occur in
    \_\_\_\_\_\_\_\_\_\_\_\_ development, and development of ovaries and
    the uterus occurs in \_\_\_\_\_\_\_\_\_\_\_\_\_\_\_\_ development.
  \end{enumerate}
\item
  The Fetal Circulatory System

  \begin{enumerate}
  \def\labelenumii{\arabic{enumii}.}
  \item
    The fetal circulation is integrated with what during prenatal
    development?
  \item
    What is a shunt?

    \begin{enumerate}
    \def\labelenumiii{\alph{enumiii})}
    \item
      Why is a shunt necessary in fetal development?
    \end{enumerate}
  \item
    What provides the fetus with oxygen while the respiratory system
    can't oxygenate itself?
  \item
    What semifunctional organ does the ductus venosus mostly skip over?
  \item
    What do the foramen ovale and ductus arteriosus allow the
    circulatory system to skip over?
  \item
    \includegraphics[width=3.77083in,height=3.28242in,alt={This figure shows a baby in the center of the image. To the left, is a panel showing the umbilical cord and how blood is supplied to the baby in the womb. Two panels on the right show the circulation of blood inside the baby's body.}]{images/media/image256.jpeg}Why
    is the blood in the aorta only partially oxygenated?
  \item
    How does the deoxygenated fetal blood reach the umbilical arteries?
  \end{enumerate}
\item
  Other Organ Systems

  \begin{enumerate}
  \def\labelenumii{\arabic{enumii}.}
  \item
    What takes over erythrocyte production from the liver during fetal
    development?
  \item
    Sensory organs develop during what weeks of development?

    \begin{enumerate}
    \def\labelenumiii{\alph{enumiii})}
    \item
      What is meconium, and where does it accumulate?
    \end{enumerate}
  \item
    What is quickening and when is it felt?
  \item
    What purpose does vernix caseosa serve in childbirth?
  \item
    Why are full term babies not seen with lanugo, but some premature
    babies have it?
  \item
    Why is it important for the fetus to produce surfactant during weeks
    21-30?
  \end{enumerate}
\end{enumerate}

28.4. Maternal Changes During Pregnancy, Labor, and Birth

\begin{enumerate}
\def\labelenumi{\Roman{enumi}.}
\item
  \includegraphics[width=2.46875in,height=3.29167in,alt={This figure shows a woman's body and marks the size of the uterus as it grows throughout pregnancy.}]{images/media/image257.jpeg}Background

  \begin{enumerate}
  \def\labelenumii{\arabic{enumii}.}
  \item
    How long is a full-term pregnancy?
  \item
    How long is each of the trimesters of pregnancy?
  \end{enumerate}
\item
  Effects of Hormones

  \begin{enumerate}
  \def\labelenumii{\arabic{enumii}.}
  \item
    During weeks 7-12 what primarily generates the hormones that nourish
    the blastocyst?
  \item
    When does the placenta take over as the primary endocrine organ of
    pregnancy?
  \item
    Estrogens climb throughout pregnancy what actions do they have?
  \item
    How does relaxin prepare the mother's body for childbirth?
  \item
    Why is thyroid hormone increased during pregnancy?
  \item
    Prolactin stimulates enlargement for what gland?
  \end{enumerate}
\item
  Weight Gain

  \begin{enumerate}
  \def\labelenumii{\arabic{enumii}.}
  \item
    What is the most obvious sign of pregnancy?
  \item
    What contributes to the weight gain?
  \item
    When does a mother need to consume more calories to maintain a
    healthy pregnancy?
  \end{enumerate}
\item
  Changes in Organ Systems During Pregnancy

  \begin{enumerate}
  \def\labelenumii{\Alph{enumii}.}
  \item
    Digestive and Urinary System Changes

    \begin{enumerate}
    \def\labelenumiii{\arabic{enumiii}.}
    \item
      What may be responsible for ``morning sickness?''

      \begin{enumerate}
      \def\labelenumiv{\alph{enumiv})}
      \item
        When does it usually subside?
      \end{enumerate}
    \item
      Why does heartburn occur in later stages of pregnancy?
    \item
      Why is urination more frequent?
    \end{enumerate}
  \item
    Circulatory System Changes

    \begin{enumerate}
    \def\labelenumiii{\arabic{enumiii}.}
    \item
      Why does blood volume increase with pregnancy?
    \end{enumerate}
  \item
    Respiratory System Changes

    \begin{enumerate}
    \def\labelenumiii{\arabic{enumiii}.}
    \item
      What happens to the respiratory minute volume during pregnancy?

      \begin{enumerate}
      \def\labelenumiv{\alph{enumiv})}
      \item
        Why is this change necessary?
      \end{enumerate}
    \item
      Why does lightening help with dyspnea during pregnancy?
    \end{enumerate}
  \item
    Integumentary System Changes

    \begin{enumerate}
    \def\labelenumiii{\arabic{enumiii}.}
    \item
      Why do stretch mark appear during pregnancy?
    \item
      What causes darkening of the areolae and the linea nigra?
    \end{enumerate}
  \end{enumerate}
\item
  \includegraphics[width=3.70764in,height=2.07639in,alt={A graph hormone concentration versus week of pregnancy shows how three hormones vary throughout pregnancy.}]{images/media/image258.jpeg}Physiology
  of Labor

  \begin{enumerate}
  \def\labelenumii{\arabic{enumii}.}
  \item
    When does parturition usually occur with multiple fetuses?
  \item
    What inhibits uterine contractions during the first several months
    of pregnancy?
  \item
    How does the ratio of estrogen to progesterone change to make the
    myometrium more sensitive to contraction signals?
  \item
    What are Braxton Hicks contractions?

    \begin{enumerate}
    \def\labelenumiii{\alph{enumiii})}
    \item
      Is this labor?
    \end{enumerate}
  \item
    Oxytocin driven contractions occur in a \_\_\_\_\_\_\_\_\_\_\_\_\_\_
    feedback loop.
  \item
    What is true labor?
  \end{enumerate}
\item
  \includegraphics[width=2.69236in,height=4.2125in,alt={This multi-part figure shows the different stages of childbirth. The top panel shows dilation, the middle panel shows birth and the bottom panel shows afterbirth delivery.}]{images/media/image259.jpeg}Stages
  of Childbirth

  \begin{enumerate}
  \def\labelenumii{\Alph{enumii}.}
  \item
    Cervical Dilation

    \begin{enumerate}
    \def\labelenumiii{\arabic{enumiii}.}
    \item
      How much must the cervix dilate to accommodate delivery?
    \item
      Dilation normally takes how long?
    \item
      How does the frequency of contractions change throughout labor?
    \item
      Why is the relaxation period between contractions necessary for
      fetal health?
    \end{enumerate}
  \item
    Expulsion Stage

    \begin{enumerate}
    \def\labelenumiii{\arabic{enumiii}.}
    \item
      When does the expulsion stage begin?
    \item
      What is the vertex presentation?
    \item
      What is breech presentation?
    \item
      Why was an episiotomy common?
    \item
      When the newborns head is presented what tasks are performed?
    \end{enumerate}
  \item
    Afterbirth

    \begin{enumerate}
    \def\labelenumiii{\arabic{enumiii}.}
    \item
      What is the afterbirth?
    \item
      How long is the postpartum period?
    \item
      Why is it important that all of the placenta is expelled?
    \item
      What purpose does involution of the uterus serve?
    \item
      How long is lochia discharge continue after pregnancy?
    \end{enumerate}
  \end{enumerate}
\end{enumerate}

28.5. Adjustments of the Infant at Birth and Postnatal Stages

\begin{enumerate}
\def\labelenumi{\Roman{enumi}.}
\item
  Background

  \begin{enumerate}
  \def\labelenumii{\arabic{enumii}.}
  \item
    Why is birth a crisis situation for a fetus?
  \item
    How long is the neonatal period?
  \end{enumerate}
\item
  Respiratory Adjustments

  \begin{enumerate}
  \def\labelenumii{\arabic{enumii}.}
  \item
    How do labor contractions constricting the umbilical cord stimulate
    breathing in the newborn?
  \item
    What respiratory and vascular adjustments are made upon the first
    breath?
  \end{enumerate}
\item
  Circulatory Adjustments

  \begin{enumerate}
  \def\labelenumii{\arabic{enumii}.}
  \item
    \includegraphics[width=3.34375in,height=2.82708in,alt={This figure illustrates the circulatory system in a newborn. The left image in both panels shows the blood circulation before birth and the right image shows the blood circulation after birth.}]{images/media/image260.jpeg}How
    do the umbilical blood vessels occlude in the absence of clamping?
  \item
    What happens to the ductus venosus?
  \item
    What closes the foramen ovale?
  \end{enumerate}
\item
  Thermoregulatory Adjustments

  \begin{enumerate}
  \def\labelenumii{\arabic{enumii}.}
  \item
    Why is it difficult for a newborn to regulate their body heat
    compared to an adult?
  \item
    What is nonshivering thermogenesis and how does it us brown adipose
    tissue to generate heat?
  \item
    How does brown fat differ from white fat?
  \end{enumerate}
\item
  Gastrointestinal and Urinary Adjustments

  \begin{enumerate}
  \def\labelenumii{\arabic{enumii}.}
  \item
    How does the sterile fetal intestinal tract develop normal bacterial
    flora?
  \item
    Why do newborns produce dilute urine?
  \end{enumerate}
\end{enumerate}

28.6. Lactation

\begin{enumerate}
\def\labelenumi{\Roman{enumi}.}
\item
  Background

  \begin{enumerate}
  \def\labelenumii{\arabic{enumii}.}
  \item
    What is lactation?
  \item
    What does breast milk provide for the infant and for the mother?
  \end{enumerate}
\item
  \includegraphics[width=3.35833in,height=4.28681in,alt={This figure shows the process of let down reflex, the process in which the brain receives sensory impulses to release the hormones necessary for producing and discharging milk to the suckling newborn.}]{images/media/image261.jpeg}Structure
  of the Lactating Breast

  \begin{enumerate}
  \def\labelenumii{\arabic{enumii}.}
  \item
    Mammary glands are modified \_\_\_\_\_\_\_\_\_\_\_\_ glands.
  \item
    What transports milk in the breast?
  \item
    Milk is secreted from \_\_\_\_\_\_\_\_\_\_\_\_\_\_, fills the
    alveoli, and is squeezed into the ducts.
  \item
    Why do Montgomery glands secrete oil?
  \end{enumerate}
\item
  The Process of Lactation

  \begin{enumerate}
  \def\labelenumii{\arabic{enumii}.}
  \item
    Prolactin has what effect on the maternal body?
  \item
    Prolactin is high during pregnancy, but no milk is produced. What
    inhibits milk production?
  \item
    How does suckling lead to milk secretion?
  \item
    What role do oxytocin play in the milk letdown reflex?
  \item
    What effect does frequent milk removal have on prolactin?
  \end{enumerate}
\item
  Changes in the Composition of Breast Milk

  \begin{enumerate}
  \def\labelenumii{\arabic{enumii}.}
  \item
    What is colostrum?

    \begin{enumerate}
    \def\labelenumiii{\alph{enumiii})}
    \item
      When is mature breast milk secreted?
    \end{enumerate}
  \item
    Colostrum is rich with immunoglobulins. What do these do for the
    newborn?
  \item
    Why is cow's milk not a sufficient replacement for breast milk?
  \item
    Milk produced adjusts to the amount needed by the infant(s). What
    happens to produced milk when suckling stops?
  \item
    What is the difference between foremilk and hindmilk?
  \item
    The laxative properties of breast milk help to expel meconium. Why
    is this necessary?

    \begin{enumerate}
    \def\labelenumiii{\alph{enumiii})}
    \item
      Why is elevated bilirubin especially dangerous to a newborn?
    \end{enumerate}
  \end{enumerate}
\end{enumerate}

28.7. Patterns of Inheritance

\begin{enumerate}
\def\labelenumi{\Roman{enumi}.}
\item
  Background
\item
  From Genotype to Phenotype

  \begin{enumerate}
  \def\labelenumii{\arabic{enumii}.}
  \item
    Each person has a \_\_\_\_\_\_\_\_ pairs of chromosomes.
  \item
    In a karyotype there are \_\_ pairs of autosomal chromosomes and
    \_\_\_ pair of sex chromosomes.
  \item
    What is the difference between a genotype and a phenotype?
  \item
    Each parent supplies how much of a person's chromosomes?
  \item
    Do alleles need to be the same on a pair of chromosomes?

    \begin{enumerate}
    \def\labelenumiii{\alph{enumiii})}
    \item
      What is it called when they are identical?
    \item
      What is it called when they are different?
    \end{enumerate}
  \item
    What is the different between a dominant and nondominant allele?
  \item
    Are all characteristic determined by a single allele?
  \end{enumerate}
\item
  \includegraphics[width=3.24722in,height=2.72778in,alt={This diagram shows the genetics experiment conducted by Mendel. The top panel shows the offspring from first generation cross and the bottom panel shows the offspring from the second generation cross.}]{images/media/image262.jpeg}Mendel's
  Theory of Inheritance

  \begin{enumerate}
  \def\labelenumii{\arabic{enumii}.}
  \item
    What did Mendel use to learn how physical characteristics were
    transferred to subsequent generations?
  \item
    Why was the tallness in the peas dominant and dwarfism recessive?

    \begin{enumerate}
    \def\labelenumiii{\alph{enumiii})}
    \item
      Tallness and dwarfism were called these variations a
      \_\_\_\_\_\_\_\_\_\_\_\_.
    \end{enumerate}
  \item
    What was the ratio observed in the second-generation offspring?
  \item
    What lead Mendel to believe in ``heritable factors'' that were
    transmitted how?
  \item
    What is codominance?
  \item
    What is random segregation?
  \item
    What is independent assortment?

    \begin{enumerate}
    \def\labelenumiii{\alph{enumiii})}
    \item
      How does independent assortment lead to more variety in offspring?
    \end{enumerate}
  \item
    Why are humans more difficult to study than pea plants?
  \end{enumerate}
\item
  \includegraphics[width=3.64583in,height=1.51042in,alt={This 2-by-2 Punnet square shows fifty percent dominant and fifty percent recessive offspring.}]{images/media/image263.jpeg}Autosomal
  Dominant Inheritance

  \begin{enumerate}
  \def\labelenumii{\arabic{enumii}.}
  \item
    What does autosomal dominant mean?
  \item
    Using a Punnett square what is the likelihood of a child inheriting
    an autosomal dominant condition from a homozygous recessive parent
    and a heterozygous parent?
  \end{enumerate}
\item
  \includegraphics[width=4.52986in,height=2.44583in,alt={In this figure, the offspring of a carrier father and carrier mother are shown. The first generation has one unaffected son, one affected daughter and one carrier son and one carrier daughter. The second generation cross shows seventy five percent unaffected and twenty five percent affected with cystic fibrosis.}]{images/media/image264.jpeg}Autosomal
  Recessive Inheritance

  \begin{enumerate}
  \def\labelenumii{\arabic{enumii}.}
  \item
    What is autosomal recessive?
  \item
    Would someone who is heterozygous display symptoms of a autosomal
    recessive disease?
  \item
    A child born to two carriers of an autosomal recessive disease has
    what percentage change of inheriting the disease?

    \begin{enumerate}
    \def\labelenumiii{\alph{enumiii})}
    \item
      What about a child born to a carrier and someone unaffected?
    \end{enumerate}
  \end{enumerate}
\item
  \includegraphics[width=2.44508in,height=3.94792in,alt={This image shows the generations resulting from an X-linked dominant, affected father in the top panel and the generations resulting from an X-linked dominant, affected mother in the bottom panel.}]{images/media/image265.jpeg}X-linked
  Dominant or Recessive Inheritance

  \begin{enumerate}
  \def\labelenumii{\arabic{enumii}.}
  \item
    What is an X-linked transmission pattern?
  \item
    If a father has an X-linked dominant condition how many of his
    daughters will inherit it? Why?

    \begin{enumerate}
    \def\labelenumiii{\alph{enumiii})}
    \item
      What is a mother has it? What is the odds of her children having
      the disease?
    \end{enumerate}
  \item
    What are some examples of X-linked recessive inheritance traits?
  \item
    Why can't males be carriers of X-linked recessive diseases?
  \end{enumerate}
\end{enumerate}

\includegraphics[width=3.41667in,height=2.58333in,alt={This figure shows the offspring from a carrier mother with the X-linked recessive inheritance.}]{images/media/image266.jpeg}

\begin{enumerate}
\def\labelenumi{\Roman{enumi}.}
\setcounter{enumi}{6}
\item
  Other Inheritance Patterns: Incomplete Dominance, Codominance, and
  Lethal Alleles

  \begin{enumerate}
  \def\labelenumii{\arabic{enumii}.}
  \item
    What is incomplete dominance?
  \item
    What is an example of incomplete dominance in humans?
  \item
    With codominance what is expressed?

    \begin{enumerate}
    \def\labelenumiii{\alph{enumiii})}
    \item
      What is an example of codominance in humans?
    \end{enumerate}
  \item
    What would a person need to inherit to have O blood type?
  \item
    What is a recessive lethal allele?

    \begin{enumerate}
    \def\labelenumiii{\alph{enumiii})}
    \item
      What is an example?
    \end{enumerate}
  \item
    Are dominant lethal inheritance patterns common?

    \begin{enumerate}
    \def\labelenumiii{\alph{enumiii})}
    \item
      Why is Huntington's passed on despite being dominant lethal?
    \end{enumerate}
  \end{enumerate}
\item
  Mutations

  \begin{enumerate}
  \def\labelenumii{\arabic{enumii}.}
  \item
    Do all mutations affect a person's phenotype?
  \item
    What can cause a mutation?
  \end{enumerate}
\item
  Chromosomal Disorders

  \begin{enumerate}
  \def\labelenumii{\arabic{enumii}.}
  \item
    What is an example of a disease caused by an incorrect number of
    chromosomes?
  \item
    What is nondisjunction and what can it result in?
  \item
    What is monosomy, and how does it relate to Turner Syndrome?
  \end{enumerate}
\end{enumerate}

This OpenStax ancillary resource is © Rice University under a CC-BY 4.0
International license; it may be reproduced or modified but must be
attributed to OpenStax, Rice University and any changes must be noted.

\end{document}
